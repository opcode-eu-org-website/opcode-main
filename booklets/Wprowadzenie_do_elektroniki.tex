% Copyright (c) 2020-2021 Robert Ryszard Paciorek <rrp@opcode.eu.org>
% 
% MIT License
% 
% Permission is hereby granted, free of charge, to any person obtaining a copy
% of this software and associated documentation files (the "Software"), to deal
% in the Software without restriction, including without limitation the rights
% to use, copy, modify, merge, publish, distribute, sublicense, and/or sell
% copies of the Software, and to permit persons to whom the Software is
% furnished to do so, subject to the following conditions:
% 
% The above copyright notice and this permission notice shall be included in all
% copies or substantial portions of the Software.
% 
% THE SOFTWARE IS PROVIDED "AS IS", WITHOUT WARRANTY OF ANY KIND, EXPRESS OR
% IMPLIED, INCLUDING BUT NOT LIMITED TO THE WARRANTIES OF MERCHANTABILITY,
% FITNESS FOR A PARTICULAR PURPOSE AND NONINFRINGEMENT. IN NO EVENT SHALL THE
% AUTHORS OR COPYRIGHT HOLDERS BE LIABLE FOR ANY CLAIM, DAMAGES OR OTHER
% LIABILITY, WHETHER IN AN ACTION OF CONTRACT, TORT OR OTHERWISE, ARISING FROM,
% OUT OF OR IN CONNECTION WITH THE SOFTWARE OR THE USE OR OTHER DEALINGS IN THE
% SOFTWARE.

\documentclass{pdfBooklets}

\title{Programowanie w elektronice: Wprowadzenie do elektroniki}
\author{%
	Projekt ,,Matematyka dla Ciekawych Świata'',\\
	Robert Ryszard Paciorek\\\normalsize\ttfamily <rrp@opcode.eu.org>
}
\date  {2022-12-09}

\makeatletter\hypersetup{
	pdftitle = {\@title}, pdfauthor = {\@author}
}\makeatother

\begin{document}

\maketitle

% Copyright (c) 2017-2020 Matematyka dla Ciekawych Świata (http://ciekawi.icm.edu.pl/)
% Copyright (c) 2017-2020 Robert Ryszard Paciorek <rrp@opcode.eu.org>
% 
% MIT License
% 
% Permission is hereby granted, free of charge, to any person obtaining a copy
% of this software and associated documentation files (the "Software"), to deal
% in the Software without restriction, including without limitation the rights
% to use, copy, modify, merge, publish, distribute, sublicense, and/or sell
% copies of the Software, and to permit persons to whom the Software is
% furnished to do so, subject to the following conditions:
% 
% The above copyright notice and this permission notice shall be included in all
% copies or substantial portions of the Software.
% 
% THE SOFTWARE IS PROVIDED "AS IS", WITHOUT WARRANTY OF ANY KIND, EXPRESS OR
% IMPLIED, INCLUDING BUT NOT LIMITED TO THE WARRANTIES OF MERCHANTABILITY,
% FITNESS FOR A PARTICULAR PURPOSE AND NONINFRINGEMENT. IN NO EVENT SHALL THE
% AUTHORS OR COPYRIGHT HOLDERS BE LIABLE FOR ANY CLAIM, DAMAGES OR OTHER
% LIABILITY, WHETHER IN AN ACTION OF CONTRACT, TORT OR OTHERWISE, ARISING FROM,
% OUT OF OR IN CONNECTION WITH THE SOFTWARE OR THE USE OR OTHER DEALINGS IN THE
% SOFTWARE.

% BEGIN: Elektronika - intro
Elektronika zajmuje się wytwarzaniem i przetwarzaniem sygnałów w postaci prądów i napięć elektrycznych.
Zjawisko prądu związane jest z przepływem ładunku (z uporządkowanym ruchem nośników ładunku), aby wystąpiło konieczna jest różnica potencjałów (napięcie) pomiędzy końcami przewodnika, prowadzi ono do neutralizacji tej różnicy.
Dlatego dla podtrzymania stałej różnicy potencjałów konieczne jest istnienie źródeł prądu, prowadzących do rozdzielania ładunków dodatnich od ujemnych.

\section{Podstawowe pojęcia}

\subsection{Napięcie elektryczne}
    Napięcie elektryczne $U$ pomiędzy punktem A i B (jakiegoś obwodu)
    jest to różnica potencjału elektrycznego w punkcie A i w punkcie B.
\subsection{Potencjał elektryczny}
    Potencjał elektryczny $V$ w punkcie A
    jest skalarną wielkością charakteryzującą pole elektryczne w danym punkcie. Odpowiada pracy którą trzeba by wykonać aby przenieść ładunek $q$ z tego punktu do nieskończoności podzielonej przez wielkość tego ładunku (jest niezależny od wartości $q$).
    W elektronice używa się wartości potencjałów względem umownego potencjału zerowego GND (co umożliwia traktowanie ich jako różnic potencjałów - napięć elektrycznych), w efekcie tego określenia "(stałe) napięcie" i "potencjał" bardzo często stosowane są zamiennie. 
\subsection{Masa}
   Masa (oznaczana jako GND) jest to
   umowny potencjał zerowy, względem którego wyraża się inne potencjały w układzie (co umożliwia traktowanie ich jako różnic potencjałów - napięć elektrycznych). Potencjał ten może być równy potencjałowi ziemi (masie ochronnej PE), bądź może być z nim nie związany (układy izolowane).

\begin{ProTip}{Schematy elektryczne wg. elektronika}
Typowo elektronicy na schematach nie rysują źródeł napięcia (np. w postaci symbolu baterii\footnote{chyba że chodzi o podkreślenie, iż dane zasilanie faktycznie odbywa się z baterii lub akumulatora}),
zamiast tego umieszczają znaczniki potencjałów zasilania (np. +5V, +3V3, Vcc, Vbus) względem masy i znaczniki masy (GND, {\Symbola ⏚}).

\vspace{3pt}Typowo potencjały wyższe umieszcza się na schemacie wyżej a niższe niżej (czyli 5V będzie na górze, a GND na dole), a przepływ prąd odbywa się w relacji od lewej do prawej i z góry na dół.
Jest to ogólna reguła, ułatwiająca czytanie schematów, nie jest ona jednak wyrocznią i trafiają się od niej odstępstwa, podyktowane zwiększeniem czytelności schematu.

\vspace{3pt}Schematy zamieszczane w tym skrypcie rysowane są według tych zasad.
\end{ProTip}

\subsection{Natężenie prądu}
    Natężenie prądu elektrycznego $I$ (określane skrótowo jako prąd)
    jest to stosunek przemieszczonego ładunku do czasu jego przepływu.

\section{Prawo Ohma}
Dla elementów liniowych (np. zwykły kawałek przewodu) zachodzi proporcjonalność natężenia prądu płynącego przez taki element do napięcia pomiędzy jego końcami: $R=\frac{U}{I}$.
Zależność ta nosi nazwę prawa Ohma\footnote{Prawo Ohma nie jest uniwersalnym prawem przyrody, a jedynie relacją empiryczną spełnioną w pewnym zakresie parametrów dla niektórych materiałów.}, a stosunek ten nazywamy oporem (rezystancję).

\section{Prawa Kirchhoffa}
Węzeł układu (sam w sobie, pomijając zjawiska pasożytnicze) nie jest w stanie gromadzić ładunku elektrycznego zatem: \emph{Suma prądów wpływających do węzła jest równa sumie prądów wypływających z tego węzła.}

Jeżeli rozważamy obwód zamknięty od punktu A z potencjałem $V_A$ to sumując napięcia na kolejnych elementach obwodu (oporach, źródłach napięciowych, etc) z uwzględnieniem ich znaku gdy wrócimy do punktu A to potencjał nadal musi wynosić $V_A$, zatem: \emph{Suma spadków napięć w zamkniętym obwodzie jest równa zeru.}
% END: Elektronika - intro

% Copyright (c) 2017-2020 Matematyka dla Ciekawych Świata (http://ciekawi.icm.edu.pl/)
% Copyright (c) 2017-2020 Robert Ryszard Paciorek <rrp@opcode.eu.org>
% 
% MIT License
% 
% Permission is hereby granted, free of charge, to any person obtaining a copy
% of this software and associated documentation files (the "Software"), to deal
% in the Software without restriction, including without limitation the rights
% to use, copy, modify, merge, publish, distribute, sublicense, and/or sell
% copies of the Software, and to permit persons to whom the Software is
% furnished to do so, subject to the following conditions:
% 
% The above copyright notice and this permission notice shall be included in all
% copies or substantial portions of the Software.
% 
% THE SOFTWARE IS PROVIDED "AS IS", WITHOUT WARRANTY OF ANY KIND, EXPRESS OR
% IMPLIED, INCLUDING BUT NOT LIMITED TO THE WARRANTIES OF MERCHANTABILITY,
% FITNESS FOR A PARTICULAR PURPOSE AND NONINFRINGEMENT. IN NO EVENT SHALL THE
% AUTHORS OR COPYRIGHT HOLDERS BE LIABLE FOR ANY CLAIM, DAMAGES OR OTHER
% LIABILITY, WHETHER IN AN ACTION OF CONTRACT, TORT OR OTHERWISE, ARISING FROM,
% OUT OF OR IN CONNECTION WITH THE SOFTWARE OR THE USE OR OTHER DEALINGS IN THE
% SOFTWARE.

% BEGIN: Elektronika - bierne
\section{Elementy bierne}

\begin{wrapfigure}{r}{0.35\textwidth}
  \begin{center}
    \vspace{-40pt}
    \includegraphics[width=0.3\textwidth]{img/elektronika/symbole}
    \vspace{-20pt}
  \end{center}
\end{wrapfigure}

\subsection{Rezystor}
\begin{teacherOnly}
	\begin{easylist}[itemize]
		& czy opór to jedyny parametr rezystora?
			&&& mamy też dokładność, stabilność temperaturową, napięciową, ...
			&&& ... ale przede wszystkim moc
		& prawo Ohma i rezystor
		& rezystancyjny dzielnik napięcia (\textbf{symulacja})
	\end{easylist}
\end{teacherOnly}

Rezystor (opornik) wprowadza do układu rezystancję związaną z swoją wartością nominalną. Typowo służy do ograniczania wartości prądu przez niego przepływającego.

Powoduje wydzielanie się energii (cieplnej) związanej z stratami na rezystancji - moc wydzielana dana jest zależnościami: $P = UI = \frac{U^2}{R} = I^2R$, czyli przy stałym napięciu przyłożonym do rezystora im większy jego opór tym mniejsza moc się wydzieli (gdyż popłynie mniejszy prąd), ale przy stałym prądzie płynącym przez rezystor moc rośnie wraz ze wzrostem oporu.

Rezystor jest elementem spełniającym prawo Ohma\footnote{Jest to zasadniczo jedyny element elektroniczny, który podlega temu prawu. Niektóre z elementów (jak kondensatory i cewki) podlegają rozszerzeniu prawa Ohma dla prądu przemiennego. Wiele innych elementów (jak np. diody i tranzystory) nie podlegają prawu Ohma.}.

\subsubsection{inne parametry rezystora}
Rzeczywisty rezystor oprócz samej wartości oporu elektrycznego charakteryzują też inne parametry, m.in. takie jak:
\begin{itemize}
\item maksymalna moc która może zostać wydzielona na danym elemencie,
\item dokładność, czyli to jak bardzo opór danego elementu może być odległy od wartości nominalnej,
\item stabilność oporu w funkcji w funkcji temperatury oraz w funkcji napięcia przyłożonego do elementu.
\end{itemize}

\subsubsection{rezystancyjny dzielnik napięcia}\label{dzielnik}

Jednym z najprostszych, użytecznych obwodów są dwa rezystory połączone szeregowo z źródłem napięcia. Układ taki nazywamy rezystancyjnym dzielnikiem napięcia. Pozwala on na uzyskanie napięcia niższego od napięcia źródła zgodnie z proporcją użytych rezystorów. Zobacz symulację: \url{http://ln.opcode.eu.org/dzielnik}.
	% https://www.falstad.com/circuit/circuitjs.html?cct=%24+1+0.000005+10.20027730826997+54+5+50%0Ag+336+320+336+336+0%0Aw+336+192+416+192+0%0Ar+416+256+416+320+0+1000%0Ag+416+320+416+336+0%0Ar+336+112+336+192+0+500%0As+416+192+416+256+0+1+false%0As+480+192+480+256+0+1+false%0Ag+480+320+480+336+0%0Ar+480+256+480+320+0+100%0Aw+416+192+480+192+0%0AR+336+112+336+80+0+0+40+12+0+0+0.5%0Aw+480+192+560+192+0%0Ap+560+192+560+320+1+0%0Ag+560+320+560+336+0%0Ar+336+192+336+320+0+500%0A38+10+0+2+24+Voltage%0A
Zwróć uwagę że napięcie wyjściowe z takiego układu jest bardzo zależne od pobieranego prądu / wielkości dołączonego obciążenia (w tym celu możesz użyć przełączników umieszczonych w symulowanym układzie), z tego powodu dzielnik rezystancyjny stosowany jest głównie w przypadkach gdy wiemy że obciążenie będzie pobierało niewielki prąd.

Rezystancyjny dzielnik napięcia jest bardzo często stosowany w celu proporcjonalnego podziału (obniżenia) napięcia wejściowego nieznanej (zmiennej) wielkości (np. celem jego pomiaru, przy użyciu miernika o ograniczonej skali),
a nie w celu uzyskania napięcia wyjściowego o konkretnej wartości (co można uzyskać w lepszy - bardziej stabilny sposób).

\subsection{Kondensator}
\begin{teacherOnly}
	\begin{easylist}[itemize]
		& "magazynuje napięcie"
		& zastosowania:
		&& stabilizacja napięcia
		&& opóźnienie zmiany jakiegoś sygnału (stała czasowa) (\textbf{symulacja})
		&& odcinanie składowej sygnały zmiennego (rozwarcie dla prądu stałego, ale przewodzi zmienny) (\textbf{symulacja})
	\end{easylist}
\end{teacherOnly}

Kondensator wprowadza do układu pojemność związaną z swoją wartością nominalną.
Pojemność wyraża zdolność do gromadzenia ładunku przez dany element - im większa pojemność tym więcej ładunku (przy takim samym przyłożonym napięciu) zgromadzi element. $C = \frac{q}{U}$

Kondensator typowo służy do ograniczania zmian napięcia (poprzez gromadzenie energii w polu elektrycznym) lub wprowadzenia opóźnienia (stałej czasowej) związanej z jego ładowaniem / rozładowywaniem.
Czas potrzebny do zmiany napięcia na kondensatorze dany jest zależnością: $\Delta T = \frac{C \cdot \Delta U}{I}$.

Zobacz symulację procesu ładowania / rozładowywania kondensatora: \url{http://ln.opcode.eu.org/cap}
	% https://www.falstad.com/circuit/circuitjs.html?cct=%24+1+0.000005+16.13108636308289+50+5+50%0Av+96+336+96+64+0+0+40+5+0+0+0.5%0AS+256+144+256+64+0+1+false+0+3%0Aw+96+64+240+64+0%0Aw+272+64+400+64+0%0Ac+256+144+256+256+0+0.00019999999999999998+0.001%0Ar+256+256+256+336+0+100%0Aw+96+336+256+336+0%0Aw+256+336+400+336+0%0Ar+400+64+400+336+0+1000%0Ao+4+128+0+4102+0.009765625+0.00009765625+0+2+4+3%0A38+4+0+0.000009999999999999999+0.00101+Capacitance%0A
(klikając na przełącznik w górnej części schematu  można wybierać pomiędzy rozładowywaniem a ładowaniem kondensatora, zwróć uwagę na różną wartość oporu użytego do tych operacji).

Innym częstym zastosowaniem jest kondensatora jest odcinanie składowej stałej – kondensator stanowi rozwarcie dla prądu stałego, ale przewodzi prąd zmienny (ze względu na prąd związany z jego ładowanie / rozładowywaniem).
Zobacz symulację: \url{http://ln.opcode.eu.org/cap_ac}
	% https://www.falstad.com/circuit/circuitjs.html?cct=%24+1+0.000005+16.13108636308289+50+5+50%0Aw+112+144+256+144+1%0Ac+256+144+256+256+0+0.00019999999999999998+-0.8719567578156338%0Ar+256+256+256+336+0+100%0Aw+112+336+256+336+0%0Av+112+336+112+144+0+1+20.000500000000002+5+0+0+0.5%0Ao+1+128+0+4102+5+0.00009765625+0+2+1+3%0A38+1+0+0.000009999999999999999+0.00101+Capacitance%0A38+4+3+0.001+40+Frequency%0A

Najistotniejszym parametrem rzeczywistych kondensatorów oprócz pojemności nominalnej jest maksymalne napięcie przy którym może pracować oprócz tego istotne mogą być parametry takie jak rezystancja wewnętrzna, maksymalna temperatura w której kondensator może pracować, żywotność tego elementu, itd.

\subsection{Cewka}
\begin{teacherOnly}
	\begin{easylist}[itemize]
		& mówiliśmy o magazynowaniu ... cewka "magazynuje prąd"
		& gdzie występuje (w jakich bardziej złożonych elementach) - transformator, przekaźnik
		& sterowanie przez tranzystor => dioda zabezpieczająca (\textbf{symulacja})
	\end{easylist}
\end{teacherOnly}

Cewka (dławik) wprowadza do układu indukcyjność związaną z swoją wartością nominalną. Samodzielnie występująca cewka typowo służy do ograniczania zmian prądu (poprzez gromadzenie energii w polu magnetycznym). Czas potrzebny zmiany prądu płynącego przez cewkę (dławik stawia opór takiej zmianie tak jak kondensator zmianie napięcia) dany jest zależnością: $\Delta T = \frac{L \cdot \Delta I}{U}$.

Głównym (ale nie jedynym) parametrem rzeczywistej cewki oprócz indukcyjności jest maksymalny prąd który może przewodzić.

\subsubsection{Przekaźniki, styczniki i transformatory}

Cewki możemy spotkać w urządzeniach takich jak przekaźniki, czy styczniki\footnote{Zasadniczo przekaźnik i stycznik jest to to samo urządzenie. Przyjmuje się rozróżnienie w nazewnictwie - przekaźniki przełączają mniejsze prądy niż styczniki.}.
Nawinięte na odpowiednim rdzeniu pełnią one tam funkcję elektromagnesu odpowiedzialnego za zmianę fizycznej pozycji styków prowadzącą do ich zwarcia lub rozwarcia (przełączania).

Innym urządzeniem opartym o cewki są transformatory - wykorzystują one kilka cewek na wspólnym rdzeniu do przekazywania energii poprzez pole magnetyczne (jedna z cewek dzięki przepływowi zmiennego prądu elektrycznego wytwarza zmienne pole magnetyczne, inna dzięki zmiennemu polu magnetycznemu wytwarza przemienny prąd elektryczny). Transformator typowo służy do zmiany napięcia lub separacji galwanicznej obwodów.

\subsubsection{Rozłączanie cewki}

Jako że cewka jest elementem który dąży do zachowania płynącego przez niego prądu, to w przypadku rozwarcia obwodu zawierającego cewkę napięcie na niej będzie rosło i bez problemów może wielokrotnie przekroczyć napięcie zasilania.
Zobacz symulację: \url{http://ln.opcode.eu.org/cewka} (rozłącz przełącznik i zaobserwuj co dzieje się z napięciem na cewce).
	% https://www.falstad.com/circuit/circuitjs.html?cct=%24+1+0.000005+12.050203812241895+46+5+50%0Av+96+336+96+144+0+0+40+5+0+0+0.5%0Ac+256+144+256+256+0+2e-8+2.185811780586783%0Ar+256+256+256+336+0+100%0Al+336+144+336+336+0+1+0.2814236307368567%0Aw+256+144+336+144+0%0Aw+336+336+256+336+0%0As+96+144+256+144+0+0+false%0Ar+96+336+256+336+0+10%0Ao+1+16+0+4354+640+0.00009765625+0+2+3+0%0A38+1+0+0.000009999999999999999+0.00101+Capacitance%0A
Zjawisko to bywa użyteczne i jest wykorzystywane w niektórych układach (np. przetwornicach podnoszących napięcie), ale często bywa też niepożądane, a nawet bardzo szkodliwe – może prowadzić do uszkadzania innych elementów w obwodzie (w szczególności elementu przełączającego).

Aby przeciwdziałać temu zjawisku można dołączyć równolegle do cewki odpowiednio mały opór, który pozwoli na rozładowanie się cewki.
Wadą takiego rozwiązania są straty związane z przewodzeniem przez ten rezystor w momencie gdy cewka jest zasilona.
Warto zauważyć że pojawiające się na cewce napięcie ma odwrotny znak (kierunek) niż spadek napięcia na tym elemencie w trakcie pracy.
Pozwala to na podłączenie równolegle z cewką elementu który przewodzi tylko w jednym kierunku\footnote{
	Nawet jeżeli element ten fizycznie jest obok elementu przełączającego powinien być podłączany równolegle do cewki a nie do elementu przełączającego.
}, w taki sposób aby w normalnym stanie nie przewodził, a po odłączeniu zasilania cewki pozwalał na jej rozładowanie.
Takim elementem jest dioda.
% END: Elektronika - bierne

\subsection{Zadania praktyczne}
	\insertZadanie{booklets-sections/electronics/zadania-10-12-podstawy.tex}{zbuduj_ladowanie_kondensatora}{}
% Copyright (c) 2017-2020 Matematyka dla Ciekawych Świata (http://ciekawi.icm.edu.pl/)
% Copyright (c) 2017-2020 Robert Ryszard Paciorek <rrp@opcode.eu.org>
% 
% MIT License
% 
% Permission is hereby granted, free of charge, to any person obtaining a copy
% of this software and associated documentation files (the "Software"), to deal
% in the Software without restriction, including without limitation the rights
% to use, copy, modify, merge, publish, distribute, sublicense, and/or sell
% copies of the Software, and to permit persons to whom the Software is
% furnished to do so, subject to the following conditions:
% 
% The above copyright notice and this permission notice shall be included in all
% copies or substantial portions of the Software.
% 
% THE SOFTWARE IS PROVIDED "AS IS", WITHOUT WARRANTY OF ANY KIND, EXPRESS OR
% IMPLIED, INCLUDING BUT NOT LIMITED TO THE WARRANTIES OF MERCHANTABILITY,
% FITNESS FOR A PARTICULAR PURPOSE AND NONINFRINGEMENT. IN NO EVENT SHALL THE
% AUTHORS OR COPYRIGHT HOLDERS BE LIABLE FOR ANY CLAIM, DAMAGES OR OTHER
% LIABILITY, WHETHER IN AN ACTION OF CONTRACT, TORT OR OTHERWISE, ARISING FROM,
% OUT OF OR IN CONNECTION WITH THE SOFTWARE OR THE USE OR OTHER DEALINGS IN THE
% SOFTWARE.

% BEGIN: Elektronika - Dioda
\section{Dioda}

\begin{wrapfigure}{r}{0.3\textwidth}
  \vspace{-33pt}
  \begin{center}
    \includegraphics[width=0.25\textwidth]{img/elektronika/diody}
  \end{center}
  \vspace{-23pt}
\end{wrapfigure}

Dioda idealna to element przewodzący prąd tylko w jednym kierunku. Symbole najpopularniejszych typów diod pokazane zostały obok. Dioda jest elementem nieliniowym – spadek napięcia na przewodzącej diodzie nie spełnia prawa Ohma i jest prawie stały (niezależny od prądu).

Rzeczywiste diody przewodzą prąd zdecydowanie chętniej w jednym kierunku niż w drugim (na ogół przewodzenie w kierunku zaporowym się pomija) ponadto charakteryzują je cechy zależne od technologi wykonania takie jak:
\begin{itemize}
\item spadek napięcia w kierunku przewodzenia (typowo dla diod krzemowych 0.6V - 0.7V, a dla diod Schottky’ego 0.3V)
\item napięcie przebicia - napięcie, które przyłożone w kierunku zaporowym powoduje znaczące przewodzenie diody w tym kierunku - w większości przypadków parametr którego nie należy przekraczać, jednak wykorzystywane (i stanowiące ich parametr) w niektórych typach diod
\item maksymalny prąd przewodzenia
\item czas przełączania (związany głównie z pasożytniczą pojemnością złącza) - zdecydowanie krótszy (około 100 ps) w diodach Schottky’ego niż w diodach krzemowych,.
\end{itemize}

\noindent
Ponadto stosowane są m.in.:
\begin{itemize}
\item diody Zenera - wykorzystuje się (charakterystyczną dla danego typu) wartość napięcia przebicia do uzyskania w układzie spadku napięcia o tej wartości,
\item diody świecące (LED) - emitujące światło w trakcie przewodzenia (na elemencie występuje stały spadek napięcia, jasność zależy od natężenia prądu),
\item fotodiody - będące detektorami oświetlenia (przewodzenie spolaryzowanej w kierunku zaporowym zależy od ilości padającego na element światła, niespolaryzowana pod wpływem oświetlenia staje się źródłem prądu).
\end{itemize}

\begin{ProTip}{{\Symbola ❢} PAMIĘTAJ {\Symbola ❢}}
Dioda jest elementem dla którego nie jest spełnione prawo Ohma. Dioda charakteryzuje się prawie stałym spadkiem napięcia w kierunku przewodzenia.

\vspace{5pt}
Dlatego, jeżeli do diody przyłożymy napięcie większe od jej napięcia przewodzenia (np. do czerwonej diody LED o spadku około 1.7V przyłożymy napięcie 5V) przez układ taki popłynie bardzo duży prąd (często równy prądowi zwarciowemu naszego źródła), co doprowadzi do zniszczenia diody.

Zobacz symulację: \url{http://ln.opcode.eu.org/led}

\vspace{5pt}
Z tego powodu diody podłączamy prawie zawsze\footnote{Istotnymi wyjątkami są: prostownik (gdzie rolę tego rezystora pełni obciążenie) oraz zasilanie diody ze źródła prądowego.} z szeregowym rezystorem służącym do ograniczenia prądu.
\end{ProTip}

\insertZadanie{booklets-sections/electronics/zadania-10-12-podstawy.tex}{prad_dioda}{}

\subsection{prostownik}

Prostownik służy do zamiany napięcia przemiennego (zmieniającego znak) na napięcie zmienne o stałym znaku.
Funkcję tą może pełnić nawet pojedyncza dioda – mamy wtedy do czynienia z prostownikiem jednopołówkowym,
	charakteryzującym się tym że napięcie na jego wyjściu spada przez połowę okresu wynosi zero – zobacz symulację \url{http://ln.opcode.eu.org/prost1}.
Lepszym i częściej stosowanym rozwiązaniem jest prostownik pełnookresowy (dwupołówkowy). Najczęstszą jego realizacją jest tzw. mostek Graetza, czyli układ 4 diod połączonych w taki sposób iż dwie z nich zawsze (w każdym punkcie napięcia wejściowego) przewodzą – zobacz symulację \url{http://ln.opcode.eu.org/prost2}.
Wadą takiego układu jest znaczny spadek napięcia na mostku, wynoszący dwukrotność spadku napięcia na pojedynczej diodzie.

\subsubsection{trójfazowe \zaawansowane{20}}
Istnieją również układy prostowników napięcia trójfazowego, charakteryzują się one m.in. niższymi tętnieniami napięcia wyjściowego – dla prostowników jednofazowych wacha się ono (pomijając spadki na diodach) od $0$ do $V_{LN}\sqrt{2}$, a dla pełnookresowego trójfazowego od $V_{LL}\sqrt{2}\sin{60}$ do $V_{LL}\sqrt{2}$ (gdzie $V_{LN}$ to napięcie skuteczne pomiędzy fazą a przewodem neutralnym, a $V_{LL} = V_{LN}\sqrt{3}$ to napięcie skuteczne międzyfazowe). Zobacz symulację: \url{http://ln.opcode.eu.org/prost3}.

\subsection{dzielnik napięcia z diodą Zenera}

W rozdziale \ref{dzielnik} omawialiśmy rezystancyjny dzielnik napięcia złożony z dwóch rezystorów. Wadą takiego układu była duża zależność napięcia wyjściowego od obciążenia. Zjawisko to można ograniczyć zastępując jeden z rezystorów (ten równolegle połączony z obciążeniem) diodą Zenera w polaryzacji zaporowej, która charakteryzuje się dość stałym spadkiem napięcia. Zobacz symulację \url{http://ln.opcode.eu.org/zener}, zauważ że nadal nie jest to rozwiązanie idealne, ale znacznie bardziej stabilne od poprzedniego.
% END: Elektronika - Dioda

\insertZadanie{booklets-sections/electronics/zadania-10-12-podstawy.tex}{napiecie_T234}{}

\begin {teacherOnly}\textbf{POKAZ: LED jako detektor światła.}
Wcześniej można pokazać poniższy schemat (bez podpisu "detektor" przez D1) i poprosić aby spróbować odgadnąć co on może robić.
Następnie omówić działanie układu, pokazać na pojedynczym LED że oświetlenie powoduje pojawienie się napięcia i pokazać że ten układ działa.

\includegraphics[width=\textwidth]{img/elektronika/zad-detektor_na_LED-pokaz}
\end{teacherOnly}

\subsection{Zadania praktyczne}
	\insertZadanie{booklets-sections/electronics/zadania-10-12-podstawy.tex}{zbuduj_spadek_napiecia_na_led}{}
	\insertZadanie{booklets-sections/electronics/zadania-10-12-podstawy.tex}{zbuduj_stabilizator_zener1}{}

% Copyright (c) 2017-2020 Matematyka dla Ciekawych Świata (http://ciekawi.icm.edu.pl/)
% Copyright (c) 2017-2020 Robert Ryszard Paciorek <rrp@opcode.eu.org>
% 
% MIT License
% 
% Permission is hereby granted, free of charge, to any person obtaining a copy
% of this software and associated documentation files (the "Software"), to deal
% in the Software without restriction, including without limitation the rights
% to use, copy, modify, merge, publish, distribute, sublicense, and/or sell
% copies of the Software, and to permit persons to whom the Software is
% furnished to do so, subject to the following conditions:
% 
% The above copyright notice and this permission notice shall be included in all
% copies or substantial portions of the Software.
% 
% THE SOFTWARE IS PROVIDED "AS IS", WITHOUT WARRANTY OF ANY KIND, EXPRESS OR
% IMPLIED, INCLUDING BUT NOT LIMITED TO THE WARRANTIES OF MERCHANTABILITY,
% FITNESS FOR A PARTICULAR PURPOSE AND NONINFRINGEMENT. IN NO EVENT SHALL THE
% AUTHORS OR COPYRIGHT HOLDERS BE LIABLE FOR ANY CLAIM, DAMAGES OR OTHER
% LIABILITY, WHETHER IN AN ACTION OF CONTRACT, TORT OR OTHERWISE, ARISING FROM,
% OUT OF OR IN CONNECTION WITH THE SOFTWARE OR THE USE OR OTHER DEALINGS IN THE
% SOFTWARE.

% BEGIN: Elektronika - Tranzystory
\section{Tranzystory}

\begin{teacherOnly}
	\begin{easylist}[itemize]
	& tranzystor bipolarny
		&& przewodzenie regulowane napięciem baza - emiter (NPN)
		&& praca w trybie klucza (stan nasycenia i zatkania) (\textbf{symulacja})
		&& praca jako wzmacniacz => POKAZAĆ w tym samym układzie co klucz i powiedzieć dlaczego tak się nie używa (rozrzut bety)
		&& rezystor bazy - po co jest? analogia do diody
		&& NPN i PNP - różnice w sposobie sterowania => różnice w zastosowaniach
	& inne rodzaje tranzystorów - (MOS)FET
		&& typy (N- vs P- oraz wzbogacony vs zubożony) i sposób sterowania (\textbf{symulacja})
		&& izolowana bramka - sterowanie z 3 pozycyjnego przełącznika 5V NC 0V (zapytać o stan NC) => \textbf{POKAZAĆ} że zostaje w stanie w jakim był, dlatego często sterowanie z pół-mostka, wspomnieć o ESD
	& mostek i pół-mostek H (\textbf{symulacja} w oparciu o schemat z switchami)
		&& do czego służy cały (silniki), a do czego można użyć pół (wyjście o ustalonych poziomach)?
	& wspomnieć o triakach – przełączanie w obwodach AC
	\end{easylist}
\end{teacherOnly}

Tranzystor jest to element o regulowanym elektrycznie przewodzeniu prądu (oporze), często wykorzystywany do wzmacniania sygnałów lub jako przełącznik elektroniczny (klucz tranzystorowy).
Klucz jest układem przełączającym wykorzystującym dwa skrajne stany pracy tranzystora - zatkania (tranzystor nie przewodzi), nasycenia (tranzystor przewodzi z minimalnymi ograniczeniami).

\subsection{NPN}
Prąd przepływający pomiędzy kolektorem a emiterem jest funkcją prądu przepływającego pomiędzy bazą a emiterem: $I_C = \beta I_B$.
Napięcie pomiędzy kolektorem a emiterem wynosi: $U_{CE} = U_{zasilania} - I_C \cdot R_{load}$.
Napięcie to nie może jednak spaść poniżej wartości minimalnej wynoszącej około 0.2V, gdy z powyższych zależności wynikałby taki spadek to tranzystor pracuje w stanem nasycenia i $U_{CE} \approx 0.2V$.

\begin{wrapfigure}{r}{0.6\textwidth}
  \begin{center}
    \includegraphics[width=0.55\textwidth]{img/elektronika/npn_pnp}
    \vspace{-10pt}
  \end{center}
\end{wrapfigure}

Aby wprowadzić tranzystor NPN w stan zatkania należy podać na jego bazę potencjał mniejszy lub równy potencjałowi emitera (zakładamy że potencjał kolektora jest nie mniejszy niż emitera - co ma miejsce w typowych warunkach polaryzacji tranzystora NPN), czyli $U_{BE} \leq 0$.

Aby wprowadzić tranzystor NPN w stan nasycenia należy na jego bazę wprowadzić potencjał większy od potencjałów emitera i kolektora, uzyskuje się to poprzez wprowadzenie do tranzystora prądu bazy $I_B \gg \frac{U_{zasilania}}{\beta R_{load}}$.

Zobacz symulację pokazującą pracę tranzystora NPN w trybie klucza: \url{http://ln.opcode.eu.org/npn}.
Zwróć uwagę na wartości napięć i prądów.

Zobacz co dzieje się przy próbie podłączenia bazy tranzystora do potencjału znacznie wyższego niż potencjał emitera – złącze baza-emiter jest takim samym złączem z jakim mamy do czynienia w diodzie i tak jak w przypadku diody występuje na nim stały spadek napięcia (nie działa tu prawo Ohma). Dlatego aby ograniczyć prąd płynący tą gałęzią i zapobiec zniszczeniu tranzystora konieczne jest zastosowanie rezystora.

\begin{teacherOnly}
Pokazać pracę w trybie wzmacniacza w tym samym układzie – zmieniać suwakiem napięcie wejściowe Vin w zakresie 0.2 - 1.6V => prąd kolektora 100 razy większy od prądu bazy (wzmocnienie).
Powiedzieć dlaczego tego układu wzmacniacza się nie używa.
\end{teacherOnly}

\subsection{PNP}
Podobnie jak w NPE tyle że prąd przepływający pomiędzy emiterem a kolektorem jest funkcją prądu przepływającego pomiędzy emiterem a bazą.

Aby wprowadzić tranzystor PNP w stan zatkania należy podać na jego bazę potencjał większy lub równy potencjałowi emitera (zakładamy że potencjał emitera jest nie mniejszy niż kolektora - co ma miejsce w typowych warunkach polaryzacji tranzystora PNP), czyli $U_{BE} \geq 0$.

Aby wprowadzić tranzystor PNP w stan nasycenia należy na jego bazę wprowadzić potencjał mniejszy od potencjałów emitera i kolektora, uzyskuje się to poprzez wyprowadzenie z tranzystora prądu bazy $I_B \gg \frac{U_{zasilania}}{\beta R_{load}}$.

Zobacz symulację pokazującą pracę tranzystora PNP w trybie klucza: \url{http://ln.opcode.eu.org/pnp}.
Zwróć uwagę na podobieństwa i różnice w stosunku do tranzystora NPN –
	w obu wypadkach tranzystor przewodzi gdy płynie prąd bazy, ale ma on różne kierunki (w NPN wpływa on bazą do tranzystora, a w PNP wypływa z niego),
	w obu wypadkach tranzystor zostaje zatkany gdy potencjał bazy zrówna się z potencjałem emitera (ale w NPN potencjał emitera jest typowo najniższym z potencjałów w układzie, często równym masie, a w PNP najwyższym, często równym potencjałowi zasilania).
Zauważ także, że tutaj również potrzebujemy rezystora ograniczającego prąd bazy.

\subsection{N-MOSFET}

\begin{wrapfigure}{r}{0.7\textwidth}
  \begin{center}
    \vspace{-40pt}
    \includegraphics[width=0.65\textwidth]{img/elektronika/mosfet}
    \vspace{-20pt}
  \end{center}
\end{wrapfigure}

Prąd przepływający pomiędzy drenem (\emph{drain}) a źródłem (\emph{source}) jest funkcją napięcia pomiędzy bramką (\emph{gate}) a źródłem (potencjału bramki względem źródła - $U_{GS}$), bramka jest izolowana (nie płynie przez nią prąd).

W kierunku dren → źródło tranzystor ten przewodzi gdy $U_{GS} > U_{GS (th)}$, natomiast w przeciwnym kierunku przewodzi zawsze. Dla tranzystorów N-MOSFET z kanałem wzbogacanym (\emph{enhancement}) $U_{GS (th)} > 0$, a z kanałem zubożonym (\emph{depletion}) $U_{GS (th)} < 0$.

Konkretna wartość $U_{GS (th)}$ zależna jest od konkretnego modelu tranzystora, innym istotnym parametrem związanym z sterowaniem tranzystorem jest maksymalna i minimalna dopuszczalna wartość napięcia $U_{GS}$.

Aby wprowadzić tranzystor MOSFET w stan zatkania należy podać $U_{GS} < U_{GS (th)}$. Dla tranzystorów:
\begin{itemize}
\item N-MOSFET z kanałem wzbogaconym wystarczy obniżyć potencjał bramki do wartości niewiele wyższej niż potencjał źródła
\item N-MOSFET z kanałem zubożonym musi to być wartość poniżej potencjału źródła.
\end{itemize}
Aby wprowadzić tranzystor MOSFET w stan przewodzenia należy podać $U_{GS} \gg U_{GS (th)}$.

\subsection{P-MOSFET}
Podobnie jak N-MOSFET tyle że:
\begin{itemize}
\item regulowane jest przewodzenie w kierunku źródło → dren (a w kierunku dren → źródło przewodzi zawsze),
\item przewodzenie w kierunku źródło → dren ma miejsce gdy $U_{GS} < U_{GS (th)}$,
\item dla tranzystorów z kanałem wzbogacanym (\emph{enhancement}) $U_{GS (th)} < 0$, a z kanałem zubożonym (\emph{depletion}) $U_{GS (th)} > 0$.
\end{itemize}

Aby wprowadzić tranzystor MOSFET w stan zatkania należy podać $U_{GS} < U_{GS (th)}$. Dla tranzystorów:
\begin{itemize}
\item P-MOSFET z kanałem zubożonym wystarczy obniżyć potencjał bramki do wartości niewiele wyższej niż potencjał źródła
\item P-MOSFET z kanałem wzbogaconym musi to być wartość poniżej potencjału źródła.
\end{itemize}
Aby wprowadzić tranzystor MOSFET w stan przewodzenia należy podać $U_{GS} \gg U_{GS (th)}$.

\vspace{12pt}

Zobacz symulację pokazującą pracę tranzystorów MOSFET w trybie klucza: \url{http://ln.opcode.eu.org/mosfet}.
Zauważ podobieństwo w sterowaniu do trnzystorów NPN i PNP (N-MOSFET przewodzi gdy potencjał bramki odpowiednio wyższy od drenu, P-MOSFET gdy odpowiednio niższy, obciążenie N-MOSFET włączane analogicznie jak NPN, a P-MOSFET jak PNP),
	zauważ różnice (bramka jest izolowanie, nie płynie nią prąd\footnote{z pominięciem prądu związanego z przeładowaniem pojemności (pasożytniczego kondensatora)}, nie ma zatem potrzeby używania tam rezystora).

\teacher{Opowiedzieć o zagadnieniu niepodłączonej bramki - pozostaje to co było, potem antena. Wspomnieć o ESD.}
% END: Elektronika - Tranzystory

% BEGIN: Elektronika - mostek H
\subsection{Mostek H}

\begin{wrapfigure}{r}{0.2\textwidth}
  \begin{center}
    \vspace{-33pt}
    \includegraphics[width=0.17\textwidth]{img/elektronika/mostek_H_switche}
    \vspace{-27pt}
  \end{center}
\end{wrapfigure}

Mostek H jest to układ (oparty o 4 przełączniki, których rolę mogą pełnić klucze tranzystorowe) pozwalający na zmianę polaryzacji zasilania podłączonego do niego odbiornika. Układ taki złożony jest z dwóch identycznych gałęzi (S1 + S2 oraz S3 + S4, każda włączona pomiędzy dwoma biegunami zasilania). Pojedyncza taka gałąź nazywana jest pół-mostkiem i składa się z dwóch kluczy które powinny być sterowane przeciwstawnie (aby wyeliminować możliwość zwarcia zasilania z masą). Układ pół-mostka może być wykorzystywany także samodzielnie jako uniwersalny układ klucza pozwalającego na załączanie odbiornika zarówno od strony napięcia dodatniego jak i od strony masy (w zależności od sposobu jego podłączenia) lub przełączania dwóch odbiorników (jednego umieszczonego pomiędzy zasilaniem a wyjściem mostka, a drugiego pomiędzy wyjściem a masą).

Rolę kluczy (przełączników) w ramach mostka mogą pełnić tranzystory PNP (jako S1, S3) i NPN (jako S2, S4) albo analogicznie tranzystory P-MOSFET i N-MOSFET.

\begin{teacherOnly}
Można pokazać: \url{http://ln.opcode.eu.org/mostek}
Ale trzeba zwrócić uwagę na ograniczenie symulatora (niepodłączone to potencjał masy).
\end{teacherOnly}
% END: Elektronika - mostek H

\subsection{Wzmacniacz}

Omawiając poszczególne typy tranzystorów skupialiśmy się na ich pracy w roli przełącznika (klucza), działającego w dwóch stanach – przewodzenia (nasycenia) i zastkania.
Jednak tranzystor będąc elementem o regulowanym przewodzeniu może zostać wykorzystany także do wzmacniania sygnałów, czyli wytworzenia na swoim wyjściu sygnału proporcjonalnego do sygnału wejściowego tyle że wzmocnionego.
Wzmacnianiu może ulegać sygnał napięciowy lub prądowy (najprostszym przypadkiem jest wzmocnienie prądu bazy jako prądu kolektora $I_C = \beta I_B$ w tranzystorze bipolarnym).
Więcej o różnych układach pracy tranzystora w roli wzmacniacza można przeczytać na \url{http://vip.opcode.eu.org/#Wzmacniacz_sygnału}.

\vspace{5pt}

\begin{wrapfigure}{r}{0.24\textwidth}
  \begin{center}
    \vspace{-25pt}
    \includegraphics[width=0.21\textwidth]{img/elektronika/wzmacniacz_operacyjny-nieodwracający}
    \vspace{-47pt}
  \end{center}
\end{wrapfigure}

Często do wzmacniania sygnału zamiast pojedynczego tranzystora wykorzystujemy układy scalone (złożone z wielu tranzystorów) nazywane wzmacniaczami operacyjnymi.
Cechują się one bardzo dużym wzmocnieniem różnicy napięcia pomiędzy swoimi wejściami, pożądane wzmocnienie uzyskuje się dobierając zewnętrzne elementy pętli ujemnego sprzężenia zwrotnego
	(w najprostszym przypadku na jedno wejście podajemy sygnał wejściowy, a na drugie odpowiednio przeskalowany przy pomocy dzielnika rezystancyjnego sygnał wyjściowy).
Więcej na ich temat można przeczytać na \url{http://vip.opcode.eu.org/#Wzmacniacz_operacyjny}.

\begin{ProTip}{Przełączanie AC}
Tranzystory stosowane są powszechnie do przełączania w obwodach prądu stałego. Istnieją także elementy półprzewodnikowe mogące pełnić funkcję przełączającą w obwodach prądu przemiennego - współcześnie są to przede wszystkim triaki.
\end{ProTip}

\begin{teacherOnly}
\noindent\begin{minipage}[t]{0.6\textwidth}
\strong{POKAZ: tranzystor jako klucz}\\
Pokazać i wyjasnić działanie układu pokazanego na schemacie. Pokazać jak zmienia się prąd w obwodzie i spadek napięcia w zależności od nastawy potencjometru.

\end{minipage}
\hfill
\begin{minipage}[t]{0.35\textwidth}
\vspace{-10pt}
\includegraphics[width=\textwidth]{img/elektronika/zad-npn-pokaz}
\end{minipage}
\end{teacherOnly}

\subsection{Zadania praktyczne}
	\insertZadanie{booklets-sections/electronics/zadania-13-tranzystor.tex}{zbuduj_klucz_npn1}{}
	\insertZadanie{booklets-sections/electronics/zadania-13-tranzystor.tex}{zbuduj_klucz_npn2}{}
	\insertZadanie{booklets-sections/electronics/zadania-13-tranzystor.tex}{zbuduj_stabilizator_zener2}{}
	\insertZadanie{booklets-sections/electronics/zadania-13-tranzystor.tex}{wzmacniacz_operacyjny}{}

\clearpage
\begin{teacherOnly}
	\begin{easylist}[itemize]
	& elektronika cyfrowa - podstawy
		&& algebra bool'a, NKB
		&& poziomy logiczne TTL, logika odwrócona
		&& bramki logiczne, symbole (\textbf{symulacja})
		&& bufor (wyjście) trójstanowe -> do czego to służy -> kilka układów nadających do wspólnej magistrali (\textbf{symulacja})
		&& wyjścia open colektor/drain -> wymuszenie niskiego stanu linii przez jeden z elementów (\textbf{symulacja})
		&& jak działa bramka - zacząć od NOT'a (półmostek), następnie NAND lub NOR (\textbf{symulacja} z analizą działania i ustaleniem co to za bramka), wspomnieć dlaczego elektronika lubi bramki zanegowane (bo takie wychodzą)
	\end{easylist}
\end{teacherOnly}

% BEGIN: Elektronika - Bramki
\section{Bramki}
\begin{wrapfigure}{r}{0.7\textwidth}
  \begin{center}
    \vspace{-40pt}
    \includegraphics[width=0.65\textwidth]{img/elektronika/bramki}
    \vspace{-20pt}
  \end{center}
\end{wrapfigure}

Bramki są układami elektronicznymi realizującymi podstawowe, opisane powyżej funkcje logiczne. Obok zostały przedstawione podstawowe symbole poszczególnych bramek w wariancie dwu wejściowym, spotkać się można także z symbolami z zanegowanymi wejściami - w takiej konwencji np. bramka AND reprezentowana jest przez NOR z zanegowanymi wejściami. Bramki (z wyjątkiem buforów oraz bramki NOT), mogą występować także w wariantach wielo-wejściowych (ze względu na łączność podstawowych operacji nie ma wątpliwości co don wyniku jaki powinna dawać np. 8 wejściowa bramka OR). Na ogół w pojedynczym układzie scalonym znajduje się kilka jednakowych bramek.

Zobacz symulację działania różnych bramek logicznych: \url{http://ln.opcode.eu.org/bramki} (H oznacza stan wysoki, czyli prawdę, L stan niski czyli fałsz, klikając na H/L przy wejściach można zmieniać ich stan).

\subsection{trój-stanowe}
Typowa bramka wymusza (w sposób silny) na swoim wyjściu stan wysoki lub niski, co uniemożliwia bezpośrednie łączenie wyjść bramek.
Bramki trój-stanowe mają możliwość skonfigurowania wyjścia w stan \emph{wysokiej impedancji} czyli nie wymuszania żadnej jego wartości.
Sterowanie załączeniem bądź wyłączeniem wyjścia (przełączeniem w stan wysokiej impedancji) odbywa się przy pomocy zewnętrznego sygnału sterującego "output enabled" ("OE"), sygnał ten może występować w postaci prostej i zanegowanej.
Pozwala to na podłączanie do jednej linii wielu bramek i decydowaniu która z nich będzie nią sterować.

\subsection{open collector / drain}
Są kolejnym rodzajem bramek których wyjścia można podłączać do wspólnej linii. Układy te posiadają wyjście w postaci tranzystora zwierającego linię wyjściową do masy, z tego względu samodzielnie zapewniają jedynie stan niski wyjścia (są w stanie wymusić stan niski, ale nie mają możliwości wymuszenia stanu wysokiego).

Stan wysoki musi zostać zapewniony zewnętrznym rezystorem podciągającym. Pozwala to stosować na takiej linii inny poziom stanu wysokiego niż na wejściach takiej bramki oraz pozwala na sterowanie wspólnej linii przez wiele bramek (czyli łączenie wyjść bramek, jednak w odróżnieniu od bramek trój-stanowych nie wymaga dodatkowych sygnałów sterujących).

\begin{wrapfigure}{r}{0.7\textwidth}
  \begin{center}
    \vspace{-20pt}
    \includegraphics[width=0.65\textwidth]{img/elektronika/open_drain}
    \vspace{-20pt}
  \end{center}
\end{wrapfigure}
Na schemacie obok przedstawiono dwa układy (U1 i U2) typu open-drain sterujące wspólną linią wyjściową w układzie \emph{suma na drucie}. Jeżeli jeden z podłączonych do linii układów będzie miał wewnętrzne wyjście ("ctrl\textit{X}") w stanie wysokim to jego wyjście OC będzie zwarte do masy (negacja na tranzystorze N-MOS lub NPN), wtedy też cała linia będzie w stanie niskim.

Zobacz symulację lininii z bramkami trójstanowymi (stan wysokiej impedancji symulowany za pomocą przełącznika) oraz linii open-colektor: \url{http://ln.opcode.eu.org/ster}

\begin{wrapfigure}{r}{0.61\textwidth}
  \begin{center}
    \vspace{-45pt}
    \includegraphics[width=0.59\textwidth]{img/elektronika/bramki_cmos}
    \vspace{-35pt}
  \end{center}
\end{wrapfigure}
\subsection{budowa wewnętrzna}
Przedstawiony powyżej układ sumy na drucie jest bardzo prostą (jedno tranzystorową) realizacją bramki logicznej realizującą funkcję logiczną NOT OR (z punktu widzenia wejść \textit{ctrl1} i \textit{ctrl2} oraz wyjścia \textit{Out}).
W podobny sposób można zrealizować bramkę AND (negując wejścia, np. przy pomocy jednego traznzystora).
Jeszcze bardziej uproszczoną realizację można uzyskać stosując diody pozwalające na wpływanie prądu do węzła (funkcja OR) lub wypływanie z niego (funkcja AND).

Po prawej przedstawione zostały schematy ideowe inwertera, dwóch podstawowych bramek (NOR i NAND) oraz bramki transmisyjnej (bufora 3 stanowego) w technologii CMOS.

Działanie tych bramek (za wyjątkiem transmisyjnej) polega na otwieraniu tranzystorów podłączonych do napięcia które chcemy otrzymać na wyjściu, a zamykaniu prowadzących do napięcia przeciwnego. W szczególności bramka NOT stanowi pół-mostek H pomiędzy stanem wysokim a stanem niskim.

Dzięki zastosowaniu tranzystorów PMOS polaryzowanych Vdd oraz NMOS polaryzowanych GND obie gałęzie operują na tym samym sygnale wejściowym (nie jest wymagana jego negacja). Szeregowe łączenie tranzystorów zapewnia że należy otworzyć oba aby otworzyć daną drogę, a równoległe że otwarcie danej drogi powodowane jest otwarciem pojedynczego tranzystora. Dzięki zastosowaniu technologi MOS i podłączaniu wejść bramki tylko do bramek tranzystorów wejścia praktycznie nie pobierają prądu (istotnym wyjątkiem jest chwila zmiany sygnału).

Działanie bramki transmisyjnej polega na przepuszczaniu lub nie (w zależności od stanu wejścia sterującego) sygnału z wejścia na wyjście. Bramka taka nie regeneruje sygnału. Ponadto w uproszczonym (jedno tranzystorowym) rozwiązaniu prowadzi ona do degradacji sygnału wartość w przybliżeniu równą napięciu progowemu tranzystora. Dlatego też na ogół występuje wraz z bramką NOT (bufor 3 stanowy z negacją) lub dwiema szeregowo połączonymi bramkami NOT (bufor 3 stanowy bez negacji).

Zobacz symulację budowy bramek: NOT (\url{http://ln.opcode.eu.org/not}), NAND (\url{http://ln.opcode.eu.org/nand}) i NOR (\url{http://ln.opcode.eu.org/nor}).
% END: Elektronika - Bramki

\subsection{Zadania praktyczne}
	\insertZadanie{booklets-sections/electronics/zadania-20-30-cyfrowa.tex}{zbuduj_przerzutnik_z_bramek}{}
	\insertZadanie{booklets-sections/electronics/zadania-20-30-cyfrowa.tex}{zbuduj_bramka_z_tranzystorow}{}
% BEGIN: Elektronika - Przerzutniki i rejestry
\section{Przerzutniki i rejestry}

\subsection{przerzutniki i ich budowa}

RS Flip-flop (RS Latch) jest podstawowym układem służącym do zapamiętania jednego bitu informacji. Posiada on dwa wejścia (set i reset) i dwa wyjścia (Q i NOT Q), wejścia mogą reagować na stan wysoki (oznaczane jako S i R) lub niski (oznaczane jako wejścia zanegowane ~S i ~R), jedno z wyjść może być jedynie wewnętrzne (nie wyprowadzone na zewnątrz układu). Podanie stanu wysokiego na wejście S (niskiego na ~S) powoduje wystawienie stanu wysokiego na wyjściu Q, a podanie stanu wysokiego na wejście R (niskiego na ~R) powoduje wystawienie stanu niskiego na wyjściu Q. Stan na wyjściu Q nie zmienia się po zmianie wejść S i R na stan niski (zostaje zapamiętany).

Zobacz i przeanalizuj symulację działania zatrzasku RS: \url{http://ln.opcode.eu.org/rs} z wejściami zanegowanymi.

\subsection{zatrzask a przerzutnik}

Zatrzask jest elementem reagującym na poziom sygnału na wejściu "enable" (E). W przypadku nie zanegowanego wejścia E, jeżeli jest ono w stanie wysokim sygnał na wyjściach (Q i NOT Q) jest funkcją sygnałów wejściowych, natomiast stan niski wejścia E blokuje zmianę sygnału wyjściowego (zostaje on zapamiętany).

Przerzutnik jest elementem reagującym na zbocze sygnału na wejściu "clock" (CLK). W zależności od konstrukcji może reagować na zbocze narastające, opadające albo na oba (wtedy na jednym realizuje odczyt wejść a na drugim zmianę stanu wyjść).

\subsection{zatrzask i przerzutnik D}

\begin{wrapfigure}{r}{0.7\textwidth}
  \begin{center}
    \vspace{-25pt}
    \includegraphics[width=0.65\textwidth]{img/elektronika/przerzutnikD}
    \vspace{-25pt}
  \end{center}
\end{wrapfigure}

Posiada jedno wejścia sygnałowe "data" (D) oraz wejście "enable" (E) w przypadku zatrzasku lub wejście "clock" (CLK) w przypadku przerzutnika. Może także posiadać asynchroniczne (niezależne od stanu wejścia E / CLK) wejścia reset i set (zanegowane lub proste). Obniżenie sygnału E lub zbocze sygnału CLK powodują zapamiętanie (i wystawienie na wyjściu Q) stanu wejścia D.

Zobacz symulację zatrzasku typu D zbudowanego z bramek NAND: \url{http://ln.opcode.eu.org/zatrzask} (możesz zmieniać stan wejścia D klikając na nie, zegar zmienia się automatycznie).
Zwróć uwagę iż przy wysokim stanie sygnału zegara (enable) stan wyjścia odpowiada stanowi wejścia (zatrzask jest przeźroczysty),
	natomiast przy niskim stanie zegara wyjście nie reaguje na zmiany stanu wejścia i znajduje się w takim stanie w jakim było wejście w chwili opadającego zbocza sygnału zegarowego.

Zobacz symulację przerzutnika D złożonego z dwóch zatrzasków: \url{http://ln.opcode.eu.org/przerzutnik}.
Zauważ że w żadnej fazie sygnału zegarowego nie jest on przeźroczysty (wyjście Q nie zależy od obecnego stanu wejścia D).
Zwróć uwagę że sygnał wejściowy zostanie zapamiętany i wystawiony na wyjście Q w momencie opadającego zbocza sygnału zegarowego.

\subsection{rejestry}
Mianem rejestru n-bitowego określa się zespół n przerzutników (rzadziej zatrzasków), często z uwspólnionym sterowaniem (sygnały clock, set, reset, etc) służący do zapamiętania n-bitowej wartości (liczby). W zależności od sposobu zapisu i odczytu można wyróżnić:

\subsubsection{rejestry równoległe}
Posiadają taką samą liczbę wejść jak i wyjść, sygnał na i-tym wyjściu jest bezpośrednio powiązany z sygnałem z i-tego wejścia (jest sygnałem zapamiętanym z tego wejścia).

Zobacz symulację rejestru równoległego zbudowanego z przerzutników typu D: \url{http://ln.opcode.eu.org/rejestr1} (stan wszystkich wejść zostanie zapamiętany i przepisany na wyjścia w chwili narastającego zbocza zegara).

\subsubsection{rejestry szeregowe serial-input}
Z kolejnymi sygnałami zegarowymi odczytywany jest stan wejścia szeregowego, a stan poprzedni przenoszony jest do kolejnego przerzutnika w ramach rejestru. W ten sposób po n cyklach zegara n-bitowy rejestr ma zapisaną nową zawartość. Często posiada zespolony z nim rejestr równoległy zapobiegający zmianie stanu wyjść w trakcie ładowania danych z wejścia szeregowego przepisanie danych z rejestru przesuwnego do rejestru odpowiedzialnego za sterowanie wyjściami sterowane jest osobnym sygnałem zegarowym.

Zobacz symulację prostego rejestru z wejściem szeregowym (bez zatrzasku/rejestru wyjściowego): \url{http://ln.opcode.eu.org/rejestr2}.
Zauważ że stan wyjść zmienia się na bieżąco w trakcie szeregowego wpisywania wartości do rejestru.

Zobacz symulację rejestru z wejściem szeregowym i rejestrem wyjściowym: \url{http://ln.opcode.eu.org/rejestr3}.
Zauważ, że stan wyjść zmienia się na skutek osobnego sygnału, który może zostać wygenerowany po zakończeniu szeregowego zapisu do rejestru.

\subsubsection{rejestry szeregowe paraller-input serial-output}
Z kolejnymi sygnałami zegarowymi na wyjście szeregowe wystawiany jest stan kolejnego z rejestrów wejściowych. Wariant asynchroniczny posiada osobny sygnał powodujący odczyt wejść do rejestru (sygnał działa jak "enable" w zatrzaskach). Wariant synchroniczny posiada sygnał decydujący o tym czy na zboczu zegara dokonywany jest odczyt wejść czy też przesuwanie zawartości rejestru umożliwiający odczyt z wyjścia szeregowego.

\subsubsection{liczniki}
Z kolejnymi sygnałami zegarowymi zwiększana jest o jeden wartość rejestru. Prostszy w budowie licznik asynchroniczny ma większe (i w dodatku rosnące wraz z bitowością licznika) ograniczenia dotyczące szybkości zliczania od licznika synchronicznego, ze względu na opóźnienie z jakim dochodzi zliczany sygnał (CLK) do kolejnych stopni licznika.
% END: Elektronika - Przerzutniki i rejestry

% BEGIN: Elektronika - Transmisja - sterowanie linią
\section{Transmisja - sterowanie linią}
\subsection{bufory}

Bufor jest to układ przekazujący sygnał logiczny z wejścia na wyjście. Bufor może służyć do:
\begin{itemize}
\item regeneracji sygnału,
\item uniemożliwieniu wprowadzenia sygnału z jego strony wyjściowej na wejściową,
\item decydowania o jego przepuszczeniu lub nie (trój-stanowy),
\item decydowania o kierunku przepuszczenia sygnału (dwa trój-stanowe albo trój-stanowy dwukierunkowy),
\item konwersji na linię open-collector / open-drain,
\item negacji sygnału (niektóre bufory realizują funkcję NOT).
\end{itemize}

\subsection{enkodery}

Enkoder "n to m" jest to układ o n wejściach, który na swoim m bitowym wyjściu wystawia numer (typowo) wejścia o najwyższym numerze, które znajduje się w stanie niskim. Możliwe są też warianty wystawiające numer pierwszego (a nie ostatniego) w kolejności wejścia lub wybierające wejście ze stanem wysokim.

Jako że wejścia numerowane są zazwyczaj od zera do 2m to układ najczęściej posiada dodatkowe wyjście informujące że którekolwiek z wejść jest w stanie aktywnym. Typowo numer wystawiany jest w postaci NKB, ale możliwe są inne kodowania.

Układ pozwala na redukcję ilości wejść potrzebnych do obsługi n-bitowego sygnału w którym tylko jeden bit może być ustawiony lub w którym można pozwolić sobie na obsługę kolejnych linii z kasowaniem ich bitu (np. wektor przerwań z priorytetyzacją).

\subsection{dekodery}

Dekoder "m to n" jest układem o działaniu przeciwnym do enkodera. Aktywuje on wyjście o numerze odpowiadającym wartości na m-bitowym wejściu adresowym. Typowo posiada także wejście zezwolenia na aktywację wyjść, które może zostać użyte do podłączenia informacji że którekolwiek z wejść enkodera było w stanie aktywnym lub do podłączenia sygnału danych z multipleksowanej linii celem jej demultipleksacji.

Przykład użycia enkodera i dekodera do obsługi matrycy przełączników (klawiatury) można zobaczyć na symulacji: \url{http://ln.opcode.eu.org/matrix}.

\subsection{(de)multipleksery cyfrowy}

Multiplekser cyfrowy (jednokierunkowy) na wyjście kopiuje stan wskazanego (poprzez adres podany na wejście adresowe) wejścia. W przypadku braku sygnału "enable" w zależności od rozwiązania wyjście pozostanie w stanie niskim lub wysokiej impedancji.

Demultiplekser cyfrowy (jednokierunkowy) to zazwyczaj układ dekodera w którym na wejście enabled podany jest sygnał z multipleksowanej linii. Nie wybrane wyjścia pozostają w stanie niskim lub wysokim (zależnie od użycia nieodwracającego lub odwracającego dekodera). Cyfrowe demultipleksery z 3 stanowym wyjściem są rzadkością. Demultipleksację można rozwiązać także przy pomocy odpowiednio sterowanych (np. z dekodera adresu) buforów trój-stanowych lub dwu-wejściowych multiplekserów.

\subsection{(de)multipleksery analogowy}

Multiplekser analogowy (dwukierunkowy) działa na zasadzie przełącznika łączącego wskazane wejście z wyjściem, dzięki elektrycznemu "zwarciu" (na ogół rezystancja takiego zwarcia to kilkadziesiąt omów) wejścia z wyjściem transmisja może odbywać się w obu kierunkach. Pozwala to także na wykorzystanie tego samego układu w roli multipleksera i demultipleksera.
% END: Elektronika - Transmisja - sterowanie linią


% BEGIN: Elektronika - Typy transmisji
\section{Topologie i typy transmisji}

W zależności od układu fizycznych połączeń komunikujących się urządzeń wyróżnia się różne topologie sieci.
Na schemacie poniżej przedstawione zostały główne topologie połączeń:

\begin{itemize}
\item \strong{magistrala} (linear bus) -- wszystkie urządzenia są podłączone do jednej linii (wspólnego medium transmisyjnego), okablowanie nie wyróżnia punktu centralnego
\item \strong{łańcuch} (daisy chain) -- struktura okablowania podobna jak w magistrali, ale medium transmisyjne jest podzielone (połączenie n urządzeń składa się z n-1 łączy punkt-punkt pomiędzy urządzeniami)
\item \strong{pierścień} (ring) -- topologia daisy chain w której końce są połączone, uodparnia to na pojedyncze uszkodzenie
\item \strong{gwiazda} (star) -- wszystkie podłączenia biorą początek w węźle centralnym, w zależności od konstrukcji węzła centralnego może być realizowana w oparciu o współdzielone medium lub połączenia punkt-punkt
\item \strong{krata} (mesh) -- każde urządzenie ma bezpośrednie połączenie punkt-punkt do każdego innego urządzenia (połączenie n urządzeń wymaga n(n-1)/2 połączeń punkt punkt)
\end{itemize}

\begin{center}
    \includegraphics[width=0.7\textwidth]{img/elektronika/topologie}
\end{center}

Ponadto występują topologie mieszane złożone z opisanych powyżej: gwiazda wielokrotna (tzn. taka gdzie niektóre z węzłów stanowią punkty centralne kolejnych gwiazd), magistrala lub ring pomiędzy punktami centralnymi gwiazd, gwiazda w której w węzłach występują magistrale lub pierścienie, itd.

\vspace{7pt}

Wyróżnić można także typy transmisji:
\begin{itemize}
\item \strong{simplex} -- umożliwia tylko transmisję jednokierunkową
\item \strong{half-duplex} -- umożliwia transmisję dwukierunkową, ale tylko w jedną stronę równocześnie
\item \strong{full-duplex} -- umożliwia pełną transmisję dwukierunkową (równoczesne nadawanie i odbiór)
\end{itemize}

\section{Magistrala równoległa}

\begin{center}
    \includegraphics[width=0.85\textwidth]{img/elektronika/magistrala_rownolegla}
\end{center}
Magistrala równoległa jest zespołem linii, wraz z układami nimi sterującymi, umożliwiającym równoległe przesyłanie danych (w jednym czasie / takcie zegara na magistrali wystawiane / przesyłane jest całe n-bitowe słowo).
Można wyróżnić szyny sterującą (kierunek przypływu, żądania obsługi, etc), adresową (adres układu który ma prawo nadawać) i danych (przesyłane dane). Często szyna adresowa współdzieli linie transmisyjne z szyną danych.
Do realizacji magistrali (celem umożliwiania podłączenia wielu układów) stosuje się zazwyczaj bufory trój-stanowe, a do zapewnienia współdzielonej szyny żądania obsługi (interrupt request) często układy typu open-collector.

Typowy układ realizacji magistrali half-duplex ze współdzielonymi liniami danych i adresu przestawiony został na schemacie zamieszczonym obok.
Zadaniem dekodera adresu jest ustalenie czy wystawiony na magistrali adres (w trakcie wysokiego stanu linii "Adres / Not Dane") jest adresem danego urządzenia i zapamiętanie tej informacji do czasu wystawienia nowego adresu. Informacja ta jest wykorzystywana do sterowania dwukierunkowym buforem trój-stanowym (jako sygnał enable).
O kierunku działania bufora decyduje sygnał "Read / Not Write". Przy magistralach o ustalonym protokole transmisyjnym sterowanie kierunkiem może być zależne od wykonywanej komendy (po ustawieniu adresu urządzenie odczytuje z magistrali polecenie i w zależności od niego steruje kierunkiem bufora - odczytuje lub zapisuje dane na magistralę).
Zastosowanie kilku linii typu OC do odbierania żądań obsługi pozwala (na podstawie tego które z tych linii znalazły się w stanie niskim na identyfikację urządzenia lub grupy urządzeń, z której niektóre zgłaszają żądanie obsługi.

W przypadku prostych urządzeń wejścia / wyjścia zamiast buforu dwukierunkowego może być umieszczony np.
jednokierunkowy bufor (lub n-bitowy rejestr) z wyjściami trój-stanowymi, który wystawia dane na magistralę w oparciu o sygnał zapisu na magistralę (WR) oraz zegar (clock) albo
n-bitowy rejestr do którego zapisywane są dane z magistrali w oparciu o sygnał RD i Clock.

\section{Magistrala szeregowa}

\begin{center}
    \includegraphics[width=0.85\textwidth]{img/elektronika/magistrala_szeregowa}
\end{center}
W magistrali szeregowej dane przesyłane są bit po bicie w pojedynczej linii. Podobnie jak w magistrali równoległej oprócz linii danych mogą występować także linie sterujące. Prostą realizację magistrali szeregowej zapewniają rejestry przesuwne.

Przykładowy układ realizacji magistrali simplex (jednokierunkowej) z rozdzielonymi szynami danych i adresową został na schemacie zamieszczonym obok.
W prezentowanym przykładzie oprócz adresu master wystawia trzy sygnały - dane, zegar i strobe. Z każdym taktem zegara na linii danych wystawiany jest kolejny bit który jest wczytywany do zespołu rejestrów. Sygnał strobe służy do przepisania wartości z rejestrów przesuwnych do rejestrów wyjściowych, takie rozwiązanie zapobiega zmianom wyjść w trakcie przesyłania nowych danych poprzez szynę szeregową, jest ono jednak opcjonalne.

W zależności od konstrukcji dekodera adresu szyna adresowa może być równoległa (w najprostszym przypadku - przez całą transmisję do danego urządzenia jego adres musi być wystawiony na szynie a dekoder jest układem bramek NOT i wielowejściowej bramki AND) lub szeregowa (w takim wypadku powinna posiadać własny zegar lub sygnał informujący o nadawaniu adresu z taktami zegara głównego, a dekoder powinien być wyposażony w rejestr przesuwny do odebrania i przechowywania aktualnego adresu z magistrali). Natomiast jeżeli magistrala byłaby oparta tylko na połączonych szeregowo rejestrach (wyjście serial-out do wejścia serial-in) to szyna adresowa nie jest potrzebna, ale konieczne może być każdorazowe wpisanie wszystkich wartości na szynę (czas zapisu rośnie z ilością podłączonych rejestrów).

\section{Standardowe interfejsy}

Istnieje wiele zestandaryzowanych interfejsów zarówno szeregowych jak i równoległych, wśród najważniejszych należy wymienić:

\begin{center} \includegraphics[width=0.47\textwidth]{img/elektronika/spi}    \includegraphics[width=0.47\textwidth]{img/elektronika/twi} \end{center}

\subsection{SPI (Serial Peripheral Interface)}
    jest to szeregowa magistrala full-duplex działająca w układzie master-slave złożona z linii zegarowej (SCLK), nadawania przez mastera (MOSI), odbioru przez mastera (MISO) oraz linii służących do aktywacji urządzenia slave (SS / CS). 

\begin{wrapfigure}{r}{0.48\textwidth} \begin{center} \vspace{-20pt} \includegraphics[width=0.43\textwidth]{img/elektronika/onewire} \vspace{-20pt} \end{center} \end{wrapfigure}

\subsection{I2C (TWI)}
    jest to szeregowa magistrala half-duplex złożona z linii sygnałowej (SDA) i zegara (SCL) posiadająca zdefiniowany format ramki wraz z adresowaniem. Z wyjątkiem bitu startu i stopu stan linii danych może zmieniać się tylko przy niskim stanie linii zegarowej.
    Nadajniki są typu open-drain przez co realizowany jest iloczyn na drucie, co pozwala na wykrywanie kolizji (jeżeli dany nadajnik nie nadaje zera a linia jest w stanie zera to nadaje także ktoś inny). Pozwala to także na uzyskanie układów multimaster, pomimo iż typowo na magistrali takiej występuje tylko jeden układ master (nadający sygnał zegara i inicjujący transmisję). 

\subsection{1 wire (one-wire)}
    jest to szeregowa magistrala half-duplex złożona z jedynie z linii sygnałowej (która może także służyć do zasilania urządzeń) posiadająca zdefiniowany format ramki wraz z adresowaniem. Standardowe nadawanie jest realizowane jako open-drain (wyjątkiem jest nadawanie tzw power-byte). 

\subsection{USART}
    jest to uniwersalny synchroniczny i asynchroniczny nadajnik i odbiornik, umożliwia realizację szeregowej transmisji synchronicznej (zgodnie z zegarem) lub asynchronicznej (wykrywanie początku ramki na podstawie linii danych). Interfejs korzysta z rozdzielonych linii nadajnika i odbiornika (wyjście danych TxD oraz wejście danych RxD, co umożliwia realizację transmisji full-duplex) oraz może korzystać z dodatkowych sygnałów sterujących (wyjście RTS informujące o gotowości do odbioru oraz wejście CTS informacji o gotowości odbioru / zezwolenia na nadawanie). Niekiedy dostępne jest także wyjście załączenia nadajnika używane do pracy w trybie half-duplex (linie TxD i RxD połączone buforem trój-stanowym nadajnika).

    \begin{center} \includegraphics[width=0.94\textwidth]{img/elektronika/uart1} \end{center}
    Interfejs ten najczęściej wykorzystywany jest w trybie asynchronicznym jako UART. W połączeniach UART zarówno nadajnik jak i odbiornik muszą mieć ustawione takie same parametry transmisji (szybkość, znaczenie 9 bitu (typowo bit parzystości, ale może także oznaczać np. pole adresowe), itp).
    Głównymi standardami elektrycznymi dla UART są: poziomy napięć układów elektronicznych używających tych portów (3.3V, 5V), RS-232 (w pełnym wariancie używa sygnałów kontroli przepływu, poziom logiczny 1 wynosi od -15V do -3V, a poziom logiczny 0 od +3V do +15V), RS-422 (transmisja różnicowa full-duplex pomiędzy dwoma urządzeniami) i RS-485 (transmisja różnicowa half-duplex w oparciu o szynę łączącą wiele urządzeń, kompatybilny elektrycznie z RS-422), możliwia jest też transmisja światłowodowa i bezprzewodowa.

    \begin{center} \resizebox{0.87\textwidth}{!}{\includegraphics[trim={0 0 0 9cm},clip]{img/elektronika/uart2}} \end{center}
    \begin{center} \resizebox{0.87\textwidth}{!}{\includegraphics[trim={0 6cm 0 0},clip]{img/elektronika/uart2}} \end{center}
% END: Elektronika - Typy transmisji

\begin{ProTip}[breakable]{Rezystory terminujące \zaawansowane{30}}
Niektóre  ze standardów interfejsów komunikacyjnych przewidują kończenie swoich magistral rezystorem terminującym.
Zastosowanie takiego rezystora ma na celu eliminację odbić sygnału, które mogłyby powstać na końcu linii transmisyjnej.

\vspace{7pt}

Zjawisko to występuje w przypadku \textit{linii długich}, czyli takich których długość jest zbliżona lub większa od długości fali związanej z przesyłanym sygnałem.
Jeżeli rozważymy np. impuls o czasie trwania 1$\mu \rm s$ to zajmie on na kablu odcinek o długości około 200m (zależy to od prędkości rozchodzenia się fali elektromagnetycznej w ośrodku który stanowi dany przewód).
Zatem dla sygnału 1MHz (czyli takiego gdzie pojedyncze impulsy są właśnie długości 1$\mu \rm s$) przewód o długości kilkuset metrów będzie linią długą.

\vspace{7pt}

Odbicia te wynikają z faktu, iż przemieszczanie się sygnału (np. naszego impulsu 5V o czasie trwania 1$\mu \rm s$) wzdłuż przewodu związane jest z
	ładowaniem kolejnych pojemności pasożytniczych, związanych z odcinkiem przewodu do którego dociera sygnał.
Dzieje się to kosztem rozładowania pojemności odcinka przewodu który sygnał już opuścił.

W momencie gdy sygnał trafia na koniec przewodnika nie ma możliwości rozładowania tej pojemności na kolejny odcinek przewodu, więc ładunek z nią związany „rozpływa się po kablu” powodując powstanie odbicia.
Odbicie takie (biegnące od końca przewodu w stronę nadajnika) nakłada się na kolejne impulsy naszego sygnału (biegnące od nadajnika) i powoduje zakłócenia w ich odbiorze (interpretacji).

Zastosowanie odpowiedniego rezystora na końcu linii pozwala na rozładowanie tej pojemności (tak jakby był tam kolejny nieskończenie długi odcinek przewodu) i eliminację odbicia.
Wartość tej rezystancji jest charakterystyczna dla danego przewodu i określana przez parametr nazywany \textit{impedancją falową}.

Rezystor terminujący stanowi obciążenie dla nadajnika i powinien on być stosowany tylko na końcach magistrali, czyli np. na ostatnich urządzeniach podłączonych do magistrali (a nie przy każdym urządzeniu do niej włączonym).

\vspace{7pt}

Jeżeli długość linii jest dużo mniejsza (dla sygnałów prostokątnych przyjmuje się że około 20-40 razy) od długości odcinka jaką zajmuje pojedynczy impuls (np. linia długości 3m, dla przykładowego sygnału 1MHz)
to nie ma potrzeby stosowania rezystorów terminujących (często nawet gdy w ogólności dany standard je przewiduje), gdyż stan całej linii jest spójny i wymuszany przez nadajnik (nie jest to przypadek linii długiej).

\vspace{7pt}

Standard I2C nie przewiduje rezystorów terminujących (i nie powinny być w nim używane, zwłaszcza że są to linie open drain i powstawałby dzielnik z rezystorem pull-up).
Wynika to z tego iż przy maksymalnej prędkości tego interfejsu za linie długie należałoby uznać odcinki co najmniej kilkunastometrowe, a z innych względów standard ten posiada ograniczenie do kilku metrów.

Standard RS-485 przewiduje stosowanie rezystorów terminujących 120 Ω, jednak w przypadku krótkich połączeń i/lub małych prędkości transmisji mogą one być pominięte.
\end{ProTip}

\subsection{Zadania praktyczne}
	\insertZadanie{booklets-sections/electronics/zadania-20-30-cyfrowa.tex}{zbuduj_rejestr_przesuwny}{}
% Copyright (c) 2017-2020 Matematyka dla Ciekawych Świata (http://ciekawi.icm.edu.pl/)
% Copyright (c) 2017-2020 Robert Ryszard Paciorek <rrp@opcode.eu.org>
% 
% MIT License
% 
% Permission is hereby granted, free of charge, to any person obtaining a copy
% of this software and associated documentation files (the "Software"), to deal
% in the Software without restriction, including without limitation the rights
% to use, copy, modify, merge, publish, distribute, sublicense, and/or sell
% copies of the Software, and to permit persons to whom the Software is
% furnished to do so, subject to the following conditions:
% 
% The above copyright notice and this permission notice shall be included in all
% copies or substantial portions of the Software.
% 
% THE SOFTWARE IS PROVIDED "AS IS", WITHOUT WARRANTY OF ANY KIND, EXPRESS OR
% IMPLIED, INCLUDING BUT NOT LIMITED TO THE WARRANTIES OF MERCHANTABILITY,
% FITNESS FOR A PARTICULAR PURPOSE AND NONINFRINGEMENT. IN NO EVENT SHALL THE
% AUTHORS OR COPYRIGHT HOLDERS BE LIABLE FOR ANY CLAIM, DAMAGES OR OTHER
% LIABILITY, WHETHER IN AN ACTION OF CONTRACT, TORT OR OTHERWISE, ARISING FROM,
% OUT OF OR IN CONNECTION WITH THE SOFTWARE OR THE USE OR OTHER DEALINGS IN THE
% SOFTWARE.

\section{Standardowe interfejsy}

Istnieje wiele zestandaryzowanych interfejsów zarówno szeregowych jak i równoległych, wśród najważniejszych należy wymienić:

\begin{center} \includegraphics[width=0.47\textwidth]{img/elektronika/spi}    \includegraphics[width=0.47\textwidth]{img/elektronika/twi} \end{center}

\subsection{SPI (Serial Peripheral Interface)}
    jest to szeregowa magistrala full-duplex działająca w układzie master-slave złożona z linii zegarowej (SCLK), nadawania przez mastera (MOSI), odbioru przez mastera (MISO) oraz linii służących do aktywacji urządzenia slave (SS / CS). 

\begin{wrapfigure}{r}{0.48\textwidth} \begin{center} \vspace{-20pt} \includegraphics[width=0.43\textwidth]{img/elektronika/onewire} \vspace{-20pt} \end{center} \end{wrapfigure}

\subsection{I2C (TWI)}
    jest to szeregowa magistrala half-duplex złożona z linii sygnałowej (SDA) i zegara (SCL) posiadająca zdefiniowany format ramki wraz z adresowaniem. Z wyjątkiem bitu startu i stopu stan linii danych może zmieniać się tylko przy niskim stanie linii zegarowej.
    Nadajniki są typu open-drain przez co realizowany jest iloczyn na drucie, co pozwala na wykrywanie kolizji (jeżeli dany nadajnik nie nadaje zera a linia jest w stanie zera to nadaje także ktoś inny). Pozwala to także na uzyskanie układów multimaster, pomimo iż typowo na magistrali takiej występuje tylko jeden układ master (nadający sygnał zegara i inicjujący transmisję). 

\subsection{1 wire (one-wire)}
    jest to szeregowa magistrala half-duplex złożona z jedynie z linii sygnałowej (która może także służyć do zasilania urządzeń) posiadająca zdefiniowany format ramki wraz z adresowaniem. Standardowe nadawanie jest realizowane jako open-drain (wyjątkiem jest nadawanie tzw power-byte). 

\subsection{USART}
    jest to uniwersalny synchroniczny i asynchroniczny nadajnik i odbiornik, umożliwia realizację szeregowej transmisji synchronicznej (zgodnie z zegarem) lub asynchronicznej (wykrywanie początku ramki na podstawie linii danych). Interfejs korzysta z rozdzielonych linii nadajnika i odbiornika (wyjście danych TxD oraz wejście danych RxD, co umożliwia realizację transmisji full-duplex) oraz może korzystać z dodatkowych sygnałów sterujących (wyjście RTS informujące o gotowości do odbioru oraz wejście CTS informacji o gotowości odbioru / zezwolenia na nadawanie). Niekiedy dostępne jest także wyjście załączenia nadajnika używane do pracy w trybie half-duplex (linie TxD i RxD połączone buforem trój-stanowym nadajnika).

    \begin{center} \includegraphics[width=0.94\textwidth]{img/elektronika/uart1} \end{center}
    Interfejs ten najczęściej wykorzystywany jest w trybie asynchronicznym jako UART. W połączeniach UART zarówno nadajnik jak i odbiornik muszą mieć ustawione takie same parametry transmisji (szybkość, znaczenie 9 bitu (typowo bit parzystości, ale może także oznaczać np. pole adresowe), itp).
    Głównymi standardami elektrycznymi dla UART są: poziomy napięć układów elektronicznych używających tych portów (3.3V, 5V), RS-232 (w pełnym wariancie używa sygnałów kontroli przepływu, poziom logiczny 1 wynosi od -15V do -3V, a poziom logiczny 0 od +3V do +15V), RS-422 (transmisja różnicowa full-duplex pomiędzy dwoma urządzeniami) i RS-485 (transmisja różnicowa half-duplex w oparciu o szynę łączącą wiele urządzeń, kompatybilny elektrycznie z RS-422), możliwa jest też transmisja światłowodowa i bezprzewodowa.

    \begin{center} \resizebox{0.87\textwidth}{!}{\includegraphics[trim={0 0 0 9cm},clip]{img/elektronika/uart2}} \end{center}
    \begin{center} \resizebox{0.87\textwidth}{!}{\includegraphics[trim={0 6cm 0 0},clip]{img/elektronika/uart2}} \end{center}
% END: Elektronika - Typy transmisji

\begin{ProTip}[breakable]{Rezystory terminujące \zaawansowane{30}}
Niektóre  ze standardów interfejsów komunikacyjnych przewidują kończenie swoich magistral rezystorem terminującym.
Zastosowanie takiego rezystora ma na celu eliminację odbić sygnału, które mogłyby powstać na końcu linii transmisyjnej.

\vspace{7pt}

Zjawisko to występuje w przypadku \textit{linii długich}, czyli takich których długość jest zbliżona lub większa od długości fali związanej z przesyłanym sygnałem.
Jeżeli rozważymy np. impuls o czasie trwania 1$\mu \rm s$ to zajmie on na kablu odcinek o długości około 200m (zależy to od prędkości rozchodzenia się fali elektromagnetycznej w ośrodku który stanowi dany przewód).
Zatem dla sygnału 1MHz (czyli takiego gdzie pojedyncze impulsy są właśnie długości 1$\mu \rm s$) przewód o długości kilkuset metrów będzie linią długą.

\vspace{7pt}

Odbicia te wynikają z faktu, iż przemieszczanie się sygnału (np. naszego impulsu 5V o czasie trwania 1$\mu \rm s$) wzdłuż przewodu związane jest z
	ładowaniem kolejnych pojemności pasożytniczych, związanych z odcinkiem przewodu do którego dociera sygnał.
Dzieje się to kosztem rozładowania pojemności odcinka przewodu który sygnał już opuścił.

W momencie gdy sygnał trafia na koniec przewodnika nie ma możliwości rozładowania tej pojemności na kolejny odcinek przewodu, więc ładunek z nią związany „rozpływa się po kablu” powodując powstanie odbicia.
Odbicie takie (biegnące od końca przewodu w stronę nadajnika) nakłada się na kolejne impulsy naszego sygnału (biegnące od nadajnika) i powoduje zakłócenia w ich odbiorze (interpretacji).

Zastosowanie odpowiedniego rezystora na końcu linii pozwala na rozładowanie tej pojemności (tak jakby był tam kolejny nieskończenie długi odcinek przewodu) i eliminację odbicia.
Wartość tej rezystancji jest charakterystyczna dla danego przewodu i określana przez parametr nazywany \textit{impedancją falową}.

Rezystor terminujący stanowi obciążenie dla nadajnika i powinien on być stosowany tylko na końcach magistrali, czyli np. na ostatnich urządzeniach podłączonych do magistrali (a nie przy każdym urządzeniu do niej włączonym).

\vspace{7pt}

Jeżeli długość linii jest dużo mniejsza (dla sygnałów prostokątnych przyjmuje się że około 20-40 razy) od długości odcinka jaką zajmuje pojedynczy impuls (np. linia długości 3m, dla przykładowego sygnału 1MHz)
to nie ma potrzeby stosowania rezystorów terminujących (często nawet gdy w ogólności dany standard je przewiduje), gdyż stan całej linii jest spójny i wymuszany przez nadajnik (nie jest to przypadek linii długiej).

\vspace{7pt}

Standard I2C nie przewiduje rezystorów terminujących (i nie powinny być w nim używane, zwłaszcza że są to linie open drain i powstawałby dzielnik z rezystorem pull-up).
Wynika to z tego iż przy maksymalnej prędkości tego interfejsu za linie długie należałoby uznać odcinki co najmniej kilkunastometrowe, a z innych względów standard ten posiada ograniczenie do kilku metrów.

Standard RS-485 przewiduje stosowanie rezystorów terminujących 120 Ω, jednak w przypadku krótkich połączeń i/lub małych prędkości transmisji mogą one być pominięte.
\end{ProTip}


% Copyright (c) 2017-2020 Matematyka dla Ciekawych Świata (http://ciekawi.icm.edu.pl/)
% Copyright (c) 2017-2020 Robert Ryszard Paciorek <rrp@opcode.eu.org>
% 
% MIT License
% 
% Permission is hereby granted, free of charge, to any person obtaining a copy
% of this software and associated documentation files (the "Software"), to deal
% in the Software without restriction, including without limitation the rights
% to use, copy, modify, merge, publish, distribute, sublicense, and/or sell
% copies of the Software, and to permit persons to whom the Software is
% furnished to do so, subject to the following conditions:
% 
% The above copyright notice and this permission notice shall be included in all
% copies or substantial portions of the Software.
% 
% THE SOFTWARE IS PROVIDED "AS IS", WITHOUT WARRANTY OF ANY KIND, EXPRESS OR
% IMPLIED, INCLUDING BUT NOT LIMITED TO THE WARRANTIES OF MERCHANTABILITY,
% FITNESS FOR A PARTICULAR PURPOSE AND NONINFRINGEMENT. IN NO EVENT SHALL THE
% AUTHORS OR COPYRIGHT HOLDERS BE LIABLE FOR ANY CLAIM, DAMAGES OR OTHER
% LIABILITY, WHETHER IN AN ACTION OF CONTRACT, TORT OR OTHERWISE, ARISING FROM,
% OUT OF OR IN CONNECTION WITH THE SOFTWARE OR THE USE OR OTHER DEALINGS IN THE
% SOFTWARE.

% BEGIN: Elektronika - Układy programowalne
\section{Układy programowalne}
\begin{teacherOnly}
	\begin{easylist}[itemize]
	& układy programowalne
		&& programowalna logika
			&& komórka pamięci może realizować dowolną funkcję logiczną
		&& procesory i mikro-kontrolery
	\end{easylist}
\end{teacherOnly}

\subsection{układy o programowalnej strukturze (PLD)}

Są to układy w których programowany jest układ bramek, przerzutników, itp. "umieszczanych" w kości oraz ich połączeń.

Program dla takich układów tworzony jest w Hardware Description Language (najczęściej VHDL lub Verilog) i zamiast wykonywanego kodu opisuje strukturę układu logicznego (połączenia bramek, tablice prawdy, etc), która następnie jest programowana w fizycznej kości.

Najprostszym przykładem układu o programowalnej strukturze logicznej jest układ pamięci $2^n$ bitowej z n-bitową szyną adresową adresującą pojedyncze bity - pozwala on na realizację dowolnej funkcji logicznej o n wejściach i pojedynczym wyjściu.

Do kategorii tej zaliczają się układy typu:
\begin{itemize}
	\item SPLD
	\begin{itemize}
		\item PLE - programowalna matryca bramek OR
		\item PAL i GAL - programowalna matryca AND z dodatkowymi bramkami OR (często także obudowana rejestrami i multiplekserami na wyjściach)
		\item PLA - programowalne matryce AND i OR
	\end{itemize}
	\item CPLD
	\item FPGA - programowalny element pamięciowy (możliwość zdefiniowania dowolnej - na ogół 4 wejściowej - funkcji w każdym elemencie logicznym, programowalne połączenia między elementami logicznymi i pinami, itd)
\end{itemize}

\subsection{systemy procesorowe}
Są to systemy realizujące ciąg instrukcji pobieranych z jakiejś pamięci.

System taki składa się z procesora odpowiedzialnego za interpretację i wykonywanie kolejnych instrukcji oraz pamięci z której pobierane są instrukcje i dane (może to być jedna pamięć, mogą to być rozdzielone pamięci). Do kategorii tej zaliczają się zarówno typowe systemy komputerowe, systemy obliczeniowe jak i różnego rodzaju programowalne mikrokontrolery.

Procesor pracuje w cyklach rozkazowych, w ramach których przetwarza pojedynczą instrukcję. Cykl taki może trwać od 1 do kilku lub więcej cykli zegarowych i typowo składa się z następujących kroków:
\begin{enumerate}
	\item pobranie instrukcji z pamięci - realizowane jest poprzez wystawienie na szynę adresową zawartości \emph{licznika programu} (zawierające adres instrukcji do wykonania) oraz wygenerowanie cyklu odczytu z pamięci, po wykonaniu odczytu danych następuje ich zapamiętanie w \emph{rejestrze instrukcji} oraz zwiększenie wartości \emph{licznika programu} o jeden;\\
		(zawartość rejestru \emph{licznika programu} po resecie procesora określa skąd pobierana będzie pierwsza instrukcja, pod takim adresem zazwyczaj umieszczana jest jakaś pamięć typu ROM lub flash)
	\item dekodowanie instrukcji - układ dekodera (np. oparty o PLA) dokonuje zdekodowania instrukcji znajdującej się w \emph{rejestrze instrukcji} i konfiguracji procesora w zależności od jej kodu i (opcjonalnie) jej argumentów; może to być np.:
	\begin{itemize}
		\item odpowiednie ustawienie multiplekserów pomiędzy rejestrami a jednostką ALU oraz wystawienie odpowiedniego kod operacji dla ALU (celem wykonania operacji arytmetycznej na wartościach rejestrów)
		\item wystawienie zawartości wskazanego rejestru na szynę adresową, podłączenie wskazanego rejestru do szyny danych oraz skonfigurowanie operacji odczytu/zapisu (celem wykonania odczytu lub zapisu wartości rejestru z/do pamięci)
	\end{itemize}
	\item wykonanie instrukcji - realizacja wcześniej zdekodowanej instrukcji zgodnie z ustawioną konfiguracją procesora
\end{enumerate}

Instrukcje skoku polegają na załadowaniu nowej wartości do \emph{licznika programu}, w przypadku skoków warunkowych jest to uzależnione od stanu \emph{rejestru flag}, które ustawiane są w oparciu o wynik ostatniej operacji wykonywanej przez ALU.

Przedstawiony model działania jest przykładowym i w rzeczywistym procesorze może to wyglądać odmiennie - np. długość instrukcji może być większa niż długość słowa używanego przez procesor / szerokość szyny danych co rozbudowuje fazę pobierania instrukcji z pamięci, mogą występować instrukcje bardziej złożone (np. operacje wykonywane z argumentem pobieranym z pamięci a nie rejestru), może także występować więcej faz (np. poprzez wydzielenie faz dostępu do pamięci, czy zapisywania wyników działania instrukcji).
Procesor może także działać potokowo, czyli nakładać na siebie kolejne fazy wykonywania różnych instrukcji (np. w czasie wykonywania jednej instrukcji realizować pobieranie kolejnej).

\subsubsection{Mikrokontrolery}

Mikrokontroler jest układem typu "System on Chip" zawierającym w jednym układzie procesor, pamięć RAM, układy wejścia-wyjścia (np. GPIO, porty szeregowe typu USART, SPI, I2C, przetworniki ADC), pamięć typu Flash (dla programu).
% END: Elektronika - Układy programowalne


\newcommand{\UrlToPythonRefToDRYKISS}{http://www.opcode.eu.org/Wprowadzenie_do_programowania_w_Pythonie.pdf}
% Copyright (c) 2017-2020 Matematyka dla Ciekawych Świata (http://ciekawi.icm.edu.pl/)
% Copyright (c) 2017-2020 Robert Ryszard Paciorek <rrp@opcode.eu.org>
% 
% MIT License
% 
% Permission is hereby granted, free of charge, to any person obtaining a copy
% of this software and associated documentation files (the "Software"), to deal
% in the Software without restriction, including without limitation the rights
% to use, copy, modify, merge, publish, distribute, sublicense, and/or sell
% copies of the Software, and to permit persons to whom the Software is
% furnished to do so, subject to the following conditions:
% 
% The above copyright notice and this permission notice shall be included in all
% copies or substantial portions of the Software.
% 
% THE SOFTWARE IS PROVIDED "AS IS", WITHOUT WARRANTY OF ANY KIND, EXPRESS OR
% IMPLIED, INCLUDING BUT NOT LIMITED TO THE WARRANTIES OF MERCHANTABILITY,
% FITNESS FOR A PARTICULAR PURPOSE AND NONINFRINGEMENT. IN NO EVENT SHALL THE
% AUTHORS OR COPYRIGHT HOLDERS BE LIABLE FOR ANY CLAIM, DAMAGES OR OTHER
% LIABILITY, WHETHER IN AN ACTION OF CONTRACT, TORT OR OTHERWISE, ARISING FROM,
% OUT OF OR IN CONNECTION WITH THE SOFTWARE OR THE USE OR OTHER DEALINGS IN THE
% SOFTWARE.

\section{Projektowanie \zaawansowane{20}}

\ifdefined\TextRefToDRYKISS\else
\ifdefined\UrlToPythonRefToDRYKISS
	\newcommand{\TextRefToDRYKISS}{, o których była mowa przy omawianiu \href{\UrlToPythonRefToDRYKISS}{bibliotek w pythonie},}
\else
	% Copyright (c) 2018-2020 Matematyka dla Ciekawych Świata (http://ciekawi.icm.edu.pl/)
% Copyright (c) 2018-2020 Robert Ryszard Paciorek <rrp@opcode.eu.org>
% 
% MIT License
% 
% Permission is hereby granted, free of charge, to any person obtaining a copy
% of this software and associated documentation files (the "Software"), to deal
% in the Software without restriction, including without limitation the rights
% to use, copy, modify, merge, publish, distribute, sublicense, and/or sell
% copies of the Software, and to permit persons to whom the Software is
% furnished to do so, subject to the following conditions:
% 
% The above copyright notice and this permission notice shall be included in all
% copies or substantial portions of the Software.
% 
% THE SOFTWARE IS PROVIDED "AS IS", WITHOUT WARRANTY OF ANY KIND, EXPRESS OR
% IMPLIED, INCLUDING BUT NOT LIMITED TO THE WARRANTIES OF MERCHANTABILITY,
% FITNESS FOR A PARTICULAR PURPOSE AND NONINFRINGEMENT. IN NO EVENT SHALL THE
% AUTHORS OR COPYRIGHT HOLDERS BE LIABLE FOR ANY CLAIM, DAMAGES OR OTHER
% LIABILITY, WHETHER IN AN ACTION OF CONTRACT, TORT OR OTHERWISE, ARISING FROM,
% OUT OF OR IN CONNECTION WITH THE SOFTWARE OR THE USE OR OTHER DEALINGS IN THE
% SOFTWARE.

\begin{ProTip}[breakable]{Reguły DRY i KISS}
\textbf{„Don't Repeat Yourself”} (\textit{nie powtarzaj się}) jest jedną z dwóch głównych reguł programistycznych (ale ma także pewne zastosowania w innych dziedzinach techniki).
Zaleca ona unikanie potarzania tych samych czynności, czy też tworzenia takich samych, a nawet analogicznych, podobnych fragmentów kodu.

Narzędziami ułatwiającymi realizację tego celu są m.in.:
\begin{itemize}
\item systemy i skrypty służące automatyzacji różnego rodzaju czynności (takich jak np. kompilacja, instalacja, aktualizacja, monitoring działania) –
      zarówno systemy takie jak make, cmake, doxygen ale również wszystkie drobne skrypty (np. shellowe czy pythonowe) tworzone w tym celu w codziennej pracy informatyka
\item elementy składniowe (m.in. takie jak pętle i funkcje) oraz mechanizmy (np. polimorfizm) dostępne w językach programowania pozwalające na eliminację powtórzeń kodu
\item biblioteki, moduły, itp pozwalające na współdzielenie tych samych rozwiązań, tego samego kodu, pomiędzy różnymi projektami
\item elementy biblioteki systemowej pozwalające na wywoływanie innych programów (np. exec) i komunikację z nimi (np. poprzez strumienie wejścia/wyjścia)
\end{itemize}

Unikanie powtórzeń takiego samego lub (co często nawet gorsze) tylko nieznacznie zmienionego kodu jest też szczególnie istotne ze względu na łatwość utrzymania kodu
– np. jakąś poprawkę wprowadza się tylko w odpowiednio sparametryzowanej funkcji, a nie kilkunastu podobnych (ale nie identycznych, ze względu na brak parametryzacji) fragmentach kodu.

W zastosowaniach nie programistycznych przejawia się często wydzielaniem modułów i dążeniem do ich powtarzalności, redukcji ilości ich typów (np. dzięki parametryzacji, czy konfigurowalności).

\vspace{7pt}

Drugą, nawet chyba ważniejszą, z tych dwóch reguł jest \textbf{„Keep It Simple, Stupid”} (niekiedy \textit{Keep It Small and Simple}), którą można streścić jako \textit{proste jest lepsze}.
Reguła KISS jest bardziej ogólna (można nawet powiedzieć że wynika z niej reguła DRY), posiada dużo szersze pole zastosowań (także nie technicznych) i może być uważana za implementację \textit{Brzytwy Ockhama} w inżynierii.
Zaleca ona m.in.:
\begin{itemize}
\item tworzenie przejrzystych, czytelnych i prostych rozwiązań (zarówno pod względem samego projektu, koncepcji, jak też ich implementacji, wykonania)
\item wybór rozwiązania prostszego spośród (równie) skutecznych rozwiązań jakiegoś problemu
\item myślenie o łatwości późniejszego utrzymania i serwisu tworzonego rozwiązania (czy to kodu programu, czy urządzenia elektronicznego, a nawet budynku)
\end{itemize}

\end{ProTip}


	\newcommand{\TextRefToDRYKISS}{}
\fi\fi

Reguły DRY i KISS\TextRefToDRYKISS{} mają zastosowanie także w elektronice, a zwłaszcza projektowaniu układów elektronicznych i różnego rodzaju systemów automatyki.
Przy projektowaniu elektroniki trochę trudniej niż w programowaniu jest zachować regułę DRY (zwłaszcza przy tworzeniu projektów płytek drukowanych \textit{PCB}),
	jednak należy dążyć do tego – wydzielać powtarzające się elementy do pod-schematów, modularyzować tworzony projekt, itd.

Kurs ten poświęcony jest głównie elektronice cyfrowej, a ta współcześnie opiera się na wykorzystaniu układów programowalnych.
Zatem dalsze rady będą dotyczyły głównie projektowania systemu/urządzenia opartego na jakimś mikrokontrolerze
(mimo to wiele z nich można zastosować także przy projektowaniu innych układów elektronicznych i nie tylko).
\\
Przy tworzeniu projektów tego typu systemów/urządzeń należy mieć szczególnie na uwadze:
\begin{itemize}
	\item trzymanie się standardów i modułowość:
		\begin{itemize}
			\item należy stosować standardowe, popularne, otwarte protokoły komunikacyjne, takie jak I$^2$C, RS-485/Modbus, Ethernet/IP
			\item należy dokonywać podziału system na moduły funkcjonalne i określamy interfejsy pomiędzy nimi, zasadę tę należy stosować rekurencyjnie do wszystkich większych/bardziej złożonych jego elementów
			\item należy unikać projektowania modułów „na miarę” (lepiej mieć n+2 jednakowych modułów niż n każdy innego rodzaju)
			\item gdy oczekujemy większej niezawodności to należy zastosować redundancję – np. podwójne układy zasilania i komunikacji (dwa porty RS-485 lub dwa porty Ethernet) w każdym z urządzeń (modułów) składających się na system
			
			\item protokoły komunikacyjne, prędkość transmisji, etc należy dobierać z sporym zapasem (rzędu nawet 50-70\%), przewidując pojawienie się kolejnych rejestrów, funkcji, itp.
			\item należy przewidzieć rezerwę miejsca w modułach (np. dostępnych wejść)
		\end{itemize}
	\item dokumentacja, wersjonowanie:
		\begin{itemize}
			\item należy stosować wersjonowanie nie tylko kodu, ale również schematów, projektów PCB, dokumentacji, etc związanych z naszym projektem
			\item należy także oznaczać wersje związane z wykonanymi prototypami (w taki sposób aby można było je jednoznacznie powiązać z fizycznie wykonanym prototypem) i trzymać je co najmniej do zakończenia życia tych prototypów
			\item należy umieszczać identyfikator wersji na tworzonych płytkach PCB
			\item należy tworzyć dokumentację, np. samo użycie modbus nie rozwiązuje kwestii dokumentacji komunikacji – konieczna jest rzetelna tabela rejestrów
		\end{itemize}
	\item zachowanie prostoty:
		\begin{itemize}
			\item jeżeli używany mikrokontroler ma wbudowaną funkcję podciągania wejść – używać jej zamiast zewnętrznych rezystorów pull-up / pull-down,
				jeżeli ma tylko pull-up to przyciski robić jako zwierane do masy aby móc z niej skorzystać
			\item pamiętać że często (wbrew początkowej intuicji) sterowanie czymś przy pomocy wyjścia cyfrowego (zwłaszcza gdy wymagany jest do tego zewnętrzny tranzystor) prostsze jest od strony masy,
				czyli poprzez odłączanie/podłączanie do sterowanego układu masy (a nie dodatniego bieguna zasilania)
			\item warto także ograniczyć liczbę wykorzystywanych rodzin i modeli mikrokontrolerów
				– w przypadku nie seryjnej produkcji oszczędności z zastosowania np. bardzo małego mikrokontrolera zostaną zatarte „kosztami” związanymi z trudnościami w jego użyciu (brak pinów do debugowania, mała pamięć programu, itd.)
				oraz związanymi z rozwijaniem projektów na różnych mikroprocesorach (bo do kolejnego był już za mały)
		\end{itemize}
	\item łatwość diagnostyki i serwisu:
		\begin{itemize}
			\item należy zapewnić reset układów peryferyjnych wraz z resetem mikrokontrolera (np. poprzez użycie jednego z pinów GPIO mikrokontrolera w tej roli),
				dotyczy to także przypadków gdy naszymi urządzeniami peryferyjnymi są inne mikrokontrolery
			\item należy zapewnić łatwą możliwość przeprogramowania mikrokontrolera (bez potrzeby rozmontowywania układu, czy też wyjmowania z niego mikrokontrolera),
				a jeżeli aktualizacja wbudowanego oprogramowania odbywa się standardowo w jakiś wyżej poziomowy sposób to należy zapewnić zabezpieczenie przed awarią w jej trakcie,
				np. poprzez umieszczanie go na wymiennej karcie SD lub dostęp (np. poprzez UART) do sprzętowego bootloadera danego mikrokontrolera, lub zewnętrzny dostęp do programowania odpowiedniej pamięci
			\item warto zapewnić 2-3 diody sygnalizacyjne informujące o stanie pracy / awarii naszego urządzenia
			
			\item należy zapewniać możliwości naprawy i modyfikacji poszczególnych elementów systemu – przede wszystkim poprzez zapewnienie dostępu do nich
				(a gdy będzie on nie możliwy lub bardzo trudny poprzez położenie zapasowych przewodów), wykonywanie połączeń w łatwo dostępnych miejscach, itd.
			\item należy zachowywać kompatybilność wsteczną zawsze wtedy gdy tylko to jest możliwe (nowe urządzenie, nowa wersja muszą pracować w istniejącej sieci, nowa wersja musi w prosty sposób móc zastąpić poprzednią)
			\item należy konsekwentnie trzymać się określonych interfejsów i protokołów, jest to szczególnie ważne w niskopoziomowych (trudno aktualizowalnych, debugowalnych, występujących w dużej liczbie egzemplarzy) urządzeniach
		\end{itemize}
	\item nie tworzenie „potworków” (bo to się będzie mściło):
		\begin{itemize}
			\item należy unikać sztucznego ograniczania funkcjonalności tworzonego urządzenia (co jest nagminnie czynione w urządzeniach powszechnie dostępnych na rynku), na przykład:
				\begin{itemize}
					\item jeżeli urządzenia ma złącze Ethernet i używa je np. do wystawienia jakiegoś WWW to należy udostępnić pełną funkcjonalność monitoringu/konfiguracji tego urządzenia przez to TCP/IP z użyciem standardowych protokołów (np. Modbus TCP)a nie wymagać do tego osobnego modułu
					\item jeżeli urządzenia ma system operacyjny (np. Linuxa) należy zapewnić możliwość pełnego dostępu do niego (z prawami root'a) – także użytkownikowi, jeżeli je od nas kupił to on jest jego właścicielem
				\end{itemize}
			\item z drugiej strony należy jednak unikać upychania funkcjonalności do granic możliwości
				(lepiej pozwolić „zmarnować się” kilku nóżkom mikrokontrolera niż zrezygnować z czytelności, powtarzalności czy możliwości diagnostycznych na rzecz np. obsłużenia kilku dodatkowych IO)
			\item należy unikać oszczędzania kilka złotych minimalizując ponad miarę rozmiar PCB czy eliminując jakieś złącza lub je ograniczając
				(na złączach oprócz sygnałów należy wystawiać też potrzebne zasilania/masy w odpowiednich ilościach)
			
			\item należy walczyć z z problemem plątaniny kabli już na etapie założeń projektowych poprzez:
				\begin{itemize}
					\item stosowanie modularności – np. 1 sterownik Ethernet/IP (lub co najmniej RS-485/Modbus) na niezbyt dużą grupę wejść (np. jeden panel przycisków) / wyjść (np. jedną grupę przekaźników)
					\item nie oszczędzanie miejsca na PCB i funduszy na gniazdka przyłączeniowe (nie robić przylutowywanej na stałe wiązki kabli, pamiętać o masach i zasilaniach)
					\item nie oszczędzanie miejsca na PCB na otwory montażowe (najlepiej na wszystkich modułach wykonywać je w identycznych miejscach - z myślą o przyszłej obudowie, a nie tam gdzie popadnie)
					\item w przypadku wykonywania większych układów, złożonych z kilku sterowników rozważyć stosowanie korytek grzebieniowych do ukrycia plątaniny kabli
					\item w przypadku stosowania magistrali równoległej rozważyć multipleksowanie linii adresowej i danych
				\end{itemize}
			\item należy pamiętać że źle zastosowane technologie, które mają służyć ułatwieniu serwisowania i obsługi systemu (koryta kablowe, stelaże/szafy rack 19", obudowy z mocowaniem na szynę DIN / TH-35) mogą przynieść odwrotny skutek,
				a liczy efekt w postaci dobrze zaprojektowanego, czytelnego, dobrze działającego, serwisowalnego, rozbudowywalnego urządzenia/systemu a nie konkretnie zastosowane technologie
		\end{itemize}
\end{itemize}


\section{Wykład wideo}
\input{booklets-sections/electronics/wykład-video-1.tex}
\input{booklets-sections/electronics/wykład-video-2.tex}

\section{Literatura dodatkowa \zaawansowane{10}}
% Copyright (c) 2018-2020 Matematyka dla Ciekawych Świata (http://ciekawi.icm.edu.pl/)
% Copyright (c) 2018-2020 Robert Ryszard Paciorek <rrp@opcode.eu.org>
% 
% MIT License
% 
% Permission is hereby granted, free of charge, to any person obtaining a copy
% of this software and associated documentation files (the "Software"), to deal
% in the Software without restriction, including without limitation the rights
% to use, copy, modify, merge, publish, distribute, sublicense, and/or sell
% copies of the Software, and to permit persons to whom the Software is
% furnished to do so, subject to the following conditions:
% 
% The above copyright notice and this permission notice shall be included in all
% copies or substantial portions of the Software.
% 
% THE SOFTWARE IS PROVIDED "AS IS", WITHOUT WARRANTY OF ANY KIND, EXPRESS OR
% IMPLIED, INCLUDING BUT NOT LIMITED TO THE WARRANTIES OF MERCHANTABILITY,
% FITNESS FOR A PARTICULAR PURPOSE AND NONINFRINGEMENT. IN NO EVENT SHALL THE
% AUTHORS OR COPYRIGHT HOLDERS BE LIABLE FOR ANY CLAIM, DAMAGES OR OTHER
% LIABILITY, WHETHER IN AN ACTION OF CONTRACT, TORT OR OTHERWISE, ARISING FROM,
% OUT OF OR IN CONNECTION WITH THE SOFTWARE OR THE USE OR OTHER DEALINGS IN THE
% SOFTWARE.

\begin{itemize}
\item \textit{SSH jako VPN} (\url{http://blog.opcode.eu.org/2020/06/09/ssh_jako_vpn.html}) – opis konfiguracji tuneli SSH
\item \textit{Linux - podręcznik administratora sieci} (\url{http://www.interklasa.pl/portal/index/subjectpages/informatyka/linuxadm.pdf})
\item \textit{Introduction to TCP/IP} (\url{https://www.coursera.org/learn/tcpip/home/welcome}) – kurs na coursera.org
\end{itemize}


\ZadaniaRozwiazaniaAuto[norepeat=true]

\copyrightFooter{
	© Matematyka dla Ciekawych Świata, 2017-2021.\\
	© Robert Ryszard Paciorek <rrp@opcode.eu.org>, 2003-2021.\\
	Kopiowanie, modyfikowanie i redystrybucja dozwolone pod warunkiem zachowania informacji o autorach.
}
\end{document}
