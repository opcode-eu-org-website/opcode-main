% Copyright (c) 2017-2020 Matematyka dla Ciekawych Świata (http://ciekawi.icm.edu.pl/)
% Copyright (c) 2017-2020 Robert Ryszard Paciorek <rrp@opcode.eu.org>
% 
% MIT License
% 
% Permission is hereby granted, free of charge, to any person obtaining a copy
% of this software and associated documentation files (the "Software"), to deal
% in the Software without restriction, including without limitation the rights
% to use, copy, modify, merge, publish, distribute, sublicense, and/or sell
% copies of the Software, and to permit persons to whom the Software is
% furnished to do so, subject to the following conditions:
% 
% The above copyright notice and this permission notice shall be included in all
% copies or substantial portions of the Software.
% 
% THE SOFTWARE IS PROVIDED "AS IS", WITHOUT WARRANTY OF ANY KIND, EXPRESS OR
% IMPLIED, INCLUDING BUT NOT LIMITED TO THE WARRANTIES OF MERCHANTABILITY,
% FITNESS FOR A PARTICULAR PURPOSE AND NONINFRINGEMENT. IN NO EVENT SHALL THE
% AUTHORS OR COPYRIGHT HOLDERS BE LIABLE FOR ANY CLAIM, DAMAGES OR OTHER
% LIABILITY, WHETHER IN AN ACTION OF CONTRACT, TORT OR OTHERWISE, ARISING FROM,
% OUT OF OR IN CONNECTION WITH THE SOFTWARE OR THE USE OR OTHER DEALINGS IN THE
% SOFTWARE.

\IfStrEq{\dbEntryID}{}{
	\section{Zadania dodatkowe}
	\insertZadanie{\currfilepath}{wypisz_mape}{}
	\insertZadanie{\currfilepath}{parsuj_klucz_wartosc_cpp}{}
	\insertZadanie{\currfilepath}{podnapis_utf}{}
	\insertZadanie{\currfilepath}{funkcja_niepodzielne_cpp}{}
	\insertZadanie{\currfilepath}{funkcja_wypisz_tablice}{}
	\insertZadanie{\currfilepath}{licz_powtorzenia_cpp}{}
}


\dbEntryBegin{wypisz_mape}\if1\dbEntryCheckResults
Napisz funkcję \Verb#wypiszMape# (szablon funkcji) która wypisuje mapę dowolnych typów. Na przykład dla wywołania:
\begin{CodeFrame*}[cpp]{}
std::map<std::string, float> a = { {"xy", 1.3}, {"qw", 16.3} };
std::map<int, std::string> b = { {1, "a"}, {2, "b"} };
wypiszMape(a);
wypiszMape(b);
\end{CodeFrame*}
\vspace{-8pt}Program powinien wypisać:
\vspace{-8pt}\begin{Verbatim}
qw → 16.3
xy → 1.3
1 → a
2 → b
\end{Verbatim}
\fi

\dbEntryBegin{parsuj_klucz_wartosc_cpp}\if1\dbEntryCheckResults
Napisz funkcję która konwertuje listę napisów postaci \Verb#klucz=wartosc# na mapę.
Funkcja musi dodawać kolejne napisy do mapy w taki sposób że część przed znakiem równości stanowi klucz, a część po znaku równości stanowi wartość.
Funkcja powinna modyfikować mapę otrzymaną (przez wskaźnik lub referencję) jako swój argument. Na przykład dla wywołania:
\begin{CodeFrame*}[cpp]{}
std::list<std::string> l = {"aa=13", "b=Ala=kot", "f=xyz"};
std::map<std::string,std::string> m;
parsuj(l, m);
for (auto& i : m)  std::cout << i.first << " → " << i.second << "\n";
\end{CodeFrame*}
\vspace{-8pt}Program powinien wypisać:
\vspace{-8pt}\begin{Verbatim}
aa → 13
b → Ala=kot
f → xyz
\end{Verbatim}
\fi

\dbEntryBegin{podnapis_utf}\if1\dbEntryCheckResults
Zmodyfikuj rozwiązanie zadanie \ref{podnapis} tak aby poprawnie obsługiwało znaki kodowane jako UTF8.
\\ \textit{
	Wskazówka 1: Zobacz opis kodowania UTF-8 na \url{https://en.wikipedia.org/wiki/UTF-8},
	zauważ że w bajtach stanowiących kontynuację znaku pierwsze dwa bity mają wartość 10,
	natomiast pierwszy bajt znaku nigdy nie ma takiej wartości najstarszych bitów.}
\\ \textit{
	Wskazówka 2: Zauważ, że aby odnaleźć pierwszy znak do wypisania, musisz przejść po napisie od samego początku.}
\fi

% PwES domowe(?):

\dbEntryBegin{funkcja_niepodzielne_cpp}\if1\dbEntryCheckResults
Napisz funkcję która wypisze liczby z zakresu od 0 do 20 nie podzielne przez wartość określoną w jej argumencie.
\fi

\dbEntryBegin{funkcja_wypisz_tablice}\if1\dbEntryCheckResults
Napisz funkcję, która przyjmuje dwa argumenty: tablicę C (czyli wskaźnik na zerowy element) liczb naturalnych i jej długość.
Funkcja ma wypisać każdy element tablicy oraz zwiększyć o jeden wartość tego elementu w oryginalnej tablicy.
Napisz \ul[black]{dwa warianty} tej funkcji:
\begin{enumerate}[label=\alph*)]
\item operujący arytmetyką wskaźnikową
\item operujący tablicami (czyli używający operatora dostępu do elementu tablicy: \Verb#[]#)
\end{enumerate}
\fi

\dbEntryBegin{licz_powtorzenia_cpp}\if1\dbEntryCheckResults
Napisz funkcję \Verb{zlicz} która dla podanej listy napisów policzy powtórzenia jej elementów. Na przykład dla wywołania:
\cpp#std::list<std::string> l = {"AX", "B", "AX"}; zlicz(l);# program powinien wypisać:
\vspace{-8pt}\begin{Verbatim}
AX wystepuje 2 razy
B wystepuje 1 razy
\end{Verbatim}
\vspace{-8pt}\textit{Wskazówka: Użyj mapy, w której element będzie stanowił klucz, a krotność jego wystąpień wartość.}
\fi
