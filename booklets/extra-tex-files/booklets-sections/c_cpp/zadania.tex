% Copyright (c) 2017-2020 Matematyka dla Ciekawych Świata (http://ciekawi.icm.edu.pl/)
% Copyright (c) 2017-2020 Robert Ryszard Paciorek <rrp@opcode.eu.org>
% 
% MIT License
% 
% Permission is hereby granted, free of charge, to any person obtaining a copy
% of this software and associated documentation files (the "Software"), to deal
% in the Software without restriction, including without limitation the rights
% to use, copy, modify, merge, publish, distribute, sublicense, and/or sell
% copies of the Software, and to permit persons to whom the Software is
% furnished to do so, subject to the following conditions:
% 
% The above copyright notice and this permission notice shall be included in all
% copies or substantial portions of the Software.
% 
% THE SOFTWARE IS PROVIDED "AS IS", WITHOUT WARRANTY OF ANY KIND, EXPRESS OR
% IMPLIED, INCLUDING BUT NOT LIMITED TO THE WARRANTIES OF MERCHANTABILITY,
% FITNESS FOR A PARTICULAR PURPOSE AND NONINFRINGEMENT. IN NO EVENT SHALL THE
% AUTHORS OR COPYRIGHT HOLDERS BE LIABLE FOR ANY CLAIM, DAMAGES OR OTHER
% LIABILITY, WHETHER IN AN ACTION OF CONTRACT, TORT OR OTHERWISE, ARISING FROM,
% OUT OF OR IN CONNECTION WITH THE SOFTWARE OR THE USE OR OTHER DEALINGS IN THE
% SOFTWARE.

\IfStrEq{\dbEntryID}{}{
	\insertZadanie{\currfilepath}{helloworld1}{}
	\insertZadanie{\currfilepath}{helloworld2}{}
	\insertZadanie{\currfilepath}{podnapis}{}
	\insertZadanie{\currfilepath}{podnapis2}{}
	\insertZadanie{\currfilepath}{podnapis_wskazniki}{}
	\insertZadanie{\currfilepath}{wypisz_mape}{}
	\insertZadanie{\currfilepath}{parsuj_klucz_wartosc_cpp}{}
	\insertZadanie{\currfilepath}{podnapis_utf}{}
}

\dbEntryBegin{helloworld1}\if1\dbEntryCheckResults
Napisz program wypisujący "Hello World".
\fi

\dbEntryBegin{helloworld1-rozwiazanie}\if1\dbEntryCheckResults
\begin{minted}[frame=single]{c}
#include <stdio.h>

int main() {
	printf("Hello World\n");
}
\end{minted}

\noindent Zwróć uwagę na:
\begin{itemize}
\item włączenie odpowiedniego pliku nagłówkowego
\item umieszczenie kodu wewnątrz funkcji main
\end{itemize}
\fi


\dbEntryBegin{helloworld2}\if1\dbEntryCheckResults
Zmodyfikuj program z zadania \ref{helloworld1} tak aby z użyciem pętli wypisywał ten napis 10 razy.
\fi

\dbEntryBegin{helloworld2-rozwiazanie}\if1\dbEntryCheckResults
\begin{minted}[frame=single]{c}
#include <stdio.h>

int main() {
	for (int i=0; i<10; ++i)
		printf("Hello World\n");
}
\end{minted}

\noindent Zwróć uwagę na użycie pętli :-)
\fi


\dbEntryBegin{podnapis}\if1\dbEntryCheckResults
Napisz funkcję \Verb#wypisz#, która z użyciem pętli będzie wypisywała poszczególne znaki podanego napisu od wskazanej do wskazanej pozycji. Przyjmij że napis składa się tylko z znaków ASCII.

Wywołanie \Verb#wypisz("! Hello World !", 3, 6);# powinno spowodować wypisanie \Verb#ello#.
\fi

\dbEntryBegin{podnapis-rozwiazanie}\if1\dbEntryCheckResults
\begin{minted}[frame=single]{c}
#include <stdio.h>

void wypisz(const char* n, int b, int e) {
	for (int i=b; i<=e; ++i)
		printf("%c", n[i]);
	printf("\n");
}

int main(){
	wypisz("! Hello World !", 3, 6);
}
\end{minted}

\noindent Zwróć uwagę na:
\begin{itemize}
\item definicję i użycie funkcji przyjmującej argumenty
\item iterownaie po elementach tablicy (napisu) i odwołanie się do i-tego elementu operatorem nawiasu kwadratowego
\end{itemize}
\fi


\dbEntryBegin{podnapis2}\if1\dbEntryCheckResults
Zmodyfikuj program z zadania \ref{podnapis2} tak aby funkcja, w przypadku gdy argument określający ostatni znak podnapisu do wypisania jest większy niż długość napisu, kończyła swoje działanie w momencie napotkania końca napisu.

Wywołanie \Verb#wypisz("! Hello World !", 10, 116);# powinno spowodować wypisanie \Verb#rld !# i niczego więcej.
\fi

\dbEntryBegin{podnapis2-rozwiazanie}\if1\dbEntryCheckResults
\begin{minted}[frame=single]{c}
void wypisz(const char* n, int b, int e) {
	for (int i=b; i<=e && n[i]; ++i)
		printf("%c", n[i]);
	printf("\n");
}
\end{minted}

\noindent Zwróć uwagę na dodatkowy warunek w pętli sprawdzający czy aktualny znak nie ma wartości zero (czyli czy nie oznacza końca napisu) łączony operatorem logicznego and z warunkiem dotyczącym numeru znaku do wypisania.
\fi


\dbEntryBegin{podnapis_wskazniki}\if1\dbEntryCheckResults
Zmodyfikuj program z zadania \ref{podnapis2} tak aby funkcja korzystała z arytmetyki wskaźnikowej, zamiast iteracji po numerze znaku w napisie (pętla ma korzystać z wskaźników i ich porównywania, zamiast operowania na numerze/indeksie znaku!).
\fi

\dbEntryBegin{podnapis_wskazniki-rozwiazanie}\if1\dbEntryCheckResults
\begin{minted}[frame=single]{c}
void wypisz(char* n, int b, int e) {
	char* end = n+e;
	for (n=n+b; n<=end && *n; ++n)
		printf("%c", *n);
	printf("\n");
}
\end{minted}

\noindent Zwróć uwagę na:
\begin{itemize}
\item zdefiniowanie zmiennej end, będącej wskaźnikiem na ostatni element podnapisu przed pętla
\item operację inkrementacji wskaźnika (\Verb$++n$) przy każdym obiegu pętli
\item odwołanie do elementu wskazywanego poprzez \Verb$*n$
\end{itemize}
\fi


\dbEntryBegin{wypisz_mape}\if1\dbEntryCheckResults
Napisz funkcję \Verb#wypiszMape# (szablon funkcji) która wypisuje mapę dowolnych typów. Na przykład dla wywołania:
\begin{CodeFrame*}[cpp]{}
std::map<std::string, float> a = { {"xy", 1.3}, {"qw", 16.3} };
std::map<int, std::string> b = { {1, "a"}, {2, "b"} };
wypiszMape(a);
wypiszMape(b);
\end{CodeFrame*}
\vspace{-8pt}Program powinien wypisać:
\vspace{-8pt}\begin{Verbatim}
qw → 16.3
xy → 1.3
1 → a
2 → b
\end{Verbatim}
\fi

\dbEntryBegin{wypisz_mape-rozwiazanie}\if1\dbEntryCheckResults
\begin{minted}[frame=single]{c}
#include <iostream>
#include <map>

template <typename A, typename B> void wypiszMape(const std::map<A, B>& m) {
	for (const auto& i : m) {
		std::cout << i.first << " → " << i.second << "\n";
	}
}

int main() {
	std::map<std::string, float> a = { {"xy", 1.3}, {"qw", 16.3} };
	std::map<int, std::string> b = { {1, "a"}, {2, "b"} };
	wypiszMape(a);
	wypiszMape(b);
}
\end{minted}

\noindent Zwróć uwagę na:
\begin{itemize}
\item deklarację funkcji w postaci szablonu z dwoma parametrami - odpowiadającymi parametrom w szablonie jakim jest std::map
\item użycie pętli for typu foreach iterującej po elementach otrzymanej mapy
\item użycie stałych referencji zarówno w argumencie funkcji, jak i w iteracji po mapie - zapobiega to niepotrzebnemu kopiownaiu mapy i jej elementów
\end{itemize}
\fi


\dbEntryBegin{parsuj_klucz_wartosc_cpp}\if1\dbEntryCheckResults
Napisz funkcję która konwertuje listę napisów postaci \Verb#klucz=wartosc# na mapę.
Funkcja musi dodawać kolejne napisy do mapy w taki sposób że część przed znakiem równości stanowi klucz, a część po znaku równości stanowi wartość.
Funkcja powinna modyfikować mapę otrzymaną (przez wskaźnik lub referencję) jako swój argument. Na przykład dla wywołania:
\begin{CodeFrame*}[cpp]{}
std::list<std::string> l = {"aa=13", "b=Ala=kot", "f=xyz"};
std::map<std::string,std::string> m;
parsuj(l, m);
for (auto& i : m)  std::cout << i.first << " → " << i.second << "\n";
\end{CodeFrame*}
\vspace{-8pt}Program powinien wypisać:
\vspace{-8pt}\begin{Verbatim}
aa → 13
b → Ala=kot
f → xyz
\end{Verbatim}
\fi

\dbEntryBegin{parsuj_klucz_wartosc_cpp-rozwiazanie}\if1\dbEntryCheckResults
\begin{minted}[frame=single]{c}
void parsuj(const std::list<std::string>& l, std::map<std::string,std::string>& m) {
	for (auto& i : l) {
		auto x = i.find("=");
		m[ i.substr(0,x) ] = i.substr(x+1);
	}
}
\end{minted}
\fi


\dbEntryBegin{podnapis_utf}\if1\dbEntryCheckResults
Zmodyfikuj rozwiązanie zadanie \ref{podnapis2} tak aby poprawnie obsługiwało znaki kodowane jako UTF8.
\\ \textit{
	Wskazówka 1: Zobacz opis kodowania UTF-8 na \url{https://en.wikipedia.org/wiki/UTF-8},
	zauważ że w bajtach stanowiących kontynuację znaku pierwsze dwa bity mają wartość 10,
	natomiast pierwszy bajt znaku nigdy nie ma takiej wartości najstarszych bitów.}
\\ \textit{
	Wskazówka 2: Zauważ, że aby odnaleźć pierwszy znak do wypisania, musisz przejść po napisie od samego początku.}
\fi

\dbEntryBegin{podnapis_utf-rozwiazanie}\if1\dbEntryCheckResults
\begin{minted}[frame=single]{c}
#include <stdio.h>

void wypisz(const char* n, int b, int e) {
	int bajt = 0;
	for (int znak = 0; znak <= e && n[bajt]; ++znak) {
		do {
			if (znak >= b)
				printf("%c", n[bajt]);
			++bajt;
		} while ((n[bajt] & 0xc0) == 0x80);
	}
	printf("\n");
}

int main(){
	wypisz("!←Ąełło World !", 3, 6);
}
\end{minted}
\fi
