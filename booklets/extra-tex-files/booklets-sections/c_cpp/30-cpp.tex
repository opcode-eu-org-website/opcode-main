% Copyright (c) 2017-2020 Matematyka dla Ciekawych Świata (http://ciekawi.icm.edu.pl/)
% Copyright (c) 2017-2020 Robert Ryszard Paciorek <rrp@opcode.eu.org>
% 
% MIT License
% 
% Permission is hereby granted, free of charge, to any person obtaining a copy
% of this software and associated documentation files (the "Software"), to deal
% in the Software without restriction, including without limitation the rights
% to use, copy, modify, merge, publish, distribute, sublicense, and/or sell
% copies of the Software, and to permit persons to whom the Software is
% furnished to do so, subject to the following conditions:
% 
% The above copyright notice and this permission notice shall be included in all
% copies or substantial portions of the Software.
% 
% THE SOFTWARE IS PROVIDED "AS IS", WITHOUT WARRANTY OF ANY KIND, EXPRESS OR
% IMPLIED, INCLUDING BUT NOT LIMITED TO THE WARRANTIES OF MERCHANTABILITY,
% FITNESS FOR A PARTICULAR PURPOSE AND NONINFRINGEMENT. IN NO EVENT SHALL THE
% AUTHORS OR COPYRIGHT HOLDERS BE LIABLE FOR ANY CLAIM, DAMAGES OR OTHER
% LIABILITY, WHETHER IN AN ACTION OF CONTRACT, TORT OR OTHERWISE, ARISING FROM,
% OUT OF OR IN CONNECTION WITH THE SOFTWARE OR THE USE OR OTHER DEALINGS IN THE
% SOFTWARE.

\section{C++}

Jak być może zauważyliście programowanie w C jest dość mocno niżej poziomowe od programowania np. w Pythonie.
A co za tym idzie często trudniejsze i wymagające więcej wysiłku i czasu.
Natomiast wykonanie takiego skompilowanego do kodu maszynowego programu jest znacznie szybsze.
Jeżeli w jakimś zastosowaniu potrzebujemy kompilowalnego, szybkiego języka takiego jak C,
	ale z wyżejpoziomowymi mechanizmami, z jakimimożna spotkać się na przykład w Pythonie warto zwrócić uwagę na C++.
Język ten oferuje między innymi:
\begin{itemize}
	\item dynamiczne typowanie podobne do znanego z pythona z uzycie słowa kluczowego \Verb$auto$,
	\item programowanie generyczne (niezależne od typów) z użyciem szablonów,
	\item pętlę for iterującą po elementach kolekcji, wsparcie dla programowania obiektowego, funkcji typu lambda.
	\item listy i słowniki i to znacznie bardziej rozbudowane niż te z którymi mieliśmy doczynienia w Pythonie, 
		możemy mieć np. słownik z wieloma jednakowymi kluczami, zbiór samych unikalnych kluczy, itd.,
\end{itemize}

Język C powstał z rozszerzenia języka B w 1972 roku, natomiast C++ jest zapoczątkowanym w 1979 roku rozszerzeniem języka C.
Pierwszy oficjalny standard C pochodzi z 1989 roku (\textit{ANSI X3.159-1989} / \textit{ISO/IEC 9899:1990})\footnote{Przy opracowywaniu tego standardu od języka C oddzieliła się część opisująca zagadnienia specyficzne dla środowisk unixowych w postaci standardu POSIX (\textit{Portable Operating System Interface}) opisywanego w serii standardów IEEE Std 1003, publikowanych od 1988 roku.}, a C++ z 1998 roku (\textit{ISO/IEC 14882-1998}).
Języki te od czasu swojego powstania, a następnie ustandaryzowania rozwijają się niezależnie.
Obecnie, w 2020 roku są to dwa niezależne języki (C++17 z 2017 roku i C18 z 2018 roku), jednak cały czas bardzo bliskie sobie – jednym z założeń C++ jest zachowanie maksymalnej kompatybilności z C\footnote{
	Większość różnic jest do tego stopnia pomijalna (zwłaszcza biorąc pod uwagę pewną liberalność kompilatorów), że typowy program C, można skompilować jako kod C++ i będzie on poprawnie działającym programem.
	Często nawet jest w 100\% formalnie poprawnym kodem C++ (ale niekoniecznie w stylu tego języka).
}.
C++ to nie tylko „C z klasami”, język ten oferuje wiele usprawnień w stosunku co do C (często adaptowanych do C w kolejnych wersjach).

Część poniższego kodu zakłada że używany jest C++ w wersji co najmniej 11, zatem do jego kompilacji powinno być użyte np. polecenie \Verb#g++ -std=c++11 plik.cpp# (lub \Verb#clang++ -std=c++11 plik.cpp#), które utworzy plik wykonywalny \Verb#a.out# (można go uruchomić poprzez \Verb#./a.out#).

\subsection{programowanie obiektowe}

\begin{CodeFrame*}[cpp]{}
#include <iostream>
#include <stdint.h>

struct NazwaStruktury {
    // pola składowe
    int a;
    std::string d;
    
    // zmienna statyczna, wspólna dla wszystkich obiektów tej klasy
    static int x;
    
    // stała
    static const int y = 7;
    
    // pola binarne (jedno i trzy bitowe)
    uint8_t mA :1;
    uint8_t mB :3;
    
    // metody składowe
    void wypisz() {
        std::cout << " a=" << a << " d=" << d << "\n";
    }
    
    // deklaracja metody, definicja musi być podana gdzieś indziej
    int getSum(int b) ;
    
    /// metody statyczna
    static void info() {
        std::cout << "INFO\n";
    }
    
    // konstruktor i destruktor
    NazwaStruktury(int aa = 0) {
        std::cout << "konstruktor\n";
        a = aa;
        d = "abc ...";
    }
    ~NazwaStruktury() {
        // potrzebny gdy klasa tworzy jakieś
        // obiekty które nalezy usuwać, itp
        std::cout << "destruktor\n";
    }
};

// definicja zmiennej statycznej z nadaniem jej wartości
// jest to niezbędne aby była ona widoczna ...
int NazwaStruktury::x = 13;

// wcześniej zdeklarowane metody możemy definiować także poza deklaracją klasy
int NazwaStruktury::getSum(int b) {
    return a + b;
}

int main() {
    // korzystanie ze struktur
    NazwaStruktury s;
    s.a = 45;
    s.wypisz();

    // korzystanie z metod statycznych
    NazwaStruktury::info();
    // a także poprzez obiekt danej klasy
    s.info();
}
\end{CodeFrame*}

\subsection{tablice zmiennej długości}

Język C od wersji C99 pozwala na korzystanie z tablic zmiennej długości (\textit{VLA}), czyli tablic których rozmiar nie jest stałą czasu kompilowania a zmienną - np.:

\begin{CodeFrame*}[c]{}
void xxx(int n) {
    float vals[n];
    v[0] = 21;
    /* ... */
}
\end{CodeFrame*}

C++ tablic zmiennej długości w stylu C99 C++ oficjalnie nie obsługuje, przy czym niektóre z kompilatorów dopuszczają użycie VLA w C++.
C++ posiada za to typ std:vector pozwalający na definiowanie tablic, których rozmiar można łatwo (z punktu widzenia programisty, niekoniecznie maszyny wykonującej ten kod) zmieniać nawet po utworzeniu tablicy:
\begin{CodeFrame*}[cpp]{}
void xxx(int n) {
    std::vector<float> vals(n);
    v[0] = 21;
    /* ... */
}
\end{CodeFrame*}


\subsection{listy}

Biblioteka standardowa C++ (a dokładniej jej fragment określany mianem STL) dostarcza także obsługę list:

\begin{CodeFrame*}[cpp]{}
#include <iostream>
#include <list>

int main() {
    std::list<int> l;
    
    // dodanie elementu na końcu
    l.push_back(17);
    l.push_back(13);
    l.push_back(3);
    l.push_back(27);
    l.push_back(21);
    // dodanie elementu na początku
    l.push_front(8);
    
    // wypisanie liczby elementów
    std::cout << "size=" << l.size()<< "\n";
    
    // wypisanie pierwszego i ostatniego elementu
    std::cout << "first=" << l.front() << " last=" << l.back() << "\n";
    
    // usuniecie ostatniego elementu
    l.pop_back();
    
    // posortowanie listy
    l.sort();
    
    // odwrócenie kolejności elementów
    l.reverse();
    
    // usuniecie pierwszego elementu
    l.pop_front();
    
    for (std::list<int>::iterator i = l.begin(); i != l.end(); ++i) {
        // wypisanie wszystkich elementów
        std::cout << *i << "\n";
        // możliwe jest także:
        //  - usuwanie elementu wskazanego przez iterator
        //  - wstawianie elementu przed wskazanym przez iterator
    }
}
\end{CodeFrame*}

W przypadku C++ listy implementowane są jako listy a nie tablice wskaźników, więc operacje wstawiania na początku i w środku są szybkie, ale operacja uzyskania n-tego elementu jest powolna.

\subsection{mapy}

Biblioteka standardowa C++ oferuje także kontener umożliwiający przechowywanie danych w postaci par klucz-wartość, gdzie wartość identyfikowana jest unikalnym kluczem (podobnie jak w pythonowych słownikach):

\begin{CodeFrame*}[cpp]{}
#include <iostream>
#include <map>

int main() {
    std::map<std::string, int> m;
    
    m["a"] = 6;
    m["cd"] = 9;
    std::cout << m["a"] << " " << m["ab"] << "\n";
    
    // wyszukanie elementu po kluczu
    std::map<std::string, int>::iterator iter = m.find("cd");
    // sprawdzenie czy istnieje
    if (iter != m.end()) {
        // wypisanie pary - klucz wartość
        std::cout << iter->first << " => " << iter->second << "\n";
        // usunięcie elementu
        m.erase(iter);
    }
    
    m["a"] = 45;
    
    // wypisanie całej mapy
    for (iter = m.begin(); iter != m.end(); ++iter)
        std::cout << iter->first << " => " << iter->second << "\n";
    // jak widać mapa jest wewnętrznie posortowana
}
\end{CodeFrame*}

Mapa \cpp{std::map} nie zachowuje kolejności wkładania elementów, natomiast jest zawsze posortowana.
C++ oferuje też inne rodzaje map (np. nie posortowaną \cpp{std::unordered_map}, czy też nie wymagającą unikalności klucza \cpp{std::multimap}).

\insertZadanie{booklets-sections/c_cpp/zadania.tex}{parsuj_klucz_wartosc_cpp}{}

\subsection{Więcej c-plus-plusa ...}

\subsubsection{referencje}

Referencje są zasadniczo wskaźnikami, których używamy jak zwykłych zmiennych (bez stosowania operatora \cpp{*} w celu operowania na wartości wskaźnika).
W odróżnieniu od wskaźników nie możemy bezpośrednio operować na adresie referencji (np. spowodować aby wskazywała na inną zmienną).
Kontynuując przykład z modyfikacją zmiennej przekazanej jako argument funkcji, z użyciem referencji kod ten może wyglądać następująco:

\begin{CodeFrame*}[c]{}
void ff(int& a) { // zwróć uwagę na & oznaczający że będzie to referencja
    a = 15;
}
int main() {
    x = 10;
    ff(x);
    printf("%d\n", x); // wypisze 15
}
\end{CodeFrame*}

\subsubsection{iteratory}

W powyższych przykładach użycia list i map w C++ warto zwrócić uwagę na użycie iteratorów pozwalających na pobieranie kolejnych wartości z tych kontenerów:

\begin{CodeFrame*}[cpp]{}
void wypiszListe(std::list<int> l) {
    for (std::list<int>::iterator i = l.begin(); i != l.end(); ++i) {
        std::cout << *i << "\n";
    }
}
\end{CodeFrame*}

Iterator zwracają niektóre z metod obiektów reprezentujących te kontenery, np. \cpp{.begin()} zwraca iterator na pierwszy element.
Zwiększanie iteratora odbywa się z użyciem operatorów \cpp{++}.
Wyjście poza zakres (zwiększenie iteratora wskazującego na ostatni element kolekcji) nie powoduje rzucenia wyjątku, za to iterator przyjmuje specjalną wartość oznaczającą koniec.
Iterator o tej wartości zwracany jest przez metodę \cpp{.end()} (lub \cpp{.rend()} przy iterowaniu w przeciwną stronę).

\subsubsection{typ auto}

Współczesny C++ oferuje także specjalny typ \cpp{auto} zwalniający programistę z konieczności jawnego definiowania typu zmiennej do której przypisywana jest od razu jakaś wartość z określonym typem. Możemy napisać np. \cpp{auto x = 5;}, ale nie możemy napisać \cpp{auto x; x = 5;}. Typ ten jest użyteczny np. do obsługi iteratorów, pozwalając powyższą pętle zapisać bez \cpp{std::list<int>::iterator} jako:

\begin{CodeFrame*}[cpp]{}
void wypiszListe(std::list<int> l) {
    for (auto i = l.begin(); i != l.end(); ++i) {
        std::cout << *i << "\n";
    }
}
\end{CodeFrame*}

\subsubsection{pętla for(each)}

C++ udostępnia także inną składnię pętli \cpp{for} pozwalającą na iterowanie po wszystkich elementach kolekcji takich jak listy, mapy, itp. Będącą odpowiednikiem pętli \Verb#foreach# znanej z niektórych języków programowania, czy też pythonowskiej pętli \python{for}:

\begin{CodeFrame*}[cpp]{}
void wypiszListe(std::list<int> l) {
    for (auto i : l) {
        std::cout << i << "\n";
    }
}
\end{CodeFrame*}

Zamiast \cpp{auto i} możemy napisać \cpp{auto& i} aby otrzymać dostęp przez referencję (wtedy wykonanie przypisania wartości do i, np \cpp{i = 0}, spowoduje modyfikację elementu listy).

Warto zauważyć także, że w odróżnieniu od wcześniejszej pętli nie operujemy tutaj na iteratorach, a na wartościach / referencjach do wartości z kontenera.

\subsubsection{szablony}

C++ pozwala też definiować szablony funkcji oraz klas, dzięki którym kompilator będzie mógł wytworzyć funkcje/klasy dla potrzebnych typów w oparciu o ten szablon (zdefiniowany dla ogólnego typu).
Na przykład powyższa funkcja wypisująca listy zdefiniowana jest tylko dla list zawierających liczby całkowite.
Jednak takie funkcje dla dowolnych typów obsługiwanych przez cout-owy operator \cpp{<<} (np. liczb zmiennoprzecinkowych, napisów, ...) będą wyglądały tak samo.
Dzięki mechanizmowi szablonów możemy napisać:

\begin{CodeFrame*}[cpp]{}
template <typename T> void wypiszListe(std::list<T>& l) {
    for (auto i : l) {
        std::cout << i << "\n";
    }
}
\end{CodeFrame*}

I następnie używać jej dla różnych typów list:

\begin{CodeFrame*}[cpp]{}
int main() {
    std::list<int> x={1, 3, 7, 2, 3};
    wypiszListe(x);
    
    std::list<float> z={2.7, 5.0, 3.1, 3.9};
    wypiszListe(z);
}
\end{CodeFrame*}

\insertZadanie{booklets-sections/c_cpp/zadania.tex}{wypisz_mape}{}

\subsubsection{lambda}

\begin{CodeFrame*}[cpp]{}
int main() {
    int x = 1, y = 1;

    // ta lamba będzie używać:
    //  wartości x z chwili wywołania (i jej zmiana bedzie widoczna na zewnątrz)
    //  wartości y z chwili utworzenia
    auto moja_lambda = [&x, y](int z) { x += z * y; return 11; };

    moja_lambda(2);
    std::cout << x << std::endl; // 3 bo x = 1 + 2 * 1

    x = 0;
    y = 0;

    int z = moja_lambda(2);
    std::cout << x << " " << z << std::endl; // 2 bo x = 0 + 2 * 1
}
\end{CodeFrame*}

C++ pozwala także na definiowanie i używanie lamb.
Definicja taka skłąda się z listy przechwytywanych zmiennych, listy argumentów i ciała funkcji. Lista przechwytywania może określiać przechwytywanie przez wartość lub przez referencję. W pierwszym przypadku wartość zmiennej z miejsca utworzenia funkcji zostanie w niej "zamrożona", czyli jej dalsze zmiany nie będą widoczne w wyołaniach lambdy.' W drugim przypadku lmba będzie widzieć zawsze aktualną wartość, a zmiany tej zmiennej wewnątrz lambdy będą widoczne takze na zewnątrz. Lista argumentów i ciało funkcji działa jak w zwykłych funkcjach. Lamba może zwrać lub może nie zwracać wartość z użyciem return.

\subsubsection{jeszcze więcej ... \zaawansowane{20}}

Jest to tylko wzmianka o różnych ciekawych aspektach współczesnego C++. Język ten pozwala na dużo więcej (np. przeciążanie operatorów dla naszych typów danych - możemy np. sumować obiekty naszych klas przy pomocy \cpp{+}), również biblioteka standardowa oferuje więcej interesujących typów (np. zbiory \cpp{std:set}), a używanie szablonów nie sprowadza się jedynie do przedstawionego prostego przypadku szablonu funkcji.

Tematyką C i C++ możnaby wypełnić cały tej długości kurs, a nie jedynie jeden blok zajęć. W naszym kursie ograniczyliśmy się jedynie do omówienia podstawowych zagadnień i krótkiego przeglądu bardziej zaawansowanych możliwości. Zachęcam do nawet pobieżnego zapoznania się np. z „C++ reference” dostępnym pod adresem: \url{https://en.cppreference.com/w/cpp} i korzystania z tej dokumentacji przy tworzeniu kodu C++.
