% Copyright (c) 2017-2020 Matematyka dla Ciekawych Świata (http://ciekawi.icm.edu.pl/)
% Copyright (c) 2003-2020 Robert Ryszard Paciorek <rrp@opcode.eu.org>
% 
% MIT License
% 
% Permission is hereby granted, free of charge, to any person obtaining a copy
% of this software and associated documentation files (the "Software"), to deal
% in the Software without restriction, including without limitation the rights
% to use, copy, modify, merge, publish, distribute, sublicense, and/or sell
% copies of the Software, and to permit persons to whom the Software is
% furnished to do so, subject to the following conditions:
% 
% The above copyright notice and this permission notice shall be included in all
% copies or substantial portions of the Software.
% 
% THE SOFTWARE IS PROVIDED "AS IS", WITHOUT WARRANTY OF ANY KIND, EXPRESS OR
% IMPLIED, INCLUDING BUT NOT LIMITED TO THE WARRANTIES OF MERCHANTABILITY,
% FITNESS FOR A PARTICULAR PURPOSE AND NONINFRINGEMENT. IN NO EVENT SHALL THE
% AUTHORS OR COPYRIGHT HOLDERS BE LIABLE FOR ANY CLAIM, DAMAGES OR OTHER
% LIABILITY, WHETHER IN AN ACTION OF CONTRACT, TORT OR OTHERWISE, ARISING FROM,
% OUT OF OR IN CONNECTION WITH THE SOFTWARE OR THE USE OR OTHER DEALINGS IN THE
% SOFTWARE.

\section{Zmienne i podstawowe operacje}

Język C wymaga określania typu zmiennej w momencie jej definiowania.

\begin{CodeFrame*}[c]{}
// liczba całkowita ze znakiem
int     liczbaA = -34;
// liczba rzeczywista (pojedynczej precyzji)
float   liczbaB = 673.1;
// 8 bitowa liczba całkowita bez znaku, wymaga pliku nagłówkowego inttypes.h
uint8_t liczbaC = 0xf3;

// zmienna napisowa "C NULL-end string"
char* napisA = "q we";
\end{CodeFrame*}

Dodawanie, mnożenie, odejmowanie zapisuje się i działają one tak jak w normalnej matematyce, dzielenie zapisuje się przy pomocy ukośnika i zależnie od typów na których operuje jest ono dzieleniem całkowitym lub zmiennoprzecinkowym.

\begin{CodeFrame*}[c]{}
double a = 12.7, b = 3, c, d, e;
int x = 5, y = 6, z;

// dodawanie, mnożenie, odejmowanie zapisuje się
// i działają one tak jak w normalnej matematyce:
e = (a + b) * 4 - y;

// dzielenie zależy od typów argumentów
d = a / b; // będzie dzieleniem zmiennoprzecinkowym bo a i b są typu float
c = x / y; // będzie dzieleniem całkowitym bo z i y są zmiennymi typu int
b = (int)a / (int)b; // będzie dzieleniem całkowitym
a = (double)x / (double)y; // będzie dzieleniem zmiennoprzecinkowym

// reszta z dzielenia (tylko dla argumentów całkowitych)
z = x % y;

// wypisanie wyników
printf("%d %f %f %f %f %f\n", z, e, d, c, b, a);

// operacje logiczne:
// ((a większe równe od 0) AND (b mniejsze od 2)) OR (z równe 5)
z = (a>=0 && b<2) || z == 5;
// negacja logiczna z
x = !z;

printf("%d %d\n", z, x);

// operacje binarne:
// bitowy OR 0x0f z 0x11 i przesunięcie wyniku o 1 w lewo
x = (0x0f | 0x11) << 1;
// bitowy XOR 0x0f z 0x11
y = (0x0f ^ 0x11);
// negacja bitowa wyniku bitowego AND 0xfff i 0x0f0
z = ~(0xfff & 0x0f0);

printf("%x %x %x\n", x, y, z);
\end{CodeFrame*}

\begin{ProTip}{printf() i puts()}
Funkcja \cpp{printf()} wypisuje napis określony przez pierwszy argument, podstawiają pod elementy typu \Verb#%x# wartości kolejnych argumentów (np. zmiennych) odpowiednio je interpretując, np.:
\begin{itemize}
	\item \Verb#%x# - liczba dziesiętna,
	\item \Verb#%x# - liczba szesnastkowa,
	\item \Verb#%f# - liczba zmiennoprzecinkowa,
	\item \Verb#%s# - napis,
	\item \Verb#%c# - pojedynczy znak.
\end{itemize}
Funkcja ta nie dodaje do wypisywanego napisu znaku nowej linii, więc jeżeli chcemy przejść do nowej linii to musi on być umieszczony w nim w sposób jawny jako \Verb#\n# (niekiedy jako \Verb#\r\n# - protokoły sieciowe, Windows, ...).
Szczegółowy opis \cpp{printf()} oraz więcej napisów formatujących można znaleźć w \Verb#man 3 printf#.

\vspace{8pt}
Do wypisywania samego napisu (bez możliwości podstawienia zmiennych, itd.) posłużyć może funkcja \cpp{puts()}, która wypisuje podany napis dodając znak nowej linii.
\end{ProTip}


\section{Przepływ sterowania w programie - skoki, warunki, pętle, funkcje}

Licznik programu (program counter, instruction pointer lub instruction address register) jest rejestrem procesora który określa adres następnej (w niektórych architekturach aktualnej) instrukcji która ma zostać przetworzona procesor.

Skoki bezwarunkowe, instrukcje warunkowe, pętle, wywołania funkcji są realizowane poprzez modyfikację licznika programu. W przypadku wywołań funkcji dodatkowo wykonywane są operacje związane z obsługą stosu (zachowywaniem stanu rejestrów, umieszczaniem argumentów na stosie, ...). Instrukcja goto (realizująca skok bezwarunkowy) jest pełnoprawną instrukcją skoku, jedyną wadą jej stosowania jest to że przy niewłaściwym / zbyt częstym wykorzystywaniu (zamiast wywołań funkcji, warunków i pętli) kod programu staje się mniej czytelny.

W większości przypadków pętle realizowane są na poziomie kodu maszynowego jako zestaw instrukcji (np. inkrementacji zmiennej, sprawdzania warunku, skoku), jednak w niektórych rozwiązaniach pętle (np. typu "powtórz n razy") mogą być realizowane sprzętowo przy pomocy pojedynczej instrukcji.

\subsection{Punkt startu}

Jako że program komputerowy jest sekwencją wykonywanych instrukcji musi rozpoczynać się od określonego miejsca.
W przypadku kodu C/C++ punktem startu jest funkcja \cpp{main()}. Zakończenie tej funkcji oznacza zakończenie programu, a wartość przez nią zwracana odpowiedzialna jest za tzw. kod powrotu przekazany procesowi wywołującemu program. 

\subsection{Podstawowe konstrukcje}

Język C oferuje kilka pętli – \cpp{for}, \cpp{while}, \cpp{do - while} oraz instrukcję warunkową \cpp{if} i instrukcję wyboru \cpp{switch}.
Możliwe jest też korzystanie z operatora warunkowego:

\begin{CodeFrame*}[c]{}
warunek ? wartosc_gdy_prawda : wartosc_gdy_falsz.
\end{CodeFrame*}
Gdzie zarówno \Verb#wartosc_gdy_prawda#, jak i  \Verb#wartosc_gdy_falsz#, mogą być wartością jak też wyrażeniem obliczającym jakąś wartość (wyrażeniem matematycznym, wywołaniem funkcji, itd.).

\begin{CodeFrame*}[c]{}
#include <stdio.h> // włączenie pliku nagłówkowego

int main() {
    int i, j, k;
    
    // instrukcja waunkowa if - else
    if (i<j) {
        puts("i<j");
    } else if (j<k) {
        puts("i>=j AND j<k");
    } else {
        puts("i>=j AND j>=k");
    }
    
    // podstawowe operatory logiczne
    if (i<j || j<k)
        puts("i<j OR j<k");
    // innymi operatorami logicznymi są && (AND), ! (NOT)
    
    // pętla for
    for (i=2; i<=9; ++i) {
        if (i==3)
            continue; // pominięcie tego kroku pętli
        if (i==7)
            break; // wyjście z pętli
        printf(" a: %d\n", i);
    }
    
    // pętla while
    while (i>0) {
        printf(" b: %d\n", --i);
    }
    
    // pętla do - while
    do {
        printf(" c: %d\n", ++i);
    } while (i<2);
    
    // instrukcja wyboru switch
    switch(i) {
        case 1:
            puts("i==1");
            break;
        default:
            puts("i!=1");
            break;
    }
}
\end{CodeFrame*}

\subsection{Własne funkcje}

\begin{CodeFrame*}[c]{}
#include <stdio.h>

// funkcja bezargumentowa niezwracająca wartości
void f1() {
    puts("ABC");
}

// funkcja dwuargumentowa zwracająca wartość
int f2(int a, int b) {
    puts("F2");
    return a*2.5 + b;
}

int main() {
    f1();
    int a = f2(3, 6);
    // zwracaną wartość można wykorzystać (jak wyżej) lub zignorować
    printf("%d\n", a);
}
\end{CodeFrame*}

\section{Złożone typy danych}

\subsection{Struktury}
Struktura jest złożonym typem danych służącym do grupowania powiązanych ze sobą logicznie zmiennych. Zmienne wchodzące w skład struktury (pola) identyfikowane są nazwami i mogą być różnych typów. Struktura zajmuje ciągły obszar pamięci, w którym umieszczane są wartości kolejnych pól. 

\begin{CodeFrame*}[c]{}
#include <stdio.h>

struct Struktura {
    int a, b;
    double c;
};

int main() {
    struct Struktura s;
    s.a = 13;
    s.c = 17.3;
    printf("%f\n", s.a + s.c);
}
\end{CodeFrame*}

\subsection{Tablice}
Tablica jest strukturą danych w której elementy (takiego samego typu) są ułożone w porządku liniowym i są dostępne za pomocą indeksów (kluczy). Typowo tablica indeksowana jest liczbami całkowitymi nie ujemnymi oraz zajmuje ciągły obszar pamięci.

\begin{CodeFrame*}[c]{}
#include <stdio.h>

int main() {
    int t[4] = {1, 8, 3, 2};
    printf("%d -> \n", t[2]);
    t[2] = 55;
    printf("    %d \n", t[2]);
}
\end{CodeFrame*}

\subsection{Napisy}
Napisy w C są w istocie tablicami elementów typu \cpp{char}. Pojedynczy znak reprezentowany jest poprzez jeden element tablicy (dla znaków kodowanych jednobajtowo) lub grupę takich elementów (dla znaków kodowanych wielobajtowo, np. polskich znaczków w UTF8). Koniec napisu oznaczany jest przez element o wartości zero (\cpp{NULL}).

W C pojedynczy znak napisu (czyli np. \cpp{char x = napis[i]} albo \cpp{char x = 'A';}) nie jest napisem – jest liczbą (zauważ różnicę między pojedynczymi a podwójnymi cudzysłowami). Możemy go wypisać z użyciem \cpp{printf()} jako wartość numeryczną poprzez \Verb#%d# lub jako znak poprzez \Verb#%c#: \cpp{printf("%d <=> %c", 'A', 'A');}.

\section{Funkcja main}

Jak już zauważyliśmy funkcja \cpp{main()} zwraca wartość całkowitą. Jest to kod powrotu programu, który służy do informowania procesu wywołującego nasz program o tym czy zakończył się on sukcesem czy porażką. W przypadku sukcesu powinien zwrócić 0, a niezerowa wartość oznacza niepowodzenie (można użyć różnych wartości do sygnalizowania różnego rodzaju niepowodzeń - np. polecenie grep inaczej sygnalizuje nie znalezienie podanego wzorca, a inaczej brak pliku który miało przeszukać).

Funkcja \cpp{main()} może przyjmować także argumenty, dzięki którym program może odebrać parametry przekazane w linii poleceń.
Drugi argument zazwyczaj nazywany \textit{argv} jest tablicą napisów \cpp{char *},
	której elementy stanowią kolejne słowa (ciągi znaków oddzielane niezabezpieczonymi spacjami) składające się na linię polecenia w wyniku którego został uruchomiony program (czyli nazwa polecenia, opcje i argumenty)

Pierwszy argument jest liczbą całkowita typu \cpp{int} i określa ilość elementów tablicy przekazanej jako drugi argument.

\begin{CodeFrame*}[c]{}
#include <stdio.h>

int main(int argc, char* argv[]) {
	for (int i=0; i<argc; ++i)
		printf("element %d to: %s\n", i, argv[i]);
	
	return argc - 2; // kod powrotu uzależniamy od ilości argumentów
}
\end{CodeFrame*}

Przykład działania (linie zaczynające się od \Verb#$# są wprowadzonymi komendami, pozostałe linie to output uruchomionych poleceń):
\begin{Verbatim}
$ ./a.out a b; echo "Kod powrotu $?"
element 0 to: ./a.out
element 1 to: a
element 2 to: b
Kod powrotu 1
$ ./a.out a; echo "Kod powrotu $?"
element 0 to: ./a.out
element 1 to: a
Kod powrotu 0
\end{Verbatim}

Zauważ że tablica otrzymana jako drugi argumnent zawsze ma co najmniej jeden element - nazwę uruchomionego polecenia.

\begin{ProTip}{Porada}
Argumenty linii poleceń są na ogół dużo lepszą metodą odbierania danych od użytkownika niż zadawanie pytań i oczekiwanie na wprowadzenie danych z klawiatury na standardowe wejście – umożliwiają łatwe wykorzystanie naszych programów w bardziej złożonych poleceniach, czy skryptach. Zastanów się jakby wyglądało programowanie w bashu gdyby komendy takie jak ls, grep, find, itd oczekiwały na wprowadzenie swoich opcji i argumentów na standardowym wejściu.
\end{ProTip}
