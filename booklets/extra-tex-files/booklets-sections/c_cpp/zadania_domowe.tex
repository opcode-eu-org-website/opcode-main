% Copyright (c) 2017-2020 Matematyka dla Ciekawych Świata (http://ciekawi.icm.edu.pl/)
% Copyright (c) 2017-2020 Robert Ryszard Paciorek <rrp@opcode.eu.org>
% 
% MIT License
% 
% Permission is hereby granted, free of charge, to any person obtaining a copy
% of this software and associated documentation files (the "Software"), to deal
% in the Software without restriction, including without limitation the rights
% to use, copy, modify, merge, publish, distribute, sublicense, and/or sell
% copies of the Software, and to permit persons to whom the Software is
% furnished to do so, subject to the following conditions:
% 
% The above copyright notice and this permission notice shall be included in all
% copies or substantial portions of the Software.
% 
% THE SOFTWARE IS PROVIDED "AS IS", WITHOUT WARRANTY OF ANY KIND, EXPRESS OR
% IMPLIED, INCLUDING BUT NOT LIMITED TO THE WARRANTIES OF MERCHANTABILITY,
% FITNESS FOR A PARTICULAR PURPOSE AND NONINFRINGEMENT. IN NO EVENT SHALL THE
% AUTHORS OR COPYRIGHT HOLDERS BE LIABLE FOR ANY CLAIM, DAMAGES OR OTHER
% LIABILITY, WHETHER IN AN ACTION OF CONTRACT, TORT OR OTHERWISE, ARISING FROM,
% OUT OF OR IN CONNECTION WITH THE SOFTWARE OR THE USE OR OTHER DEALINGS IN THE
% SOFTWARE.


\subsection{Zadania domowe}
\vspace{3pt}

\begin{Zadanie}{ --- 1 pkt}{funkcja_niepodzielne_cpp}
Napisz funkcję która wypisze liczby z zakresu od 0 do 20 nie podzielne przez wartość określoną w jej argumencie.
\end{Zadanie}

\begin{Zadanie}{ --- 2 pkt}{funkcja_wypisz_tablice}
Napisz funkcję, która przyjmuje dwa argumenty: tablicę C (czyli wskaźnik na zerowy element) liczb naturalnych i jej długość.
Funkcja ma wypisać każdy element tablicy oraz zwiększyć o jeden wartość tego elementu w oryginalnej tablicy.
Napisz \ul[black]{dwa warianty} tej funkcji:
\begin{enumerate}[label=\alph*)]
\item operujący arytmetyką wskaźnikową
\item operujący tablicami (czyli używający operatora dostępu do elementu tablicy: \Verb#[]#)
\end{enumerate}
\end{Zadanie}

\begin{Zadanie}{ --- 3 pkt}{licz_powtorzenia_cpp}
Napisz funkcję \Verb{zlicz} która dla podanej listy napisów policzy powtórzenia jej elementów. Na przykład dla wywołania:
\cpp#std::list<std::string> l = {"AX", "B", "AX"}; zlicz(l);# program powinien wypisać:
\vspace{-8pt}\begin{Verbatim}
AX wystepuje 2 razy
B wystepuje 1 razy
\end{Verbatim}
\vspace{-8pt}\textit{Wskazówka: Użyj mapy, w której element będzie stanowił klucz, a krotność jego wystąpień wartość.}
\end{Zadanie}
