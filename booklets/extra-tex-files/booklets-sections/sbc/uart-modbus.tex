% Copyright (c) 2021 Robert Ryszard Paciorek <rrp@opcode.eu.org>
% 
% MIT License
% 
% Permission is hereby granted, free of charge, to any person obtaining a copy
% of this software and associated documentation files (the "Software"), to deal
% in the Software without restriction, including without limitation the rights
% to use, copy, modify, merge, publish, distribute, sublicense, and/or sell
% copies of the Software, and to permit persons to whom the Software is
% furnished to do so, subject to the following conditions:
% 
% The above copyright notice and this permission notice shall be included in all
% copies or substantial portions of the Software.
% 
% THE SOFTWARE IS PROVIDED "AS IS", WITHOUT WARRANTY OF ANY KIND, EXPRESS OR
% IMPLIED, INCLUDING BUT NOT LIMITED TO THE WARRANTIES OF MERCHANTABILITY,
% FITNESS FOR A PARTICULAR PURPOSE AND NONINFRINGEMENT. IN NO EVENT SHALL THE
% AUTHORS OR COPYRIGHT HOLDERS BE LIABLE FOR ANY CLAIM, DAMAGES OR OTHER
% LIABILITY, WHETHER IN AN ACTION OF CONTRACT, TORT OR OTHERWISE, ARISING FROM,
% OUT OF OR IN CONNECTION WITH THE SOFTWARE OR THE USE OR OTHER DEALINGS IN THE
% SOFTWARE.

\ifdefined\mysection
	\def\useExternalSection{TRUE}
\else
	\newcommand\mysection[1]{\section{#1}}
	\newcommand\mysubsection[1]{\subsection{#1}}
\fi

\mysection{UART}

\mysubsection{I$^2$C vs UART}

I$^2$C jest bardzo popularną magistralą "lokalną" używaną do połączenia CPU z układami peryferyjnymi (którymi mogą być np. także inne mikrokontrolery).
Ma ona ograniczony zasięg – typowo kilkanaście / kilkadziesiąt cm, raczej nie spotyka się kilkumetrowych rozwiązań.
Na większych odległościach tego typu funkcje pełni UART\footnote{którego używaliśmy już m.in. do programowania mikrokontrolera STM32 i do wypisywania wiadomości "diagnostycznych" przez STM32},
	najczęściej w wariancie RS485, pozwalającego na łączenie do 32 urządzeń na jednej linii.
W odróżnieniu od I2C sam UART nie określa żadnego sposobu adresacji i wykorzystywane są do tego inne protokoły, których powyżej UART/RS485 jest naprawdę bardzo dużo.
Chyba najpopularniejszym jest Modbus RTU.


\mysubsection{Modbus – adresacja i identyfikacja ramki}

Modbus jest otwartym, prostym protokołem komunikacyjnym występującym w kilku odmianach – RTU, ASCII, TCP.
Dwie pierwsze wykorzystują łącze szeregowe typu UART, trzecia używa pakietów IP/TCP.

\noindent
Protokół Modbus:
\begin{enumerate}
	\item jest protokołem master-slave, czyli jest wyraźnie określone, które urządzenie inicjalizuje transmisję, a które jedynie odpowiada na otrzymane żądania
	\item zapewnia adresację urządzeń (8 bitowy adres, zakres 1–247)
	\item określa sposób dostępu do danych w urządzeniu (określany 8 bitowym kodem funkcji) i adresację tych danych (16 bitowy adres rejestru / wejścia binarnego / ...)
	\item określa sposób identyfikacji początku i końca ramki
\end{enumerate}

Ten ostatni punkt jest szczególnie istotny przy przesyłaniu danych z użyciem łącza typu UART, gdzie identyfikowany jest tylko początek/koniec bajtu, ale nie ma określonego sposobu identyfikacji początku/końca grupy bajtów którą jest np. ramka jakiegoś protokołu.
\vspace{7pt}
\\
Modbus ASCII liczby reprezentujące adresy, kody funkcji i dane zapisuje w postaci tekstowej (jako liczby szesnastkowe), więc do identyfikacji początku i końca ramki mogą posłużyć inne niż (0-9A-F) znaki ASCII – początek ramki oznacza dwukropek (\Verb$:$), a koniec ciąg \Verb$\r\n$.
\vspace{3pt}
\\
Modbus RTU\footnote{którego obsługa jest wymagana przez standard dla urządzeń zgodnych z Modbus i jest on dużo popularniejszy niż wariant ASCII} przesyła te wartości liczbowe binarnie (czyli po prostu w postaci danej liczby), zatem nie ma tutaj żadnej wolnej wartości, którą można by użyć jako znacznik początku / końca ramki. W tym przypadku służą do tego zależności czasowe:
\begin{itemize}
	\item odstęp pomiędzy bajtami w ramce nie może przekroczyć 1.5 czasu trwania transmisji jednego bajtu
	\item odstęp pomiędzy ramkami musi wynosić przynajmniej 3.5 czasu trwania transmisji jednego bajtu
\end{itemize}

\noindent
Pełną specyfikację protokołu można znaleźć na \url{https://modbus.org/specs.php}.

\mysubsection{Modbus w Linuxie}

Obsługę protokołu Modbus RTU po stronie Linuxa może zapewnić program \Verb$mbpoll$ pozwalający zarówno na odczyt (modbusowe funkcje 1, 2, 3 i 4) jak i zapis (modbusowe funkcje 5 i 6) danych z/do urządzenia modbus TCP lub RTU.

Jeżeli ten zestaw funkcjonalności jest dla nas niewystarczający lub po prostu potrzebujemy zintegrować obsługę \textit{Modbus} z naszym kodem programu możemy użyć bezpośrednio biblioteki \textit{libmodbus} (na której bazuje \textit{mbpoll}).

\ifdefined\useExternalSection
	\let\useExternalSection\undefined
\else
	\let\mysection\undefined
	\let\mysubsection\undefined
\fi
