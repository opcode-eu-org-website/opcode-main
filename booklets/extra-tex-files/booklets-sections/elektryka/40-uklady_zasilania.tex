% Copyright (c) 2011-2021 Robert Ryszard Paciorek <rrp@opcode.eu.org>
% 
% MIT License
% 
% Permission is hereby granted, free of charge, to any person obtaining a copy
% of this software and associated documentation files (the "Software"), to deal
% in the Software without restriction, including without limitation the rights
% to use, copy, modify, merge, publish, distribute, sublicense, and/or sell
% copies of the Software, and to permit persons to whom the Software is
% furnished to do so, subject to the following conditions:
% 
% The above copyright notice and this permission notice shall be included in all
% copies or substantial portions of the Software.
% 
% THE SOFTWARE IS PROVIDED "AS IS", WITHOUT WARRANTY OF ANY KIND, EXPRESS OR
% IMPLIED, INCLUDING BUT NOT LIMITED TO THE WARRANTIES OF MERCHANTABILITY,
% FITNESS FOR A PARTICULAR PURPOSE AND NONINFRINGEMENT. IN NO EVENT SHALL THE
% AUTHORS OR COPYRIGHT HOLDERS BE LIABLE FOR ANY CLAIM, DAMAGES OR OTHER
% LIABILITY, WHETHER IN AN ACTION OF CONTRACT, TORT OR OTHERWISE, ARISING FROM,
% OUT OF OR IN CONNECTION WITH THE SOFTWARE OR THE USE OR OTHER DEALINGS IN THE
% SOFTWARE.

\section{Układy zasilania}

\subsection{Związek z potencjałem ziemi (układy sieci - IT, TT, TN)}

Wyróżnia się kilka \href{https://en.wikipedia.org/wiki/Earthing_system}{układów sieci zasilającej}. Poszczególne układy posiadają kilku literowe oznaczenia w których:
\begin{itemize}
	\item pierwsza litera (\strong{I} lub \strong{T}) oznacza relację punktu neutralnego transformatora i potencjału ziemi:
	\begin{itemize}
		\item \strong{I} – punkt neutralny transformatora izolowany od ziemi (lub połączony poprzez dużą impedancję)
		\begin{itemize}
			\item uzwojenie wtórne transformatora może nie posiadać punktu neutralnego (np. może być połączone w trójkąt)
			\item przewód neutralny nie musi (ale może) występować w instalacji
			\item nie da się odróżnić przewodu neutralnego od pojedynczego przewodu fazowego – w obu wypadkach potencjał względem ziemi jest nieustalony
			\item znaczenie mają jedynie napięcia pomiędzy przewodami fazowymi oraz fazowymi i neutralnym (jeżeli występuje)
		\end{itemize}
		\item \strong{T} – punkt neutralny transformatora połączony z ziemią
		\begin{itemize}
			\item uzwojenie wtórne transformatora musi posiadać punkt neutralny (typowo jest uzwojeniem typu gwiazda)
			\item przewód neutralny ma potencjał równy potencjałowi ziemi, zatem inne napięcia odnoszą się także do potencjału ziemi (np. w przewodzie fazowym jest 230V względem Ziemi)
		\end{itemize}
	\end{itemize}
	\item druga litera (\strong{T} lub \strong{N}) oznacza sposób połączenia części normalnie nieprzewodzących z potencjałem ziemi:
		\item \strong{T} – połączenie za pomocą lokalnego uziomu lub uziomu grupowego (obejmującego kilka odbiorników) nie połączonego z uziomem punktu neutralnego transformatora
		\item \strong{N} – połączenie poprzez uziom punktu neutralnego transformatora, w tym wypadku występują kolejne oznaczenia informujące w jaki sposób realizowane jest to połączenie:
		\begin{itemize}
			\item \strong{S} - przy pomocy osobnego przewodu ochronnego PE dochodzącego do stacji transformatorowej
			\item \strong{C} - przy pomocy wspólnego przewodu ochronnego i neutralnego PEN\footnote{\label{PEN}Przewód PEN powinien mieć przekrój minimum 10mm$^2$ Cu lub 16mm$^2$ Al}$^,$\footnote{Układ nie jest dopuszczony do stosowania w budynkach na terenie Polski.}
			\item \strong{C-S} - na części trasy (od stacji transformatorowej do punktu rozdziału PEN\footnoteref{PEN}, np. w złączu lub rozdzielni głównej\footnote{Punkt taki powinien być uziemiony.})
			w postaci wspólnego przewodu ochronnego i neutralnego PEN, dalej w postaci osobnego przewodu ochronnego PE
		\end{itemize}
\end{itemize}
%
Wyróżnia się układy:
\begin{itemize}
	\item \strong{IT} – punkt neutralny transformatora izlowoany lub nie wystepuje, uziemienia urządzeń łączone lokalnie do ziemi
	\item \strong{TT} – punkt neutralny transformatora połączony z ziemią, uziemienia urządzeń łączone lokalnie do ziemi
	\item \strong{TN} – punkt neutralny transformatora połączony z ziemią, uziemienia urządzeń łączone do uziemienia punktu neutralnego transformatora, przy pomocy:
	\begin{itemize}
		\item przewodu PE  – układ \strong{TN-S}
		\item przewodu PEN – układ \strong{TN-C}
		\item przewodu PEN (bliżej transformatora) i PE (bliżej odbiornika) – układ \strong{TN-C-S}
	\end{itemize}
\end{itemize}

\subsubsection{układ IT}
\begin{itemize}
	\item Dzięki izolowaniu wszystkich przewodów roboczych od potencjału ziemi pierwsze zwarcie doziemne\footnote{bez względu na to czy przewodu fazowego czy neutralnego} w tym układzie związane jest z przepływem niskiego prądu zwarciowego.
	\begin{itemize}
		\item Wartość prądu takiego zwarcia zależy od wielkości sieci (wielkości pojemności sieć-ziemia) i często jest rzędu kilku mA, czyli może nie spowodować nawet zadziałania RCD.
		\item Napięcie które po takim zwarciu będzie występować pomiędzy pozostałymi przewodami a ziemią zależy od tego czy zwarciu uległ przewód neutralny czy fazowy
			(bo ziemia zaczyna pełnić jego funkcje, czyli możemy mieć np. napięcie międzyfazowe między ziemią a dowolną z nie zwartych faz).
		\item Można powiedzieć że w momencie wystąpienia pierwszego zwarcia doziemnego układ przechodzi w układ TT (przy czym impedancja uziemienia "punktu neutralnego", powstała na skutek zwarcia, jest nieznanej jakości).
	\end{itemize}
	\item Układ może być (i często jest) stosowany:
	\begin{itemize}
		\item w celu zwiększenia niezawodności (np. na salach operacyjnych) - pierwsze zwarcie jest sygnalizowane, ale nie powoduje wyłączenia; układ uzyskiwany z lokalnego trafo separacyjnego 1:1 zasilanego z „normalnej” sieci TN
		\item w celu zwiększenia bezpieczeństwa (przeciw iskrzeniu, np. w kopalniach) - prąd pierwszego zwarcia jest niewielki, więc zmniejsza ryzyko powstania iskry; pierwsze zwarcie powoduje automatyczne wyłączenie obwodu
		\item w przypadkach gdzie „trudno znaleźć ziemię” w celu uziemienia punktu neutralnego (np. na statkach)
	\end{itemize}
	\item Układ pozwala na rezygnację z przewodu neutralnego (i stosowanie wyłącznie napięć międzyfazowych, ale nie jest to wymogiem\footnote{
			Rezygnacja z przewodu neutralnego i stosowanie tylko napięć międzyfazowych eliminuje także problem przerwania przewodu neutralnego skutkującego ryzykiem pojawienia się napięcia międzyfazowego na odbiornikach jednofazowych.
		}.
	\item Układ ten wymaga stosowania zabezpieczeń dwupolowych (dla odbiorów zasilanych faza-neutralny lub faza-faza).
	\item W układzie powinny być stosowane układy sygnalizujące wystąpienie pierwszego zwarcia doziemnego / układy kontroli stanu izolacji.
	\item Główną wadą takiego układu jest gorsza (wyższa) niż w TN (a często także niż w TT) impedancja pętli podwójnego zwarcia doziemnego, co może prowadzić do braku odpowiednio szybkiego wyłączenia takiego zwarcia.
\end{itemize}

\subsection{prąd zwarcia}

\begin{center}\includegraphics{img/elektryka/prad_zwarciowy}\end{center}

Przy projektowaniu instalacji i doborze zabezpieczeń istotna rolę odgrywa prąd zwarcia.
Oblicza się go w oparciu o dane transformatora i parametry linii przesyłowych (grubość i materiał kabli, co przekłada się na ich rezystancję) zgodnie z poniższymi zasadami:

\begin{itemize}
	\item dane do obliczeń – parametry transformatora:
	\begin{itemize}
		\item prąd nominalny $I_n$, czyli prąd uzwojenia wtórego przy nominalnym obciążeniu transformatora
		\item nominalne napięcie strony pierwotnej $U_P$ oraz strony wtórnej $U_n$
		\item napięcie zwarcia $U_k$, czyli napięcie na uzwojeniu pierwotnym powodujące przepływ nominalnego prądu na zwartym uzwojeniu wtórnym,
			typowo podane w postaci $U_k{\rm [\%]}$, czyli wyrażone jako procent nominalnego napięcia uzwojenia pierwotnego:
				$$U_k{\rm [\%]}=\frac{U_p}{U_k}$$
	\end{itemize}
	\item prąd zwarcia $I_{k1}$ na zaciskach transformatora (czyli w w obwodzie zamykanym przez S1) obliczamy w oparciu o parametry T1:
		$$I_{k1} = \frac{I_n}{U_k[\%]}$$
	\item rezystancję wewnętrzną $R_w$ transformatora obliczamy w oparciu o napięcie nominalne strony wtórnej $U_n$ i prąd $I_{k1}$:
		$$R_w = \frac{U_n}{I_{k1}} = \frac{U_n \cdot U_k{\rm [\%]}}{I_n}$$
	\item dzięki temu możemy obliczyć prąd zwarcia instalacji:
		$$I_{k2} = \frac{U_n}{Rw + Ri} = \frac{U_n}{U_n \cdot U_k[\%]/I_n + R_i}$$
\end{itemize}

\noindent\strong{Przykład:}
\begin{itemize}
	\item trafo na 400V ($U_n=400{\rm ~V}$) o mocy 1.2MW ($I_n=1800{\rm ~A}$) i napięciu zwarcia $U_k{\rm [\%]}=6\%$
	\item szyna Cu o przekroju $800{\rm ~mm^2}$ i długości 1m ($R_i = 0.00004545~\Omega$)
	\item zatem $I_{k2} = 400 / (400 \cdot 0.06/1800 + 0.00004545) = 29.898{\rm ~kA}$
\end{itemize}
\vspace{8pt}

Jest to prosty sposób obliczania spodziewanego prądu zwarcia.
Obliczenia w projekcie powinny zawsze uwzględniać wszystkie czynniki i być wykonywane zgodnie z normami.
Autor nie ponosi odpowiedzialności za skutki błędnego zastosowania tego schematu obliczeń.

Po wykonaniu instalacji (oraz okresowo) dokonuje się pomiarów impedancji pętli zwarcia, które mają na celu ustalenie rzeczywistego (minimalnego) prądu zwarcia.
Jest on istotny ze względu na skuteczność zadziałania zabezpieczeń nadprądowych.
Konieczność wykonania pomiarów a nie tylko oparcia się na obliczeniu wynika m.in. z faktu iż mogły pojawić się w instalacji nieprzewidziane rezystancje (np. niedokręcony zacisk śrubowy).
