% Copyright (c) 2021 Robert Ryszard Paciorek <rrp@opcode.eu.org>
% 
% MIT License
% 
% Permission is hereby granted, free of charge, to any person obtaining a copy
% of this software and associated documentation files (the "Software"), to deal
% in the Software without restriction, including without limitation the rights
% to use, copy, modify, merge, publish, distribute, sublicense, and/or sell
% copies of the Software, and to permit persons to whom the Software is
% furnished to do so, subject to the following conditions:
% 
% The above copyright notice and this permission notice shall be included in all
% copies or substantial portions of the Software.
% 
% THE SOFTWARE IS PROVIDED "AS IS", WITHOUT WARRANTY OF ANY KIND, EXPRESS OR
% IMPLIED, INCLUDING BUT NOT LIMITED TO THE WARRANTIES OF MERCHANTABILITY,
% FITNESS FOR A PARTICULAR PURPOSE AND NONINFRINGEMENT. IN NO EVENT SHALL THE
% AUTHORS OR COPYRIGHT HOLDERS BE LIABLE FOR ANY CLAIM, DAMAGES OR OTHER
% LIABILITY, WHETHER IN AN ACTION OF CONTRACT, TORT OR OTHERWISE, ARISING FROM,
% OUT OF OR IN CONNECTION WITH THE SOFTWARE OR THE USE OR OTHER DEALINGS IN THE
% SOFTWARE.

\section{Silniki}

Silnik jest urządzeniem do zamiany energii elektrycznej na energię mechaniczną (prawie zawsze) ruchu obrotowego (który potem może być zamieniany na inne postaci ruchu).
W konstrukcji mechanicznej silnika wyróżnia się:
\begin{itemize}
	\item \strong{wirnik} – element obracający się
	\item \strong{stojan} – element pozostający nieruchomo wewnątrz którego obraca się wirnik
\end{itemize}

Możliwe konstrukcje silników wielofazowych rozważaliśmy już przy omawianiu zagadnienia prądu wielofazowego, w oparciu o obracanie magnesu trwałego przy pomocy układu cewek zasilanych z poszczególnych faz.
Pośrednio (przy rozważaniu układu podzielonej fazy) przyjrzeliśmy się nawet możliwości konstrukcji silnika jednofazowego – nie udało nam się.
Model ten dość dobrze oddaje koncepcję budowy rzeczywistych silników elektrycznych.
I w każdym typie silnika konieczne jest zapewnienie co najmniej dwóch faz, aby móc uzyskać wirujące (a nie tylko zmienne – jak przy jednej fazie) pole magnetyczne.

Dlatego też najprostszym koncepcyjnie silnikiem jest silnik trójfazowy prądu przemiennego z magnesami stałymi w wirniku i elektromagnesami w stojanie, czyli właśnie taki jak rozważaliśmy przy prądzie trójfazowym.
Zamiast magnesów trwałych na wirniku mogą być umieszczone elektromagnesy zasilane prądem stałym (doprowadzonym poprzez szczotki).
Silniki tego typu określa się mianem \strong{silników synchronicznych} gdyż ich prędkość obrotowa jest równa prędkości wirowania pola magnetycznego.

Wadą takich silników jest trudność w ich rozruchu – taki silnik (podłączony bezpośrednio do sieci elektrycznej) nie zacznie się obracać.
Spowodowane to jest tym iż pole magnetyczne w stojanie będzie zmieniało się zbyt szybko, aby pokonać bezwładność mechaniczną wirnika i wprawić go w ruch.

Stosowane jest kilka mechanizmów na rozruch takiego silnika (np. rozruch jako silnika asynchronicznego, co wyklucza stosowanie magnesów trwałych na wirniku).
Współcześnie do rozruchu tego typu silników wykorzystywane są elektroniczne przemienniki częstotliwości (falowniki),
	które pozwalają na uruchomienie takiego silnika przy wolno zmiennym polu magnetycznym stojana, jak również na płynne sterowanie prędkością obrotową silnika.

\subsection{silniki indukcyjne (asynchroniczne AC)}

Główna różnica w stosunku co do silnika synchronicznego polega na wykonaniu wirnika – znajdują się na nim uzwojenia w których (za sprawą zmiennego pola magnetycznego stojana) indukowany jest prąd (dlatego \textit{silnik indukcyjny}).
Jego przepływ powoduje powstanie pola magnetycznego wirnika (przekształcenie wirnika w magnes), dzięki czemu zaczyna się obracać.
Aby proces indukcji mógł zachodzić wirnik musi obracać się wolniej niż wirowanie pola magnetycznego stojana (czyli wolniej niż wynikałoby z częstotliwości zasilania, dlatego \textit{silnik asynchroniczny}).
Wirnik często wykonywany jest w postaci klatki (stanowiącej od razu jego uzwojenie), dlatego większość silników tego typu to \strong{silniki klatkowe}.

\noindent Więcej informacji: \url{https://www.youtube.com/watch?v=59HBoIXzX_c}

\subsection{silniki jednofazowe AC}

Jak się przekonaliśmy silnik z jedną fazą nie działa – jedna faza daje zmienne pole, ale one nie rotuje (nie determinuje kierunku obrotu, nie powoduje rozpoczęcia obrotu).
Dlatego w silnikach jednofazowych prądu przemiennego wytwarzana jest druga faza z użyciem kondensatora (który jak wiemy wprowadza przesunięcie fazowe prądu).
Koncepcyjnie są to silniki takie jak omawiane przy prądzie dwufazowym, wykonywane najczęściej jako asynchroniczne (klatkowe).

\noindent Więcej informacji: \url{https://www.youtube.com/watch?v=jNWlWzFzHi4}

\subsection{silniki prądu stałego (DC)}

W klasycznym silniku prądu stałego wirowanie pola magnetycznego uzyskuje się poprzez komutację mechaniczną.
W rozwiązaniu takim w stojanie znajdują się magnesy trwałe, a na wirniku uzwojenia.
Komutator zmienia biegunowość uzwojeń wirnik tak aby obracał się on względem stojana.

\noindent Więcej informacji: \url{https://www.youtube.com/watch?v=GQatiB-JHdI}

\subsection{bezszczotkowe silniki prądu stałego (BLDC)}

W silniku BLDC mamy nieruchome elektromagnesy (w stojanie) i magnes trwały na wirniku (czyli tak jak w silniku synchronicznym z magnesami trwałymi).
Sterownik silnika, będący elektroniczną formą komutatora, zasila kolejno cewki poszczególnych „faz” co powoduje ruch wirnika przyciąganego/odpychanego przez odpowiednie elektromagnesy (analogicznie jak w wspomnianym silnku AC).
Główna różnica polega na napięciu zasilania oraz kształcie sygnałów podawanych na cewki – w silniku BLDC będzie on na ogół przypominał prostokąt a nie sinus.

\subsection{silniki krokowe}

Sa to silniki bardzo zbliżone do BLDC – również mamy elektromagnesy w stojanie i magnesy trwałe na wirniku.
Główna różnica polega w zastosowaniu i sposobie sterowania.
Sterując BLDC zależy nam na prędkości obrotowej, a sterując silnikiem krokowym zależy nam na wykonaniu określonej liczby kroków.
Często będą występować też różnice konstrukcyjne – w ilości faz, cewek w stojanie, biegunów na wirniku, itd.
W silnikach krokowych z każdą fazą działania silnika następuje tylko niewielkie (często stanowiące niewielki ułamek pełnego obrotu) obrócenie wirnika.
Z każdym impulsem sterującym silnik wykonuje taki jeden krok.
Sposób wysterowania sprawia także że silnik taki ma duży moment trzymający (kosztem energochłonności).

\noindent Więcej informacji: \url{https://www.youtube.com/watch?v=09Mpkjcr0bo}

\subsection{podsumowanie}

Warto zauważyć upowszechnianie się elektronicznych systemów sterowania silnikami.
Falowniki pozwalają na łatwe stosowanie silików trójfazowych w obwodach jednofazowych oraz pozwalają także na stosowanie silników synchronicznych z magnesami trwałymi, łatwe kontrolowanie prędkości obrotowej silników AC.
Należy jednak zwrócić uwagę na podobieństwa w działaniu przestawionych typów silników elektrycznych (np. BLDC do synchronicznego AC z magnesami stałymi sterowanego przez falownik).

W przedstawionych tutaj opisach zostały prawie całkowicie pominięte zagadnienia związane z mechaniczną konstrukcją silników,
	a także rozwiązaniami takimi jak np. umieszczenie więcej niż dwóch biegunów na wirniku (np. 4 w kolejności NSNS), co oczywiście wpływa też na układ elektromagnesów zasilanych z poszczególnych faz.
