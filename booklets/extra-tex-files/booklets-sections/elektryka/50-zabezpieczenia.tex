% Copyright (c) 2021 Robert Ryszard Paciorek <rrp@opcode.eu.org>
% 
% MIT License
% 
% Permission is hereby granted, free of charge, to any person obtaining a copy
% of this software and associated documentation files (the "Software"), to deal
% in the Software without restriction, including without limitation the rights
% to use, copy, modify, merge, publish, distribute, sublicense, and/or sell
% copies of the Software, and to permit persons to whom the Software is
% furnished to do so, subject to the following conditions:
% 
% The above copyright notice and this permission notice shall be included in all
% copies or substantial portions of the Software.
% 
% THE SOFTWARE IS PROVIDED "AS IS", WITHOUT WARRANTY OF ANY KIND, EXPRESS OR
% IMPLIED, INCLUDING BUT NOT LIMITED TO THE WARRANTIES OF MERCHANTABILITY,
% FITNESS FOR A PARTICULAR PURPOSE AND NONINFRINGEMENT. IN NO EVENT SHALL THE
% AUTHORS OR COPYRIGHT HOLDERS BE LIABLE FOR ANY CLAIM, DAMAGES OR OTHER
% LIABILITY, WHETHER IN AN ACTION OF CONTRACT, TORT OR OTHERWISE, ARISING FROM,
% OUT OF OR IN CONNECTION WITH THE SOFTWARE OR THE USE OR OTHER DEALINGS IN THE
% SOFTWARE.

\section{Zabezpieczenia}

W instalacjach elektrycznych stosuje się różnego rodzaju zabezpieczenia. Generalnie można je podzielić na kilka typów – zabezpieczenia przed:
\begin{itemize}
	\item przepływem zbyt dużego prądu (wyłączniki nadprądowe, przeciążeniowe, bezpieczniki topikowe, ...)
	\item ucieczką prądu z instalacji (wyłączniki różnicowoprądowe, kontrolery stanu izolacji, ...)
	\item iskrzeniem
	\item przepięciami / pikami napięcia powyżej wartości znamionowej (ochronniki przepięciowe w postaci iskrowników, warystorów, ...)
\end{itemize}

\subsection{bezpiecznik topikowy}

Podstawowym rodzajem zabezpieczenia jest \href{https://pl.wikipedia.org/wiki/Bezpiecznik_topikowy}{bezpiecznik topikowy}, którego głównym elementem jest przewodzący prąd topik.
W skutek przewodzenia prądu nagrzewa się on, a jeżeli przewodzony prąd jest zbyt duży (z powodu przeciążenia lub zwarcia) ulega przepaleniu.
W bezpiecznikach takich znajdować się może także piasek lub inna substancja / mechanizm (np. w postaci emitera gazu), których celem jest przyspieszenie gaszenia łuku elektrycznego powstałego po przepaleniu topika.

\subsection{wyłącznik nadprądowy}

\href{https://pl.wikipedia.org/wiki/Wyłącznik_instalacyjny}{Wyłącznik nadprądowy}\footnote{
	W tym miejscu warto wspomnieć o rozróżnieniu pomiędzy wyłącznikiem (ma zdolność wyłączania prądów zwarciowych), rozłącznikiem (ma zdolność wyłączania prądów roboczych) i odłącznikiem (może być otwierany/zamykany jedynie w stanie bezprądowym). Związane to jest z ich budową mechaniczną i zdolnością gaszenia łuku.
} pełni w instalacji taką samą funkcję jak bezpiecznik – zabezpiecza przed zbyt dużym prądem.
Typowo składa się z dwóch członów zabudowanych w pojedynczym aparacie – zwarciowego (elektromagnetycznego, działającego natychmiastowo przy prądzie kilka-kilkanaście razy przekraczającym wartość nominalną) oraz termicznego (którego prędkość zadziałania zależy od wielkości przeciążenia). Można się spotkać z konstrukcjami w których (ze względu na pożądaną charakterystykę działania) występuje tylko jeden z tych członów.

\subsection{wyłącznik różnicowoprądowy (RCD, RCCB, GFI, GFCI, ALCI, LCDI)}

Wyłącznik różnicowoprądowy jest urządzeniem reagującym na różnicę w wartości prądu wpływająceo i wypływającego przewodami roboczymi (fazowe i neutralny\footnote{Jeżeli występuje.}).
Wystąpienie takiej różnicy oznacza iż prąd ucieka z instalacji inną drogą – wystąpiło połączenie z potencjałem ziemi.\footnote{
	Wyłącznik różnicowoprądowy wykrywa i reaguje jedynie na wypływ prądu do ziemi.
	Oznacza to że RCD nie zareaguje jeżeli osoba izolowana od ziemi podłączy się pomiędzy przewód fazowy a neutralny chroniony przez taki wyłącznik – nie ma wtedy różnicy w prądzie wpływającym i wypływającym
}.
Jeżeli jest to połączenie o małej rezystancji (powodujące przepływ prądu znacznie większego niż nominalny) mamy do czynienia z zwarciem, na które zareaguje także wyłącznik nadprądowy (bądź inny bezpiecznik).
Jeżeli jednak rezystancja jest większa (więc płynący prąd jest niewielki) zareaguje jedynie wyłącznik różnicowoprądowy.

Wyróżnia się kilka typów wyłączników różnicowoprądowych, określających typy prądu różnicowego (nie mylić z prądem roboczym płynącym przez wyłącznik) na które reaguje dany aparat:
\begin{itemize}
	\item \strong{AC} – jedynie prąd przemienny sinusoidalny 
	\item \strong{A}  – prąd przemienny sinusoidalny, prąd sinusoidalny wyprostowany jednopołówkowo i impulsowy
	\item \strong{B}  – jak typ A oraz dodatkowo: prąd stały, prądy wyższej częstotliwości oraz kombinacje prądu przemiennego i stałego\footnote{
		Typowo posiada dwa przekładniki detekcyjne – taki jak w typie A oraz osobny dla prądów stałych, działanie przekładnika dla prądu stałego zależne jest od obecności napięcia zasilania}
\end{itemize}

Wyłączniki różnicowoprądowe mogą być wykonane:
\begin{itemize}
	\item z wyzwalaniem bezpośrednim – energia potrzebna a do inicjalizacji wyzwolenia (czyli energia potrzebna do zwolnienia sprężyny) uzyskiwana jest z przekładnika Ferrantiego
	\item z wyzwalaniem elektronicznym – informacja o wartości prądu różnicowego z przekładnika jest wzmacniana elektronicznie celem wytworzenia impulsu wyzwalającego (np. aktywującego cewkę wybijakową)\footnote{
		Wyłącznik taki (w odróżnieniu od wyłącznika z wyzwalaniem bezpośrednim) wymaga dwóch biegunów zasilania do poprawnej pracy i nie jest dopuszczony do stosowania przez przepisy europejskie,
		jest za to powszechnie stosowanym rozwiązaniem w USA i Kanadzie.
	}
\end{itemize}
Więcej informacji: \url{https://bezel.com.pl/2018/08/01/budowa-i-zasada-dzialania/}

Wyłączniki różnicowoprądowe często integrowane są z wyłącznikiem nadprądowym w jedno urządzenie.
Określane są wtedy jako \textit{RCBO} lub (w USA) \textit{GFCI breaker}.

\subsection{zabezpieczenia ziemiozwarciowe}

Zabezpieczenia przed zwarciami doziemnymi w sieciach SN\footnote{Sieci SN są typowo budowane w układzie IT, czyli zwarcie pojedynczej fazy z ziemią nie powoduje przepływu dużego prądu i zadziałania zabezpieczeń nadprądowych.} oparte są na tej samej zasadzie działania co RCD – mierzony jest z użyciem przekładnika Ferrantiego prąd upływu i po przekroczeniu zadanej wartości następuje wyłączenie obwodu.

Zabezpieczenia ziemiozwarciowe (dla odpowiednio dużych prądów doziemnych) mogą być też realizowane na zasadzie obliczania prądu różnicowego w oparciu o pomiar prądów roboczych przez elektroniczny układ zabezpieczeń lub pomiar prądu w uziomie za pomocą przekładnika. % np. Micrologic 6.0 A

\subsection{urządzenia kontroli stanu izolacji w sieci IT (UKSI, IMD)}

Wyłączniki różnicowoprądowe w układzie IT (zależnie od czułości wyłącznika i prądu doziemnego charakterystycznego dla danej sieci IT) mogą reagować na pierwsze bądź drugie zwarcie doziemne\footnote{
	W przypadku drugiego także gdy któreś z nich ma miejsce przez znaczną impedancję. Na podwójne zwarcie o małej impedancji zareaguje także zabezpieczenie nadprądowe.
	RCD może jednak nie zareagować na podwójne zwarcie gdy oba zwarcia mają zbliżoną impedancję i znajdują się za RCD.
}.
Jednak zamiast wysokoczułych RCD w układzie tym na ogół stosowany jest system monitorowania stanu izolacji, którego zadaniem jest sygnalizowanie pierwszego zwarcia doziemnego (lub natychmiastowe wyłączanie zasilania w takiej sytuacji).

Działanie sytemu kontroli stanu izolacji opiera się na wprowadzaniu pomiędzy obwód a ziemię napięcia testowego i pomiarze płynącego w związku z nim prądu.\footnote{
	Dawniej była stosowna metoda pomiaru napięć pomiędzy przewodami roboczymi i ziemią. Nie pozwala ona jednak na wykrywanie uszkodzeń symetrycznych i w związku z tym nie jest zgodna z obecnymi wymaganiami dla systemów IMD.
}
Znajomość tych wartości pozwala na obliczenie wartości rezystancji izolacji\footnote{
	Należy zauważyć że pomiar ten odbywa się przy napięciu roboczym instalacji, a nie (jak pomiary okresowe) przy napięciu znacznie wyższym.
}.
Proste systemy korzystają po prostu z napięcia stałego (co jest problematyczne w sieciach AC/DC i DC), bardziej zaawansowane systemy używają bardziej złożonych impulsów pomiarowych.

Ze względu na charakter takiego pomiaru może być tylko jedno urządzenie go realizującego na całą połączoną galwanicznie sieć IT
	– prąd pomiarowy rozchodzi się po całej sieci i podłączenie kilku urządzeń IMD skutkowałoby ich wzajemnym zagłuszaniem się.
Celem ustalenia na którym z obwodów wystąpiło pogorszenie stanu izolacji (czy też pierwsze zwarcie doziemne) układ taki rozbudowuje się o układ lokalizacji zwarcia doziemnego IFL
	wykorzystuje on przepływ prądu będący efektem działania IMD, przy czym pomiar wartości tego prądu wykonywany jest dla każdego odpływu niezależnie\footnote{
		Dokładniej jest to pomiar różnicowy tego prądu na wszystkich przewodach roboczych tego odpływu (jak w RCD), inaczej mogłyby być fałszywe wskazania związane z przepływem prądu testowego przez instalację do miejsca uszkodzenia
	}.
Obwód mający obniżoną rezystancję na jednym (lub kilku) przewodach będzie charakteryzował się tym że prąd pomiarowy w nim „ginie”
	– więcej go różnica sumy prądów wpływających wszystkimi przewodami będzie różna od sumy prądów wypływających, gdyż część tego prądu odpływa do ziemi pomijając przekładnik (przez miejsce uszkodzenia).

\vspace{7pt}\noindent
Więcej informacji:
\begin{itemize}
	\item \href{https://www.youtube.com/watch?v=fs5UW-PUpLE}{Urządzenia do kontroli stanu izolacji w sieciach IT (video)}
	\item \href{https://elektroenergetyka.pl/upload/file/2020/04/wiatr_kwiecien_2020.pdf}{Ochrona przeciwporażeniowa w sieci o układzie zasilania IT}
	\item \href{https://www.promac.com.pl/wp-content/uploads/2018/11/Kontrola-izolacji-lokalizacja-IT.pdf}{Kontrola stanu izolacji i lokalizacja doziemień w sieciach nieuziemionych (układ IT)}
	\item \href{https://www.repostechnik.pl/wp-content/uploads/2017/05/HIG-IFL1_PL.pdf}{Kontrola stanów izolowanych układów z pomocą przekaźników kontroli stanu izolacji HIG – lokalizacja miejsca doziemienia}
	\item \href{https://www.studiecd.dk/cahiers_techniques/The_IT_system_earthing_in_LV.pdf}{The IT earthing system (unearthed neutral) in LV}
\end{itemize}

\subsection{układy detekcji łuku elektrycznego (AFCI)}

Zabezpieczenia AFCI pozwalają na wykrywanie łuku elektrycznego (iskrzenia), które towarzyszy wielu rodzajom uszkodzeń instalacji elektrycznych,
	a (jeżeli ma miejsce w torze normalnego przepływ prądu – np. na połączeniu dwóch fragmentów przewodu fazowego) nie jest wykrywane przez inne rodzaje zabezpieczeń.
Detekcja odbywa się w oparciu o wykrywanie zakłóceń, które takie iskrzenie wprowadza do sieci elektrycznej.

Zabezpieczenia te często są zintegrowane z członem różnicowoprądowym i/lub nadprądowym.

\subsection{zabezpieczenia przepięciowe (SPD)}

Przepięcie jest to wzrost napięcia powyżej wartości znamionowej. Przyczyną takiego wzrostu mogą być m.in. wyładowania atmosferyczne, rozłączanie obwodów zawierających cewki (np. silnki, transformatory).
Zabezpieczenia przepięciowe służą do ograniczania wartości napięcia i najczęściej realizowane są w postaci iskierników i warystorów.

Ochronniki przepięciowe muszą być stopniowane\footnote{
	Zasadniczo jako jedyne z omawianych.
	Stopniowanie nadprądowych czy RCD jest możliwe, ale nie jest niezbędne do ich działania
		– zabezpieczenie nadprądowe np. 1A może być podłączone bez żadnych wcześniejszych zabezpieczeń do transformatora o prądzie nominalnym np. 2.5kA, wystarczy że będzie posiadać odpowiednią zdolność zwarciową.
} i konieczne jest zachowanie wymaganych odległości pomiędzy kolejnymi stopniami zabezpieczeń lub stosowanie specjalnych układów ją symulujących.
Wyróżnia się następujące stopnie (typy) ochrony przepięciowej:
\begin{itemize}
	\item \strong{typ 1} (klasa I, dawniej B)   – ochrona przed bezpośrednimi lub bliskimi trafieniami piorunów;
		głównym parametrem jest szczytowa wartość prądu udarowego $I_{imp}$ (najczęściej o kształcie 10/350 µs) symulującego działanie pioruna;
		montaż w pobliżu wejścia instalacji do obiektu (w rozdzielnicy głównej)
	\item \strong{typ 2} (klasa II, dawniej C)  – ochrona przed odległymi trafieniami piorunów i przepięciami łączeniowymi;
		głównym parametrem jest szczytowa wartość prądu wyładowczego o kształcie 8/20 µs (może być podawana jako $I_n$ – wytrzymywany bez uszkodzenia wielokrotnie lub $I_{max}$ – maksymalny wytrzymywany bez uszkodzenia co najmniej jednokrotnie)
		montaż typowo w rozdzielnicach dystrybucyjnych
	\item \strong{typ 3} (klasa III, dawniej D) – ochrona wrażliwych urządzeń;
		głównymi parametrami są napięcie obwodu otwartego generatora $U_{oc}$ (1.2/50 µs);
		montaż typowo w gnieździe zasilającym (lub bliskiej mu rozdzielnicy)
\end{itemize}
Bardzo isotnym parametrem ograniczników jest poziom ochrony oznaczany jako $U_p$. Określa on maksymalną wartość przepięcia, która może wystąpić (przy poprawnym funkcjonowaniu ochronika).
Typowo elementy instalacji dopuszczają maksymalny poziom przepięcia rzędu 4kV, sprzęt AGD 2.5kV, a sprzęt elektroniczny 1.5kV.
Generalnie im wyższe wartości $I_{imp}$, $I_n$, $U_{oc}$ i niższe $U_p$ ochronnika tym lepiej.

\begin{wrapfigure}{r}{0.4\textwidth}
\includegraphics[width=0.35\textwidth]{img/elektryka/przepieciowy_3+1}
\end{wrapfigure}

Dostępne są także ochronniki kombinowane typu 1+2 oraz ochronniki z zintegrowanym elementem koordynującym pozwalającym bezpośrednio za nimi umieszczać ochronnik o wyższym typie.

Na schemacie obok został przedstawiony ochronnik typu 1+2 wykonany w układzie „3+1”, w którym ochronniki warystorowe i iskiernikowe znajdują się pomiędzy przewodami fazowymi a neutralnym, a pomiędzy przewodem neutralnym a PE znajduje się jedynie ochronnik iskiernikowy sumujący, którego zadaniem jest odprowadzić prąd udarowy z przewodu N do uziemienia.
Układ taki ma kilka zalet w porównaniu z (klasycznym) „4+0”.
Główną z nich jest umieszczenie ochronników równolegle z odbiornikami (pomiędzy fazami / fazą a neutralnym) i uniezależnienie poziomu ochrony odbiornika od napięcia na ochronniku N-PE.
Układ może być stosowany w systemach TN\footnote{Warto zaznaczyć że przy TN-C lub TN-C-S gdy rozdział następuje w tym samym miejscu co instalacja ochronnika nie za bardzo jest sens stosowania iskiernika między N a PE – i tak są metalicznie zwarte tuż obok.}, TT oraz IT.

\vspace{7pt}\noindent
Więcej informacji:
\begin{itemize}
	\item \href{https://rst.pl/ograniczniki-typu-i-ograniczniki-kombinowane-klasyfikacja-urzadzen/}{Klasyfikacja ograniczników przepięć – ograniczniki Typu i ograniczniki kombinowane}
	\item \href{https://www.dehn.pl/sites/default/files/uploads/dehn/DEHN-PL/druki/wp350_poradnik_dla_producentow_rozdzielnic.pdf}{Dobór i stosowanie SPD}
	\item \href{https://www.moeller.pl/Documentation/Katalogi/Katalog_EATON_opp_2010_PL.pdf}{Katlog „EATON” zawierający omówienie zagadnień ochrony przepięciowej}
	\item \href{https://www.dehn.pl/sites/default/files/uploads/dehn/pdf/blitzplaner/bpl2015/lpg_2015_e_complete.pdf}{Lightning protection guide}
\end{itemize}
