% Copyright (c) 2021 Robert Ryszard Paciorek <rrp@opcode.eu.org>
% 
% MIT License
% 
% Permission is hereby granted, free of charge, to any person obtaining a copy
% of this software and associated documentation files (the "Software"), to deal
% in the Software without restriction, including without limitation the rights
% to use, copy, modify, merge, publish, distribute, sublicense, and/or sell
% copies of the Software, and to permit persons to whom the Software is
% furnished to do so, subject to the following conditions:
% 
% The above copyright notice and this permission notice shall be included in all
% copies or substantial portions of the Software.
% 
% THE SOFTWARE IS PROVIDED "AS IS", WITHOUT WARRANTY OF ANY KIND, EXPRESS OR
% IMPLIED, INCLUDING BUT NOT LIMITED TO THE WARRANTIES OF MERCHANTABILITY,
% FITNESS FOR A PARTICULAR PURPOSE AND NONINFRINGEMENT. IN NO EVENT SHALL THE
% AUTHORS OR COPYRIGHT HOLDERS BE LIABLE FOR ANY CLAIM, DAMAGES OR OTHER
% LIABILITY, WHETHER IN AN ACTION OF CONTRACT, TORT OR OTHERWISE, ARISING FROM,
% OUT OF OR IN CONNECTION WITH THE SOFTWARE OR THE USE OR OTHER DEALINGS IN THE
% SOFTWARE.

Istnieje co najmniej kilka dziedzin techniki związanych z prądem elektrycznym i jego wykorzystaniem (elektrotechnika, elektronika, elektroenergetyki, ...).
Granice pomiędzy nimi bywają niekiedy dość płynne (np. stosowanie elementów elektronicznych w zastosowaniach elektroenergetyki), gdyż wszystkie zajmują się zjawiskami związanymi z przepływem prądu elektrycznego, a typowo rozróżnia je wartość prądu, napięcia, mocy (tu jednak nie ma wyraźnych granic) oraz cel w jakim ten prąd ma płynąć (przekazanie sygnału czy wykonanie pracy).

W tym artykule poruszone zostaną kwestie, które niekoniecznie mają znaczenie w niskonapięciowej elektronice cyfrowej, za to są istotnymi aspektami w instalacjach elektrycznych.
