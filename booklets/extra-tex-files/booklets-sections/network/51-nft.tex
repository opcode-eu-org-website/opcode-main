% Copyright (c) 2017-2020 Matematyka dla Ciekawych Świata (http://ciekawi.icm.edu.pl/)
% Copyright (c) 2017-2020 Robert Ryszard Paciorek <rrp@opcode.eu.org>
% 
% MIT License
% 
% Permission is hereby granted, free of charge, to any person obtaining a copy
% of this software and associated documentation files (the "Software"), to deal
% in the Software without restriction, including without limitation the rights
% to use, copy, modify, merge, publish, distribute, sublicense, and/or sell
% copies of the Software, and to permit persons to whom the Software is
% furnished to do so, subject to the following conditions:
% 
% The above copyright notice and this permission notice shall be included in all
% copies or substantial portions of the Software.
% 
% THE SOFTWARE IS PROVIDED "AS IS", WITHOUT WARRANTY OF ANY KIND, EXPRESS OR
% IMPLIED, INCLUDING BUT NOT LIMITED TO THE WARRANTIES OF MERCHANTABILITY,
% FITNESS FOR A PARTICULAR PURPOSE AND NONINFRINGEMENT. IN NO EVENT SHALL THE
% AUTHORS OR COPYRIGHT HOLDERS BE LIABLE FOR ANY CLAIM, DAMAGES OR OTHER
% LIABILITY, WHETHER IN AN ACTION OF CONTRACT, TORT OR OTHERWISE, ARISING FROM,
% OUT OF OR IN CONNECTION WITH THE SOFTWARE OR THE USE OR OTHER DEALINGS IN THE
% SOFTWARE.

\subsubsection{nft (nftables)}

Polecenie \Verb#nft list ruleset# pozwala na wylistowanie wszystkich reguł.

\begin{figure}[h!]\begin{center}\begin{adjustbox}{scale=.9}
\inputFileContent{booklets-sections/network/ilustracje/51-nft.tex}
\end{adjustbox}\end{center}
\caption{Trasa pakietu przez filtry nftables. Wskazano punkty zaczepień dla łańcuchów reguł.}
\end{figure}

\paragraph{Tabele, łańcuchy i reguły}
\begin{itemize}
	\item Reguły (\Verb#rule#) grupowane są w łańcuchy (\Verb#chains#) w ramach których przetwarzane są kolejno (do momentu napotkania reguły kończącej przetwarzanie pakietu).
	\item Łańcuchy grupowane są w tabele (\Verb#table#).
	\item Każda tabela ma określoną rodzinę obsługiwanych adresów (\Verb#family#), mogą to być:
	\begin{itemize}
		\item \Verb#inet#   (osobne lub wspólne reguły dla IPv4 i IPv6),
			\item \Verb#ip#  (reguły tylko dla IPv4),
			\item \Verb#ip6# (reguły tylko dla IPv6),
		\item \Verb#arp#    (reguły dla warstwy L2 przetwarzane przed uruchomieniem procesowania IP),
		\item \Verb#bridge# (reguły przetwarzane dla pakietów przechodzących przez softwerowy bridge),
		\item \Verb#netdev# (reguły przetwarzane w momencie wejścia ruchu na urządzenie sieciowe, urządzenie musi być określone dla łańcucha reguł, może być alternatywą dla \Verb#tc#).
		% https://wiki.nftables.org/wiki-nftables/index.php/Nftables_families
	\end{itemize}
	\item Tabel danej rodziny może być wiele, stosowane będą łańcuchu z wszystkich tych tabel (odpowiednio do ich parametrów).
	\item Tabele dla różnych rodzin mogą mieć taką samą nazwę.
\end{itemize}

\paragraph{Kierowanie ruchu do reguł}
\begin{itemize}
	\item Ruch do łańcucha może być kierowany jawnie przez regułę w innym łańcuchu lub automatycznie w oparciu o parametry danego łańcucha: typ (\Verb#type#), punkt zaczepienia (\Verb#hook#) i priorytet (\Verb#priority#).
	\item Pasujące łańcuchy (o tym samym punkcie zaczepienia) będą przetwarzane kolejno wg priorytetów do momentu napotkania reguły kończącej przetwarzanie pakietu w którymś z tych łańcuchów (lub przetworzenia wszystkich reguł).
	\item Podstawowym typem łańcucha jest \Verb#filter#. Dodatkowo mogą być użyte typy:
	\begin{itemize}
		\item \Verb#nat# –
			translacja adresów sieciowych w oparciu o śledzenie połączenie (\Verb#conntrack#),
			reguły przetwarzają tylko pierwszy pakiet połączenia, pozostałe przetwarza utworzony wpis \Verb#conntrack#,
			typ może być użyty jedynie w łańcuchach tabel związanych z protokołami IP (inet, ip, ip6) z wyjątkiem łańcucha \Verb#forward#
		\item \Verb#route# –
			zaakceptowanie w takim powoduje wyszukanie nowej trasy routingowej,
			typ może być użyty jedynie w łańcuchach wyjściowych (zaczepionych w \Verb#output#) tabel związanych z protokołami IP (inet, ip, ip6)
	\end{itemize}
	\item Dostępne punkty zaczepienia reguł zależą od rodziny:
	\begin{itemize}
		\item dla \Verb#inet#, \Verb#ip#, \Verb#ip6# i \Verb#bridge# są to:
			prerouting
			input
			forward
			output
			postrouting
		\item dla \Verb#arp# są to:
			input
			output
		\item dla \Verb#netdev# są to:
			ingress
	\end{itemize}
	\item Priorytet jest określany swobodnie i może być wartością ujemny lub dodatnią.
		Warto mieć świadomość iż śledzenie pakietów (\Verb#conntrack#) na wejściu ma priorytet -200 (jest robione przed większością innych reguł) a na wyjściu 300 (jest robione po większości innych reguł).
\end{itemize}

\paragraph{Konstrukcja poleceń}
\ 

Polecenia nft można konstruować na kilka sposobów.
%
Możemy dla każdego elementu tworzonego firewalla wywoływać polecenie nft – np poniższe polecenie utworzy tablicę ABC, w niej łańcuch XYZ i w nim jedną dwie reguły (akceptację ruchu wchodzącego interfejsem "eth0" i eth1):

\begin{CodeFrame*}[bash]{}
nft 'add table ip ABC';  nft 'add chain ip ABC XYZ'
nft 'add rule  ip ABC XYZ iifname "eth0" accept'
nft 'add rule  ip ABC XYZ iifname "eth1" accept'
\end{CodeFrame*}

Możemy też kilka poleceń nft podać w jenym uruchomieniu komendy nft, rozdzielając je średnikiem\footnote{
	Zauważ że popzrednio średnik był poza cudzysłowem (aby bash zinterpretował go jako koniec pierwszej komendy),
	a teraz musi być wewnątrz cudzyłowia lub być zabezpieczony w inny sposób (aby bash nie zinterpretował go jako koniec komendy).
}:

\begin{CodeFrame*}[bash]{}
nft 'add table ip ABC; add chain ip ABC XYZ
     add rule ip ABC XYZ iifname "eth0" accept
     add rule ip ABC XYZ iifname "eth1" accept'
\end{CodeFrame*}

Zauważ że w pierwszej linii mamy dwa polecenia nft rozdielone średnikiem, a w kolejne nie są już nim rozdzielane.
Wynika to z tego że w składni nft – podobnie jak w bashu – średnik może zostać zastąpiony nową linią.

W obu tych wypadkach dodając kolejną regułem musimy każdorazowo powtarać określenie jej lokalizacji (typ tablicy, jej nazwę i nazwę łańcucha).
Aby tego uniknąć można zastoswać zapis z blokami wydzielanymi przy pomocy nawiasów klamrowych:

\begin{CodeFrame*}[bash]{}
nft 'table ip ABC { chain XYZ { iifname "eth0" accept; iifname "eth1" accept; }; };'
\end{CodeFrame*}

Zauważ średniki po każdym z wewnętrznych poleceń i po klamerkach kończących bardziej zewnętrzne polecenia.
W przypadku zapisu wieloliniowego, gdyby występował tam znak nowej linii mogłyby być one pominięte.

Wszystkie powyższe zapisy generują identyczny układ reguł firewalla.
Zapis ten można jeszcze bardziej skompresować, ale uzyskamy wtedy też bardziej skompresowaną regułę firewalla:

\begin{CodeFrame*}[bash]{}
nft 'table ip ABC { chain XYZ { iifname {"eth0", "eth1"} accept; }; };'
\end{CodeFrame*}

\paragraph{Pliki konfiguracyjne}

\begin{Verbatim}
#!/usr/sbin/nft -f
flush ruleset

table inet filter {
	chain INPUT {
		type filter hook input priority 0; policy drop;
		
		# lo and established / invalid connections
		iifname "lo" accept
		ct state {established, related} accept
		ct state invalid reject
		
		# icmp, igmp
		meta l4proto icmp icmp type timestamp-request reject
		meta l4proto {icmp, ipv6-icmp, igmp} accept
		
		# ssh
		ip  saddr 10.40.0.0 tcp dport ssh accept
		ip6 saddr {
			2001:db8:0:a17::123,
			2001:db8:0:1313::/64
		} tcp dport ssh accept
		
		# reject all other packets with ICMP error
		reject
	}
}
\end{Verbatim}

\noindent Zauważ że zamiast powtarzać regułę dla każdego adresu:
\begin{Verbatim}
	ip6 saddr 2001:db8:0:a17::123 tcp dport ssh accept
	ip6 saddr 2001:db8:0:1313::/64 tcp dport ssh accept
\end{Verbatim}
możemy podać zbiór parametrów (np. adresów) w klamerkach w ramach jednej reguły (tak jak pokazano powyżej).
Możliwe jest także definiowanie zbiorów adresów (\Verb$set$) i odwoływanie się do nich z użyciem \Verb$@nazwa$.

Podobnie jak przy wpisywaniu poleceń nftables bespośrednio w argumentach polecenia nft, także w plikach konfiguracyjnych możemy zapisywać je zarówno w „notacji klamerkowej” (jak powyżej) jak i ciągu kolejnych poleceń - np.:

\begin{Verbatim}
#!/usr/sbin/nft -f
add table ip filter
add chain ip filter POST {type nat hook postrouting priority 100; policy accept;}
add rule ip filter POST oifname "ens4" ip saddr 10.40.0.0/24 snat to 213.135.50.250
\end{Verbatim}

\noindent Co tworzy maskowanie adresów IP z 10.40.0.0/24 na 213.135.50.250 dla ruchu wychodzącego interfejsem ens4 i jest równoważne:

\begin{Verbatim}
#!/usr/sbin/nft -f
table ip filter {
	chain POST {
		type nat hook postrouting priority 100; policy accept;
		oifname "ens4" ip saddr 10.40.0.0/24 snat to 213.135.50.250
	}
}
\end{Verbatim}

\vspace{5pt}\noindent
Od wersji 0.9.2 nftables możliwe jest też tworzenie wspólnych reguł dla udp i tcp w następujący sposób:
\\\Verb$add rule inet filter INPUT meta l4proto {tcp, udp} th dport domain$

\insertZadanie{booklets-sections/network/zadania-50_60.tex}{wlacz_forward}{}
