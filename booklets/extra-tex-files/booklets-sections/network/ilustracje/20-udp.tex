\ifdefined\inputOnlyContent\else
	\documentclass[tikz]{standalone}
	\usepackage{tikzPackets}
	\InputIfFileExists{booklets-sections/network/preambule.tex}{}{}
	\begin{document}
\fi

\begin{minipage}[t]{31\packetsBitWidth}\hspace{-0.5\packetsBitWidth}\begin{tikzpicture}[semithick]
	\packetsInit
	\tikzstyle{data}=[protocolsField, minimum height = 1.5cm, align=center, fill=black!10]
	\tikzstyle{prev}=[protocolsField, minimum height = 1.2cm, align=center, fill=black!30, node font=\footnotesize]
	
	\packetsPrintBitScale{31}
	
	\ifdefined\hideLowerHeaders\else
		\packetsPutField[prev]{32}{Nagłówek warstwy sprzętowej.}
		\packetsNextLine
		\packetsPutField[prev]{32}{Nagłówek IPv4 (z ew. opcjami) lub IPv6 (z ew. kolejnymi nagłówkami)\\ z polem „Protokół” / „Next Header” o wartości 0x11.}
		\packetsNextLine
	\fi
	
	% 0-31
	\packetsPutField{16}{Port źródłowy}
	\packetsPutField{16}{Port docelowy}
	\packetsEndLine{0}{31}
	
	% 32-63
	\packetsPutField{16}{Długość ($\rm{LEN}$)}
	\packetsPutField{16}{Suma kontrolna\footnote{Uwzględnia też wybrane pola z nagłówków IPv4 lub IPv6.}}
	\packetsEndLine{32}{63}
	
	% 64-...
	\packetsPutField[data]{32}{Dane\\\vspace{3pt}\footnotesize  długość $= (\rm{LEN} - 8) * 8$ bitów}
	\packetsEndLine{64}{\ldots}
	
	% footnotes
	\node[align=left, anchor = north west, minimum width=32\packetsBitWidth, text width=31\packetsBitWidth, xshift=-0.3cm] [] at (\packetsLastNode.south west) {
		% for correct text wrap in footer parent minipage enviromet must use width equal to this „text width”
		\makeatletter\unvbox\@mpfootins\makeatother % print minipage footnotes without „footnoterule”
	};
\end{tikzpicture}\end{minipage}

\ifdefined\inputOnlyContent\else\end{document}\fi
