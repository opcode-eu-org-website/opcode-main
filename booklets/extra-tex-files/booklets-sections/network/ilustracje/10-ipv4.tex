\ifdefined\inputOnlyContent\else
	\documentclass[tikz]{standalone}
	\usepackage{tikzPackets}
	\InputIfFileExists{booklets-sections/network/preambule.tex}{}{}
	\begin{document}
\fi

\begin{minipage}[t]{31\packetsBitWidth}\hspace{-0.5\packetsBitWidth}\begin{tikzpicture}[semithick]
	\packetsInit
	\tikzstyle{opts}=[protocolsField, minimum height = 1.5cm]
	\tikzstyle{data}=[protocolsField, minimum height = 1.5cm, align=center, fill=black!10]
	\tikzstyle{prev}=[protocolsField, minimum height = 1.2cm, align=center, fill=black!30, minimum width=32\packetsBitWidth, node font=\footnotesize]
	
	\packetsPrintBitScale{31}
	
	\ifdefined\hideLowerHeaders\else
		\packetsPutField[prev]{32}{Nagłówek warstwy sprzętowej.}
		\packetsNextLine
	\fi
	
	% 0-31
	\packetsPutField{4}{Wersja}
	\packetsPutField{4}{IHL};
	\packetsPutField{8}{QoS\footnote{
		Dokładniej dwa pola: DSCP – priorytetyzacja ruchu DiffServ i ECN – informacja o przeciążeniu
	}}
	\packetsPutField{16}{Długość całkowita ($\rm{LEN}$)}
	\packetsEndLine{0}{31}
	
	% 32-63
	\packetsPutField{32}{Fragmentacja\footnote{
		Obsługa mechanizmu podziału pakietu IP na mniejsze części (gdy nie mieści się do ramki warstwy drugiej).
		Dokładniej są to trzy pola: Identification – numer identyfikujący zbiór fragmentów,
		Flagi – informujące o zakazie lub dokonaniu fragmentacji, oraz Offset – przesunięcie fragmentu względem całości pakietu.
	}}
	\packetsEndLine{32}{63}
	
	% 64-95
	\packetsPutField{8}{Time To Live\footnote{
		Licznik zmiejszany gdy pakiet przechodzi prze router, służy eleiminacji zapętleń w sieci IP.
	}}
	\packetsPutField{8}{Protokół\footnote{
		Numer identyfikujący protokół warswy wyższej znajdujący się w polu danych.
	}}
	\packetsPutField{16}{Suma kontrolna nagłówka}
	\packetsEndLine{64}{95}
	
	% 96-127
	\packetsPutField{32}{Adres źródłowy (nadawcy)}
	\packetsEndLine{96}{127}
	
	% 128-159
	\packetsPutField{32}{Adres docelowy (odbiorcy)}
	\packetsEndLine{128}{159}
	
	% 160-...
	\packetsPutField[opts]{32}{Opcje\\\vspace{3pt}\footnotesize długość $= (\rm{IHL} - 5) \cdot 32$ bitów, mogą nie występować (gdy IHL=5)}
	\packetsEndLine{160}{\ldots}
	\packetsPutField[data]{32}{Dane\\\vspace{3pt}\footnotesize  długość $= \rm{LEN} - (\rm{IHL} \cdot 4) \cdot 8$ bitów}
	\packetsEndLine{$\geq160$}{\ldots}
	
	% footnotes
	\node[align=left, anchor = north west, minimum width=32\packetsBitWidth, text width=31\packetsBitWidth, xshift=-0.3cm] [] at (\packetsLastNode.south west) {
		% for correct text wrap in footer parent minipage enviromet must use width equal to this „text width”
		\makeatletter\unvbox\@mpfootins\makeatother % print minipage footnotes without „footnoterule”
	};
\end{tikzpicture}\end{minipage}

\ifdefined\inputOnlyContent\else\end{document}\fi
