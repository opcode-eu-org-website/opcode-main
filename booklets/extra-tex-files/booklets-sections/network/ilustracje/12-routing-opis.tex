\noindent Przykłady:
\begin{easylist}[itemize]
& Pakiet wysłany z HA1 do \Verb$10.0.3.13$:
	&& zostanie przesłany przez \textit{Router A}, który ma wpis \Verb$10.0.3.0/24 via 10.9.1.1$, do \textit{Router C}
	&& z \textit{Router C}, który ma wpis \Verb$10.0.3.0/24 via 10.9.3.1$, zostanie przekazany do \textit{Router D}
	&& \textit{Router D} przekaże go do odpowiedniego hosta (HD1) poprzez interfejs eth1

& Pakiet wysłany z HD1 do \Verb$10.0.4.33$:
	&& zostanie przesłany przez \textit{Router D}, który ma wpis \Verb$10.0.4.1/23 via 10.9.3.2$, do \textit{Router C}
	&& z \textit{Router C}, który ma wpis \Verb$10.0.4.0/24 via 10.9.1.2$, zostanie przekazany do \textit{Router A}
	&& \textit{Router A} przekaże go do odpowiedniego hosta (HA1) poprzez interfejs eth1
	
& Pakiet wysłany z HD1 do \Verb$10.0.5.13$:
	&& zostanie przesłany przez \textit{Router D}, który ma wpis \Verb$10.0.4.1/23 via 10.9.3.2$, do \textit{Router C}
	&& z \textit{Router C}, który ma wpis \Verb$10.0.5.0/24 via 10.9.3.2$, zostanie przekazany do \textit{Router B}
	&& \textit{Router B} przekaże go do odpowiedniego hosta (HB1) poprzez interfejs eth1
\end{easylist}

\noindent Zwróć uwagę że:
\begin{easylist}[itemize]
	& aby możliwa była komunikacja trasa routingowa musi być:
		&& skonfigurowana w obie strony (tak jak ma to miejsce między HA1 i HD1)
			&&& z tego powodu HE1 nie może komunikować się np. z HD1
		&& skonfigurowana zarówno na routerach brzegowych (takich jak A i D), jak i wszystkich routerach pośrednich (takich jak C)
	& \textit{Router D} posiada zagregowany wpis w tablicy routingu obejmujący zarówno sieci podłączone do \textit{Router A}, jak i do \textit{Router B},
	  ale np. wpisy w \textit{Router A} nie są i nie mogą być zagregowane
\end{easylist}
