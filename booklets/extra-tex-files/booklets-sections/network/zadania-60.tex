% Copyright (c) 2017-2020 Matematyka dla Ciekawych Świata (http://ciekawi.icm.edu.pl/)
% Copyright (c) 2017-2020 Robert Ryszard Paciorek <rrp@opcode.eu.org>
% 
% MIT License
% 
% Permission is hereby granted, free of charge, to any person obtaining a copy
% of this software and associated documentation files (the "Software"), to deal
% in the Software without restriction, including without limitation the rights
% to use, copy, modify, merge, publish, distribute, sublicense, and/or sell
% copies of the Software, and to permit persons to whom the Software is
% furnished to do so, subject to the following conditions:
% 
% The above copyright notice and this permission notice shall be included in all
% copies or substantial portions of the Software.
% 
% THE SOFTWARE IS PROVIDED "AS IS", WITHOUT WARRANTY OF ANY KIND, EXPRESS OR
% IMPLIED, INCLUDING BUT NOT LIMITED TO THE WARRANTIES OF MERCHANTABILITY,
% FITNESS FOR A PARTICULAR PURPOSE AND NONINFRINGEMENT. IN NO EVENT SHALL THE
% AUTHORS OR COPYRIGHT HOLDERS BE LIABLE FOR ANY CLAIM, DAMAGES OR OTHER
% LIABILITY, WHETHER IN AN ACTION OF CONTRACT, TORT OR OTHERWISE, ARISING FROM,
% OUT OF OR IN CONNECTION WITH THE SOFTWARE OR THE USE OR OTHER DEALINGS IN THE
% SOFTWARE.

\IfStrEq{\dbEntryID}{}{
	\insertZadanie{\currfilepath}{ustaw_adres}{}
	\insertZadanie{\currfilepath}{ustaw_route}{}
	\insertZadanie{\currfilepath}{wlacz_forward}{}
	\insertZadanie{\currfilepath}{serwer_trojkat}{}
	\insertZadanie{\currfilepath}{udp_echo}{}
}


\dbEntryBegin{ustaw_adres}\if1\dbEntryCheckResults
Napisz polecenie które ustawi adres ip \Verb#172.33.13.113# (maska sieci to \Verb#255.255.255.0#) na interfejsie \Verb#eth5#.
\fi

\dbEntryBegin{ustaw_route}\if1\dbEntryCheckResults
Napisz polecenie które ustawi trasę routingową do sieci \Verb#10.13.0.0/16# przez bramkę o adresie ip \Verb#172.33.13.13#.
\fi

\dbEntryBegin{wlacz_forward}\if1\dbEntryCheckResults
Napisz polecenia które włączą przekazywanie pakietów (routing) pomiędzy interfejsami \Verb#eth3# i \Verb#eth4#, ale nie zezwolą na przekazywanie pakietów innymi interfejsami (w tym pakietów inny interfejs $\leftrightarrow$ \Verb#eth3# / \Verb#eth4#).
\\
\textit{Wskazówka: skorzystaj z reguł filtracji pakietów}
\fi

\dbEntryBegin{serwer_trojkat}\if1\dbEntryCheckResults
Napisz (w Pythonie lub C/C++) serwer TCP, który oczekuje że klient wyśle mu liczbę, w odpowiedzi na którą serwer odeśle do tego klienta trójkąt z gwiazdek odpowiedniej wielkości.
Na przykład dla żądania klienta w postaci "3" powinien to być:
\begin{Verbatim}
*
**
***
\end{Verbatim}
\fi

\dbEntryBegin{udp_echo}\if1\dbEntryCheckResults
Powyżej znajdują się przykładowe kody wysyłający dane po UDP ("klient UDP") i odbierający dane po UDP ("serwer UDP")
oraz kod serwera usługi "echo" (odsyłającej odebrane dane do nadawcy) w wariancie TCP, którą omawialiśmy na wykładzie.

W oparciu o te informacje napisz (w Pythonie lub C/C++) program realizujący funkcję serwera echo z użyciem UDP.
\fi
