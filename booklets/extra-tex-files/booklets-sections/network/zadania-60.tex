% Copyright (c) 2017-2020 Matematyka dla Ciekawych Świata (http://ciekawi.icm.edu.pl/)
% Copyright (c) 2017-2020 Robert Ryszard Paciorek <rrp@opcode.eu.org>
% 
% MIT License
% 
% Permission is hereby granted, free of charge, to any person obtaining a copy
% of this software and associated documentation files (the "Software"), to deal
% in the Software without restriction, including without limitation the rights
% to use, copy, modify, merge, publish, distribute, sublicense, and/or sell
% copies of the Software, and to permit persons to whom the Software is
% furnished to do so, subject to the following conditions:
% 
% The above copyright notice and this permission notice shall be included in all
% copies or substantial portions of the Software.
% 
% THE SOFTWARE IS PROVIDED "AS IS", WITHOUT WARRANTY OF ANY KIND, EXPRESS OR
% IMPLIED, INCLUDING BUT NOT LIMITED TO THE WARRANTIES OF MERCHANTABILITY,
% FITNESS FOR A PARTICULAR PURPOSE AND NONINFRINGEMENT. IN NO EVENT SHALL THE
% AUTHORS OR COPYRIGHT HOLDERS BE LIABLE FOR ANY CLAIM, DAMAGES OR OTHER
% LIABILITY, WHETHER IN AN ACTION OF CONTRACT, TORT OR OTHERWISE, ARISING FROM,
% OUT OF OR IN CONNECTION WITH THE SOFTWARE OR THE USE OR OTHER DEALINGS IN THE
% SOFTWARE.

\IfStrEq{\dbEntryID}{}{
	\insertZadanie{\currfilepath}{serwer_trojkat}{}
	\insertZadanie{\currfilepath}{udp_echo}{}
	\insertZadanie{\currfilepath}{dos_echo1}{}
	\insertZadanie{\currfilepath}{dos_echo2}{}
}


\dbEntryBegin{serwer_trojkat}\if1\dbEntryCheckResults
Napisz serwer TCP, który oczekuje że klient wyśle mu liczbę, w odpowiedzi na którą serwer odeśle do tego klienta trójkąt z gwiazdek odpowiedniej wielkości.
Na przykład dla żądania klienta w postaci "3" powinien to być:
\begin{Verbatim}
*
**
***
\end{Verbatim}
\fi

\dbEntryBegin{udp_echo}\if1\dbEntryCheckResults
Powyżej znajdują się przykładowe kody wysyłający dane po UDP ("klient UDP") i odbierający dane po UDP ("serwer UDP")
oraz kod serwera usługi "echo" (odsyłającej odebrane dane do nadawcy) w wariancie TCP, którą omawialiśmy na wykładzie.

W oparciu o te informacje napisz program realizujący funkcję serwera echo z użyciem UDP.

\emph{Jeżeli nie lubisz Pythona program może być w C lub C++.}
\fi

\teacher{Zadania \ref{dos_echo1} i \ref{dos_echo2} mozna zrobić jako wspólny dla całej grupy pokaz.}

\dbEntryBegin{dos_echo1}\if1\dbEntryCheckResults
Uruchom dwie instancje serwera echo korzystającego z protokołu UDP.

Zastanów się co by się stało jeżeli jeden z tych serwerów dostałby pakiet pochodzący od drugiego z nich?

Korzystając z pakietu \textit{scapy} oraz posiadając prawa root'a możemy przy pomocy Pythona wysyłać dowolnie spreparowane pakiety IP:

\begin{CodeFrame*}[python]{}
from scapy.all import IP, IPv6, UDP, send

send(IPv6(src=sIP, dst=dIP) / UDP(sport=sPort, dport=dPort) / "ABC ... XYZ")
# powyższa funkcja utworzy (a następnie wyśle):
#  → pakiet IPv6 od sIP do dIP
#    (adresy podajemy jako napisy),
#  → w którym jest pakiet UDP z portem źródłowym sPort i docelowym dport
#    (porty podajemy jako wartości numeryczne)
#  → w którym są dane "ABC ... XYZ"

# jeżeli zamiast IPv6() użyjemy IP() będziemy używać pakietu IPv4

# możemy też zaimportować inne funkcjonalności z modułu scapy
# (np. ICMP, TCP, ...) i używać ich do budowy naszych pakietów
\end{CodeFrame*}

Modyfikując powyższy kod spróbuj wysłać sfałszowany pakiet adresowany do jednego z serwerów, który jako adres nadawcy ma podany drugi z serwerów.

\textit{Scapy nie jest elementem biblioteki standardowej pythona – konieczne może być zainstalowanie pakietu \texttt{python3-scapy} albo zainstalowanie go poprzez managera modułów pythonowoych „pip”: \texttt{pip3 install scapy-python3}.}
\fi

\dbEntryBegin{dos_echo2}\if1\dbEntryCheckResults
Zobacz co się stanie jeżeli w sfałszowanym pakiecie podasz ten sam serwer jako nadawcę i odbiorcę.

Usługa UDP-echo była kiedyś powszechnie stosowaną usługą diagnostyczną umożliwiająca testowanie połączenia sieciowego. Do tej pory ma nawet przyznany standardowy numer portu (7).
Jak myślisz dlaczego usługa UDP-echo nie jest już powszechnie dostępną na każdym komputerze podłączonym do Internetu?
\fi

