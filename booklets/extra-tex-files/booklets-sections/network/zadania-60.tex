% Copyright (c) 2017-2020 Matematyka dla Ciekawych Świata (http://ciekawi.icm.edu.pl/)
% Copyright (c) 2017-2020 Robert Ryszard Paciorek <rrp@opcode.eu.org>
% 
% MIT License
% 
% Permission is hereby granted, free of charge, to any person obtaining a copy
% of this software and associated documentation files (the "Software"), to deal
% in the Software without restriction, including without limitation the rights
% to use, copy, modify, merge, publish, distribute, sublicense, and/or sell
% copies of the Software, and to permit persons to whom the Software is
% furnished to do so, subject to the following conditions:
% 
% The above copyright notice and this permission notice shall be included in all
% copies or substantial portions of the Software.
% 
% THE SOFTWARE IS PROVIDED "AS IS", WITHOUT WARRANTY OF ANY KIND, EXPRESS OR
% IMPLIED, INCLUDING BUT NOT LIMITED TO THE WARRANTIES OF MERCHANTABILITY,
% FITNESS FOR A PARTICULAR PURPOSE AND NONINFRINGEMENT. IN NO EVENT SHALL THE
% AUTHORS OR COPYRIGHT HOLDERS BE LIABLE FOR ANY CLAIM, DAMAGES OR OTHER
% LIABILITY, WHETHER IN AN ACTION OF CONTRACT, TORT OR OTHERWISE, ARISING FROM,
% OUT OF OR IN CONNECTION WITH THE SOFTWARE OR THE USE OR OTHER DEALINGS IN THE
% SOFTWARE.

\IfStrEq{\dbEntryID}{}{
	\insertZadanie{\currfilepath}{ustaw_adres}{}
	\insertZadanie{\currfilepath}{ustaw_route}{}
	\insertZadanie{\currfilepath}{wlacz_forward}{}
	\insertZadanie{\currfilepath}{serwer_trojkat}{}
	\insertZadanie{\currfilepath}{udp_echo}{}
}

\IfStrEq{\dbEntryID}{rozwiazania}{
	\insertRozwiazanie{\currfilepath}{ustaw_adres}{}
	\insertRozwiazanie{\currfilepath}{ustaw_route}{}
	\insertRozwiazanie{\currfilepath}{wlacz_forward}{}
	\insertRozwiazanie{\currfilepath}{serwer_trojkat}{}
	\insertRozwiazanie{\currfilepath}{udp_echo}{}
}


\dbEntryBegin{ustaw_adres}\if1\dbEntryCheckResults
Napisz polecenie które ustawi adres ip \Verb#172.33.13.113# (maska sieci to \Verb#255.255.255.0#) na interfejsie \Verb#eth5#.
\fi

\dbEntryBegin{ustaw_adres-rozwiazanie}\if1\dbEntryCheckResults
\begin{Verbatim}
ip addr add 172.33.13.113/24 dev eth5
\end{Verbatim}
\fi


\dbEntryBegin{ustaw_route}\if1\dbEntryCheckResults
Napisz polecenie które ustawi trasę routingową do sieci \Verb#10.13.0.0/16# przez bramkę o adresie ip \Verb#172.33.13.13#.
\fi

\dbEntryBegin{ustaw_route-rozwiazanie}\if1\dbEntryCheckResults
\begin{Verbatim}
ip route add 10.13.0.0/16 via 172.33.13.13
\end{Verbatim}
\fi


\dbEntryBegin{wlacz_forward}\if1\dbEntryCheckResults
Napisz polecenia które włączą przekazywanie pakietów (routing) pomiędzy interfejsami \Verb#eth3# i \Verb#eth4#, ale nie zezwolą na przekazywanie pakietów innymi interfejsami (w tym pakietów inny interfejs $\leftrightarrow$ \Verb#eth3# / \Verb#eth4#).
\\
\textit{Wskazówka: skorzystaj z reguł filtracji pakietów}
\fi

\dbEntryBegin{wlacz_forward-rozwiazanie}\if1\dbEntryCheckResults
\begin{Verbatim}
for f in /proc/sys/net/ipv*/conf/*/forwarding; do echo 1 > $f; done

nft add table ip filter
nft add chain ip filter FORWARD '{ type filter hook forward priority 0; }'
nft add rule  ip filter FORWARD iifname "eth3" oifname "eth4" accept
nft add rule  ip filter FORWARD iifname "eth4" oifname "eth3" accept
nft add rule  ip filter FORWARD reject
\end{Verbatim}
\fi


\dbEntryBegin{serwer_trojkat}\if1\dbEntryCheckResults
Napisz (w Pythonie lub C/C++) serwer TCP, który oczekuje że klient wyśle mu liczbę, w odpowiedzi na którą serwer odeśle do tego klienta trójkąt z gwiazdek odpowiedniej wielkości.
Na przykład dla żądania klienta w postaci "3" powinien to być:
\begin{Verbatim}
*
**
***
\end{Verbatim}
\fi

\dbEntryBegin{serwer_trojkat-rozwiazanie}\if1\dbEntryCheckResults
\begin{Verbatim}
import socket, select, signal, sys, os

MAX_CHILD = 5
QUERY_SIZE = 3
TIMEOUT = 13
BUF_SIZE = 4096

if len(sys.argv) != 2:
    print("USAGE: " + sys.argv[0] + " listenPort", file=sys.stderr)
    exit(1);

# obsługa sygnału o zakończeniu potomka
childNum = 0
def onChildEnd(s, f):
    print("odebrano sygnał o śmierci potomka")
    global childNum
    childNum -= 1
    os.waitpid(-1, os.WNOHANG);
signal.signal(signal.SIGCHLD, onChildEnd)

# utworzenie gniazd sieciowych ... SOCK_STREAM oznacza TCP
sfd_v4 = socket.socket(socket.AF_INET,  socket.SOCK_STREAM)
sfd_v6 = socket.socket(socket.AF_INET6, socket.SOCK_STREAM)

# ustawienie opcji gniazda ... IPV6_V6ONLY=1 wyłącza korzystanie
# z tego samego socketu dla IPv4 i IPv6
sfd_v6.setsockopt(socket.IPPROTO_IPV6, socket.IPV6_V6ONLY, 1)

# przypisanie adresów ...
# '0.0.0.0' oznacza dowolny adres IPv4 (czyli to samo co INADDR_ANY)
# '::' oznacza dowolny adres IPv6 (czyli to samo co in6addr_any)
sfd_v4.bind(('0.0.0.0', int(sys.argv[1])))
sfd_v6.bind(('::',      int(sys.argv[1])))

# określenie gniazd jako używanych do odbioru połączeń przychodzących
# (długość kolejki połączeń ustawiona na wartość QUERY_SIZE)
sfd_v4.listen(QUERY_SIZE)
sfd_v6.listen(QUERY_SIZE)

# czekanie na połączenia z użyciem select() w nieskończonej pętli
while True:
    sfd, _, _ = select.select([sfd_v4, sfd_v6], [], [])
    for fd in sfd:
        #  odebranie połączenia
        sfd_c, sAddr = fd.accept()
        
        # weryfikacja ilości potomków
        if childNum >= MAX_CHILD:
            print("za dużo potomków - odrzucam połączenie od:", sAddr);
            sfd_c.send("Internal Server Error\r\n".encode())
            sfd_c.close()
            break
        
        # aby móc obsługiwać wiele połączeń rozgałęziamy proces
        pid = os.fork()
        if pid > 0:
            # rodzic - zwiększamy licznik potomków
            childNum += 1
        else:
            # potomek - obsługa danego połączenia
            print("połączenie od:", sAddr)
            while True:
                # czekanie na dane z timeout'em
                # aby zabezpieczyć się przed atakiem DoS
                rd, _, _ = select.select([sfd_c], [], [], TIMEOUT)
                if sfd_c in rd:
                    data = sfd_c.recv(BUF_SIZE)
                    if data:
                        print("odebrano od", sAddr, ":", data.decode());
\end{Verbatim}
\begin{minted}[frame=none]{python}
                        # zamist odsyłać odebrane dane poprzez sfd_c.send(data)
                        try:
                            liczba = int(data.decode())
                        except:
                            liczba = 0
                        if liczba < 2 or liczba > 10:
                            sfd_c.send("Błędne polecenie – podaj liczbę >=2 i <=10.\n".encode())
                        else:
                            trojkat = ""
                            for i in range(1, liczba+1):
                                trojkat += "*"*i+"\n"
                            sfd_c.send(trojkat.encode())
                        # generujemy na ich podstawie trójkąt i go odsyłamy używając sfd_c.send
\end{minted}
\begin{Verbatim}
                    else:
                        print("koniec połączenia od:", sAddr)
                        break
                else:
                    print("timeout połączenia od:", sAddr)
                    break
            # zamykanie połączenia
            sfd_c.shutdown(socket.SHUT_RDWR)
            sfd_c.close()
            sys.exit()
\end{Verbatim}
Kolorowaniem kodu wyróżniono zmianę w stosunku co do przykładowego kodu ze skryptu.
\fi


\dbEntryBegin{udp_echo}\if1\dbEntryCheckResults
Powyżej znajdują się przykładowe kody wysyłający dane po UDP ("klient UDP") i odbierający dane po UDP ("serwer UDP")
oraz kod serwera usługi "echo" (odsyłającej odebrane dane do nadawcy) w wariancie TCP, którą omawialiśmy na wykładzie.

W oparciu o te informacje napisz (w Pythonie lub C/C++) program realizujący funkcję serwera echo z użyciem UDP.
\fi

\dbEntryBegin{udp_echo-rozwiazanie}\if1\dbEntryCheckResults
\begin{minted}[frame=none]{python}
import socket, sys

if len(sys.argv) != 2:
    print("USAGE: " + sys.argv[0] + " listenPort", file=sys.stderr)
    exit(1);

sfd = socket.socket(socket.AF_INET6, socket.SOCK_DGRAM)
sfd.setsockopt(socket.IPPROTO_IPV6, socket.IPV6_V6ONLY, 0)
sfd.bind(('::', int(sys.argv[1])))

while True:
    data, sAddr, = sfd.recvfrom(4096)
    print("odebrano od", sAddr, ":", data.decode());
    sfd.sendto(data, sAddr)
\end{minted}
\fi

