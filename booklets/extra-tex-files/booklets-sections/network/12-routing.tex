% Copyright (c) 2017-2020 Matematyka dla Ciekawych Świata (http://ciekawi.icm.edu.pl/)
% Copyright (c) 2017-2020 Robert Ryszard Paciorek <rrp@opcode.eu.org>
% 
% MIT License
% 
% Permission is hereby granted, free of charge, to any person obtaining a copy
% of this software and associated documentation files (the "Software"), to deal
% in the Software without restriction, including without limitation the rights
% to use, copy, modify, merge, publish, distribute, sublicense, and/or sell
% copies of the Software, and to permit persons to whom the Software is
% furnished to do so, subject to the following conditions:
% 
% The above copyright notice and this permission notice shall be included in all
% copies or substantial portions of the Software.
% 
% THE SOFTWARE IS PROVIDED "AS IS", WITHOUT WARRANTY OF ANY KIND, EXPRESS OR
% IMPLIED, INCLUDING BUT NOT LIMITED TO THE WARRANTIES OF MERCHANTABILITY,
% FITNESS FOR A PARTICULAR PURPOSE AND NONINFRINGEMENT. IN NO EVENT SHALL THE
% AUTHORS OR COPYRIGHT HOLDERS BE LIABLE FOR ANY CLAIM, DAMAGES OR OTHER
% LIABILITY, WHETHER IN AN ACTION OF CONTRACT, TORT OR OTHERWISE, ARISING FROM,
% OUT OF OR IN CONNECTION WITH THE SOFTWARE OR THE USE OR OTHER DEALINGS IN THE
% SOFTWARE.

% BEGIN: routing IP
\subsection{Routing}

\begin{teacherOnly}
	\begin{easylist}[itemize]
		& pokazać tablicę routingu jako wynik polecenia ip r
		& omówić przeszukiwanie - sprawdzanie pasujących sieci, próba dopasowania najmniejszej, znaczenie 0.0.0.0/0 (::/0)
		& pokazać i omówić ping i traceroute, wspomnieć o TTL
	\end{easylist}
\end{teacherOnly}

Router kieruje każdy z pakietów do kolejnego routera lub bezpośrednio do hosta docelowego na podstawie jego adresu docelowego i tablicy routingu. Tablica taka zawiera adresy sieci wraz z adresami następnych routerów do nich prowadzących bądź wskazaniem lokalnego interfejsu sieciowego poprzez który powinny być osiągalne hosty z danej sieci. W tym celu korzysta z sprawdzania przynależności adresu do sieci, w celu ustalenia adresu następnego routera i/lub interfejsu sieciowego na który ma zostać przekazany pakiet.

Tablica przeglądana jest od wpisów najbardziej precyzyjnych, czyli z największym prefixem do wpisów najbardziej ogólnych (ostatnim wpisem jest na ogół trasa domyślna czyli sieć ::/0 dla IPv6 lub 0.0.0.0/0 dla IPv4). Dzięki czemu jeżeli kilka wpisów (sieci) z tablicy routingu pasuje do adresu docelowego z nagłówka pakietu, wybierany jest wpis najbardziej precyzyjny (o najdłuższym prefixie), a pasująca do każdego adresu trasa domyślna wybierana jest tylko gdy nie ma żadnej lepszej. Może się zdarzyć że kilka wpisów (nawet z tą samą maską) pasuje do adresu docelowego hosta, w takiej sytuacji do wyboru ścieżki używane są inne dane z tablicy routingu (takie jak metryka).

Tablice routingu mogą zawierać wpisy dodawane statycznie (wpisane do konfiguracji danego urządzenia), jak też wpisy dodawane dynamicznie w oparciu o protokołu wymiany informacji routingowych (protokoły routingu) takie jak: IGRP, OSPF, BGP. Protokoły routingu dynamicznego mogą być wykorzystywane m.in. do rozkładania obciążenia na różne łącza, zapewnienia redundancji łącz, blokowania ataków (D)DoS.

Także każdy z hostów ma tablice routingu, typowo składa się z dwóch pozycji – trasy do sieci lokalnej (tej sieci z której adres posiada dany host) wskazującej bezpośrednio na urządzenie sieciowe oraz trasy domyślnej wskazującej na router zapewniający dostęp do innych sieci, nazywany bramką (gateway). Jeżeli router nie posiada adresu w tej samej sieci co host konieczna jest dodatkowa trasa wskazująca poprzez jakie urządzenie dostępny jest router domyślny.

Oprócz opisanego powyżej routingu unicastowego (kierowania do jednego odbiorcy) realizowane są także transmisje:
\begin{itemize}
	\item \emph{anycast} – do dowolnego / najbliższego hosta o danym adresie; zasadniczo jest to transmisja unicast, tyle że adres docelowy nie jest unikalny w skali globalnej a różne routery kieruje te pakiety do różnych hostów docelowych (typowo wybierając najbliższy taki host)
	\item \emph{multicast} – do grupy hostów, w tym wypadku (multicastowy) adres IP identyfikuje "kanał nadawczy" a nie unikalny host docelowy
	\item \emph{broadcast} – do wszystkich hostów (w ramach danej sieci – nie są routowne), transmisje rozgłoszeniowe można traktować jako szczególny przypadek transmisji multicastowych w których grupa multicastowa obejmuje wszystkie hosty (można je zastąpić takimi transmisjami multicastowymi)
\end{itemize}
% END: routing IP
