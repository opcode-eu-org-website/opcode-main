% Copyright (c) 2017-2020 Matematyka dla Ciekawych Świata (http://ciekawi.icm.edu.pl/)
% Copyright (c) 2017-2020 Robert Ryszard Paciorek <rrp@opcode.eu.org>
% 
% MIT License
% 
% Permission is hereby granted, free of charge, to any person obtaining a copy
% of this software and associated documentation files (the "Software"), to deal
% in the Software without restriction, including without limitation the rights
% to use, copy, modify, merge, publish, distribute, sublicense, and/or sell
% copies of the Software, and to permit persons to whom the Software is
% furnished to do so, subject to the following conditions:
% 
% The above copyright notice and this permission notice shall be included in all
% copies or substantial portions of the Software.
% 
% THE SOFTWARE IS PROVIDED "AS IS", WITHOUT WARRANTY OF ANY KIND, EXPRESS OR
% IMPLIED, INCLUDING BUT NOT LIMITED TO THE WARRANTIES OF MERCHANTABILITY,
% FITNESS FOR A PARTICULAR PURPOSE AND NONINFRINGEMENT. IN NO EVENT SHALL THE
% AUTHORS OR COPYRIGHT HOLDERS BE LIABLE FOR ANY CLAIM, DAMAGES OR OTHER
% LIABILITY, WHETHER IN AN ACTION OF CONTRACT, TORT OR OTHERWISE, ARISING FROM,
% OUT OF OR IN CONNECTION WITH THE SOFTWARE OR THE USE OR OTHER DEALINGS IN THE
% SOFTWARE.

\subsection{Sieci bezprzewodowe}

Linux może być także klientem sieci bezprzewodowych WiFi - do ich konfiguracji przydać mogą się następujące narzędzia:
\begin{itemize}
	\item \Verb#iwconfig# –
		podstawowe operacje na interfejsie bezprzewodowym
	\item \Verb#iwlist# –
		listowanie "widocznych" sieci i informacji o nich
	\item \Verb#wpa_supplicant# –
		łączenie się z sieciami zabezpieczonymi WPA
\end{itemize}

\vspace{4pt}\noindent
Linux może być nie tylko klientem takich sieci, ale pełnić w nich też funkcję access pointu, do tego zadania pomocne mogą być narzędzia takie jak:
\begin{itemize}
	\item \Verb#hostapd# –
		uruchomienie access pointa na bazie linuxa i karty wifi
	\item \Verb#dnsmasq# –
		serwer DHCP oraz serwer maskujący DNS \\ (nie tylko dla sieci bezprzewodowych, ale często używany z hostapd)
\end{itemize}
oraz oczywiście odpowiednia konfiguracja routingu i przekazywania pakietów.
