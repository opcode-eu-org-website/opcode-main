% Copyright (c) 2017-2020 Matematyka dla Ciekawych Świata (http://ciekawi.icm.edu.pl/)
% Copyright (c) 2017-2020 Robert Ryszard Paciorek <rrp@opcode.eu.org>
% 
% MIT License
% 
% Permission is hereby granted, free of charge, to any person obtaining a copy
% of this software and associated documentation files (the "Software"), to deal
% in the Software without restriction, including without limitation the rights
% to use, copy, modify, merge, publish, distribute, sublicense, and/or sell
% copies of the Software, and to permit persons to whom the Software is
% furnished to do so, subject to the following conditions:
% 
% The above copyright notice and this permission notice shall be included in all
% copies or substantial portions of the Software.
% 
% THE SOFTWARE IS PROVIDED "AS IS", WITHOUT WARRANTY OF ANY KIND, EXPRESS OR
% IMPLIED, INCLUDING BUT NOT LIMITED TO THE WARRANTIES OF MERCHANTABILITY,
% FITNESS FOR A PARTICULAR PURPOSE AND NONINFRINGEMENT. IN NO EVENT SHALL THE
% AUTHORS OR COPYRIGHT HOLDERS BE LIABLE FOR ANY CLAIM, DAMAGES OR OTHER
% LIABILITY, WHETHER IN AN ACTION OF CONTRACT, TORT OR OTHERWISE, ARISING FROM,
% OUT OF OR IN CONNECTION WITH THE SOFTWARE OR THE USE OR OTHER DEALINGS IN THE
% SOFTWARE.

% BEGIN: TCP / UDP
\section{Komunikacja TCP/IP}

\begin{teacherOnly}
	\begin{easylist}[itemize]
		& protokoły warstwy transportowej - TCP i UDP
		&& czym się różnią?
		&& numery portów – identyfikacja usługi / procesu na hoście
	\end{easylist}
\end{teacherOnly}

W oparciu o protokół IP działają protokoły warstwy transportowej takie jak UDP, TCP, czy też (mniej znany protokół dedykowany dla  strumieniowych transmisji czasu rzeczywistego) SCTP.

Jednym z zadań tych protokołów jest identyfikowanie usługi (procesu) w ramach systemu posiadającego dany adres IP, do którego mają trafić dane.
W tym celu zarówno UDP jak i TCP na każdym z hostów wyróżniają numeryczny identyfikator dla aplikacji/procesu/usługi będącego odbiorcą czy też nadawcą informacji zwany numerem portu.

Najprostszym protokołem warstwy transmisji wydaje się być UDP, protokół ten umożliwia przesłanie informacji pomiędzy dwoma hostami IP i nie kontroluje on tego czy została ona przesłana poprawnie.
Natomiast TCP, w odróżnieniu od UDP, kontroluje to czy przesłana informacja dotarła do adresata i nie została uszkodzona, a w przypadku problemów informacja wysyłana jest ponownie. TCP w związku z tym w przeciwieństwie do UDP musi otworzyć połączenie i wykorzystywać je do kontroli poprawności przesłania informacji, wymaga zatem przesłania większej liczby pakietów (co może prowadzić do pewnych opóźnień itp).
W związku z tym TCP używany jest tam gdzie konieczna jest kontrola poprawności transmisji (oraz ponowne wysłanie zgubionego pakietu), UDP tam gdzie nie jest to potrzebne (a liczy się czas).

\inputSideBySideAsFigure
	{Datagram UDP}{booklets-sections/network/ilustracje/20-udp.tex}
	{Pakiet TCP}{booklets-sections/network/ilustracje/20-tcp.tex}
	{Struktura pakietów UDP i TCP}{ilustracja_pakiety_udp_tcp}
% END: TCP / UDP

% BEGIN: Polularne usługi
\subsection{Popularne usługi}

W ramach sieci mogą być realizowane różne usługi w oparciu o różne protokoły warstwy aplikacyjnej. Standardowe usługi posiadają zdefiniowane domyślne adresy portów dla swoich protokołów. Wśród usług i protokołów sieciowych należy wymienić przynajmniej:
\begin{itemize}
	\item DNS (Domain Name System) - odpowiedzialny za system mapujący nazwy alfanumeryczne hostów na adresy IP.
	\item mechanizmy auto konfiguracji hostów - DHCP, rozgłaszanie informacji routingowej poprzez ICMPv6 (protokół warstwy 3)
	\item WWW - udostępnianie treści z użyciem protokołu HTTP
	\item pocztę elektroniczną - przesyłanie wiadomości (protokoły SMTP, IMAP, POP)
	\item komunikację natychmiastową i telefonię IP (protokoły SIP, XMPP, IAX)
	\item SSH - zdalny, szyfrowany dostęp do systemów IT, przesył plików oraz tunelowanie innych usług
\end{itemize}

\subsubsection{Domain Name System}

DNS umożliwia mapowanie nazwy na adres IP (lub wiele adresów IP) oraz przechowywanie dodatkowych informacji na temat domeny i znajdujących się w niej usług.

Domeny posiadają budowę hierarchiczną / drzewiastą:
\begin{itemize}
	\item precyzja rośnie od prawej do lewej
	\item kolejne poziomy oddzielane są kropkami
	\item najwyższym poziomem jest kropka będąca ostatnim znakiem w pełnej nazwie domenowej (np. \texttt{ciekawi.icm.edu.pl\textbf{\color{red}{.}}}), którą najczęściej pomija się w zapisie
	\item hierarchia ta jest niezależna od hierarchii routingu i wynika z faktu posiadania/użytkowania danej (pod)domeny)
\end{itemize}
Realizacja odpowiedzi na zapytanie DNS wygląda następująco:
\begin{enumerate}
	\item host kieruje zapytanie do określonego w jego konfiguracji serwera "rozwijającego" DNS (DNS resolver),
	\item serwer taki sprawdza w swojej pamięci podręcznej czy zna odpowiedź na to zapytanie (i nie jest ona przeterminowana - nie upłynął czas TTL od odnalezienia), jeżeli nie ma jej w swojej pamięci to
	\item serwer taki zna adresy głównych serwerów DNS (root serwerów) zawierających informacje na temat serwerów obsługujących domeny najwyższego rzędu i kieruje do jednego z nich zapytanie o serwer obsługujący skrajnie prawą część adresu (np. \textit{.org}),
	\item do otrzymanego serwera kierowane jest zapytanie o większą część adresu (np. \textit{eu.org}),
	\item itd. aż do uzyskania odpowiedzi o pytany adres
\end{enumerate}

\begin{teacherOnly}
pokazać jak działa serwer dns robiąc ręcznie zapytania dig'iem o kolejne poziomy:

\begin{Verbatim}
dig NS .
dig @g.root-servers.net. NS pl.
dig @a-dns.pl. NS edu.pl.
dig @a-dns.pl. NS icm.edu.pl.
dig @ns1.agh.edu.pl. NS ciekawi.icm.edu.pl.
# dostaliśmy   CNAME www2.icm.edu.pl.  oraz brak wpisu NS
dig @ns1.agh.edu.pl. A www2.icm.edu.pl.
# dostaliśmy 213.135.59.55
\end{Verbatim}
\end{teacherOnly}

\inputSingleAsFigure[.9]{booklets-sections/network/ilustracje/20-dns.tex}{Realizacja zapytania o rekord DNS}{ilustracja_dns}

DNS przechowuje informacje w postaci rekordów mających określony typ (w większości przypadków dla danej nazwy domenowej może być zdefiniowanych wiele rekordów, tego samego lub innych typów).
Wśród najważniejszych typów rekordów należy wymienić:
\begin{itemize}
	\item \Verb@NS@   – informacja o serwerach obsługujących DNS danej domeny
	\item \Verb@A@    – mapowanie nazwy na adres IPv4
	\item \Verb@AAAA@ – mapowanie nazwy na adres IPv4
	\item \Verb@MX@   – informacja o serwerach obsługujących pocztę danej domeny
	\item \Verb@SRV@  – informacje o hoście świadczącym usługę w tej domenie (usługa określana jest w nazwie domeny o którą pytamy)
	\item \Verb@PTR@  – mapowanie adresów IP na nazwy domenowe, realizowane w specjalnym drzewie \Verb@in-addr.arpa@ (dla IPv4) lub \Verb@ip6.arpa@ (IPv6),
	                    gdzie adres IP zapisywany jest w odwróconej kolejności po bajcie dla IPv4 lub cyfrze szesnastkowej dla IPv6
	                    \teacher{Pokazać wynik polecenia \texttt{host} z adresem IPv4 i IPv6}
	\item \Verb@TXT@  – informacje dodatkowe (np. jakie serwery pocztowe, są upoważnione do wysyłania poczty z tej domeny)
\end{itemize}

\subsubsection{Standardowe numery portów}

Popularne usługi (np. www) posiadają ustalone standardowe numery portów na których nasłuchiwac będzie serwer takiej usługi (np. dla wspomnainego www jest to port 80). Informacja o numerze portu usługi może być umieszczona także w rekordzie SRV systemu DNS.
% END: Polularne usługi
