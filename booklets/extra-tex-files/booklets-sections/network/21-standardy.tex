% Copyright (c) 2017-2020 Matematyka dla Ciekawych Świata (http://ciekawi.icm.edu.pl/)
% Copyright (c) 2017-2020 Robert Ryszard Paciorek <rrp@opcode.eu.org>
% 
% MIT License
% 
% Permission is hereby granted, free of charge, to any person obtaining a copy
% of this software and associated documentation files (the "Software"), to deal
% in the Software without restriction, including without limitation the rights
% to use, copy, modify, merge, publish, distribute, sublicense, and/or sell
% copies of the Software, and to permit persons to whom the Software is
% furnished to do so, subject to the following conditions:
% 
% The above copyright notice and this permission notice shall be included in all
% copies or substantial portions of the Software.
% 
% THE SOFTWARE IS PROVIDED "AS IS", WITHOUT WARRANTY OF ANY KIND, EXPRESS OR
% IMPLIED, INCLUDING BUT NOT LIMITED TO THE WARRANTIES OF MERCHANTABILITY,
% FITNESS FOR A PARTICULAR PURPOSE AND NONINFRINGEMENT. IN NO EVENT SHALL THE
% AUTHORS OR COPYRIGHT HOLDERS BE LIABLE FOR ANY CLAIM, DAMAGES OR OTHER
% LIABILITY, WHETHER IN AN ACTION OF CONTRACT, TORT OR OTHERWISE, ARISING FROM,
% OUT OF OR IN CONNECTION WITH THE SOFTWARE OR THE USE OR OTHER DEALINGS IN THE
% SOFTWARE.

\begin{teacherOnly}
\noindent Porozmawiać także o:
\begin{easylist}[itemize]
	& adresach URL
	& tunelowaniu ruchu – pakiet IP może zawierać inny pakiet IP
		&& pakiety zagnieżdżane są jeden w drugim ... danymi pakietu UDP lub TCP może być cokolwiek ... może to być też pakiet IP
		&& \Verb#ssh -L port:hostB:portB  host#\\
			połączenie do \Verb#localhost:port#  zostanie przekierowane przez SSH do \Verb#hostB:portB#  dostępnego z host (serwera ssh)
		&& \Verb#ssh -R port:hostB:portB  host#\\
			połączenie do \Verb#host:port#  zostanie przekierowane przez SSH do \Verb#hostB:portB# dostępnego z localhosta (klienta ssh)
			(domyślnie \Verb#host# będzie odbierał połączenia na ten port tylko od samego siebie – będzie słuchał tylko na swoim localhost)
		&& \Verb#ssh -D 8080  host#\\
			dynamiczny – proxy SOCKS4 / SOCKS5 
		&&\Verb# ssh -w 1:3  host#\\
			utworzy na kliencie \Verb#tun1# i serwerze urządzenia \Verb#tun3# i użyje ich do zestawienia tunelu (ruch wysłany na \Verb#tun1# bedzie docierał do \Verb#tun3# i na odwrót tak jakby były fizycznie połączone kablem ... rolę tego kabla pełni ssh)
		&& VPN - analogicznie do ssh -w tylko bardziej automatyczne zestawianie ... np. nie trzeba się martwić o konflikty numerków tunX ... dba o to serwer VPN ...
	& czemu służą, jak działają (ogólnie) protokoły routingu dynamicznego
	& standardach Internetu - dokumenty RFC
		&& standardy de facto
		&& poważne i mniej (gołębie, ...)
		&& dość dobrze się czyta
\end{easylist}
\end{teacherOnly}
