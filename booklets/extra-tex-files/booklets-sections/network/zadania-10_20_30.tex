% Copyright (c) 2017-2020 Matematyka dla Ciekawych Świata (http://ciekawi.icm.edu.pl/)
% Copyright (c) 2017-2020 Robert Ryszard Paciorek <rrp@opcode.eu.org>
% 
% MIT License
% 
% Permission is hereby granted, free of charge, to any person obtaining a copy
% of this software and associated documentation files (the "Software"), to deal
% in the Software without restriction, including without limitation the rights
% to use, copy, modify, merge, publish, distribute, sublicense, and/or sell
% copies of the Software, and to permit persons to whom the Software is
% furnished to do so, subject to the following conditions:
% 
% The above copyright notice and this permission notice shall be included in all
% copies or substantial portions of the Software.
% 
% THE SOFTWARE IS PROVIDED "AS IS", WITHOUT WARRANTY OF ANY KIND, EXPRESS OR
% IMPLIED, INCLUDING BUT NOT LIMITED TO THE WARRANTIES OF MERCHANTABILITY,
% FITNESS FOR A PARTICULAR PURPOSE AND NONINFRINGEMENT. IN NO EVENT SHALL THE
% AUTHORS OR COPYRIGHT HOLDERS BE LIABLE FOR ANY CLAIM, DAMAGES OR OTHER
% LIABILITY, WHETHER IN AN ACTION OF CONTRACT, TORT OR OTHERWISE, ARISING FROM,
% OUT OF OR IN CONNECTION WITH THE SOFTWARE OR THE USE OR OTHER DEALINGS IN THE
% SOFTWARE.

\IfStrEq{\dbEntryID}{}{
	\insertZadanie{\currfilepath}{czy_w_sieci}{}
	\insertZadanie{\currfilepath}{routing}{}
	\insertZadanie{\currfilepath}{ping1}{}
	\insertZadanie{\currfilepath}{traceroute1}{}
	\insertZadanie{\currfilepath}{dns1}{}
	\insertZadanie{\currfilepath}{tcpdump}{}
	\insertZadanie{\currfilepath}{http1}{}
	\insertZadanie{\currfilepath}{http2}{}
	\insertZadanie{\currfilepath}{ping_utf}{}
}

\IfStrEq{\dbEntryID}{rozwiazania}{
	\insertRozwiazanie{\currfilepath}{czy_w_sieci}{}
	\insertRozwiazanie{\currfilepath}{routing}{}
	\insertRozwiazanie{\currfilepath}{ping1}{}
	\insertRozwiazanie{\currfilepath}{traceroute1}{}
	\insertRozwiazanie{\currfilepath}{dns1}{}
	\insertRozwiazanie{\currfilepath}{tcpdump}{}
	\insertRozwiazanie{\currfilepath}{http1}{}
	\insertRozwiazanie{\currfilepath}{http2}{}
	\insertRozwiazanie{\currfilepath}{ping_utf}{}
}

\dbEntryBegin{czy_w_sieci}\if1\dbEntryCheckResults
Ustal czy adres \Verb$2001:db8:0:a17::123$ należy do sieci \Verb$2001:db8::/48$. Możesz posłużyć się narzędziami do obliczania zakresów sieci IP (np. \Verb#sipcalc#) lub obliczyć to ręcznie.
\fi

\dbEntryBegin{czy_w_sieci-rozwiazanie}\if1\dbEntryCheckResults
tak --- zakres sieci 2001:06a0:0000:0000:0000:0000:0000:0000 - 2001:06a0:0000:003f:ffff:ffff:ffff:ffff
\fi


\dbEntryBegin{routing}\if1\dbEntryCheckResults
Wynik polecenia \Verb#ip -6 r# pokazującego tablicę rotingową wygląda następująco:
\begin{Verbatim}
2001:db8:0:21::/64 dev eth1  proto kernel  metric 256 
2001:db8::/32 via 2001:db8:0:21::100 dev eth2  metric 1024 
2001:db8:fff:21::/64 dev eth2  proto kernel  metric 256 
2001:db8:abc:21::/64 via 2001:db8:fff:21::1 dev eth2  metric 1024 
fe80::/64 dev eth1  proto kernel  metric 256 
fe80::/64 dev eth2  proto kernel  metric 256 
default via 2001:db8:0:21::1 dev eth1  metric 1024 
\end{Verbatim}

Ustal gdzie zostanie skierowany pakiet adresowany do \Verb$2001:db8:abc:21:123::ff3$.
\fi

\dbEntryBegin{routing-rozwiazanie}\if1\dbEntryCheckResults
Adres 2001:db8:abc:21:123::ff3 zawiera się w zakresie sieci następujących sieci występujących w tablicy routingu:
\begin{itemize}
	\item 2001:db8::/32
	\item 2001:db8:abc:21::/64 
	\item default (0::/0)
\end{itemize}

Najdłuższy prefix spośród tych sieci ma 2001:db8:abc:21::/64 (64 bity), zatem zostanie użyty wpis w tablicy routingowej dla tej sieci.

W efekcie pakiet zostanie skierowany do routera 2001:db8:fff:21::1 poprzez interfejs eth2.
\fi


\dbEntryBegin{ping1}\if1\dbEntryCheckResults
Korzystając z narzędzi służących do diagnozowania sieci sprawdź czy host \Verb#ciekawi.icm.edu.pl# jest dostępny.
Jakiego polecenia użyłeś(aś) w tym celu? Co jeszcze mówi wynik tego polecenia?

\begin{teacherOnly}
Rozwiązaniem powinno opierać się o użycie polecenia \Verb#ping#, inne działające rozwiązania też akceptujemy, ale komentując/dyskutując naprowadzamy na to że standardowo do tego używa się komendy \Verb#ping#.
\end{teacherOnly}
\fi

\dbEntryBegin{ping1-rozwiazanie}\if1\dbEntryCheckResults
\begin{Verbatim}
$ ping -c4 ciekawi.icm.edu.pl
PING www2.icm.edu.pl (213.135.59.55) 56(84) bytes of data.
64 bytes from www2.icm.edu.pl (213.135.59.55): icmp_seq=1 ttl=60 time=3.91 ms
64 bytes from www2.icm.edu.pl (213.135.59.55): icmp_seq=2 ttl=60 time=3.63 ms
64 bytes from www2.icm.edu.pl (213.135.59.55): icmp_seq=3 ttl=60 time=2.94 ms
64 bytes from www2.icm.edu.pl (213.135.59.55): icmp_seq=4 ttl=60 time=5.03 ms

--- www2.icm.edu.pl ping statistics ---
4 packets transmitted, 4 received, 0% packet loss, time 6ms
rtt min/avg/max/mdev = 2.942/3.875/5.025/0.752 ms
\end{Verbatim}

Polecenie na standardowym wyjściu wypisuje m.in.:
\begin{itemize}
	\item odpytywany adres IP i wynik odwrotnego dns'u dla niego
	\item ilość danych wysyłanych (może być użyteczne do testowania problemów z MTU)
	\item dla każdego zapytania, które dostało odpowiedź:
	\begin{itemize}
		\item ilość przesłanych otrzymanych w pakiecie
		\item serwer który odpowiedział (rev-dns i ip)
		\item numer kolejny zapytania/odpowiedzi
		\item TTL pakietu z odpowiedzią (wskazuje na ilość routerów przez które przeszedł pakiet)
		\item czas potrzebny na uzyskanie odpowiedzi
	\end{itemize}
	\item podsumowujące statystyki związane z ilością zagubionych pakietów i czasami odpowiedzi
\end{itemize}

Zamiast informacji o poszczególnych odpowiedziach polecenie ping może nic nie wypisywać (brak odpowiedzi):
\begin{Verbatim}
$ ping -c 1 1.5.2.1
PING 1.5.2.1 (1.5.2.1) 56(84) bytes of data.

--- 1.5.2.1 ping statistics ---
1 packets transmitted, 0 received, 100% packet loss, time 0ms
\end{Verbatim}
lub podawać komunikat błędu wraz z jego nadawcą (jeżeli taki został zwrócony):
\begin{Verbatim}
$ ping -c1 192.168.6.55
PING 192.168.6.55 (192.168.6.55) 56(84) bytes of data.
From 192.168.6.3 icmp_seq=1 Destination Host Unreachable

--- 192.168.6.55 ping statistics ---
1 packets transmitted, 0 received, +1 errors, 100% packet loss, time 0ms
\end{Verbatim}

Kod powrotu polecenie ping informuje o tym czy host jest osiągalny czy nie, dzięki czemu polecenie to może zostać uzyte np. w bash'owym warunku \Verb#if#:
\begin{Verbatim}
if ping -c 1 10.0.1.1 >& /dev/null; then
	echo "10.0.1.1 jest dostępny"
else
	echo "10.0.1.1 nie jest dostępny"
fi
\end{Verbatim}
\fi


\dbEntryBegin{traceroute1}\if1\dbEntryCheckResults
Korzystając z narzędzi służących do diagnozowania sieci ustal jaką trasą podróżują pakiety z Twojego komputera do \Verb#www.opcode.eu.org# oraz do \Verb#www.example.org#.
Jakiego lub jakich poleceń użyłeś(aś) w tym celu? Co jeszcze mówi ich wynik? Co możesz powiedzieć porównując uzyskane trasy?
\fi
\dbEntryBegin{traceroute1-rozwiazanie}\if1\dbEntryCheckResults
Należy skorzystać z polecenia \Verb#treacroute#, \Verb#tracepath#, \Verb#mtr# lub podobnych.

Wynik polecenia host może wyglądać następująco (ale nie będzie nic dziwnego jeżeli otrzymasz inny – trasa routingowa zależy od miejsca z którego się łączysz, może też zmieniać się z czasem)
\begin{Verbatim}
traceroute  www.example.org
traceroute to www.example.org (93.184.216.34), 30 hops max, 60 byte packets
 1  funbox.home (192.168.6.1)  0.570 ms  9.648 ms  9.624 ms
 2  192.0.0.1 (192.0.0.1)  7.981 ms  8.007 ms  8.082 ms
 3  195.117.0.225 (195.117.0.225)  8.120 ms  8.189 ms  8.223 ms
 4  poz-r1.tpnet.pl (194.204.175.94)  18.474 ms  18.512 ms  18.547 ms
 5  hbg-b1-link.telia.net (213.248.96.144)  23.234 ms  23.268 ms  23.304 ms
 6  hbg-bb4-link.telia.net (213.155.135.86)  110.655 ms hbg-bb3-link.telia.net (213.155.135.80)  104.215 ms hbg-bb4-link.telia.net (213.155.135.86)  104.056 ms
 7  * * ldn-bb4-link.telia.net (62.115.122.161)  101.809 ms
 8  * * *
 9  nyk-b6-link.telia.net (80.91.254.36)  106.471 ms nyk-b6-link.telia.net (62.115.125.63)  104.446 ms  104.562 ms
10  edgecast-ic-317660-nyk-b5.c.telia.net (62.115.147.201)  109.616 ms  109.669 ms  112.148 ms
11  152.195.69.131 (152.195.69.131)  106.924 ms 152.195.68.141 (152.195.68.141)  109.509 ms 152.195.69.139 (152.195.69.139)  106.711 ms
12  93.184.216.34 (93.184.216.34)  103.418 ms  108.218 ms  106.420 ms
13  93.184.216.34 (93.184.216.34)  106.454 ms  106.587 ms  110.858 ms
\end{Verbatim}

Podobnie jak w przypadku polecenia ping wypisywana jest informacja o odpytywanym adresie IP (i wynik odwrotnego dns'u dla niego), ilość wysyłanych danych.
Wypisywane są kolejne routery (ip i rev-dns) wraz z czasami odpowiedzi (domyślnie traceroute na każdym poziomie robi 3 zapytania).
Gwiazdki oznaczają brak odpowiedzi. Jeżeli od pewnego momentu występują same gwiazdki oznacza to że wraz z dalszym zwiększaniem TTL nie dostajemy już kolejnych komunikatów o jego przekroczeniu.
	Może to wynikać z osiągnięcia hostu docelowego, który ma zablokowany testowany port bez wysyłania komunikatu o niedostępności portu lub błędu routera który nie przekazuje poprawnie komunikatów o zbyt małym TTL.
\fi



\dbEntryBegin{dns1}\if1\dbEntryCheckResults
Ustal (wszystkie) adresy IPv4 i IPv6 serwera \Verb#www.bitbucket.org#.
Zastanów się czemu może służyć to że niektóre nazwy domenowe rozwijają się na wiele różnych adresów IP.
\fi

\dbEntryBegin{dns1-rozwiazanie}\if1\dbEntryCheckResults
Należy użyć na przykład polecenia \Verb#host www.bitbucket.org# lub poleceń \Verb#dig AAAA www.bitbucket.org# ; \Verb#dig A www.bitbucket.org#.

Wynik polecenia host może wyglądać następująco (ale nie będzie nic dziwnego jeżeli otrzymasz inne adresy – dane w DNS ulegają zmianom, niekiedy nawet dynamicznie wprowadzanym).
\begin{Verbatim}
www.bitbucket.org is an alias for bitbucket.org.
bitbucket.org has address 104.192.141.1
bitbucket.org has IPv6 address 2406:da00:ff00::22c5:2ef4
bitbucket.org has IPv6 address 2406:da00:ff00::6b17:d1f5
bitbucket.org has IPv6 address 2406:da00:ff00::22e9:9f55
bitbucket.org has IPv6 address 2406:da00:ff00::34cc:ea4a
bitbucket.org mail is handled by 1 aspmx.l.google.com.
bitbucket.org mail is handled by 5 alt1.aspmx.l.google.com.
\end{Verbatim}
Warto zauważyć że zostaliśmy poinformowani także o tym że www.bitbucket.org jest aliasem (CNAME) na bitbucket.org.

Zwracanie wielu IP jest jednym ze sposobów rozkładania obciążenia i zapewnienia redundancji. Innymi rozwiązaniami jest udzielanie różnych odpowiedzi różnym kliemntom (tak robi np. www.google.com):
\begin{Verbatim}
$ host www.google.com
www.google.com has address 172.217.20.164
www.google.com has IPv6 address 2a00:1450:401b:802::2004
\end{Verbatim}
z innego hosta:
\begin{Verbatim}
$ host www.google.com
www.google.com has address 216.58.206.4
www.google.com has IPv6 address 2a00:1450:4001:821::2004
\end{Verbatim}
Jeszcze innym jest używanie mechanizmów routingowych takich jak anycast.
\fi


\dbEntryBegin{tcpdump}\if1\dbEntryCheckResults
Korzystając z dwóch instancji programu \Verb#nc# (\Verb#netcat#) – jednej w roli serwera, drugiej w roli klienta prześlij między nimi jakieś dane.
Użyj programu \Verb#tcpdump# (z odpowiednimi opcjami) aby podsłuchać komunikację sieciową między tymi programami i zobaczyć przesyłane dane.
\fi

\dbEntryBegin{tcpdump-rozwiazanie}\if1\dbEntryCheckResults
Należy uruchomić w osobnych oknach terminala kolejno polecenia:
\begin{enumerate}
	\item \Verb#tcpdump -i lo -A port 5555# - podsłuchujemy komunikację używającą portu 5555
	\item \Verb#nc -lp 5555# - uruchamiamy nasłuch na porcie 5555
	\item \Verb#nc localhost 5555# - łączymy się na port 5555
\end{enumerate}
Dane  wpisywane w jednym z uruchomionych netcat'ów będą wypisywane w drugim i odwrotnie.
Wszystkie te dane (wraz z informacjami zawartymi w nagłówkach IP i TCP) będą wyświetlane w okienku w którym działa tcpdump.

Oczywiście możemy użyć innego numeru portu, itd.
\fi


\dbEntryBegin{http1}\if1\dbEntryCheckResults
Korzystając bezpośrednio z poleceń protokołu HTTP i programu \Verb#nc# (\Verb#netcat#) lub \Verb#telnet#, pobierz i wyświetl kod strony \Verb#http://www.opcode.eu.org/#.

\textit{Wskazówka:
	Opis protokołu HTTP odnajdziesz bez problemu w sieci.\\
	Ogólnie żądanie HTTP składa się z pierwszej linii określającej typ wykonywanej operacji, ścieżkę oraz wersję protokołu - np. \texttt{GET /abc.txt HTTP/1.1} oznacza prośbę o zwrócenie zawartości pliku \texttt{/abc.txt}.
	Następnie podawane są nagłówki, w wersji HTTP 1.1 obowiązkowy jest nagłówek „Host” określający nazwę domenową serwera - np. \texttt{Host: www.example.org}.
	Po nagłówkach występuje pusta linia po której mogą być przesłane (przy niektórych typach żądań) jakieś dane (np. z wypełnionego na stronie formularza).
}
\fi

\dbEntryBegin{http1-rozwiazanie}\if1\dbEntryCheckResults
\begin{Verbatim}
netcat www.opcode.eu.org 80
GET / HTTP/1.1
Host: www.opcode.eu.org

HTTP/1.1 200 OK
Server: nginx/1.14.2
Date: Sun, 26 Apr 2020 09:12:09 GMT
Content-Type: text/html; charset=utf-8
Content-Length: 11490
Last-Modified: Wed, 08 Apr 2020 17:36:32 GMT
Connection: keep-alive
ETag: "5e8e0ba0-2ce2"
Accept-Ranges: bytes

<?xml version="1.0" encoding="UTF-8" ?>
<html xmlns="http://www.w3.org/1999/xhtml" xml:lang="pl" lang="pl">
<head>
        <title>OpCode.eu.org - strona główna</title>
\end{Verbatim}
[...]

\vspace{7pt}
Jeżeli użyjemy innego nagłówka \Verb#Host:# w zapytaniu możemy dostać np. komunikat o przekierowaniu:
\begin{Verbatim}
netcat www.opcode.eu.org 80
GET / HTTP/1.1
Host: opcode.eu.org

HTTP/1.1 301 Moved Permanently
Server: nginx/1.14.2
Date: Tue, 21 Apr 2020 17:58:44 GMT
Content-Type: text/html
Content-Length: 185
Connection: keep-alive
Location: http://www.opcode.eu.org/

<html>
<head><title>301 Moved Permanently</title></head>
<body bgcolor="white">
<center><h1>301 Moved Permanently</h1></center>
<hr><center>nginx/1.14.2</center>
</body>
</html>
\end{Verbatim}
Przeglądarki WWW nie wyświetlają tych komunikatów tylko automatycznie przechodzą pod wskazany w nagłówku \Verb#Location:# adres.
\fi


\dbEntryBegin{http2}\if1\dbEntryCheckResults
Zadanie \ref{http1} można rozwiązać przy pomocy netcat'a bez dodatkowych opcji, jednak jeżeli stroną do pobrania byłoby np. \Verb#http://www.icm.edu.pl# to należałoby skorzystać z opcji \Verb#-C# netcat'a (w przeciwnym razie serwer zwraca błąd 400 "Bad Request").
Sprawdź co robi ta opcja i zastanów się dlaczego w przypadku niektórych serwerów jest konieczna a w przypadku innych nie? Co na ten temat mówi standard HTTP?
\fi

\dbEntryBegin{http2-rozwiazanie}\if1\dbEntryCheckResults
Wynika to z stosowania w większości protokołów sieciowych (w tym w HTTP) jako znaku linii sekwencji dwu bajtowej \Verb#\n\r# (nowa linia, powrót karetki).
Serwer WWW obsługujący domenę opcode.eu.org postępuje według filozofii nakazującej liberalne podejście do danych otrzymywanych i restrykcyjne do generowanych przez siebie (wysyłanych) i poprawnie interpretuje formalnie błędne żądanie zawierające znaki nowej linii w postaci \Verb#\n#.
Serwer obsługujący www.icm.edu.pl nie jest już tak liberalny i wymaga znaków nowej linii w postaci \Verb#\n\r#, dlatego do netcata musimy dodać opcję -C aby konwertował wprowadzone znaki nowej linii na taką postać przed wysłaniem.
\fi

\dbEntryBegin{ping_utf}\if1\dbEntryCheckResults
Zobacz czy rozwiązanie zadania \ref{ping1} zadziała gdy użyjesz nazwy serwera zawierającej polskie znaki: \Verb#licealiści.icm.edu.pl#.
Jak myślisz, dlaczego polskie znaki są tak rzadko używane w nazwach domenowych?
\fi
\dbEntryBegin{ping_utf-rozwiazanie}\if1\dbEntryCheckResults
\begin{Verbatim}
ping -c2 licealiści.icm.edu.pl
PING www2.icm.edu.pl (213.135.59.55) 56(84) bytes of data.
64 bytes from www2.icm.edu.pl (213.135.59.55): icmp_seq=1 ttl=60 time=4.20 ms
64 bytes from www2.icm.edu.pl (213.135.59.55): icmp_seq=2 ttl=60 time=3.26 ms

--- www2.icm.edu.pl ping statistics ---
2 packets transmitted, 2 received, 0% packet loss, time 2ms
rtt min/avg/max/mdev = 3.260/3.732/4.204/0.472 ms
\end{Verbatim}
Jak widać działa. Standard kodowania dowolnych znaków Unicode, tak aby mogły być użyte w nazwach domenowych został opracowany w 2003 roku (Punycode, RFC3492), jest dość powszechnie zaimplementowany.
Możemy użyć go także w pythonowej metodzie encode (np. \Verb#print( "żółw".encode("punycode") )#).

Ciężko powiedzieć dlaczego (przynajmniej w Polsce) znaki z poza ASCII są tak rzadko stosowane w nazwach domenowych.

\textit{Ciekawostka: kodowanie znaków nie ASCII w nazwie domenowej jest określone jednoznacznie, natomiast w dalszej części adresu URL niezbyt - zasadniczo zależy od konfiguracji serwera, ale standardem defacto jest tutaj UTF8).}
\fi
