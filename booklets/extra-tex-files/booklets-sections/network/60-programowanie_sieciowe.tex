% Copyright (c) 2017-2020 Matematyka dla Ciekawych Świata (http://ciekawi.icm.edu.pl/)
% Copyright (c) 2017-2020 Robert Ryszard Paciorek <rrp@opcode.eu.org>
% 
% MIT License
% 
% Permission is hereby granted, free of charge, to any person obtaining a copy
% of this software and associated documentation files (the "Software"), to deal
% in the Software without restriction, including without limitation the rights
% to use, copy, modify, merge, publish, distribute, sublicense, and/or sell
% copies of the Software, and to permit persons to whom the Software is
% furnished to do so, subject to the following conditions:
% 
% The above copyright notice and this permission notice shall be included in all
% copies or substantial portions of the Software.
% 
% THE SOFTWARE IS PROVIDED "AS IS", WITHOUT WARRANTY OF ANY KIND, EXPRESS OR
% IMPLIED, INCLUDING BUT NOT LIMITED TO THE WARRANTIES OF MERCHANTABILITY,
% FITNESS FOR A PARTICULAR PURPOSE AND NONINFRINGEMENT. IN NO EVENT SHALL THE
% AUTHORS OR COPYRIGHT HOLDERS BE LIABLE FOR ANY CLAIM, DAMAGES OR OTHER
% LIABILITY, WHETHER IN AN ACTION OF CONTRACT, TORT OR OTHERWISE, ARISING FROM,
% OUT OF OR IN CONNECTION WITH THE SOFTWARE OR THE USE OR OTHER DEALINGS IN THE
% SOFTWARE.

\section{Programowanie usług sieciowych}

\begin{teacherOnly}
	Na początek (w wariancie zintegrowanym, na samym wykładzie pomijamy do zrobienia na ćwiczeniach):
	\begin{easylist}[itemize]
		& zabawa netcat-em (serwer, klient) + tcpdum do podsłuchiwania
		& kilka słów więcej o:
			&& HTTP (+ oglądanie WWW netcatem, pamiętać o -C w netcacie)
			&& SMTP (+ sesja SMTP netcatem, pokazać łatwość fałszowania nagłówków)
	\end{easylist}
	
	Następnie (jeżeli wcześniej nie było) \strong{krótkie} wprowadzenie systemowe:
	\begin{easylist}[itemize]
		& fork
		& komunikacja międzyprocesowa
			&& sygnały
				&&& poopowiadać o kill
					&&&& czy program może nie umrzeć?
				&&& wiele różnych sygnałów ...
			&& pamięć i semafory
				&&& poopowiadać o problemie if a==1: a=0
		& subprocess w pythonie
		& multiprocessing i lock w pythonie
	\end{easylist}\vspace{4pt}
	
	Opowiedzieć o funkcji \Verb#getaddrinfo#:
	\begin{easylist}[itemize]
		& \Verb#import socket#\\
		  \Verb#socket.getaddrinfo("www.opcode.eu.org", "www")#
		&& widzimy że to lista krotek
		& \Verb#for x in socket.getaddrinfo("www.opcode.eu.org", "www"):#\\
		  \Verb#   print(x[4])#
		& jeżeli podamy numer portu:\\
		  \Verb#socket.getaddrinfo("www.opcode.eu.org", "80")#
		&& to dostaniemy różne typy gniazd sieciowych - TCP, UDP i "RAW" ...
	\end{easylist}
	Od tego przechodzimy płynnie do wysyłania po UDP.
\end{teacherOnly}


\subsection{wysyłanie danych po UDP}
\begin{CodeFrame*}[python]{}
import socket, sys

if len(sys.argv) != 3:
  print("USAGE: " + sys.argv[0] + " dstHost dstPort", file=sys.stderr)
  exit(1)

dstAddrInfo = socket.getaddrinfo(sys.argv[1], sys.argv[2])
dstAddrInfo = dstAddrInfo[0]
sfd = socket.socket(dstAddrInfo[0], socket.SOCK_DGRAM)

sfd.sendto("Ala ma kota".encode(), dstAddrInfo[4])
\end{CodeFrame*}

\subsection{odbiór danych po UDP}
\begin{CodeFrame*}[python]{}
import socket, sys

if len(sys.argv) != 2:
  print("USAGE: " + sys.argv[0] + " listenPort", file=sys.stderr)
  exit(1)

sfd = socket.socket(socket.AF_INET6, socket.SOCK_DGRAM)
sfd.setsockopt(socket.IPPROTO_IPV6, socket.IPV6_V6ONLY, 0)
sfd.bind(('::', int(sys.argv[1])))

while True:
  data, sAddr, = sfd.recvfrom(4096)
  print("odebrano od", sAddr, ":", data.decode());
\end{CodeFrame*}

\subsection{klient TCP}
\begin{CodeFrame*}[python]{}
import socket, select, sys

if len(sys.argv) != 3:
	print("USAGE: " + sys.argv[0] + " dstHost dstPort", file=sys.stderr)
	exit(1);

# struktura zawierająca adres na który wysyłamy
dstAddrInfo = socket.getaddrinfo(sys.argv[1], sys.argv[2], proto=socket.IPPROTO_TCP)

# mogliśmy uzyskać kilka adresów, więc próbujemy używać kolejnych do skutku
for aiIter in dstAddrInfo:
	try:
		print("try connect to:", aiIter[4])
		# utworzenie gniazda sieciowego ... SOCK_STREAM oznacza TCP
		sfd = socket.socket(aiIter[0], socket.SOCK_STREAM)
		# połączenie ze wskazanym adresem
		sfd.connect(aiIter[4])
	except:
		# jeżeli się nie udało ... zamykamy gniazdo
		if sfd:
			sfd.close()
		sfd = None
		# i próbujemy następny adres
		continue
	break;

if sfd == None:
	print("Can't connect", file=sys.stderr)
	exit(1);

# wysyłanie
sfd.sendall("Ala ma Kota\n".encode())

# czekanie na odbiór i odbiór
while True:
	rdfd, _, _ = select.select([sfd], [], [], 13.0)
	if sfd in rdfd:
		d = sfd.recv(4096)
		d = d.decode()
		print(d, end="")
		
		# odbiór pustego pakietu lub pakietu zawierającego
		# jedynie pustą linię kończy działanie
		if d == "" or d == "\n" or d == "\r\n":
			break
	else:
		# timeout kończy działanie
		break

# zamykanie połączenia
sfd.shutdown(socket.SHUT_RDWR)
sfd.close()
\end{CodeFrame*}

\subsection{serwer TCP}
\begin{CodeFrame*}[python]{}
import socket, select, signal, sys, os

MAX_CHILD = 5
QUERY_SIZE = 3
TIMEOUT = 13
BUF_SIZE = 4096

if len(sys.argv) != 2:
	print("USAGE: " + sys.argv[0] + " listenPort", file=sys.stderr)
	exit(1);

# obsługa sygnału o zakończeniu potomka
childNum = 0
def onChildEnd(s, f):
	print("odebrano sygnał o śmierci potomka")
	global childNum
	childNum -= 1
	os.waitpid(-1, os.WNOHANG);
signal.signal(signal.SIGCHLD, onChildEnd)

# utworzenie gniazd sieciowych ... SOCK_STREAM oznacza TCP
sfd_v4 = socket.socket(socket.AF_INET,  socket.SOCK_STREAM)
sfd_v6 = socket.socket(socket.AF_INET6, socket.SOCK_STREAM)

# ustawienie opcji gniazda ... IPV6_V6ONLY=1 wyłącza korzystanie
# z tego samego socketu dla IPv4 i IPv6
sfd_v6.setsockopt(socket.IPPROTO_IPV6, socket.IPV6_V6ONLY, 1)

# przypisanie adresów ...
# '0.0.0.0' oznacza dowolny adres IPv4 (czyli to samo co INADDR_ANY)
# '::' oznacza dowolny adres IPv6 (czyli to samo co in6addr_any)
sfd_v4.bind(('0.0.0.0', int(sys.argv[1])))
sfd_v6.bind(('::',      int(sys.argv[1])))

# określenie gniazd jako używanych do odbioru połączeń przychodzących
# (długość kolejki połączeń ustawiona na wartość QUERY_SIZE)
sfd_v4.listen(QUERY_SIZE)
sfd_v6.listen(QUERY_SIZE)

# czekanie na połączenia z użyciem select() w nieskończonej pętli
while True:
	sfd, _, _ = select.select([sfd_v4, sfd_v6], [], [])
	for fd in sfd:
		#  odebranie połączenia
		sfd_c, sAddr = fd.accept()
		
		# weryfikacja ilości potomków
		if childNum >= MAX_CHILD:
			print("za dużo potomków - odrzucam połączenie od:", sAddr);
			sfd_c.send("Internal Server Error\r\n".encode())
			sfd_c.close()
			return
		
		# aby móc obsługiwać wiele połączeń rozgałęziamy proces
		pid = os.fork()
		if pid > 0:
			# rodzic - zwiększamy licznik potomków
			childNum += 1
		else:
			# potomek - obsługa danego połączenia
			print("połączenie od:", sAddr)
			while True:
				# czekanie na dane z timeout'em
				# aby zabezpieczyć się przed atakiem DoS
				rd, _, _ = select.select([sfd_c], [], [], TIMEOUT)
				if sfd_c in rd:
					data = sfd_c.recv(BUF_SIZE)
					if data:
						print("odebrano od", sAddr, ":", data.decode());
						sfd_c.send(data)
					else:
						print("koniec połączenia od:", sAddr)
						break
				else:
					print("timeout połączenia od:", sAddr)
					break
			# zamykanie połączenia
			sfd_c.shutdown(socket.SHUT_RDWR)
			sfd_c.close()
			sys.exit()
\end{CodeFrame*}
