% Copyright (c) 2017-2020 Matematyka dla Ciekawych Świata (http://ciekawi.icm.edu.pl/)
% Copyright (c) 2017-2020 Robert Ryszard Paciorek <rrp@opcode.eu.org>
% 
% MIT License
% 
% Permission is hereby granted, free of charge, to any person obtaining a copy
% of this software and associated documentation files (the "Software"), to deal
% in the Software without restriction, including without limitation the rights
% to use, copy, modify, merge, publish, distribute, sublicense, and/or sell
% copies of the Software, and to permit persons to whom the Software is
% furnished to do so, subject to the following conditions:
% 
% The above copyright notice and this permission notice shall be included in all
% copies or substantial portions of the Software.
% 
% THE SOFTWARE IS PROVIDED "AS IS", WITHOUT WARRANTY OF ANY KIND, EXPRESS OR
% IMPLIED, INCLUDING BUT NOT LIMITED TO THE WARRANTIES OF MERCHANTABILITY,
% FITNESS FOR A PARTICULAR PURPOSE AND NONINFRINGEMENT. IN NO EVENT SHALL THE
% AUTHORS OR COPYRIGHT HOLDERS BE LIABLE FOR ANY CLAIM, DAMAGES OR OTHER
% LIABILITY, WHETHER IN AN ACTION OF CONTRACT, TORT OR OTHERWISE, ARISING FROM,
% OUT OF OR IN CONNECTION WITH THE SOFTWARE OR THE USE OR OTHER DEALINGS IN THE
% SOFTWARE.

\section{Programowanie usług sieciowych}

\subsection{wysyłanie danych po UDP}
\begin{CodeFrame*}[python]{}
import socket, sys

if len(sys.argv) != 3:
  print("USAGE: " + sys.argv[0] + " dstHost dstPort", file=sys.stderr)
  exit(1)

# pobieramy informacje o adresach na które rozwija się podana nazwa hosta / usługi
# dzięki tej funkcji jako dstHost możemy podać zarówno nazwę domenową jak i numer IP 
# (w którejś z standardowych notacji) hosta z którym chcemy się połączyć
# a jako dstPort numer portu lub nazwę usługi z /etc/services
dstAddrInfo = socket.getaddrinfo(sys.argv[1], sys.argv[2], type=socket.SOCK_DGRAM)

# funkcja ta nam zwraca listę dostępnych możliwości połączenia (np. nazwa domenowa
# może rozwijać się na wiele różnych adresów), przekazując do getaddrinfo
# opcjonalny argument type zawęziliśmy ta listę do połączeń UDP (SOCK_DGRAM)

# moglibyśmy próbować kolejnych adresów w pętli, ale w tym prostym przykładzie
# próbujemy użyć jedynie pierwszego ze zwróconych adresów
dstAddrInfo = dstAddrInfo[0]

# otwieramy gniazdo
sfd = socket.socket(dstAddrInfo[0], socket.SOCK_DGRAM)

# wysyłamy dane
sfd.sendto("Ala ma kota".encode(), dstAddrInfo[4])
\end{CodeFrame*}


\subsection{odbiór danych po UDP}
\begin{CodeFrame*}[python]{}
import socket, sys

if len(sys.argv) != 2:
  print("USAGE: " + sys.argv[0] + " listenPort", file=sys.stderr)
  exit(1)

# otwieramy gniazdo
sfd = socket.socket(socket.AF_INET6, socket.SOCK_DGRAM)

# ustawiamy opcję gniazda pozwalającą na jednoczesną obsługę IPv4 i IPv6
sfd.setsockopt(socket.IPPROTO_IPV6, socket.IPV6_V6ONLY, 0)

# ustawiamy adres i port na którym słuchamy
# adres zerowy oznacza słuchanie na wszystkich adresach IP danego hosta
sfd.bind(('::', int(sys.argv[1])))

while True:
  # czekamy na dane, gdy je otrzymamy to odbieramy
  data, sAddr, = sfd.recvfrom(4096)
  # i wypisujemy co i od kogo dostaliśmy
  print("odebrano od", sAddr, ":", data.decode());
\end{CodeFrame*}

\subsection{klient TCP}
\inputminted[frame=single,tabsize=2]{python}{vip-code-src/sieciowe/TCP_-_klient.py}
\textit{Kod do pobrania: \url{https://bitbucket.org/OpCode-eu-org/opcode-vip/raw/master/vademecum/code-src/sieciowe/TCP_-_klient.py}}

\subsection{serwer TCP}
\inputminted[frame=single,tabsize=2]{python}{vip-code-src/sieciowe/TCP_-_serwer.py}
\textit{Kod do pobrania: \url{https://bitbucket.org/OpCode-eu-org/opcode-vip/raw/master/vademecum/code-src/sieciowe/TCP_-_serwer.py}}

\setcounter{subsection}{0}
\insertZadanie{booklets-sections/network/zadania-50_60.tex}{serwer_trojkat}{}
\insertZadanie{booklets-sections/network/zadania-50_60.tex}{udp_echo}{}
\insertZadanie{booklets-sections/network/zadania-50_60.tex}{dos_echo1}{}
\insertZadanie{booklets-sections/network/zadania-50_60.tex}{dos_echo2}{}
