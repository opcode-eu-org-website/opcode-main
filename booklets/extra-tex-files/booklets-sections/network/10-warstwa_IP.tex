% Copyright (c) 2017-2020 Matematyka dla Ciekawych Świata (http://ciekawi.icm.edu.pl/)
% Copyright (c) 2017-2020 Robert Ryszard Paciorek <rrp@opcode.eu.org>
% 
% MIT License
% 
% Permission is hereby granted, free of charge, to any person obtaining a copy
% of this software and associated documentation files (the "Software"), to deal
% in the Software without restriction, including without limitation the rights
% to use, copy, modify, merge, publish, distribute, sublicense, and/or sell
% copies of the Software, and to permit persons to whom the Software is
% furnished to do so, subject to the following conditions:
% 
% The above copyright notice and this permission notice shall be included in all
% copies or substantial portions of the Software.
% 
% THE SOFTWARE IS PROVIDED "AS IS", WITHOUT WARRANTY OF ANY KIND, EXPRESS OR
% IMPLIED, INCLUDING BUT NOT LIMITED TO THE WARRANTIES OF MERCHANTABILITY,
% FITNESS FOR A PARTICULAR PURPOSE AND NONINFRINGEMENT. IN NO EVENT SHALL THE
% AUTHORS OR COPYRIGHT HOLDERS BE LIABLE FOR ANY CLAIM, DAMAGES OR OTHER
% LIABILITY, WHETHER IN AN ACTION OF CONTRACT, TORT OR OTHERWISE, ARISING FROM,
% OUT OF OR IN CONNECTION WITH THE SOFTWARE OR THE USE OR OTHER DEALINGS IN THE
% SOFTWARE.

% BEGIN: Sieci - intro
\section{Podstawy TCP/IP}

Sieci komputerowe działają na zasadzie przesyłania informacji w postaci porcji, z których każda posiada co najmniej informację o adresie odbiorcy (zwykle też nadawcy), nazywanych ramkami lub pakietami. Kierowanie pakietów w odpowiednie miejsce odbywa się na podstawie adresu pakietu i nie jest związane z fizycznym zestawianiem łącza pomiędzy nadawcą a odbiorcą - każdy pakiet jest kierowany niezależnie, a w ramach pojedynczego łącza (kanału transmisji) mogą być przekazywane pakiety adresowane do różnych odbiorców. Nazywane jest to komutacją pakietów, w odróżnieniu od komutacji łącza (która występowała np. w klasycznej, analogowej telefonii, gdzie przekaźniki w centralach dokonywały zestawienia połączeń elektrycznych między dwoma aparatami telefonicznymi).

To właśnie umieszczanie w nagłówku każdego z pakietów adresów umożliwia kierowanie ruchem takich pakietów bez wcześniejszego fizycznego zestawiania łącz. W oparciu o ten adres host jest w stanie rozpoznać czy pakiet jest przeznaczony dla niego czy nie, a przełączniki i routery mogą kierować pakiety do odpowiednich fragmentów sieci.

Jeżeli w sieci komputerowej host A chce się komunikować z hostem B, a host C z hostem D nie musimy zapewnić im osobnych łączy do przeprowadzania takiej komunikacji, tak jak to było na przykład w klasycznej telefonii analogowej, gdzie dwie takie "rozmowy" wymagałyby zestawienia osobnych fizycznych kabli od jednego abonenta do drugiego (odbywało się to za pomocą przekaźników w centralach). W sieci komputerowej hosty te będą wytwarzały odpowiednio zaadresowane pakiety i reagowały tylko na pakiety zaadresowane do nich. Dzięki temu pakiety, stanowiące osobne strumienie komunikacji, mogą być z łatwością przesyłane tym samym fizycznym łączem. Host C może słyszeć lub nie komunikację pomiędzy hostami A i B (zależy to od różnych czynników, na przykład od wykorzystywanej sieci i względnego położenia tych hostów), natomiast wie że to "nie do niego". 

Jeżeli w ramach sieci mamy jakieś urządzenie dzielące ją na mniejsze kawałki, podział ten będzie się odbywał w oparciu o adresy pakietów – urządzenie takie do danej sieci będzie przekazywało tylko pakiety adresowane do hostów w tej sieci. Mamy do czynienia z przełączaniem, czyli komutacją pakietów – w odróżnieniu od (omówionej pokrótce na przykładzie telefonii) komutacji łączy, która polegała na zestawianiu fizycznego łącza pomiędzy komunikującymi się hostami.

\subsection{Struktura warstwowa}

Protokoły sieciowe na ogół tworzą tak zwane stosy protokołów, gdzie dane opakowywane są kolejno w nagłówki kolejnych protokołów. Każdy z protokołów w takim stosie pełni dedykowane mu funkcje i na ogół nie ingeruje ani w protokoły warstwy niższej, ani w przenoszoną przez niego zawartość, czyli protokoły warstwy wyższej z danymi.

Komunikacja sieciowa typowo posiada strukturę warstwową. W modelu OSI wyróżnia się 7 warstw:
\begin{enumerate}
	\item fizyczną (pierwszą) definiującą aspekty związane z fizycznym przesyłem sygnału takie jak częstotliwości radiowe, poziomy napięć, etc.;
		określa sposób transmisji kolejnych bajtów
	\item łącza danych (drugą) definiującą aspekty związane z formatem ramki, protokoły ustalania zasad dostępu do medium transmisyjnego, itd.;
		określa sposób transmisji porcji danych pomiędzy hostami w jednej sieci
	\item sieciową (trzecią) definiującą aspekty związane z formatem pakietu, adresacją i zasady routingu umożliwiające zapewnienie łączności pomiędzy różnymi sieciami;
		określa sposoby transmisji porcji danych pomiędzy sieciami
	\item transportową (czwartą) odpowiedzialną za podział strumienia na porcje informacji, kontrolę nad poprawnością transmisji, adresację usług w ramach hosta
	\item sesji (piątą)
	\item prezentacji (szóstą)
	\item aplikacji (siódmą)
\end{enumerate}

\inputSingleAsFigure[.9]{booklets-sections/network/ilustracje/10-warstwy.tex}{Struktura warstwowa protokołów sieciowych}{ilustracja-warstwy}

\noindent
W modelu TCP/IP wyróżnia się 4 warstwy:
\begin{enumerate}
	\item Dostępu do sieci - obejmującą warstwy 1 i 2 modelu OSI
	\item Internetu - obejmującą warstwę 3 modelu OSI
	\item Transportową - obejmującą warstwę 4 modelu OSI
	\item Aplikacji - obejmującą warstwy 5, 6 i 7 modelu OSI
\end{enumerate}

\noindent
Z punktu widzenia modelu TCP/IP można powiedzieć o enkapsulacji danych kolejnych warstw w ramach warstwy niższej, czyli „surowe” dane (np. strona HTML) obudowywane są strukturą opisywaną przez warstwę aplikacji (np. nagłówkami HTTP), następnie całość ta umieszczana jest w polu danych pakietu warstwy transportowej (np. TCP), ten z kolei w polu danych pakietu IP (warstwy sieciowej), na koniec pakiet IP jest umieszczany w polu danych ramki warstwy dostępu do sieci (np. ramki ethernetowej). W ramach podróży przez kolejne sieci pakiet IP jest wyjmowany i wkładany w kolejne ramki warstwy dostępu do sieci, na ogół tylko z niewielkimi ingerencjami w zawartość tego pakietu (prawie zawsze nie dochodzącymi do pola danych pakietu TCP lub datagramu UDP, czyli nie wykraczającymi poza warstwę 4 OSI).

\begin{teacherOnly} Można porównać do podróży listu różnymi środkami transportu:
	\begin{easylist}[itemize]
		& list ma jakąś zawartość - dane które na ogół nie są sprawdzane po drodze
		& koperta ma adres nadawcy i docelowy
		& pociąg którym podróżuje ma "lokalnuy"/"sprzętowy" adres nadawcy i docelowy (stacja początkowa, końcowa)
		& później koperta może być przeniesiona do innego pociągu (ten sam typ sieci fizycznej, ale inna sieć), lub np. na statek (inny typ sieci fizycznej)
	\end{easylist}
\end{teacherOnly}

% END: Sieci - intro

% BEGIN: sieci IP
\subsection{Protokół IP}

Protokół IP (Internet Protocol) odpowiedzialny jest przede wszystkim za sposób adresacji hostów oraz reguły komutacji pakietów (routing). Jest on wspomagany przez kolejny protokół z tej rodziny - ICMP (Internet Control Message Protocol), którego zadaniem jest przekazywanie informacji kontrolnych np. o nieosiągalności hosta docelowego, odrzuceniu przetwarzania pakietu ze względu na zbyt dużą liczbę skoków (gdy wartość pola TTL z nagłówka IP wyniesie zero) a także pingi (zarówno żądanie jak i odpowiedź).

Jako że IP stworzony jest do łączenia różnych fizycznych sieci w jedną sieć logiczną (internet), to pozwala on na wydzielanie w oparciu o zasady adresacji mniejszych fragmentów sieci, nazywanych niekiedy podsieciami. Oczywiście zapewnia też mechanizmy przekazywania pakietów pomiędzy takimi podsieciami.

\inputSideBySideAsFigure
	{Pakiet IPv4}{booklets-sections/network/ilustracje/10-ipv4.tex}
	{Pakiet IPv6}{booklets-sections/network/ilustracje/10-ipv6.tex}
	{Struktura pakietów IP}{ilustracja_pakiety_ip}
% END: sieci IP
