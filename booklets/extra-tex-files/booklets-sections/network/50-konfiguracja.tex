% Copyright (c) 2017-2020 Matematyka dla Ciekawych Świata (http://ciekawi.icm.edu.pl/)
% Copyright (c) 2017-2020 Robert Ryszard Paciorek <rrp@opcode.eu.org>
% 
% MIT License
% 
% Permission is hereby granted, free of charge, to any person obtaining a copy
% of this software and associated documentation files (the "Software"), to deal
% in the Software without restriction, including without limitation the rights
% to use, copy, modify, merge, publish, distribute, sublicense, and/or sell
% copies of the Software, and to permit persons to whom the Software is
% furnished to do so, subject to the following conditions:
% 
% The above copyright notice and this permission notice shall be included in all
% copies or substantial portions of the Software.
% 
% THE SOFTWARE IS PROVIDED "AS IS", WITHOUT WARRANTY OF ANY KIND, EXPRESS OR
% IMPLIED, INCLUDING BUT NOT LIMITED TO THE WARRANTIES OF MERCHANTABILITY,
% FITNESS FOR A PARTICULAR PURPOSE AND NONINFRINGEMENT. IN NO EVENT SHALL THE
% AUTHORS OR COPYRIGHT HOLDERS BE LIABLE FOR ANY CLAIM, DAMAGES OR OTHER
% LIABILITY, WHETHER IN AN ACTION OF CONTRACT, TORT OR OTHERWISE, ARISING FROM,
% OUT OF OR IN CONNECTION WITH THE SOFTWARE OR THE USE OR OTHER DEALINGS IN THE
% SOFTWARE.

\section{Konfiguracja sieci w Linuxie}

Konfigurację interfejsów sieciowych w systemie Linux umożliwia polecenie \Verb$ip$. Przykłady użycia (ta lista w żaden sposób nie wyczerpuje dostępnych możliwości i dodatkowych opcji):
\begin{itemize}
	\item wyświetlanie i ustawianie adresów IP
		\begin{itemize}
			\item \Verb{ip addr} – wypisuje obecną konfigurację adresów i informacje o stanie interfejsu
			                 (\Verb{UP}/\Verb{DOWN} – interfejs włączony/wyłączony,
			                  \Verb{LOWER_UP}/\Verb{LOWER_DOWN} – link warstwy niższej na interfejsie / jego brak)
			\item \Verb{ip addr add ADDRESS dev INTERFACE} – dodaje adres \Verb{ADDRESS} do interfejsu \Verb{INTERFACE}
			\item \Verb{ip addr del ADDRESS dev INTERFACE} – usuwa adres \Verb{ADDRESS} z interfejsu \Verb{INTERFACE}
		\end{itemize}
	\item włączanie i wyłaczanie interfejsów
		\begin{itemize}
			\item \Verb{ip link set INTERFACE up} / \Verb{ip link set INTERFACE down} – włączenie / wyłączenie interfejsu \Verb{INTERFACE}
			\item \Verb{ip link set INTERFACE address ADDRESS} – ustawienie adresu sprzętowego urządzenia \Verb{INTERFACE} na \Verb{ADDRESS}
		\end{itemize}
	\item konfiguracja tagowanych VLANów
		\begin{itemize}
			\item \Verb{ip link add link INTERFACE name INTERFACE.VLANID type vlan id VLANID} – dodanie interfejsu związanego z tagowanym VLANem o numerze \Verb{VLANID} na interfejsie \Verb{INTERFACE}, moduł 8021q powinien zostać załadowany automatycznie
			\item \Verb{ip link del INTERFACE.VLANID type vlan} – usunięcie interfejsu INTERFACE.VLANID (związanego z tagowanym VLANem VLANID na interfejsie INTERFACE)
		\end{itemize}
	\item konfiguracja BRIDGE (programowego switcha)
		\begin{itemize}
			\item \Verb{ip link add INTERFACE type bridge} – dodanie interfejsu bridge'owego o nazwie INTERFACE
			\item \Verb{ip link set SLAVE master INTERFACE}  – włączenie interfejsu SLAVE w skład bridge'owego INTERFACE
			\item \Verb{ip link set SLAVE nomaster} – wyłaczenie interfejsu SLAVE z bridge'a
			\item \Verb{ip link show master INTERFACE} – wyświetlanie portów składowych bridge'a o nazwie INTERFACE
			\item przydatne może być także polecenie \Verb{bridge}
		\end{itemize}
	\item konfiguracja BONDów (interfejsów agregujących inne w grupę celem zwiększenia prędkości lub niezawodności)
		\begin{itemize}
			\item \Verb{ip link add INTERFACE type bond} – dodanie interfejsu bondingowego o nazwie INTERFACE
			\item \Verb{ip link set SLAVE master INTERFACE}  – włączenie interfejsu SLAVE w skład bondingu INTERFACE
			\item \Verb{ip link set SLAVE nomaster} - wyłaczenie interfejsu SLAVE z bondingu
			\item \Verb{ip link show master INTERFACE} – wyświetlanie portów składowych interfejsu bondingowego o nazwie INTERFACE
		\end{itemize}
	\item konfiguracja routingu
		\begin{itemize}
			\item \Verb{ip [-6] route} – wyświetlanie informacji na temat tras routingowych dla IPv4 (gdy wywołany bez opcji \Verb{-6}) / IPv6 (gdy wywołany z opcją \Verb{-6})
			\item \Verb{ip route add NETWORK via GATEWAY dev INTERFACE} – dodanie trasy routingowej do sieci \Verb{NETWORK} poprzez router o adresie \Verb{GATEWAY} na interfejsie \Verb{INTERFACE}
			\item \Verb{ip route del NETWORK via GATEWAY dev INTERFACE} – usunięcie trasy routingowej do sieci \Verb{NETWORK} ...
		\end{itemize}
\end{itemize}

Często dostępne są także klasyczne polecenia:
\begin{itemize}
	\item \Verb{ifconfig}
		włączanie i wyłączanie interfejsów sieciowych (up i down), ustawianie adresu IP i wyświetlanie informacji o interfejsach.
	\item \Verb{route}
		konfiguracja tras routingowych
	\item \Verb{vconfig}
		dodawanie i usuwanie obsługi wskazanych VLANów z danego interfejsu
	\item \Verb{brctl}
		konfiguracja programowego switcha ethernetowego pomiędzy interfejsami (bridge)
	\item \Verb{ifenslave}
		konfiguracja bondów
\end{itemize}

Innym przydatnym poleceniem z pakietu iproute2 jest \Verb#tc#, które służy do konfiguracji ustawień kontroli przepływu (np. kolejkowania) na interfejsach sieciowych.

Warto zaznaczyć iż konfiguracja dokonywana poleceniami takimi jak \Verb#ip#, \Verb#tc#, \Verb#ifconfig#, itp. jest konfiguracją typu \textit{runtime}, czyli jest tracona po wyłączeniu systemu.
Aby konfiguracja sieci była trwała należy polecenia takie zapisać w postaci skryptu uruchamianego przy starcie systemu lub skorzystać z systemowych plików konfiguracyjnych związanych z siecią.

\insertZadanie{booklets-sections/network/zadania-50_60.tex}{ustaw_adres}{}
\insertZadanie{booklets-sections/network/zadania-50_60.tex}{ustaw_route}{}

\subsection{Konfiguracja DNS}

Za zamianę nazw domenowych na adresy IP odpowiadają funkcje biblioteki standardowej C. Korzysta ona do tego celu z konfiguracji zawartej w pliku \Verb#/etc/resolv.conf#.
Powinien on zawierać co najmniej jeden wpis postaci \Verb#nameserver ip_serwera_dns#, określający serwer rozwiązujący nazwy DNS do którego będziemy kierować nasze zapytania.
Wpisów tych może być kilka co pozwala na określenie serwerów uzywany w przypadku niedostępności podstawowego (obecnie używane są maksymalnie 3).

Dodatkowo plik ten może posiadać wpisy
	\Verb#domain# określający domenę lokalną (jeżeli nie jest tu określona a hostname zawiera domenę to używana jest ta z hostname; jeżeli nie chcemy używać można określić na \Verb#.#) oraz
	\Verb#search# określający listę domen do przeszukiwania.
Określają one domeny, króre będą dodawane jako surfix do domeny o którą się pytamy. Na przykład gdy mamy \Verb#domain abc.def#, a pytamy się o \Verb#xyz# (bez kropki w środku lub na końcu), biblioteka najpierw spróbuje ustalić adres \Verb#xyz.abc.def.# a potem \Verb#xyz.#

Plik ten pozwala ustawić także inne opcje związane z odpytywaniem DNS - szczegóły w \Verb#man 5 resolv.conf#.

Innym plikiem związanym z rozwijaniem nazw jest \Verb#/etc/hosts#, który stanowi bazę mapowań nazw na numery IP.
Jest on użyteczny dla lokalnie definiowanych nazw i adresów.
Wpisy w nim zawarte mają priorytet wyższy od informacji z DNS (jeżeli host został znaleziony w tym pliku nie jest wykonywane zapytanie do serwera rozwijającego DNS).


\subsection{konfiguracja automatyczna}

W zależności od ustawień sieci do której podłączony jest konfigurowany host możliwe jest także skorzystanie z konfiguracji automatycznej DHCP i/lub autokonfiguracji IPv6.

\subsubsection{DHCP}

DHCP jest protokołem typu klient-serwer, pozwalającym klientowi uzyskać informacje na temat konfiguracji sieci takie jak adres ip, długość prefixu, trasy routingowe (w szczególności adres bramki domyślnej), adresy serwerów DNS zarówno dla IPv4, jak i IPv6.

Do pobrania konfiguracji z serwera DHCP i jej ustawienia służy najczęściej polecenie \Verb#dhclient# (dostępne są inne implementacje klienta dhcp, np: \Verb#udhcpc#, \Verb#dhcpcd#).
Z ważniejszych opcji należy wspomnieć o:\\
	\hspace*{1cm} \Verb#-6# – korzystanie z DHCPv6, czyli DHCP dla protokołu IPv6,\\
	\hspace*{1cm} \Verb#-n# – nie ustawianie / używanie pobranej konfiguracji,\\
	\hspace*{1cm} \Verb#-d# – nie przechodzenie w tło (włącza też \Verb#-v#),\\
	\hspace*{1cm} \Verb#-v# – wypisywanie większej informacji o działaniu programu.

Dostępne są też różne narzędzia diagnostyczne związane z DHCP, np: \Verb#dhcping#, \Verb#dhcp-probe#.
Linux może pełnić także funkcję serwera DHCP, przy pomocy aplikacji takich jak np.: 
	\textit{isc-dhcp-server}, \textit{udhcpd}, \textit{dnsmasq}, \textit{odhcp6c}, \textit{dhcpy6d}, \textit{wide-dhcpv6}.

\subsubsection{IPv6 autoconf}

Innym sposobem automatycznej konfiguracji interfejsów sieciowych, wprowadzonym w IPv6 jest autokonfiguracja w oparciu o adresy link-local generowane w oparciu o MAC adres karty sieciowej.
	Polega ona na tym że dla podsieci będących LAN'em przydzielana jest pula z maską /64 co umożliwia tworzenie unikalnych numerów IP w oparciu o (niepowtarzalne) numery sprzętowe MAC.
	64 bitowy prefiks sieci jest informacją rozgłaszaną przy pomocy ICMPv6 przez routery (mechanizm radvd), a host dokleja do niego część go identyfikującą związaną z adresem link-local.
	Radvd rozgłasza także informacje routingowe (takie jak adres bramy - dhcpv6 tego nie potrafi), niestety nie da się rozgłaszać w ten sposób innej od standardowej dla LAN długości prefixu.

Linux domyślnie ma włączony ten mechanizm, można go jednak wyłączyć poprzez \Verb#echo 0 > /proc/sys/net/ipv6/conf/${IFACE}/autoconf#, gdzie \Verb#${IFACE}# oznacza interfejs na którym chcemy wyłączyć ten mechanizm.


\subsection{Konfiguracja w proc}

Konieczne / przydatne może być dokonywanie pewnych ustawień poprzez jądrowe systemy plików \Verb$/proc$ i \Verb$/sys$.
Najczęstszym przypadkiem jest włączenie przekazywania pakietów pomiędzy interfejsami poprzez:

\begin{minted}{bash}
for f in /proc/sys/net/ipv*/conf/*/forwarding; do echo 1 > $f; done
\end{minted}
(powyższy jednolinijkowiec włącza forwading pakietów IP dla IPv4 i IPv6 na wszystkich interfejsach)

Innym przykładem jest pokazane wcześniej wyłączenie automatycznej konfiguracji IPv6, przydatne gdy chcemy korzystać tylko z ręcznie przydzielanych adresów.

Z opisem poszczególnych ustawień w ramach systemu \Verb$/proc/sys$ (w tym tych poświęceonych obsłudze sieci z \Verb$/proc/sys/net$) można zapoznać się m.in. na stronie \url{https://sysctl-explorer.net/}.


\subsection{Filtracja pakietów}

Oprócz wyżej omówionej konfiguracji interfejsów i tras routingowych, często potrzebna jest konfiguracja jądrowych mechanizmów filtracji pakietów.

Filtracja pakietów umożliwia m.in. ignorowanie (\textit{drop}) lub odrzucenie z komunikatem błędu (\textit{reject}) niepożądanego ruchu sieciowego
	– zarówno wchodzącego, wychodzącego, jak i przekazywanego (jeżeli uruchomiona jest funkcjonalność routera poprzez wpisanie wartości 1 do \Verb$/proc/sys/net/ipv*/conf/*/forwarding$).
Pozwala także na śledzenie połączeń (np. w celu innego traktowania połączeń nawiązanych niż nowych)
	i manipulowanie przechodzącymi pakietami (np. modyfikację adresów IP i numerów portów w ramach mechanizmów \textit{snat} modyfikującego adresy źródłowe i \textit{dnat} modyfikujące adresy docelowe).

Do konfiguracji filtracji pakietów służy polecenie \Verb#nft#.
Na starszych systemach \Verb#nft# może być niedostępny, wtedy można korzystać z poleceń:
\begin{itemize}
	\item \Verb{iptables}, \Verb{ip6tables}
		konfiguracja filtrów działających na pakietach IP (\Verb{iptables} dla IPv4, \Verb{ip6tables} dla IPv6), filtracja może odbywać się m.in. w oparciu o źródłowe i docelowe adresy IP, numery portów, protokół warstwy transportowej, interfejsy oraz mechanizm śledzenia połączeń; umożliwia także konfigurację translacji adresów (NAT).
	\item \Verb{ebtables}
		konfiguracja filtrów działających na poziomie switcha ethernetowego, filtracja może odbywać się m.in. w oparciu o źródłowe i docelowe interfejsy i adresy sprzętowe.
	\item \Verb{arptables}
		konfiguracja filtrów związanych z protokołem ARP (zamiany adresów IP na adresy sprzętowe)
\end{itemize}

Konfiguracja filtracji pakietów dokonywana z użyciem poleceń \Verb#nft#, \Verb#iptables# jest konfiguracją \textit{runtime} i jest tracona po wyłączeniu systemu.
