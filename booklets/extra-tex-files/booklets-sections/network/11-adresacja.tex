% Copyright (c) 2017-2020 Matematyka dla Ciekawych Świata (http://ciekawi.icm.edu.pl/)
% Copyright (c) 2017-2020 Robert Ryszard Paciorek <rrp@opcode.eu.org>
% 
% MIT License
% 
% Permission is hereby granted, free of charge, to any person obtaining a copy
% of this software and associated documentation files (the "Software"), to deal
% in the Software without restriction, including without limitation the rights
% to use, copy, modify, merge, publish, distribute, sublicense, and/or sell
% copies of the Software, and to permit persons to whom the Software is
% furnished to do so, subject to the following conditions:
% 
% The above copyright notice and this permission notice shall be included in all
% copies or substantial portions of the Software.
% 
% THE SOFTWARE IS PROVIDED "AS IS", WITHOUT WARRANTY OF ANY KIND, EXPRESS OR
% IMPLIED, INCLUDING BUT NOT LIMITED TO THE WARRANTIES OF MERCHANTABILITY,
% FITNESS FOR A PARTICULAR PURPOSE AND NONINFRINGEMENT. IN NO EVENT SHALL THE
% AUTHORS OR COPYRIGHT HOLDERS BE LIABLE FOR ANY CLAIM, DAMAGES OR OTHER
% LIABILITY, WHETHER IN AN ACTION OF CONTRACT, TORT OR OTHERWISE, ARISING FROM,
% OUT OF OR IN CONNECTION WITH THE SOFTWARE OR THE USE OR OTHER DEALINGS IN THE
% SOFTWARE.

% BEGIN: adresacja IP
\subsection{Adresacja IP}

Adresy hostów (nazywane adresami IP) są to 32-bitowe (w IPv4) lub 128-bitowe (w IPv6) liczby.
Adresy IPv4 zapisywane są najczęściej w notacji kropkowo-dziesiętnej, gdzie każdy bajt (ciąg 8 bitów) zapisywany jest jako liczba dziesiętna rozdzielana kropką od pozostałych. Adresy IPv6 zapisywane są zazwyczaj w notacji dwukropokowej, polegającej na zapisywaniu 16 bitowych części adresu liczbami szesnastkowymi oddzielanymi dwukropkiem, dodatkowo jeden ciąg zer (o długości będącej wielokrotnością 16 bitów) może być skompresowany (pominięty) co daje w zapisie dwa dwukropki \Verb$::$.

\subsubsection{Długość prefixu i maska}

Adresy hostów grupuje się w adresy sieci, bazując na jednakowym (bitowo) początku takiego adresu (zwanym adresem sieci lub prefixem). Ilość bitów stanowiących adres sieci w danym adresie IP nazywana jest długością prefixu i zapisywana jest zazwyczaj po ukośniku\footnote{
	Jest to notacja \textit{CIDR}. Przed wprowadzeniem tego mechanizmu w IPv4 funkcjonował klasowy sposób routingu (\textit{classful}), gdzie wielkość maski była deteminowana wartością pierwszych bitów adresu – ale to już historia.
}. Na przykład zapis \Verb$2001:db8::a17/48$ oznacza że pierwsze 48 bity stanowią adres sieci a kolejne $128-48 = 80$ bitów stanowi adres hosta w tej sieci.

Długość prefixu jednoznacznie określa maskę danej podsieci, czyli liczbę odpowiadającą długości adresu (32 bity lub 128 bitów), złożoną z ciągu jedynek o długości prefixu oraz ciągu zer (o długości adresu hosta). W przypadku IPv4 spotykane jest także podawanie maski sieci w notacji kropkowo-dziesiętnej zamiast długości prefixu.

\begin{Verbatim}[frame=single,label=\textrm{\bfseries Przykład},commandchars=\\\{\},commentchar=\%]
adres IPv4 zapisany z informacją o długości prefixu: 10.23.45.56/{\color{violet}13}, czyli:
adres: 10.23.45.56 = {\color{blue}0000101000010}{\color{brown}1110010110100111000}
maska: 255.248.0.0 = {\color{blue}1111111111111}{\color{brown}0000000000000000000}
                     {\color{blue}prefix sieci }{\color{brown}adres hosta w sieci}
adres sieci        = {\color{blue}0000101000010}{\color{brown}0000000000000000000} = 10.16.0.0\vspace{4pt}
{\footnotesize{}Adres sieci obliczany jest jako bitowy AND pomiędzy adresem i maską.}\vspace{4pt}
{\footnotesize{}Maska zawsze jest złożona z ciągu samych bitów o wartości {\color{blue}1} a następnie o wartości {\color{brown}0}.}
{\footnotesize{}Bitów o wartości {\color{blue}1} jest tyle ile wynosi długość prefixu (podawana po /), czyli w tym przykładzie {\color{violet}13}.}\vspace{4pt}
{\footnotesize{}W IPv6 działa to analogicznie, tyle że adres ma 128 bitów długości, stosuje się notacje dwukropkową}
{\footnotesize{}zamiast kropkowo-dziesiętnej i nie stosuje się jawnego zapisu maski (a jedynie długość prefixu).}
\end{Verbatim}

Sieć może zostać podzielona na mniejsze sieci (z większą wartością prefixu), jak też grupa sieci może zostać zagregowana w jedną większą ($2^n$ raza) sieć (z prefixem mniejszym o n). Agregacja hostów i sieci w większe całości jest wykorzystywana w mechanizmach routingu, co pozwala na redukcję wielkości tablic routingowych.

\subsubsection{Przynależność do sieci}
Adres sieci zapisuje się typowo z wyzerowanymi bitami stanowiącymi adres hosta (czyli po dokonaniu bitowego \emph{and} z maską danej sieci) oraz podaną informacją o długości prefixu, dla powyższego przykładu będzie to \Verb$2001:db8::/48$. Informacja taka jest wystarczająca do sprawdzenia czy dowolny inny adres IP należy do tej sieci czy nie.

\label{czy_w_sieci_kod}\begin{CodeFrame*}[python][fontsize=\footnotesize]{}
from ipaddress import *

# adres

adr     = ip_interface("2001:0db8::17:15")
adr_int = int(adr.ip)
print("Adress IPv6 jest 128 bitową liczbą całkowitą np.:")
print("  " + str(adr.ip) + " == " + hex(adr_int) + "\n")

# sieć - maska i długość prefixu

net         = ip_interface("::/112");
netmask     = net.network.netmask
netmask_int = int(netmask)
net_preflen = net.network.prefixlen

print("Maska podsieci IPv6 jest 128 bitową liczbą całkowitą np.:")
print("  " + str(netmask) + " == " + hex(netmask_int) + "\n")
print("Jako że maska jest liczbą, która zapisana binarnie, zawsze zawiera ciągły ciąg bitów")
print("o wartości 1, a po nim ciągły ciąg bitów o wartości 0 (mogą być zerowej długości), to")
print("często stosowany jest zapis polegający na podawaniu długości prefiksu: /" + str(net_preflen))
print("jest to ilość bitów o wartości 1 w masce, czyli im większy prefix tym mniejsza sieć.\n")

# adres w sieci

adr2     = ip_interface("2001:0db8::17:15/112");
net2     = adr2.network
net2_int = int(net2.network_address)

print("Aby obliczyć adres sieci (czyli wspólną dla wszystkich hostów w danej sieci część")
print("adresu IP) należy wykonać binarny AND pomiędzy adresem IP hosta a maską podsieci.")
print("Dla powyższego przykładu:")
print("  " + hex(netmask_int & adr_int) + " == " + str(net2) + " == " + hex(net2_int) + "\n")

# aby sprawdzić czy adres IP należy do danej sieci trzeba obliczyć adres sieci tego hosta
# w oparciu o maskę sieci którą sprawdzamy
def sprawdzSiec(n, a):
	nn = int(a) & int(n.netmask)
	if nn == int(n.network_address):
		print(str(a) + " należy do sieci " + str(n))
	else:
		print(str(a) + " NIE należy do sieci " + str(n))

sprawdzSiec(net2, ip_interface("2001:0db8::17:ab13").ip)
sprawdzSiec(net2, ip_interface("2001:0db8::13:a").ip)
\end{CodeFrame*}
% END: adresacja IP
