% Copyright (c) 2017-2020 Matematyka dla Ciekawych Świata (http://ciekawi.icm.edu.pl/)
% Copyright (c) 2017-2020 Robert Ryszard Paciorek <rrp@opcode.eu.org>
% 
% MIT License
% 
% Permission is hereby granted, free of charge, to any person obtaining a copy
% of this software and associated documentation files (the "Software"), to deal
% in the Software without restriction, including without limitation the rights
% to use, copy, modify, merge, publish, distribute, sublicense, and/or sell
% copies of the Software, and to permit persons to whom the Software is
% furnished to do so, subject to the following conditions:
% 
% The above copyright notice and this permission notice shall be included in all
% copies or substantial portions of the Software.
% 
% THE SOFTWARE IS PROVIDED "AS IS", WITHOUT WARRANTY OF ANY KIND, EXPRESS OR
% IMPLIED, INCLUDING BUT NOT LIMITED TO THE WARRANTIES OF MERCHANTABILITY,
% FITNESS FOR A PARTICULAR PURPOSE AND NONINFRINGEMENT. IN NO EVENT SHALL THE
% AUTHORS OR COPYRIGHT HOLDERS BE LIABLE FOR ANY CLAIM, DAMAGES OR OTHER
% LIABILITY, WHETHER IN AN ACTION OF CONTRACT, TORT OR OTHERWISE, ARISING FROM,
% OUT OF OR IN CONNECTION WITH THE SOFTWARE OR THE USE OR OTHER DEALINGS IN THE
% SOFTWARE.

\dbEntryBegin{rfc1924}\if1\dbEntryCheckResults
Zapoznaj się z RFC1924 i napisz program konwertujący adresy IPv6 pomiędzy notacją dwukropkową a notacją base-85 zgodną z tą specyfikacją.

\textit{Wskazówka: do odczytu adresu w standardowej notacji dwukropkowej możesz użyć narzędzi z modułu \texttt{ipaddress}}
\fi

% PwES domowe (?):

\dbEntryBegin{czy_w_sieci_ipv4}\if1\dbEntryCheckResults
Ustal czy host o adresie IPv4 192.168.65.20 należy do sieci 192.168.33.15/19.
\fi

\dbEntryBegin{czy_w_sieci_ipv6}\if1\dbEntryCheckResults
Ustal czy host o adresie IPv6 2001:6a0:0:21::60:2 należy do sieci 2001:6a0:0:10::/58.
\fi


\dbEntryBegin{adresy_serwerow_dns}\if1\dbEntryCheckResults
Ustal adresy serwerów DNS posiadających informację o domenie \emph{gov}. Podaj polecenie którego użyłeś.
\fi

\dbEntryBegin{trasy_pakietow}\if1\dbEntryCheckResults
Polecenie \Verb#ip r# pokazało następują tablicę routingu:

\begin{Verbatim}
default via 192.168.29.2 dev eth0.2 
192.168.29.192/27 dev eth0.2  proto kernel  scope link  src 192.168.29.193
172.16.16.0/27 via 172.16.18.2 dev tun5 
172.16.16.48/28 dev wlan0  proto kernel  scope link  src 172.16.16.49 
172.16.18.0/30 dev tun5  proto kernel  scope link  src 172.16.18.1 
192.168.29.0/24 dev eth0  proto kernel  scope link  src 192.168.29.1 
\end{Verbatim}
Ustal trasę (urządzenie którym zostanie wysłany pakiet oraz jeżeli jest potrzebny to adres routera do którego będzie przesyłany) dla następujacych adresów IP:
\begin{itemize}
	\item 8.8.8.8
	\item 192.168.29.202
	\item 172.16.16.15
\end{itemize}
\fi

\dbEntryBegin{ustaw_adres}\if1\dbEntryCheckResults
Napisz polecenie które ustawi adres ip \Verb#172.33.13.113# (maska sieci to \Verb#255.255.255.0#) na interfejsie \Verb#eth5#.
\fi

\dbEntryBegin{ustaw_route}\if1\dbEntryCheckResults
Napisz polecenie które ustawi trasę routingową do sieci \Verb#10.13.0.0/16# przez bramkę o adresie ip \Verb#172.33.13.13#.
\fi

\dbEntryBegin{wlacz_forward}\if1\dbEntryCheckResults
Napisz polecenia które włączą przekazywanie pakietów (routing) pomiędzy interfejsami \Verb#eth3# i \Verb#eth4#, ale nie zezwolą na przekazywanie pakietów innymi interfejsami
\fi

\dbEntryBegin{ustaw_route}\if1\dbEntryCheckResults
Napisz serwer UDP lub TCP (określ który wariant wybrałeś/wybrałaś), który na ciąg znaków \texttt{ip} wysłany przez klienta odeśle do niego informację o jego numerze IP.
\fi
