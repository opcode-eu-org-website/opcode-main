% Copyright (c) 2017-2020 Matematyka dla Ciekawych Świata (http://ciekawi.icm.edu.pl/)
% Copyright (c) 2017-2020 Robert Ryszard Paciorek <rrp@opcode.eu.org>
% 
% MIT License
% 
% Permission is hereby granted, free of charge, to any person obtaining a copy
% of this software and associated documentation files (the "Software"), to deal
% in the Software without restriction, including without limitation the rights
% to use, copy, modify, merge, publish, distribute, sublicense, and/or sell
% copies of the Software, and to permit persons to whom the Software is
% furnished to do so, subject to the following conditions:
% 
% The above copyright notice and this permission notice shall be included in all
% copies or substantial portions of the Software.
% 
% THE SOFTWARE IS PROVIDED "AS IS", WITHOUT WARRANTY OF ANY KIND, EXPRESS OR
% IMPLIED, INCLUDING BUT NOT LIMITED TO THE WARRANTIES OF MERCHANTABILITY,
% FITNESS FOR A PARTICULAR PURPOSE AND NONINFRINGEMENT. IN NO EVENT SHALL THE
% AUTHORS OR COPYRIGHT HOLDERS BE LIABLE FOR ANY CLAIM, DAMAGES OR OTHER
% LIABILITY, WHETHER IN AN ACTION OF CONTRACT, TORT OR OTHERWISE, ARISING FROM,
% OUT OF OR IN CONNECTION WITH THE SOFTWARE OR THE USE OR OTHER DEALINGS IN THE
% SOFTWARE.

\subsubsection{iptables}

\teacher{Nie omawiamy na zajęciach. Iptables odchodzi do przeszłości. Może w ogóle usunąć ze skryptu?}

Iptables wykorzystuje kilka tablic reguł (najistotniejszymi są \Verb{filter} i \Verb{nat}). Tablica może zostać określona przy pomocy opcji \Verb{-t}, jeżeli nie użyto tej opcji operacje będą wykonywane na tablicy \Verb{filter}. Zależności pomiędzy poszczególnymi łańcuchami i tablicami przedstawia (uproszczony) diagram przejścia pakietu przez mechanizm iptables:

\begin{figure}[h!]
\ifdefined\inputOnlyContent\else
	\documentclass[tikz]{standalone}
	\usetikzlibrary{positioning,arrows.meta}
	\begin{document}
\fi

\begin{tikzpicture}[->, >={Stealth[length=8pt,width=6pt]}, node distance=0.6cm, semithick]
	\tikzstyle{invisibleNode}=[inner sep=0, outer sep = 0pt, minimum size=0]
	\tikzstyle{info1}=[above, align=center]
	\tikzstyle{info2}=[below=0.8cm, align=center]
	\tikzstyle{base}=[align=center, minimum height=3.3em, minimum width=8.6em]
	\tikzstyle{inout}=[base]
	\tikzstyle{routing}=[draw, base]
	\tikzstyle{table_raw}=[draw, base]
	\tikzstyle{table_mangle}=[draw, base]
	\tikzstyle{table_nat}=[draw, base]
	\tikzstyle{table_filter}=[draw, base]
	
	\node[inout] (INPUT) {pakiet\\przychodzący};
	\node[table_raw] (RAW_PREROUTING) [below = of INPUT] {raw\\PREROUTING};
	\node[routing] (TRACK1) [below = of RAW_PREROUTING] {śledzenie\\stanu połączeń};
	\node[table_mangle] (MANGLE_PREROUTING) [below = of TRACK1] {mangle\\PREROUTING};
		\node[table_nat] (NAT_PREROUTING) [right = 2.5cm of MANGLE_PREROUTING] {nat\\PREROUTING};
		\node[routing] (ROUTING1) [below = of NAT_PREROUTING] {wybór trasy\\routingowej};
	
	\node[table_mangle] (MANGLE_INPUT) [below = 2.7cm of MANGLE_PREROUTING] {mangle\\INPUT};
		\node[table_mangle] (MANGLE_FORWARD) [below = 1.7cm of ROUTING1] {mangle\\FOWRARD};
	\node[table_filter] (FILTER_INPUT) [below = of MANGLE_INPUT] {filter\\INPUT};
	\node[table_nat] (NAT_INPUT) [below = of FILTER_INPUT] {nat\\INPUT};
	\node[inout] (RECIVE) [below = of NAT_INPUT] {przetwarzanie\\lokalne};
		\node[table_filter] (FILTER_FORWARD) [below = of MANGLE_FORWARD] {filter\\FORWARD};
	
	\node[inout] (GENERATE) [right = 7.6cm of INPUT] {pakiet\\generowany lokalnie};
	\node[routing] (ROUTING2) [below = of GENERATE] {wybór trasy\\routingowej};
	\node[table_raw] (RAW_OUTPUT) [below = of ROUTING2] {raw\\OUTPUT};
	\node[routing] (TRACK2) [below = of RAW_OUTPUT] {śledzenie\\stanu połączeń};
	\node[table_mangle] (MANGLE_OUTPUT) [below = of TRACK2] {mangle\\OUTPUT};
	\node[table_nat] (NAT_OUTPUT) [below = of MANGLE_OUTPUT] {nat\\OUTPUT};
	\node[routing] (ROUTING3) [below = of NAT_OUTPUT] {wybór trasy\\routingowej};
	\node[table_filter] (FILTER_OUTPUT) [below = of ROUTING3] {filter\\OUTPUT};
	\node[table_mangle] (MANGLE_POSTROUTING) [below = of FILTER_OUTPUT] {mangle\\POSTROUTING};
	
	\node[table_nat] (NAT_POSTROUTING) [below = 1.7cm of MANGLE_POSTROUTING] {nat\\POSTROUTING};
	\node[inout] (OUTPUT) [below = of NAT_POSTROUTING] {pakiet\\wychodzący};
	
	\draw (INPUT) edge (RAW_PREROUTING);
	\draw (RAW_PREROUTING) edge (TRACK1);
	\draw (TRACK1) edge (MANGLE_PREROUTING);
		\draw (MANGLE_PREROUTING) edge node[info1] {nie z\\localhost} (NAT_PREROUTING);
		\draw (MANGLE_PREROUTING) node[info2] {z localhost} edge (MANGLE_INPUT);
	\draw (NAT_PREROUTING) edge (ROUTING1);
		\draw (ROUTING1) node[info2] {nie do localhost} edge (MANGLE_FORWARD);
		\node[invisibleNode] (ROUTING1a) [left = 3.6cm of ROUTING1] {};
		\draw[-] (ROUTING1) edge node[info1] {do localhost} (ROUTING1a);
		\draw (ROUTING1a) -| (MANGLE_INPUT);
	\draw (MANGLE_INPUT) edge (FILTER_INPUT);
	\draw (FILTER_INPUT) edge (NAT_INPUT);
	\draw (NAT_INPUT) edge (RECIVE);
	
	\draw (MANGLE_FORWARD) edge (FILTER_FORWARD);
	\draw (FILTER_FORWARD) |- (MANGLE_POSTROUTING);
	
	\draw (GENERATE) edge (ROUTING2);
	\draw (ROUTING2) edge (RAW_OUTPUT);
	\draw (RAW_OUTPUT) edge (TRACK2);
	\draw (TRACK2) edge (MANGLE_OUTPUT);
	\draw (MANGLE_OUTPUT) edge (NAT_OUTPUT);
	\draw (NAT_OUTPUT) edge (ROUTING3);
	\draw (ROUTING3) edge (FILTER_OUTPUT);
	\draw (FILTER_OUTPUT) edge (MANGLE_POSTROUTING);
	
	\draw (MANGLE_POSTROUTING) node[info2] {nie do localhost} edge (NAT_POSTROUTING);
	\node[invisibleNode] (MANGLE_POSTROUTINGa) [right = 0.7cm of MANGLE_POSTROUTING] {};
	\node[invisibleNode] (OUTPUTa) [right = 0.7cm of OUTPUT] {};
	\draw[-] (MANGLE_POSTROUTINGa) -| (MANGLE_POSTROUTING);
	\draw[-] (MANGLE_POSTROUTINGa) edge node[below,rotate=90] {do localhost} (OUTPUTa);
	\draw (OUTPUTa) |- (OUTPUT);
	\draw (NAT_POSTROUTING) edge (OUTPUT);
\end{tikzpicture}

\ifdefined\inputOnlyContent\else\end{document}\fi

\caption{Trasa pakietu przez filtry iptables. Wskazano punkty zaczepień nazwy łańcuchów.}
\end{figure}

W każdej z tablic występuje kilka różnych łańcuchów reguł. Każdy łańcuch posiada akcję domyślną, która może zostać ustawiona komendą \Verb{iptables [-t TABLICA] -P ŁAŃCUCH AKCJA}.
Reguły do wskazanego łańcucha (w wskazanej tablicy) mogą być dodawane/usuwane z użyciem komend:
\begin{itemize}
	\item \Verb{iptables [-t TABLICA] -A|-D ŁAŃCUCH REGUŁA} – dodanie (\Verb{-A}) lub usunięcie (\Verb{-D}) reguły
	\item \Verb{iptables [-t TABLICA] -I ŁAŃCUCH POZYCJA REGUŁA} – wstawienie reguły na wskazaną pozycję
	\item \Verb{iptables [-t TABLICA] -F ŁAŃCUCH} – usuniecie wszystkich reguł z łańcucha
\end{itemize}

Reguły składają się ze zbioru dopasowań (filtrów) w postaci opcji do komendy iptables oraz akcji podawanej po opcji \Verb{-j}, do najistotniejszych filtrów należą:
\begin{itemize}
	\item \Verb{-s ADRES} – pasuje gdy adres źródłowy w pakiecie zgadza się z podaną siecią IP (lub pojedynczym adresem)
	\item \Verb{-d ADRES} – pasuje gdy adres docelowy w pakiecie zgadza się z podaną siecią IP (lub pojedynczym adresem)
	\item \Verb{-p PROTOKÓŁ --dport PORT} – pasuje gdy pakiet zawiera w sobie pakiet wskazanego protokołu (np. tcp, udp) i adresowany jest na wskazany numer portu
	\item \Verb{-i INTERFEJS} – pasuje gdy pakiet przyszedł wskazanym interfejsem sieciowym
	\item \Verb{-o INTERFEJS} – pasuje gdy pakiet wychodzi wskazanym interfejsem sieciowym
\end{itemize}

Najistotniejszymi akcjami jest ACCEPT (zaakceptowanie/przepuszczenie pakietu przez łańcuch), REJECT (odrzucenie pakietu z wygenerowaniem komunikatu błędu poprzez ICMP), DROP (zapomnienie o pakiecie / ciche zignorowanie) oraz LOG (zapisanie informacji do logu).

Przykład konfiguracji iptables:

\begin{CodeFrame*}[bash]{}
# polityki domyślne
iptables -P INPUT DROP
iptables -P FORWARD ACCEPT
iptables -P OUTPUT ACCEPT

# interfejs lokalny oraz połączenia nawiązane
iptables -A INPUT -i lo -j ACCEPT
iptables -A INPUT -m state --state ESTABLISHED -j ACCEPT
iptables -A INPUT -m state --state INVALID -j REJECT

# SSH
iptables -A INPUT -p tcp --dport ssh -s 0.0.0.0/0 -j sshguard
iptables -A INPUT -p tcp --dport ssh -s 0.0.0.0/0 -j ACCEPT

## ICMP
iptables -A INPUT -p icmp -j ACCEPT

## RESZTA
iptables -A INPUT -j REJECT
\end{CodeFrame*}

Do wyświetlenia wszystkich reguł można użyć komendy \Verb{iptables-save} / \Verb{ip6tables-save}. Generuje ona skrypt który może zostać wczytany przy pomocy \Verb{iptables-restore} / \Verb{ip6tables-restore}.
