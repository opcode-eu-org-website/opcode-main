% Copyright (c) 2017-2020 Matematyka dla Ciekawych Świata (http://ciekawi.icm.edu.pl/)
% Copyright (c) 2017-2020 Robert Ryszard Paciorek <rrp@opcode.eu.org>
% 
% MIT License
% 
% Permission is hereby granted, free of charge, to any person obtaining a copy
% of this software and associated documentation files (the "Software"), to deal
% in the Software without restriction, including without limitation the rights
% to use, copy, modify, merge, publish, distribute, sublicense, and/or sell
% copies of the Software, and to permit persons to whom the Software is
% furnished to do so, subject to the following conditions:
% 
% The above copyright notice and this permission notice shall be included in all
% copies or substantial portions of the Software.
% 
% THE SOFTWARE IS PROVIDED "AS IS", WITHOUT WARRANTY OF ANY KIND, EXPRESS OR
% IMPLIED, INCLUDING BUT NOT LIMITED TO THE WARRANTIES OF MERCHANTABILITY,
% FITNESS FOR A PARTICULAR PURPOSE AND NONINFRINGEMENT. IN NO EVENT SHALL THE
% AUTHORS OR COPYRIGHT HOLDERS BE LIABLE FOR ANY CLAIM, DAMAGES OR OTHER
% LIABILITY, WHETHER IN AN ACTION OF CONTRACT, TORT OR OTHERWISE, ARISING FROM,
% OUT OF OR IN CONNECTION WITH THE SOFTWARE OR THE USE OR OTHER DEALINGS IN THE
% SOFTWARE.

\subsubsection{iptables}

\teacher{Nie omawiamy na zajęciach. Iptables odchodzi do przeszłości. Może w ogóle usunąć ze skryptu?}

Iptables wykorzystuje kilka tablic reguł (najistotniejszymi są \Verb{filter} i \Verb{nat}). Tablica może zostać określona przy pomocy opcji \Verb{-t}, jeżeli nie użyto tej opcji operacje będą wykonywane na tablicy \Verb{filter}. Zależności pomiędzy poszczególnymi łańcuchami i tablicami przedstawia (uproszczony) diagram przejścia pakietu przez mechanizm iptables:

\begin{figure}[h!]\begin{center}\begin{adjustbox}{scale=.9}
\inputFileContent{booklets-sections/network/ilustracje/52-iptables.tex}
\end{adjustbox}\end{center}
\caption{Trasa pakietu przez filtry iptables. Wskazano punkty zaczepień nazwy łańcuchów.}
\end{figure}

W każdej z tablic występuje kilka różnych łańcuchów reguł. Każdy łańcuch posiada akcję domyślną, która może zostać ustawiona komendą \Verb{iptables [-t TABLICA] -P ŁAŃCUCH AKCJA}.
Reguły do wskazanego łańcucha (w wskazanej tablicy) mogą być dodawane/usuwane z użyciem komend:
\begin{itemize}
	\item \Verb{iptables [-t TABLICA] -A|-D ŁAŃCUCH REGUŁA} – dodanie (\Verb{-A}) lub usunięcie (\Verb{-D}) reguły
	\item \Verb{iptables [-t TABLICA] -I ŁAŃCUCH POZYCJA REGUŁA} – wstawienie reguły na wskazaną pozycję
	\item \Verb{iptables [-t TABLICA] -F ŁAŃCUCH} – usuniecie wszystkich reguł z łańcucha
\end{itemize}

Reguły składają się ze zbioru dopasowań (filtrów) w postaci opcji do komendy iptables oraz akcji podawanej po opcji \Verb{-j}, do najistotniejszych filtrów należą:
\begin{itemize}
	\item \Verb{-s ADRES} – pasuje gdy adres źródłowy w pakiecie zgadza się z podaną siecią IP (lub pojedynczym adresem)
	\item \Verb{-d ADRES} – pasuje gdy adres docelowy w pakiecie zgadza się z podaną siecią IP (lub pojedynczym adresem)
	\item \Verb{-p PROTOKÓŁ --dport PORT} – pasuje gdy pakiet zawiera w sobie pakiet wskazanego protokołu (np. tcp, udp) i adresowany jest na wskazany numer portu
	\item \Verb{-i INTERFEJS} – pasuje gdy pakiet przyszedł wskazanym interfejsem sieciowym
	\item \Verb{-o INTERFEJS} – pasuje gdy pakiet wychodzi wskazanym interfejsem sieciowym
\end{itemize}

Najistotniejszymi akcjami jest ACCEPT (zaakceptowanie/przepuszczenie pakietu przez łańcuch), REJECT (odrzucenie pakietu z wygenerowaniem komunikatu błędu poprzez ICMP), DROP (zapomnienie o pakiecie / ciche zignorowanie) oraz LOG (zapisanie informacji do logu).

Przykład konfiguracji iptables:

\begin{CodeFrame*}[bash]{}
# polityki domyślne
iptables -P INPUT DROP
iptables -P FORWARD ACCEPT
iptables -P OUTPUT ACCEPT

# interfejs lokalny oraz połączenia nawiązane
iptables -A INPUT -i lo -j ACCEPT
iptables -A INPUT -m state --state ESTABLISHED -j ACCEPT
iptables -A INPUT -m state --state INVALID -j REJECT

# SSH
iptables -A INPUT -p tcp --dport ssh -s 0.0.0.0/0 -j sshguard
iptables -A INPUT -p tcp --dport ssh -s 0.0.0.0/0 -j ACCEPT

## ICMP
iptables -A INPUT -p icmp -j ACCEPT

## RESZTA
iptables -A INPUT -j REJECT
\end{CodeFrame*}

Do wyświetlenia wszystkich reguł można użyć komendy \Verb{iptables-save} / \Verb{ip6tables-save}. Generuje ona skrypt który może zostać wczytany przy pomocy \Verb{iptables-restore} / \Verb{ip6tables-restore}.
