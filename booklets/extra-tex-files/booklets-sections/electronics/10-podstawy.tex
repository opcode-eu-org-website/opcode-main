% Copyright (c) 2017-2020 Matematyka dla Ciekawych Świata (http://ciekawi.icm.edu.pl/)
% Copyright (c) 2017-2020 Robert Ryszard Paciorek <rrp@opcode.eu.org>
% 
% MIT License
% 
% Permission is hereby granted, free of charge, to any person obtaining a copy
% of this software and associated documentation files (the "Software"), to deal
% in the Software without restriction, including without limitation the rights
% to use, copy, modify, merge, publish, distribute, sublicense, and/or sell
% copies of the Software, and to permit persons to whom the Software is
% furnished to do so, subject to the following conditions:
% 
% The above copyright notice and this permission notice shall be included in all
% copies or substantial portions of the Software.
% 
% THE SOFTWARE IS PROVIDED "AS IS", WITHOUT WARRANTY OF ANY KIND, EXPRESS OR
% IMPLIED, INCLUDING BUT NOT LIMITED TO THE WARRANTIES OF MERCHANTABILITY,
% FITNESS FOR A PARTICULAR PURPOSE AND NONINFRINGEMENT. IN NO EVENT SHALL THE
% AUTHORS OR COPYRIGHT HOLDERS BE LIABLE FOR ANY CLAIM, DAMAGES OR OTHER
% LIABILITY, WHETHER IN AN ACTION OF CONTRACT, TORT OR OTHERWISE, ARISING FROM,
% OUT OF OR IN CONNECTION WITH THE SOFTWARE OR THE USE OR OTHER DEALINGS IN THE
% SOFTWARE.

% BEGIN: Elektronika - intro
Elektronika zajmuje się wytwarzaniem i przetwarzaniem sygnałów w postaci prądów i napięć elektrycznych.
Zjawisko prądu związane jest z przepływem ładunku (z uporządkowanym ruchem nośników ładunku), aby wystąpiło konieczna jest różnica potencjałów (napięcie) pomiędzy końcami przewodnika, prowadzi ono do neutralizacji tej różnicy.
Dlatego dla podtrzymania stałej różnicy potencjałów konieczne jest istnienie źródeł prądu, prowadzących do rozdzielania ładunków dodatnich od ujemnych.

\section{Podstawowe pojęcia}

\subsection{Napięcie elektryczne}
    Napięcie elektryczne $U$ pomiędzy punktem A i B (jakiegoś obwodu)
    jest to różnica potencjału elektrycznego w punkcie A i w punkcie B.
\subsection{Potencjał elektryczny}
    Potencjał elektryczny $V$ w punkcie A
    jest skalarną wielkością charakteryzującą pole elektryczne w danym punkcie. Odpowiada pracy którą trzeba by wykonać aby przenieść ładunek $q$ z tego punktu do nieskończoności podzielonej przez wielkość tego ładunku (jest niezależny od wartości $q$).
    W elektronice używa się wartości potencjałów względem umownego potencjału zerowego GND (co umożliwia traktowanie ich jako różnic potencjałów - napięć elektrycznych), w efekcie tego określenia "(stałe) napięcie" i "potencjał" bardzo często stosowane są zamiennie. 
\subsection{Masa}
   Masa (oznaczana jako GND) jest to
   umowny potencjał zerowy, względem którego wyraża się inne potencjały w układzie (co umożliwia traktowanie ich jako różnic potencjałów - napięć elektrycznych). Potencjał ten może być równy potencjałowi ziemi (masie ochronnej PE), bądź może być z nim nie związany (układy izolowane).

\begin{ProTip}{Schematy elektryczne wg. elektronika}
Typowo elektronicy na schematach nie rysują źródeł napięcia (np. w postaci symbolu baterii\footnote{chyba że chodzi o podkreślenie, iż dane zasilanie faktycznie odbywa się z baterii lub akumulatora}),
zamiast tego umieszczają znaczniki potencjałów zasilania (np. +5V, +3V3, Vcc, Vbus) względem masy i znaczniki masy (GND, {\Symbola ⏚}).

\vspace{3pt}Typowo potencjały wyższe umieszcza się na schemacie wyżej a niższe niżej (czyli 5V będzie na górze, a GND na dole), a przepływ prąd odbywa się w relacji od lewej do prawej i z góry na dół.
Jest to ogólna reguła, ułatwiająca czytanie schematów, nie jest ona jednak wyrocznią i trafiają się od niej odstępstwa, podyktowane zwiększeniem czytelności schematu.

\vspace{3pt}Schematy zamieszczane w tym skrypcie rysowane są według tych zasad.
\end{ProTip}

\subsection{Natężenie prądu}
    Natężenie prądu elektrycznego $I$ (określane skrótowo jako prąd)
    jest to stosunek przemieszczonego ładunku do czasu jego przepływu.

\section{Prawo Ohma}
Dla elementów liniowych (np. zwykły kawałek przewodu) zachodzi proporcjonalność natężenia prądu płynącego przez taki element do napięcia pomiędzy jego końcami: $R=\frac{U}{I}$.
Zależność ta nosi nazwę prawa Ohma\footnote{Prawo Ohma nie jest uniwersalnym prawem przyrody, a jedynie relacją empiryczną spełnioną w pewnym zakresie parametrów dla niektórych materiałów.}, a stosunek ten nazywamy oporem (rezystancję).

\section{Prawa Kirchhoffa}
Węzeł układu (sam w sobie, pomijając zjawiska pasożytnicze) nie jest w stanie gromadzić ładunku elektrycznego zatem: \emph{Suma prądów wpływających do węzła jest równa sumie prądów wypływających z tego węzła.}

Jeżeli rozważamy obwód zamknięty od punktu A z potencjałem $V_A$ to sumując napięcia na kolejnych elementach obwodu (oporach, źródłach napięciowych, etc) z uwzględnieniem ich znaku gdy wrócimy do punktu A to potencjał nadal musi wynosić $V_A$, zatem: \emph{Suma spadków napięć w zamkniętym obwodzie jest równa zeru.}
% END: Elektronika - intro
