% Copyright (c) 2017-2020 Matematyka dla Ciekawych Świata (http://ciekawi.icm.edu.pl/)
% Copyright (c) 2017-2020 Robert Ryszard Paciorek <rrp@opcode.eu.org>
% 
% MIT License
% 
% Permission is hereby granted, free of charge, to any person obtaining a copy
% of this software and associated documentation files (the "Software"), to deal
% in the Software without restriction, including without limitation the rights
% to use, copy, modify, merge, publish, distribute, sublicense, and/or sell
% copies of the Software, and to permit persons to whom the Software is
% furnished to do so, subject to the following conditions:
% 
% The above copyright notice and this permission notice shall be included in all
% copies or substantial portions of the Software.
% 
% THE SOFTWARE IS PROVIDED "AS IS", WITHOUT WARRANTY OF ANY KIND, EXPRESS OR
% IMPLIED, INCLUDING BUT NOT LIMITED TO THE WARRANTIES OF MERCHANTABILITY,
% FITNESS FOR A PARTICULAR PURPOSE AND NONINFRINGEMENT. IN NO EVENT SHALL THE
% AUTHORS OR COPYRIGHT HOLDERS BE LIABLE FOR ANY CLAIM, DAMAGES OR OTHER
% LIABILITY, WHETHER IN AN ACTION OF CONTRACT, TORT OR OTHERWISE, ARISING FROM,
% OUT OF OR IN CONNECTION WITH THE SOFTWARE OR THE USE OR OTHER DEALINGS IN THE
% SOFTWARE.

% bierne i dioda

\dbEntryBegin{zadanie_cewka_i_dioda}\if1\dbEntryCheckResults
Wróć do symulacji związanej z cewką \url{http://ln.opcode.eu.org/cewka} i spróbuj dodać diodę równolegle do cewki taki sposób aby wyeliminować powstawanie przepięć.
Zobacz też że dodanie diody równolegle z elementem przełączającym nie pozwala na rozwiązanie problemu.
\fi

\dbEntryBegin{zadanie_mostek}\if1\dbEntryCheckResults
\noindent\begin{minipage}[b]{0.6\textwidth}
Zbuduj układ przedstawiony na schemacie (nazywany mostkiem Gretza) i zauważ że polaryzacja wyjścia (podłączonego do woltomierza) jest niezależna od polaryzacji wejścia (czyli od tego czy biegun dodatni będzie przyłączony do in1 czy in2, a ujemny odpowiednio in2 lub in1).
\end{minipage}
\hfill
\begin{minipage}[b]{0.35\textwidth}
\includegraphics[width=\textwidth]{img/elektronika/zad-mostek}
\end{minipage}
\fi

% cyfrowe

\dbEntryBegin{zadanie_zidentyfikuj_uklad}\if1\dbEntryCheckResults
\noindent\begin{minipage}[t]{\textwidth}
	\noindent\parbox[b]{0.7\textwidth}{
		Otrzymałeś układ logiczny w obudowie DIP14, o układzie wyprowadzeń pokazanym na rysunku obok\footnote{
			Warto zauważyć że sposób numerowania pinów jest standardowy dla danego typu obudowy, natomiast funkcje poszczególnych pinów różnią się w zależności od danego układu i są opisywane w jego dokumentacji).
		}.
		Nóżki numer 10 i 9 są wejściami pewnej bramki logicznej, której wyjście jest na nóżce numer 8. Sporządź tablicę prawdy dla tej bramki i zidentyfikuj co to za bramka.
		\vspace{0.25cm}
	}\hfill\parbox[b]{0.25\textwidth}{
		\includegraphics[height=0.22\textwidth,angle=90,origin=c]{img/elektronika/DIP14-zadanie}
		\vspace{-0.5cm}
	}
\end{minipage}
\fi
