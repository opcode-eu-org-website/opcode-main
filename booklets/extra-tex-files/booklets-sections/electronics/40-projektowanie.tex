% Copyright (c) 2017-2020 Matematyka dla Ciekawych Świata (http://ciekawi.icm.edu.pl/)
% Copyright (c) 2017-2020 Robert Ryszard Paciorek <rrp@opcode.eu.org>
% 
% MIT License
% 
% Permission is hereby granted, free of charge, to any person obtaining a copy
% of this software and associated documentation files (the "Software"), to deal
% in the Software without restriction, including without limitation the rights
% to use, copy, modify, merge, publish, distribute, sublicense, and/or sell
% copies of the Software, and to permit persons to whom the Software is
% furnished to do so, subject to the following conditions:
% 
% The above copyright notice and this permission notice shall be included in all
% copies or substantial portions of the Software.
% 
% THE SOFTWARE IS PROVIDED "AS IS", WITHOUT WARRANTY OF ANY KIND, EXPRESS OR
% IMPLIED, INCLUDING BUT NOT LIMITED TO THE WARRANTIES OF MERCHANTABILITY,
% FITNESS FOR A PARTICULAR PURPOSE AND NONINFRINGEMENT. IN NO EVENT SHALL THE
% AUTHORS OR COPYRIGHT HOLDERS BE LIABLE FOR ANY CLAIM, DAMAGES OR OTHER
% LIABILITY, WHETHER IN AN ACTION OF CONTRACT, TORT OR OTHERWISE, ARISING FROM,
% OUT OF OR IN CONNECTION WITH THE SOFTWARE OR THE USE OR OTHER DEALINGS IN THE
% SOFTWARE.

\section{Projektowanie \zaawansowane{20}}

\ifdefined\TextRefToDRYKISS\else
\ifdefined\UrlToPythonRefToDRYKISS
	\newcommand{\TextRefToDRYKISS}{, o których była mowa przy omawianiu \href{\UrlToPythonRefToDRYKISS}{bibliotek w pythonie},}
\else
	% Copyright (c) 2018-2020 Matematyka dla Ciekawych Świata (http://ciekawi.icm.edu.pl/)
% Copyright (c) 2018-2020 Robert Ryszard Paciorek <rrp@opcode.eu.org>
% 
% MIT License
% 
% Permission is hereby granted, free of charge, to any person obtaining a copy
% of this software and associated documentation files (the "Software"), to deal
% in the Software without restriction, including without limitation the rights
% to use, copy, modify, merge, publish, distribute, sublicense, and/or sell
% copies of the Software, and to permit persons to whom the Software is
% furnished to do so, subject to the following conditions:
% 
% The above copyright notice and this permission notice shall be included in all
% copies or substantial portions of the Software.
% 
% THE SOFTWARE IS PROVIDED "AS IS", WITHOUT WARRANTY OF ANY KIND, EXPRESS OR
% IMPLIED, INCLUDING BUT NOT LIMITED TO THE WARRANTIES OF MERCHANTABILITY,
% FITNESS FOR A PARTICULAR PURPOSE AND NONINFRINGEMENT. IN NO EVENT SHALL THE
% AUTHORS OR COPYRIGHT HOLDERS BE LIABLE FOR ANY CLAIM, DAMAGES OR OTHER
% LIABILITY, WHETHER IN AN ACTION OF CONTRACT, TORT OR OTHERWISE, ARISING FROM,
% OUT OF OR IN CONNECTION WITH THE SOFTWARE OR THE USE OR OTHER DEALINGS IN THE
% SOFTWARE.

\begin{ProTip}[breakable]{Reguły DRY i KISS}
\textbf{„Don't Repeat Yourself”} (\textit{nie powtarzaj się}) jest jedną z dwóch głównych reguł programistycznych (ale ma także pewne zastosowania w innych dziedzinach techniki).
Zaleca ona unikanie potarzania tych samych czynności, czy też tworzenia takich samych, a nawet analogicznych, podobnych fragmentów kodu.

Narzędziami ułatwiającymi realizację tego celu są m.in.:
\begin{itemize}
\item systemy i skrypty służące automatyzacji różnego rodzaju czynności (takich jak np. kompilacja, instalacja, aktualizacja, monitoring działania) –
      zarówno systemy takie jak make, cmake, doxygen ale również wszystkie drobne skrypty (np. shellowe czy pythonowe) tworzone w tym celu w codziennej pracy informatyka
\item elementy składniowe (m.in. takie jak pętle i funkcje) oraz mechanizmy (np. polimorfizm) dostępne w językach programowania pozwalające na eliminację powtórzeń kodu
\item biblioteki, moduły, itp pozwalające na współdzielenie tych samych rozwiązań, tego samego kodu, pomiędzy różnymi projektami
\item elementy biblioteki systemowej pozwalające na wywoływanie innych programów (np. exec) i komunikację z nimi (np. poprzez strumienie wejścia/wyjścia)
\end{itemize}

Unikanie powtórzeń takiego samego lub (co często nawet gorsze) tylko nieznacznie zmienionego kodu jest też szczególnie istotne ze względu na łatwość utrzymania kodu
– np. jakąś poprawkę wprowadza się tylko w odpowiednio sparametryzowanej funkcji, a nie kilkunastu podobnych (ale nie identycznych, ze względu na brak parametryzacji) fragmentach kodu.

W zastosowaniach nie programistycznych przejawia się często wydzielaniem modułów i dążeniem do ich powtarzalności, redukcji ilości ich typów (np. dzięki parametryzacji, czy konfigurowalności).

\vspace{7pt}

Drugą, nawet chyba ważniejszą, z tych dwóch reguł jest \textbf{„Keep It Simple, Stupid”} (niekiedy \textit{Keep It Small and Simple}), którą można streścić jako \textit{proste jest lepsze}.
Reguła KISS jest bardziej ogólna (można nawet powiedzieć że wynika z niej reguła DRY), posiada dużo szersze pole zastosowań (także nie technicznych) i może być uważana za implementację \textit{Brzytwy Ockhama} w inżynierii.
Zaleca ona m.in.:
\begin{itemize}
\item tworzenie przejrzystych, czytelnych i prostych rozwiązań (zarówno pod względem samego projektu, koncepcji, jak też ich implementacji, wykonania)
\item wybór rozwiązania prostszego spośród (równie) skutecznych rozwiązań jakiegoś problemu
\item myślenie o łatwości późniejszego utrzymania i serwisu tworzonego rozwiązania (czy to kodu programu, czy urządzenia elektronicznego, a nawet budynku)
\end{itemize}

\end{ProTip}


	\newcommand{\TextRefToDRYKISS}{}
\fi\fi

Reguły DRY i KISS\TextRefToDRYKISS{} mają zastosowanie także w elektronice, a zwłaszcza projektowaniu układów elektronicznych i różnego rodzaju systemów automatyki.
Przy projektowaniu elektroniki trochę trudniej niż w programowaniu jest zachować regułę DRY (zwłaszcza przy tworzeniu projektów płytek drukowanych \textit{PCB}),
	jednak należy dążyć do tego – wydzielać powtarzające się elementy do pod-schematów, modularyzować tworzony projekt, itd.

Kurs ten poświęcony jest głównie elektronice cyfrowej, a ta współcześnie opiera się na wykorzystaniu układów programowalnych.
Zatem dalsze rady będą dotyczyły głównie projektowania systemu/urządzenia opartego na jakimś mikrokontrolerze
(mimo to wiele z nich można zastosować także przy projektowaniu innych układów elektronicznych i nie tylko).
\\
Przy tworzeniu projektów tego typu systemów/urządzeń należy mieć szczególnie na uwadze:
\begin{itemize}
	\item trzymanie się standardów i modułowość:
		\begin{itemize}
			\item należy stosować standardowe, popularne, otwarte protokoły komunikacyjne, takie jak I$^2$C, RS-485/Modbus, Ethernet/IP
			\item należy dokonywać podziału system na moduły funkcjonalne i określamy interfejsy pomiędzy nimi, zasadę tę należy stosować rekurencyjnie do wszystkich większych/bardziej złożonych jego elementów
			\item należy unikać projektowania modułów „na miarę” (lepiej mieć n+2 jednakowych modułów niż n każdy innego rodzaju)
			\item gdy oczekujemy większej niezawodności to należy zastosować redundancję – np. podwójne układy zasilania i komunikacji (dwa porty RS-485 lub dwa porty Ethernet) w każdym z urządzeń (modułów) składających się na system
			
			\item protokoły komunikacyjne, prędkość transmisji, etc należy dobierać z sporym zapasem (rzędu nawet 50-70\%), przewidując pojawienie się kolejnych rejestrów, funkcji, itp.
			\item należy przewidzieć rezerwę miejsca w modułach (np. dostępnych wejść)
		\end{itemize}
	\item dokumentacja, wersjonowanie:
		\begin{itemize}
			\item należy stosować wersjonowanie nie tylko kodu, ale również schematów, projektów PCB, dokumentacji, etc związanych z naszym projektem
			\item należy także oznaczać wersje związane z wykonanymi prototypami (w taki sposób aby można było je jednoznacznie powiązać z fizycznie wykonanym prototypem) i trzymać je co najmniej do zakończenia życia tych prototypów
			\item należy umieszczać identyfikator wersji na tworzonych płytkach PCB
			\item należy tworzyć dokumentację, np. samo użycie modbus nie rozwiązuje kwestii dokumentacji komunikacji – konieczna jest rzetelna tabela rejestrów
		\end{itemize}
	\item zachowanie prostoty:
		\begin{itemize}
			\item jeżeli używany mikrokontroler ma wbudowaną funkcję podciągania wejść – używać jej zamiast zewnętrznych rezystorów pull-up / pull-down,
				jeżeli ma tylko pull-up to przyciski robić jako zwierane do masy aby móc z niej skorzystać
			\item pamiętać że często (wbrew początkowej intuicji) sterowanie czymś przy pomocy wyjścia cyfrowego (zwłaszcza gdy wymagany jest do tego zewnętrzny tranzystor) prostsze jest od strony masy,
				czyli poprzez odłączanie/podłączanie do sterowanego układu masy (a nie dodatniego bieguna zasilania)
			\item warto także ograniczyć liczbę wykorzystywanych rodzin i modeli mikrokontrolerów
				– w przypadku nie seryjnej produkcji oszczędności z zastosowania np. bardzo małego mikrokontrolera zostaną zatarte „kosztami” związanymi z trudnościami w jego użyciu (brak pinów do debugowania, mała pamięć programu, itd.)
				oraz związanymi z rozwijaniem projektów na różnych mikroprocesorach (bo do kolejnego był już za mały)
		\end{itemize}
	\item łatwość diagnostyki i serwisu:
		\begin{itemize}
			\item należy zapewnić reset układów peryferyjnych wraz z resetem mikrokontrolera (np. poprzez użycie jednego z pinów GPIO mikrokontrolera w tej roli),
				dotyczy to także przypadków gdy naszymi urządzeniami peryferyjnymi są inne mikrokontrolery
			\item należy zapewnić łatwą możliwość przeprogramowania mikrokontrolera (bez potrzeby rozmontowywania układu, czy też wyjmowania z niego mikrokontrolera),
				a jeżeli aktualizacja wbudowanego oprogramowania odbywa się standardowo w jakiś wyżej poziomowy sposób to należy zapewnić zabezpieczenie przed awarią w jej trakcie,
				np. poprzez umieszczanie go na wymiennej karcie SD lub dostęp (np. poprzez UART) do sprzętowego bootloadera danego mikrokontrolera, lub zewnętrzny dostęp do programowania odpowiedniej pamięci
			\item warto zapewnić 2-3 diody sygnalizacyjne informujące o stanie pracy / awarii naszego urządzenia
			
			\item należy zapewniać możliwości naprawy i modyfikacji poszczególnych elementów systemu – przede wszystkim poprzez zapewnienie dostępu do nich
				(a gdy będzie on nie możliwy lub bardzo trudny poprzez położenie zapasowych przewodów), wykonywanie połączeń w łatwo dostępnych miejscach, itd.
			\item należy zachowywać kompatybilność wsteczną zawsze wtedy gdy tylko to jest możliwe (nowe urządzenie, nowa wersja muszą pracować w istniejącej sieci, nowa wersja musi w prosty sposób móc zastąpić poprzednią)
			\item należy konsekwentnie trzymać się określonych interfejsów i protokołów, jest to szczególnie ważne w niskopoziomowych (trudno aktualizowalnych, debugowalnych, występujących w dużej liczbie egzemplarzy) urządzeniach
		\end{itemize}
	\item nie tworzenie „potworków” (bo to się będzie mściło):
		\begin{itemize}
			\item należy unikać sztucznego ograniczania funkcjonalności tworzonego urządzenia (co jest nagminnie czynione w urządzeniach powszechnie dostępnych na rynku), na przykład:
				\begin{itemize}
					\item jeżeli urządzenia ma złącze Ethernet i używa je np. do wystawienia jakiegoś WWW to należy udostępnić pełną funkcjonalność monitoringu/konfiguracji tego urządzenia przez to TCP/IP z użyciem standardowych protokołów (np. Modbus TCP)a nie wymagać do tego osobnego modułu
					\item jeżeli urządzenia ma system operacyjny (np. Linuxa) należy zapewnić możliwość pełnego dostępu do niego (z prawami root'a) – także użytkownikowi, jeżeli je od nas kupił to on jest jego właścicielem
				\end{itemize}
			\item z drugiej strony należy jednak unikać upychania funkcjonalności do granic możliwości
				(lepiej pozwolić „zmarnować się” kilku nóżkom mikrokontrolera niż zrezygnować z czytelności, powtarzalności czy możliwości diagnostycznych na rzecz np. obsłużenia kilku dodatkowych IO)
			\item należy unikać oszczędzania kilka złotych minimalizując ponad miarę rozmiar PCB czy eliminując jakieś złącza lub je ograniczając
				(na złączach oprócz sygnałów należy wystawiać też potrzebne zasilania/masy w odpowiednich ilościach)
			
			\item należy walczyć z z problemem plątaniny kabli już na etapie założeń projektowych poprzez:
				\begin{itemize}
					\item stosowanie modularności – np. 1 sterownik Ethernet/IP (lub co najmniej RS-485/Modbus) na niezbyt dużą grupę wejść (np. jeden panel przycisków) / wyjść (np. jedną grupę przekaźników)
					\item nie oszczędzanie miejsca na PCB i funduszy na gniazdka przyłączeniowe (nie robić przylutowywanej na stałe wiązki kabli, pamiętać o masach i zasilaniach)
					\item nie oszczędzanie miejsca na PCB na otwory montażowe (najlepiej na wszystkich modułach wykonywać je w identycznych miejscach - z myślą o przyszłej obudowie, a nie tam gdzie popadnie)
					\item w przypadku wykonywania większych układów, złożonych z kilku sterowników rozważyć stosowanie korytek grzebieniowych do ukrycia plątaniny kabli
					\item w przypadku stosowania magistrali równoległej rozważyć multipleksowanie linii adresowej i danych
				\end{itemize}
			\item należy pamiętać że źle zastosowane technologie, które mają służyć ułatwieniu serwisowania i obsługi systemu (koryta kablowe, stelaże/szafy rack 19", obudowy z mocowaniem na szynę DIN / TH-35) mogą przynieść odwrotny skutek,
				a liczy efekt w postaci dobrze zaprojektowanego, czytelnego, dobrze działającego, serwisowalnego, rozbudowywalnego urządzenia/systemu a nie konkretnie zastosowane technologie
		\end{itemize}
\end{itemize}
