\begin{teacherOnly}
	\begin{easylist}[itemize]
	& elektronika cyfrowa - podstawy
		&& algebra bool'a, NKB
		&& poziomy logiczne TTL, logika odwrócona
		&& bramki logiczne, symbole (\textbf{symulacja})
		&& bufor (wyjście) trójstanowe -> do czego to służy -> kilka układów nadających do wspólnej magistrali (\textbf{symulacja})
		&& wyjścia open colektor/drain -> wymuszenie niskiego stanu linii przez jeden z elementów (\textbf{symulacja})
		&& jak działa bramka - zacząć od NOT'a (półmostek), następnie NAND lub NOR (\textbf{symulacja} z analizą działania i ustaleniem co to za bramka), wspomnieć dlaczego elektronika lubi bramki zanegowane (bo takie wychodzą)
	\end{easylist}
\end{teacherOnly}

% BEGIN: Elektronika - Bramki
\section{Bramki}
\begin{wrapfigure}{r}{0.7\textwidth}
  \begin{center}
    \vspace{-40pt}
    \includegraphics[width=0.65\textwidth]{img/elektronika/bramki}
    \vspace{-20pt}
  \end{center}
\end{wrapfigure}

Bramki są układami elektronicznymi realizującymi podstawowe, opisane powyżej funkcje logiczne. Obok zostały przedstawione podstawowe symbole poszczególnych bramek w wariancie dwu wejściowym, spotkać się można także z symbolami z zanegowanymi wejściami - w takiej konwencji np. bramka AND reprezentowana jest przez NOR z zanegowanymi wejściami. Bramki (z wyjątkiem buforów oraz bramki NOT), mogą występować także w wariantach wielo-wejściowych (ze względu na łączność podstawowych operacji nie ma wątpliwości co don wyniku jaki powinna dawać np. 8 wejściowa bramka OR). Na ogół w pojedynczym układzie scalonym znajduje się kilka jednakowych bramek.

Zobacz symulację działania różnych bramek logicznych: \url{http://ln.opcode.eu.org/bramki} (H oznacza stan wysoki, czyli prawdę, L stan niski czyli fałsz, klikając na H/L przy wejściach można zmieniać ich stan).
	% https://www.falstad.com/circuit/circuitjs.html?cct=%24+1+0.000005+1.500424758475255+50+5+50%0AM+96+176+144+176+0+2.5%0AL+-32+160+-48+160+0+0+false+5+0%0AL+96+368+80+368+0+0+false+5+0%0A150+-32+176+96+176+0+2+0+5%0A152+-32+256+96+256+0+2+0+5%0A154+96+352+224+352+0+2+0+5%0AI+96+80+224+80+0+0.5+5%0AM+96+256+144+256+0+2.5%0AM+224+352+272+352+0+2.5%0AL+96+80+80+80+0+0+false+5+0%0AM+224+80+272+80+0+2.5%0AL+-32+192+-48+192+0+0+false+5+0%0AL+-32+240+-48+240+0+0+false+5+0%0AL+-32+272+-48+272+0+0+false+5+0%0AL+96+336+80+336+0+0+false+5+0%0A151+240+176+368+176+0+2+5+5%0A153+240+256+368+256+0+2+5+5%0AL+240+240+224+240+0+0+false+5+0%0AL+240+272+224+272+0+0+false+5+0%0AL+240+192+224+192+0+0+false+5+0%0AL+240+160+224+160+0+0+false+5+0%0AM+368+176+416+176+0+2.5%0AM+368+256+416+256+0+2.5%0A

\subsection{trój-stanowe}
Typowa bramka wymusza (w sposób silny) na swoim wyjściu stan wysoki lub niski, co uniemożliwia bezpośrednie łączenie wyjść bramek.
Bramki trój-stanowe mają możliwość skonfigurowania wyjścia w stan \emph{wysokiej impedancji} czyli nie wymuszania żadnej jego wartości.
Sterowanie załączeniem bądź wyłączeniem wyjścia (przełączeniem w stan wysokiej impedancji) odbywa się przy pomocy zewnętrznego sygnału sterującego "output enabled" ("OE"), sygnał ten może występować w postaci prostej i zanegowanej.
Pozwala to na podłączanie do jednej linii wielu bramek i decydowaniu która z nich będzie nią sterować.

\subsection{open collector / drain}
Są kolejnym rodzajem bramek których wyjścia można podłączać do wspólnej linii. Układy te posiadają wyjście w postaci tranzystora zwierającego linię wyjściową do masy, z tego względu samodzielnie zapewniają jedynie stan niski wyjścia (są w stanie wymusić stan niski, ale nie mają możliwości wymuszenia stanu wysokiego).

Stan wysoki musi zostać zapewniony zewnętrznym rezystorem podciągającym. Pozwala to stosować na takiej linii inny poziom stanu wysokiego niż na wejściach takiej bramki oraz pozwala na sterowanie wspólnej linii przez wiele bramek (czyli łączenie wyjść bramek, jednak w odróżnieniu od bramek trój-stanowych nie wymaga dodatkowych sygnałów sterujących).

\begin{wrapfigure}{r}{0.7\textwidth}
  \begin{center}
    \vspace{-20pt}
    \includegraphics[width=0.65\textwidth]{img/elektronika/open_drain}
    \vspace{-20pt}
  \end{center}
\end{wrapfigure}
Na schemacie obok przedstawiono dwa układy (U1 i U2) typu open-drain sterujące wspólną linią wyjściową w układzie \emph{suma na drucie}. Jeżeli jeden z podłączonych do linii układów będzie miał wewnętrzne wyjście ("ctrl\textit{X}") w stanie wysokim to jego wyjście OC będzie zwarte do masy (negacja na tranzystorze N-MOS lub NPN), wtedy też cała linia będzie w stanie niskim.

Zobacz symulację lininii z bramkami trójstanowymi (stan wysokiej impedancji symulowany za pomocą przełącznika) oraz linii open-colektor: \url{http://ln.opcode.eu.org/ster}
	% https://www.falstad.com/circuit/circuitjs.html?cct=%24+1+0.000005+1.500424758475255+50+5+50%0As+16+160+80+160+0+0+false%0Aw+80+160+112+160+0%0Aw+80+272+112+272+0%0As+16+272+80+272+0+1+false%0As+16+384+80+384+0+1+false%0Aw+80+384+112+384+0%0Aw+112+272+176+272+2%0Aw+112+160+112+272+1%0Aw+112+272+112+384+1%0Ar+-48+160+16+160+0+0.001%0Ar+-48+272+16+272+0+0.001%0Ar+-48+384+16+384+0+0.001%0AR+432+208+384+208+0+0+40+0.5+0+0+0.5%0AR+512+80+512+48+0+0+40+5+0+0+0.5%0At+400+176+432+176+0+1+-4.499999993000001+9.999999717180685e-10+100%0Ar+336+176+400+176+0+10000%0Aw+432+192+432+208+0%0AL+336+176+320+176+0+1+false+0.5+5%0Aw+432+288+432+304+0%0Ar+336+272+400+272+0+10000%0At+400+272+432+272+0+1+-4.499999993000001+9.999999717180685e-10+100%0Aw+432+160+512+160+0%0Aw+432+256+512+256+0%0Ar+512+160+512+80+0+10000%0Aw+512+160+512+256+0%0Aw+512+256+576+256+2%0AL+336+272+320+272+0+1+false+0.5+5%0AL+-48+160+-64+160+0+1+false+5+0.5%0AL+-48+272+-64+272+0+0+false+5+0.5%0AL+-48+384+-64+384+0+0+false+5+0.5%0AL+336+368+320+368+0+1+false+0.5+5%0Aw+432+352+512+352+0%0At+400+368+432+368+0+1+-4.499999993000001+9.999999717180685e-10+100%0Ar+336+368+400+368+0+10000%0Aw+432+384+432+400+0%0Aw+512+256+512+352+0%0AR+432+304+384+304+0+0+40+0.5+0+0+0.5%0AR+432+400+384+400+0+0+40+0.5+0+0+0.5%0A

\begin{wrapfigure}{r}{0.61\textwidth}
  \begin{center}
    \vspace{-45pt}
    \includegraphics[width=0.59\textwidth]{img/elektronika/bramki_cmos}
    \vspace{-35pt}
  \end{center}
\end{wrapfigure}
\subsection{budowa wewnętrzna}
Przedstawiony powyżej układ sumy na drucie jest bardzo prostą (jedno tranzystorową) realizacją bramki logicznej realizującą funkcję logiczną NOT OR (z punktu widzenia wejść \textit{ctrl1} i \textit{ctrl2} oraz wyjścia \textit{Out}).
W podobny sposób można zrealizować bramkę AND (negując wejścia, np. przy pomocy jednego traznzystora).
Jeszcze bardziej uproszczoną realizację można uzyskać stosując diody pozwalające na wpływanie prądu do węzła (funkcja OR) lub wypływanie z niego (funkcja AND).

Po prawej przedstawione zostały schematy ideowe inwertera, dwóch podstawowych bramek (NOR i NAND) oraz bramki transmisyjnej (bufora 3 stanowego) w technologii CMOS.

Działanie tych bramek (za wyjątkiem transmisyjnej) polega na otwieraniu tranzystorów podłączonych do napięcia które chcemy otrzymać na wyjściu, a zamykaniu prowadzących do napięcia przeciwnego. W szczególności bramka NOT stanowi pół-mostek H pomiędzy stanem wysokim a stanem niskim.

Dzięki zastosowaniu tranzystorów PMOS polaryzowanych Vdd oraz NMOS polaryzowanych GND obie gałęzie operują na tym samym sygnale wejściowym (nie jest wymagana jego negacja). Szeregowe łączenie tranzystorów zapewnia że należy otworzyć oba aby otworzyć daną drogę, a równoległe że otwarcie danej drogi powodowane jest otwarciem pojedynczego tranzystora. Dzięki zastosowaniu technologi MOS i podłączaniu wejść bramki tylko do bramek tranzystorów wejścia praktycznie nie pobierają prądu (istotnym wyjątkiem jest chwila zmiany sygnału).

Działanie bramki transmisyjnej polega na przepuszczaniu lub nie (w zależności od stanu wejścia sterującego) sygnału z wejścia na wyjście. Bramka taka nie regeneruje sygnału. Ponadto w uproszczonym (jedno tranzystorowym) rozwiązaniu prowadzi ona do degradacji sygnału wartość w przybliżeniu równą napięciu progowemu tranzystora. Dlatego też na ogół występuje wraz z bramką NOT (bufor 3 stanowy z negacją) lub dwiema szeregowo połączonymi bramkami NOT (bufor 3 stanowy bez negacji).

Zobacz symulację budowy bramek: NOT (\url{http://ln.opcode.eu.org/not}), NAND (\url{http://ln.opcode.eu.org/nand}) i NOR (\url{http://ln.opcode.eu.org/nor}).
	% https://www.falstad.com/circuit/circuitjs.html?cct=%24+0+0.000005+10.20027730826997+50+5+50%0Af+208+176+272+176+1+1.5+0.02%0Af+208+272+272+272+0+1.5+0.02%0Aw+272+192+272+224+0%0Aw+272+224+272+256+0%0Aw+208+176+208+224+0%0Aw+208+224+208+272+0%0AL+208+224+160+224+0+1+false+5+0%0AM+272+224+336+224+0+2.5%0AR+272+160+272+112+0+0+40+5+0+0+0.5%0Ag+272+288+272+320+0%0A;
	% https://www.falstad.com/circuit/circuitjs.html?cct=%24+0+0.000005+10.20027730826997+50+5+50%0Af+288+128+352+128+1+1.5+0.02%0Af+288+224+352+224+0+1.5+0.02%0Aw+352+144+352+176+0%0Aw+352+176+352+208+0%0AM+352+176+416+176+0+2.5%0Af+288+288+352+288+0+1.5+0.02%0Aw+352+240+352+272+0%0Ag+352+304+352+336+0%0Aw+288+128+288+224+0%0Af+192+128+256+128+1+1.5+0.02%0Aw+256+80+256+112+0%0Aw+256+80+352+80+0%0Aw+352+80+352+112+0%0AR+256+80+192+80+0+0+40+5+0+0+0.5%0Aw+192+128+192+288+0%0Aw+192+288+288+288+0%0AL+288+224+128+224+0+1+false+5+0%0AL+192+288+128+288+0+1+false+5+0%0Aw+256+144+256+176+0%0Aw+256+176+352+176+0%0A;
	% https://www.falstad.com/circuit/circuitjs.html?cct=%24+0+0.000005+10.20027730826997+50+5+50%0Af+272+144+336+144+1+1.5+0.02%0Aw+336+160+336+192+0%0AR+336+128+336+80+0+0+40+5+0+0+0.5%0Af+272+208+336+208+1+1.5+0.02%0Aw+336+224+336+240+0%0Aw+176+144+272+144+0%0AL+176+144+128+144+0+0+false+5+0%0AL+272+208+128+208+0+0+false+5+0%0Aw+336+240+336+256+0%0Aw+336+256+336+272+0%0Aw+240+256+240+272+0%0Aw+176+144+176+304+0%0Aw+272+208+272+304+0%0Ag+336+320+336+352+0%0Ag+240+320+240+352+0%0Aw+240+256+336+256+0%0Aw+240+272+240+288+0%0Af+176+304+240+304+0+1.5+0.02%0AM+336+256+400+256+0+2.5%0Af+272+304+336+304+0+1.5+0.02%0Aw+336+272+336+288+0%0A;
% END: Elektronika - Bramki
