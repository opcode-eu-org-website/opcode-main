% Copyright (c) 2017-2020 Matematyka dla Ciekawych Świata (http://ciekawi.icm.edu.pl/)
% Copyright (c) 2017-2020 Robert Ryszard Paciorek <rrp@opcode.eu.org>
% 
% MIT License
% 
% Permission is hereby granted, free of charge, to any person obtaining a copy
% of this software and associated documentation files (the "Software"), to deal
% in the Software without restriction, including without limitation the rights
% to use, copy, modify, merge, publish, distribute, sublicense, and/or sell
% copies of the Software, and to permit persons to whom the Software is
% furnished to do so, subject to the following conditions:
% 
% The above copyright notice and this permission notice shall be included in all
% copies or substantial portions of the Software.
% 
% THE SOFTWARE IS PROVIDED "AS IS", WITHOUT WARRANTY OF ANY KIND, EXPRESS OR
% IMPLIED, INCLUDING BUT NOT LIMITED TO THE WARRANTIES OF MERCHANTABILITY,
% FITNESS FOR A PARTICULAR PURPOSE AND NONINFRINGEMENT. IN NO EVENT SHALL THE
% AUTHORS OR COPYRIGHT HOLDERS BE LIABLE FOR ANY CLAIM, DAMAGES OR OTHER
% LIABILITY, WHETHER IN AN ACTION OF CONTRACT, TORT OR OTHERWISE, ARISING FROM,
% OUT OF OR IN CONNECTION WITH THE SOFTWARE OR THE USE OR OTHER DEALINGS IN THE
% SOFTWARE.

\IfStrEq{\dbEntryID}{}{
	\ifdefined\noExtraInfoMode\else
		\subsection{Tranzystory}
	\fi
	\insertZadanie{\currfilepath}{prad_R8_R10}{}
	\insertZadanie{\currfilepath}{polmostek_H}{}
}

\IfStrEq{\dbEntryID}{praktyczne}{
	\ifdefined\noExtraInfoMode\else
		\subsection{Tranzystory}
	\fi
	\insertZadanie{\currfilepath}{zbuduj_klucz_npn1}{}
	\insertZadanie{\currfilepath}{zbuduj_klucz_npn2}{}
	\insertZadanie{\currfilepath}{zbuduj_stabilizator_zener2}{}
}


%
% zadania teoretyczne
%

\dbEntryBegin{prad_R8_R10}\if1\dbEntryCheckResults
Oszacuj wartość prądu płynącego przez R8 oraz wartość prądu płynącego przez R10. Odpowiedź krótko uzasadnij.
	\vspace{-6.5mm}\begin{center} \includegraphics[width=0.55\textwidth]{img/elektronika/zadania_teoretyczne-tranzystory_bipolarne} \end{center}
\fi
\dbEntryBegin{prad_R8_R10-rozwiazanie}\if1\dbEntryCheckResults
\textbf{R8:} Prąd bazy wynosi około 0.5mA, czyli tranzystor przewodzi.
             Prąd kolektora "nastawiany" przez tranzystor (wynikający z prądu bazy), przy założeniu najsłabszego wzmocnienia wynosi 25mA (minimalne $h_{FE}$ tego tranzystora to około 50).
             Prąd kolektora wynikający z wartości R8 to około 14mA, czyli znacznie mniejszy niż prąd "nastawiany" przez tranzystor.
             Zatem tranzystor nie ogranicza prądu – praca w nasyceniu, prąd płynący przez R8 wynosi około 14mA.\\
\textbf{R10:} Prąd bazy nie płynie (baza na potencjale wyższym niż emiter w PNP), czyli tranzystor przewodzi.
              Zatem prąd płynący przez R10 wynosi około 0mA.
\fi

\dbEntryBegin{polmostek_H}\if1\dbEntryCheckResults
Narysuj schemat układu półmostka H z zastosowaniem tranzystorów bipolarnych (NPN i PNP) jako elementów przełączających.
\fi
\dbEntryBegin{polmostek_H-rozwiazanie}\if1\dbEntryCheckResults
\begin{center}\includegraphics[width=0.68\textwidth]{img/elektronika/zad-rozw-półmostek_bipolarne}\end{center}
\fi


%
% zadania praktyczne
%

\dbEntryBegin{zbuduj_klucz_npn1}\if1\dbEntryCheckResults
\noindent\begin{minipage}[b]{0.6\textwidth}
Zastanów się co przedstawia układ przedstawiony na schemacie obok. Skonstruuj go i zobacz jak działa.

Czy przez obciążenie (rezystor \textit{LOAD} o wartości 1k) płynie prąd gdy rezystor \textit{Rb} poprzez przełącznik \textit{S1} podłączony jest do napięcia zasilającego, a czy płynie gdy podłączony jest do masy?

Zmierz wartość prądu płynącego przez \textit{Rb} i płynącego przez \textit{LOAD} w obu wypadkach. Zastanów się do czego może być użyty taki układ?

\textit{Wskazówka: Zamiast użyć przełącznika możesz po prostu przełączać kabelek na płytce stykowej.}
\end{minipage}
\hfill
\begin{minipage}[b]{0.35\textwidth}
\includegraphics[width=\textwidth]{img/elektronika/zad-npn1}
\end{minipage}
\fi


\dbEntryBegin{zbuduj_klucz_npn2}\if1\dbEntryCheckResults
\noindent\begin{minipage}[b]{0.5\textwidth}
Zmodyfikuj układ z zadania \ref{zbuduj_klucz_npn1} aby wyglądał jak na schemacie obok (obecnie S1 przełącza pomiędzy napięciem z dzielnika R1/R2 oraz stanem niepodłączonym).

Oblicz ile powinno wynosić napięcie wyjściowe z dzielnika R1/R2? Czy rzeczywiste napięcie zgadza się z tym co obliczyłeś?

Wykonaj ponownie pomiary prądu płynącego przez \textit{Rb} i płynącego przez \textit{LOAD} w obu stanach S1.
Jak wprowadzone zmiany wpłyneły na zachowanie układu?
\end{minipage}
\hfill
\begin{minipage}[b]{0.45\textwidth}
\includegraphics[width=\textwidth]{img/elektronika/zad-npn2}
\end{minipage}
\fi


\dbEntryBegin{zbuduj_stabilizator_zener2}\if1\dbEntryCheckResults
\noindent\begin{minipage}[b]{0.7\textwidth}
Zbuduj układ stabilizacji napięcia w oparciu o diodę Zenera i tranzystor przedstawiony na schemacie obok.

Zastanów się nad sposobem działania tego układu – w tym celu dokonaj pomiarów napięcia wyjściowego oraz napięcia na bazie tranzystora w zależności od napięcia wejściowego.

Zobacz jak na napięcie wyjściowe wpływa wielkość obciążenia symulowanego przez R2 (pamiętaj aby nie ustawiać zbyt małej rezystancji, bo przekroczysz maksymalny prąd dozwolony dla użytego tranzystora).

W czym ukłąd ten jest lepszy od układu z zadania \ref{zbuduj_stabilizator_zener1}? Zastanów się dlaczego.
\end{minipage}
\hfill
\begin{minipage}[b]{0.25\textwidth}
\includegraphics[width=\textwidth]{img/elektronika/zad-stabilizator_zener2}
\end{minipage}
\fi


