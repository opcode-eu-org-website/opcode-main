% Copyright (c) 2017-2020 Matematyka dla Ciekawych Świata (http://ciekawi.icm.edu.pl/)
% Copyright (c) 2017-2020 Robert Ryszard Paciorek <rrp@opcode.eu.org>
% 
% MIT License
% 
% Permission is hereby granted, free of charge, to any person obtaining a copy
% of this software and associated documentation files (the "Software"), to deal
% in the Software without restriction, including without limitation the rights
% to use, copy, modify, merge, publish, distribute, sublicense, and/or sell
% copies of the Software, and to permit persons to whom the Software is
% furnished to do so, subject to the following conditions:
% 
% The above copyright notice and this permission notice shall be included in all
% copies or substantial portions of the Software.
% 
% THE SOFTWARE IS PROVIDED "AS IS", WITHOUT WARRANTY OF ANY KIND, EXPRESS OR
% IMPLIED, INCLUDING BUT NOT LIMITED TO THE WARRANTIES OF MERCHANTABILITY,
% FITNESS FOR A PARTICULAR PURPOSE AND NONINFRINGEMENT. IN NO EVENT SHALL THE
% AUTHORS OR COPYRIGHT HOLDERS BE LIABLE FOR ANY CLAIM, DAMAGES OR OTHER
% LIABILITY, WHETHER IN AN ACTION OF CONTRACT, TORT OR OTHERWISE, ARISING FROM,
% OUT OF OR IN CONNECTION WITH THE SOFTWARE OR THE USE OR OTHER DEALINGS IN THE
% SOFTWARE.

% BEGIN: Elektronika - Tranzystory
\section{Tranzystory}

Tranzystor jest to element o regulowanym elektrycznie przewodzeniu prądu (oporze), często wykorzystywany do wzmacniania sygnałów lub jako przełącznik elektroniczny (klucz tranzystorowy).
Klucz jest układem przełączającym wykorzystującym dwa skrajne stany pracy tranzystora - zatkania (tranzystor nie przewodzi), nasycenia (tranzystor przewodzi z minimalnymi ograniczeniami).

\subsection{NPN}
Prąd przepływający pomiędzy kolektorem a emiterem jest funkcją prądu przepływającego pomiędzy bazą a emiterem: $I_C = \beta I_B$.
Napięcie pomiędzy kolektorem a emiterem wynosi: $U_{CE} = U_{zasilania} - I_C \cdot R_{load}$.
Napięcie to nie może jednak spaść poniżej wartości minimalnej wynoszącej około 0.2V, gdy z powyższych zależności wynikałby taki spadek to tranzystor pracuje w stanem nasycenia i $U_{CE} \approx 0.2V$.

\begin{wrapfigure}{r}{0.6\textwidth}
  \begin{center}
    \includegraphics[width=0.55\textwidth]{img/elektronika/npn_pnp}
    \vspace{-10pt}
  \end{center}
\end{wrapfigure}

Aby wprowadzić tranzystor NPN w stan zatkania należy podać na jego bazę potencjał mniejszy lub równy potencjałowi emitera (zakładamy że potencjał kolektora jest nie mniejszy niż emitera - co ma miejsce w typowych warunkach polaryzacji tranzystora NPN), czyli $U_{BE} \leq 0$.

Aby wprowadzić tranzystor NPN w stan nasycenia należy na jego bazę wprowadzić potencjał większy od potencjałów emitera i kolektora, uzyskuje się to poprzez wprowadzenie do tranzystora prądu bazy $I_B \gg \frac{U_{zasilania}}{\beta R_{load}}$.

Zobacz symulację pokazującą pracę tranzystora NPN w trybie klucza: \url{http://ln.opcode.eu.org/npn}.
Zwróć uwagę na wartości napięć i prądów.

Zobacz co dzieje się przy próbie podłączenia bazy tranzystora do potencjału znacznie wyższego niż potencjał emitera – złącze baza-emiter jest takim samym złączem z jakim mamy do czynienia w diodzie i tak jak w przypadku diody występuje na nim stały spadek napięcia (nie działa tu prawo Ohma). Dlatego aby ograniczyć prąd płynący tą gałęzią i zapobiec zniszczeniu tranzystora konieczne jest zastosowanie rezystora.

\subsection{PNP}
Podobnie jak w NPE tyle że prąd przepływający pomiędzy emiterem a kolektorem jest funkcją prądu przepływającego pomiędzy emiterem a bazą.

Aby wprowadzić tranzystor PNP w stan zatkania należy podać na jego bazę potencjał większy lub równy potencjałowi emitera (zakładamy że potencjał emitera jest nie mniejszy niż kolektora - co ma miejsce w typowych warunkach polaryzacji tranzystora PNP), czyli $U_{BE} \geq 0$.

Aby wprowadzić tranzystor PNP w stan nasycenia należy na jego bazę wprowadzić potencjał mniejszy od potencjałów emitera i kolektora, uzyskuje się to poprzez wyprowadzenie z tranzystora prądu bazy $I_B \gg \frac{U_{zasilania}}{\beta R_{load}}$.

Zobacz symulację pokazującą pracę tranzystora PNP w trybie klucza: \url{http://ln.opcode.eu.org/pnp}.
Zwróć uwagę na podobieństwa i różnice w stosunku do tranzystora NPN –
	w obu wypadkach tranzystor przewodzi gdy płynie prąd bazy, ale ma on różne kierunki (w NPN wpływa on bazą do tranzystora, a w PNP wypływa z niego),
	w obu wypadkach tranzystor zostaje zatkany gdy potencjał bazy zrówna się z potencjałem emitera (ale w NPN potencjał emitera jest typowo najniższym z potencjałów w układzie, często równym masie, a w PNP najwyższym, często równym potencjałowi zasilania).
Zauważ także, że tutaj również potrzebujemy rezystora ograniczającego prąd bazy.

\subsection{N-MOSFET}

\begin{wrapfigure}{r}{0.7\textwidth}
  \begin{center}
    \vspace{-40pt}
    \includegraphics[width=0.65\textwidth]{img/elektronika/mosfet}
    \vspace{-20pt}
  \end{center}
\end{wrapfigure}

Prąd przepływający pomiędzy drenem (\emph{drain}) a źródłem (\emph{source}) jest funkcją napięcia pomiędzy bramką (\emph{gate}) a źródłem (potencjału bramki względem źródła - $U_{GS}$), bramka jest izolowana (nie płynie przez nią prąd).

W kierunku dren → źródło tranzystor ten przewodzi gdy $U_{GS} > U_{GS (th)}$, natomiast w przeciwnym kierunku przewodzi zawsze. Dla tranzystorów N-MOSFET z kanałem wzbogacanym (\emph{enhancement}) $U_{GS (th)} > 0$, a z kanałem zubożonym (\emph{depletion}) $U_{GS (th)} < 0$.

Konkretna wartość $U_{GS (th)}$ zależna jest od konkretnego modelu tranzystora, innym istotnym parametrem związanym z sterowaniem tranzystorem jest maksymalna i minimalna dopuszczalna wartość napięcia $U_{GS}$.

Aby wprowadzić tranzystor MOSFET w stan zatkania należy podać $U_{GS} < U_{GS (th)}$. Dla tranzystorów:
\begin{itemize}
\item N-MOSFET z kanałem wzbogaconym wystarczy obniżyć potencjał bramki do wartości niewiele wyższej niż potencjał źródła
\item N-MOSFET z kanałem zubożonym musi to być wartość poniżej potencjału źródła.
\end{itemize}
Aby wprowadzić tranzystor MOSFET w stan przewodzenia należy podać $U_{GS} \gg U_{GS (th)}$.

\subsection{P-MOSFET}
Podobnie jak N-MOSFET tyle że:
\begin{itemize}
\item regulowane jest przewodzenie w kierunku źródło → dren (a w kierunku dren → źródło przewodzi zawsze),
\item przewodzenie w kierunku źródło → dren ma miejsce gdy $U_{GS} < U_{GS (th)}$,
\item dla tranzystorów z kanałem wzbogacanym (\emph{enhancement}) $U_{GS (th)} < 0$, a z kanałem zubożonym (\emph{depletion}) $U_{GS (th)} > 0$.
\end{itemize}

Aby wprowadzić tranzystor MOSFET w stan zatkania należy podać $U_{GS} < U_{GS (th)}$. Dla tranzystorów:
\begin{itemize}
\item P-MOSFET z kanałem zubożonym wystarczy obniżyć potencjał bramki do wartości niewiele wyższej niż potencjał źródła
\item P-MOSFET z kanałem wzbogaconym musi to być wartość poniżej potencjału źródła.
\end{itemize}
Aby wprowadzić tranzystor MOSFET w stan przewodzenia należy podać $U_{GS} \gg U_{GS (th)}$.

\vspace{12pt}

Zobacz symulację pokazującą pracę tranzystorów MOSFET w trybie klucza: \url{http://ln.opcode.eu.org/mosfet}.
Zauważ podobieństwo w sterowaniu do trnzystorów NPN i PNP (N-MOSFET przewodzi gdy potencjał bramki odpowiednio wyższy od drenu, P-MOSFET gdy odpowiednio niższy, obciążenie N-MOSFET włączane analogicznie jak NPN, a P-MOSFET jak PNP),
	zauważ różnice (bramka jest izolowanie, nie płynie nią prąd\footnote{z pominięciem prądu związanego z przeładowaniem pojemności (pasożytniczego kondensatora)}, nie ma zatem potrzeby używania tam rezystora).
% END: Elektronika - Tranzystory

\clearpage

% BEGIN: Elektronika - mostek H
\subsection{Mostek H}

\begin{wrapfigure}{r}{0.2\textwidth}
  \begin{center}
    \vspace{-33pt}
    \includegraphics[width=0.17\textwidth]{img/elektronika/mostek_H_switche}
    \vspace{-27pt}
  \end{center}
\end{wrapfigure}

Mostek H jest to układ (oparty o 4 przełączniki, których rolę mogą pełnić klucze tranzystorowe) pozwalający na zmianę polaryzacji zasilania podłączonego do niego odbiornika. Układ taki złożony jest z dwóch identycznych gałęzi (S1 + S2 oraz S3 + S4, każda włączona pomiędzy dwoma biegunami zasilania). Pojedyncza taka gałąź nazywana jest pół-mostkiem i składa się z dwóch kluczy które powinny być sterowane przeciwstawnie (aby wyeliminować możliwość zwarcia zasilania z masą). Układ pół-mostka może być wykorzystywany także samodzielnie jako uniwersalny układ klucza pozwalającego na załączanie odbiornika zarówno od strony napięcia dodatniego jak i od strony masy (w zależności od sposobu jego podłączenia) lub przełączania dwóch odbiorników (jednego umieszczonego pomiędzy zasilaniem a wyjściem mostka, a drugiego pomiędzy wyjściem a masą).

Rolę kluczy (przełączników) w ramach mostka mogą pełnić tranzystory PNP (jako S1, S3) i NPN (jako S2, S4) albo analogicznie tranzystory P-MOSFET i N-MOSFET.
% END: Elektronika - mostek H

\subsection{Wzmacniacz}

Omawiając poszczególne typy tranzystorów skupialiśmy się na ich pracy w roli przełącznika (klucza), działającego w dwóch stanach – przewodzenia (nasycenia) i zastkania.
Jednak tranzystor będąc elementem o regulowanym przewodzeniu może zostać wykorzystany także do wzmacniania sygnałów, czyli wytworzenia na swoim wyjściu sygnału proporcjonalnego do sygnału wejściowego tyle że wzmocnionego.
Wzmacnianiu może ulegać sygnał napięciowy lub prądowy (najprostszym przypadkiem jest wzmocnienie prądu bazy jako prądu kolektora $I_C = \beta I_B$ w tranzystorze bipolarnym).
Więcej o różnych układach pracy tranzystora w roli wzmacniacza można przeczytać na \url{http://vip.opcode.eu.org/#Wzmacniacz_sygnału}.

\vspace{5pt}

\begin{wrapfigure}{r}{0.24\textwidth}
  \begin{center}
    \vspace{-25pt}
    \includegraphics[width=0.21\textwidth]{img/elektronika/wzmacniacz_operacyjny-nieodwracający}
    \vspace{-27pt}
  \end{center}
\end{wrapfigure}

Często do wzmacniania sygnału zamiast pojedynczego tranzystora wykorzystujemy układy scalone (złożone z wielu tranzystorów) nazywane wzmacniaczami operacyjnymi.
Cechują się one bardzo dużym wzmocnieniem różnicy napięcia pomiędzy swoimi wejściami, pożądane wzmocnienie uzyskuje się dobierając zewnętrzne elementy pętli ujemnego sprzężenia zwrotnego
	(w najprostszym przypadku na jedno wejście podajemy sygnał wejściowy, a na drugie odpowiednio przeskalowany przy pomocy dzielnika rezystancyjnego sygnał wyjściowy).
Zobacz symulację: \url{http://ln.opcode.eu.org/opamp} i \url{http://ln.opcode.eu.org/opamp_loop}. Więcej na ich temat można przeczytać na \url{http://vip.opcode.eu.org/#Wzmacniacz_operacyjny}.

\subsection{Przełączanie AC}
Tranzystory stosowane są powszechnie do przełączania w obwodach prądu stałego. Istnieją także elementy półprzewodnikowe mogące pełnić funkcję przełączającą w obwodach prądu przemiennego - współcześnie są to przede wszystkim triaki.
