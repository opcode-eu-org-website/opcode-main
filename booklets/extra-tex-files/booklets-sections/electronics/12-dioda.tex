% Copyright (c) 2017-2020 Matematyka dla Ciekawych Świata (http://ciekawi.icm.edu.pl/)
% Copyright (c) 2017-2020 Robert Ryszard Paciorek <rrp@opcode.eu.org>
% 
% MIT License
% 
% Permission is hereby granted, free of charge, to any person obtaining a copy
% of this software and associated documentation files (the "Software"), to deal
% in the Software without restriction, including without limitation the rights
% to use, copy, modify, merge, publish, distribute, sublicense, and/or sell
% copies of the Software, and to permit persons to whom the Software is
% furnished to do so, subject to the following conditions:
% 
% The above copyright notice and this permission notice shall be included in all
% copies or substantial portions of the Software.
% 
% THE SOFTWARE IS PROVIDED "AS IS", WITHOUT WARRANTY OF ANY KIND, EXPRESS OR
% IMPLIED, INCLUDING BUT NOT LIMITED TO THE WARRANTIES OF MERCHANTABILITY,
% FITNESS FOR A PARTICULAR PURPOSE AND NONINFRINGEMENT. IN NO EVENT SHALL THE
% AUTHORS OR COPYRIGHT HOLDERS BE LIABLE FOR ANY CLAIM, DAMAGES OR OTHER
% LIABILITY, WHETHER IN AN ACTION OF CONTRACT, TORT OR OTHERWISE, ARISING FROM,
% OUT OF OR IN CONNECTION WITH THE SOFTWARE OR THE USE OR OTHER DEALINGS IN THE
% SOFTWARE.

% BEGIN: Elektronika - Dioda
\section{Dioda}
\begin{teacherOnly}
	\begin{easylist}[itemize]
		& przewodzi w jednym kierunku
		& element nieliniowy -> łamie prawo Ohma
		& dioda prostownicza i dioda świecąca
			&& rezystor przy diodzie świecącej -> na diodzie stały spadek napięcia w kierunku przewodzenia (\textbf{symulacja})
			&& dioda prostownicza – prostownik jedno połówkowy i mostek prostowniczy (\textbf{symulacja})
		& czy może przewodzić w drugą stronę -> dioda Zenera
		& dzielnik napięcia z diodą Zenera (\textbf{symulacja})
		& fotodioda
		& dioda LED jako detektor światła (jeżeli będzie czas!)
	\end{easylist}
\end{teacherOnly}

Dioda idealna to element przewodzący prąd tylko w jednym kierunku. Rzeczywiste diody przewodzą prąd zdecydowanie chętniej w jednym kierunku niż w drugim (na ogół przewodzenie w kierunku zaporowym się pomija) ponadto charakteryzują je cechy zależne od technologi wykonania takie jak:
\begin{itemize}
\item spadek napięcia w kierunku przewodzenia (typowo dla diod krzemowych 0.6V - 0.7V, a dla diod Schottky’ego 0.3V)
\item napięcie przebicia - napięcie, które przyłożone w kierunku zaporowym powoduje znaczące przewodzenie diody w tym kierunku - w większości przypadków parametr którego nie należy przekraczać, jednak wykorzystywane (i stanowiące ich parametr) w niektórych typach diod
\item maksymalny prąd przewodzenia
\item czas przełączania (związany głównie z pasożytniczą pojemnością złącza) - zdecydowanie krótszy (około 100 ps) w diodach Schottky’ego niż w diodach krzemowych,.
\end{itemize}

\begin{wrapfigure}{r}{0.3\textwidth}
  \begin{center}
    \includegraphics[width=0.25\textwidth]{img/elektronika/diody}
  \end{center}
\end{wrapfigure}

\noindent
Ponadto stosowane są m.in.:
\begin{itemize}
\item diody Zenera - wykorzystuje się (charakterystyczną dla danego typu) wartość napięcia przebicia do uzyskania w układzie spadku napięcia o tej wartości,
\item diody świecące (LED) - emitujące światło w trakcie przewodzenia (na elemencie występuje stały spadek napięcia, jasność zależy od natężenia prądu),
\item fotodiody - będące detektorami oświetlenia (przewodzenie spolaryzowanej w kierunku zaporowym zależy od ilości padającego na element światła, niespolaryzowana pod wpływem oświetlenia staje się źródłem prądu).
\end{itemize}

\begin{ProTip}{{\Symbola ❢} PAMIĘTAJ {\Symbola ❢}}
Dioda jest elementem dla którego nie jest spełnione prawo Ohma. Dioda charakteryzuje się prawie stałym spadkiem napięcia w kierunku przewodzenia.

\vspace{5pt}
Dlatego, jeżeli do diody przyłożymy napięcie większe od jej napięcia przewodzenia (np. do czerwonej diody LED o spadku około 1.7V przyłożymy napięcie 5V) przez układ taki popłynie bardzo duży prąd (często równy prądowi zwarciowemu naszego źródła), co doprowadzi do zniszczenia diody.

Zobacz symulację: \url{http://ln.opcode.eu.org/led}

\vspace{5pt}
Z tego powodu diody podłączamy prawie zawsze\footnote{Istotnymi wyjątkami są: prostownik (gdzie rolę tego rezystora pełni obciążenie) oraz zasilanie diody ze źródła prądowego.} z szeregowym rezystorem służącym do ograniczenia prądu.
\end{ProTip}

\subsection{prostownik}

Prostownik służy do zamiany napięcia przemiennego (zmieniającego znak) na napięcie zmienne o stałym znaku.
Funkcję tą może pełnić nawet pojedyncza dioda – mamy wtedy do czynienia z prostownikiem jednopołówkowym,
	charakteryzującym się tym że napięcie na jego wyjściu spada przez połowę okresu wynosi zero – zobacz symulację \url{http://ln.opcode.eu.org/prost1}.
Lepszym i częściej stosowanym rozwiązaniem jest prostownik pełnookresowy (dwupołówkowy). Najczęstszą jego realizacją jest tzw. mostek Graetza, czyli układ 4 diod połączonych w taki sposób iż dwie z nich zawsze (w każdym punkcie napięcia wejściowego) przewodzą – zobacz symulację \url{http://ln.opcode.eu.org/prost2}.
Wadą takiego układu jest znaczny spadek napięcia na mostku, wynoszący dwukrotność spadku napięcia na pojedynczej diodzie.

Istnieją również układy prostowników napięcia trójfazowego, charakteryzują się one m.in. niższymi tętnieniami napięcia wyjściowego – dla prostowników jednofazowych wacha się ono (pomijając spadki na diodach) od $0$ do $V_{LN}\sqrt{2}$, a dla pełnookresowego trójfazowego od $V_{LL}\sqrt{2}\sin{60}$ do $V_{LL}\sqrt{2}$ (gdzie $V_{LN}$ to napięcie skuteczne pomiędzy fazą a przewodem neutralnym, a $V_{LL} = V_{LN}\sqrt{3}$ to napięcie skuteczne międzyfazowe). Zobacz symulację: \url{http://ln.opcode.eu.org/prost3}.

\subsection{dzielnik napięcia z diodą Zenera}

W rozdziale \ref{dzielnik} omawialiśmy rezystancyjny dzielnik napięcia złożony z dwóch rezystorów. Wadą takiego układu była duża zależność napięcia wyjściowego od obciążenia. Zjawisko to można ograniczyć zastępując jeden z rezystorów (ten równolegle połączony z obciążeniem) diodą Zenera w polaryzacji zaporowej, która charakteryzuje się dość stałym spadkiem napięcia. Zobacz symulację \url{http://ln.opcode.eu.org/zener}, zauważ że nadal nie jest to rozwiązanie idealne, ale znacznie bardziej stabilne od poprzedniego.
% END: Elektronika - Dioda

\begin{teacherOnly}
\noindent\begin{minipage}[t]{0.6\textwidth}
\strong{POKAZ: LED jako detektor światła}\\
Wcześniej można pokazać poniższy schemat (bez podpisu "detektor" prz D1) i poprosić aby spróbować odgadnąć co on może robić.
Następnie omówić działanie układu, pokazać na pojedynczym LED że oświetlenie powoduje pojawienie się napięcia i pokazać że ten układ działa.
\end{minipage}
\hfill
\begin{minipage}[t]{0.35\textwidth}
\vspace{-10pt}
\includegraphics[width=\textwidth]{img/elektronika/zad-detektor_na_LED-pokaz}
\end{minipage}
\end{teacherOnly}
