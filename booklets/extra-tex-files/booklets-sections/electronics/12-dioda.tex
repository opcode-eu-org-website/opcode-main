% Copyright (c) 2017-2020 Matematyka dla Ciekawych Świata (http://ciekawi.icm.edu.pl/)
% Copyright (c) 2017-2020 Robert Ryszard Paciorek <rrp@opcode.eu.org>
% 
% MIT License
% 
% Permission is hereby granted, free of charge, to any person obtaining a copy
% of this software and associated documentation files (the "Software"), to deal
% in the Software without restriction, including without limitation the rights
% to use, copy, modify, merge, publish, distribute, sublicense, and/or sell
% copies of the Software, and to permit persons to whom the Software is
% furnished to do so, subject to the following conditions:
% 
% The above copyright notice and this permission notice shall be included in all
% copies or substantial portions of the Software.
% 
% THE SOFTWARE IS PROVIDED "AS IS", WITHOUT WARRANTY OF ANY KIND, EXPRESS OR
% IMPLIED, INCLUDING BUT NOT LIMITED TO THE WARRANTIES OF MERCHANTABILITY,
% FITNESS FOR A PARTICULAR PURPOSE AND NONINFRINGEMENT. IN NO EVENT SHALL THE
% AUTHORS OR COPYRIGHT HOLDERS BE LIABLE FOR ANY CLAIM, DAMAGES OR OTHER
% LIABILITY, WHETHER IN AN ACTION OF CONTRACT, TORT OR OTHERWISE, ARISING FROM,
% OUT OF OR IN CONNECTION WITH THE SOFTWARE OR THE USE OR OTHER DEALINGS IN THE
% SOFTWARE.

% BEGIN: Elektronika - Dioda
\section{Dioda}
\begin{teacherOnly}
	\begin{easylist}[itemize]
		& przewodzi w jednym kierunku
		& element nieliniowy -> łamie prawo Ohma
		& dioda prostownicza i dioda świecąca
			&& rezystor przy diodzie świecącej -> na diodzie stały spadek napięcia w kierunku przewodzenia (\textbf{symulacja})
			&& dioda prostownicza – prostownik jedno połówkowy i mostek prostowniczy (\textbf{symulacja})
		& czy może przewodzić w drugą stronę -> dioda Zenera
		& dzielnik napięcia z diodą Zenera (\textbf{symulacja})
		& fotodioda
		& dioda LED jako detektor światła (jeżeli będzie czas!)
	\end{easylist}
\end{teacherOnly}

Dioda idealna to element przewodzący prąd tylko w jednym kierunku. Rzeczywiste diody przewodzą prąd zdecydowanie chętniej w jednym kierunku niż w drugim (na ogół przewodzenie w kierunku zaporowym się pomija) ponadto charakteryzują je cechy zależne od technologi wykonania takie jak:
\begin{itemize}
\item spadek napięcia w kierunku przewodzenia (typowo dla diod krzemowych 0.6V - 0.7V, a dla diod Schottky’ego 0.3V)
\item napięcie przebicia - napięcie, które przyłożone w kierunku zaporowym powoduje znaczące przewodzenie diody w tym kierunku - w większości przypadków parametr którego nie należy przekraczać, jednak wykorzystywane (i stanowiące ich parametr) w niektórych typach diod
\item maksymalny prąd przewodzenia
\item czas przełączania (związany głównie z pasożytniczą pojemnością złącza) - zdecydowanie krótszy (około 100 ps) w diodach Schottky’ego niż w diodach krzemowych,.
\end{itemize}

\begin{wrapfigure}{r}{0.3\textwidth}
  \begin{center}
    \includegraphics[width=0.25\textwidth]{img/elektronika/diody}
  \end{center}
\end{wrapfigure}

\noindent
Ponadto stosowane są m.in.:
\begin{itemize}
\item diody Zenera - wykorzystuje się (charakterystyczną dla danego typu) wartość napięcia przebicia do uzyskania w układzie spadku napięcia o tej wartości,
\item diody świecące (LED) - emitujące światło w trakcie przewodzenia (na elemencie występuje stały spadek napięcia, jasność zależy od natężenia prądu),
\item fotodiody - będące detektorami oświetlenia (przewodzenie spolaryzowanej w kierunku zaporowym zależy od ilości padającego na element światła, niespolaryzowana pod wpływem oświetlenia staje się źródłem prądu).
\end{itemize}

\begin{ProTip}{{\Symbola ❢} PAMIĘTAJ {\Symbola ❢}}
Dioda jest elementem dla którego nie jest spełnione prawo Ohma. Dioda charakteryzuje się prawie stałym spadkiem napięcia w kierunku przewodzenia.

\vspace{5pt}
Dlatego, jeżeli do diody przyłożymy napięcie większe od jej napięcia przewodzenia (np. do czerwonej diody LED o spadku około 1.7V przyłożymy napięcie 5V) przez układ taki popłynie bardzo duży prąd (często równy prądowi zwarciowemu naszego źródła), co doprowadzi do zniszczenia diody.

Zobacz symulację: \url{http://ln.opcode.eu.org/led}
	% https://www.falstad.com/circuit/circuitjs.html?cct=%24+1+0.000005+10.20027730826997+50+1+50%0A172+336+176+336+128+0+7+3.5300000000000002+5+-2+0+0.5+Voltage%0Aw+336+320+336+352+1%0Ag+336+352+336+368+0%0Ad+336+256+336+320+2+default-led%0Ar+336+176+336+256+0+1000%0Aw+336+256+400+256+0%0Aw+336+320+400+320+0%0Ap+400+256+400+320+1+0%0Aw+336+176+400+176+0%0As+400+176+400+256+0+1+false%0A

\vspace{5pt}
Z tego powodu diody podłączamy prawie zawsze\footnote{Istotnymi wyjątkami są: prostownik (gdzie rolę tego rezystora pełni obciążenie) oraz zasilanie diody ze źródła prądowego.} z szeregowym rezystorem służącym do ograniczenia prądu.
\end{ProTip}

\subsection{prostownik}

Prostownik służy do zamiany napięcia przemiennego (zmieniającego znak) na napięcie zmienne o stałym znaku.
Funkcję tą może pełnić nawet pojedyncza dioda – mamy wtedy do czynienia z prostownikiem jednopołówkowym,
	charakteryzującym się tym że napięcie na jego wyjściu spada przez połowę okresu wynosi zero – zobacz symulację \url{http://ln.opcode.eu.org/prost1}.
	% https://www.falstad.com/circuit/circuitjs.html?cct=%24+1+0.000005+13.097415321081861+55+5+50%0Av+112+320+112+96+0+1+40+5+0+0+0.5%0Ar+416+96+416+320+0+640%0Ad+112+96+416+96+2+default%0Aw+112+320+416+320+0%0Ao+0+64+0+4102+5+0.0125+0+2+1+0%0A
Lepszym i częściej stosowanym rozwiązaniem jest prostownik pełnookresowy (dwupołówkowy). Najczęstszą jego realizacją jest tzw. mostek Graetza, czyli układ 4 diod połączonych w taki sposób iż dwie z nich zawsze (w każdym punkcie napięcia wejściowego) przewodzą – zobacz symulację \url{http://ln.opcode.eu.org/prost2}.
	% https://www.falstad.com/circuit/circuitjs.html?cct=%24+1+0.000005+10.20027730826997+53+5+50%0Av+160+352+160+64+0+1+40+5+0+0+0.5%0Aw+160+64+304+64+0%0Aw+304+64+304+128+0%0Ad+304+128+368+192+2+default%0Ad+304+256+368+192+2+default%0Ad+240+192+304+128+2+default%0Ad+240+192+304+256+2+default%0Aw+304+256+304+352+0%0Aw+304+352+160+352+0%0Aw+240+192+240+288+0%0Aw+368+192+416+192+0%0Aw+240+288+416+288+0%0Ar+416+192+416+288+0+100%0Ax+463+248+557+251+4+20+load%0Ao+0+64+0+4098+5+0.05+0+2+12+0%0A
Wadą takiego układu jest znaczny spadek napięcia na mostku, wynoszący dwukrotność spadku napięcia na pojedynczej diodzie.

Istnieją również układy prostowników napięcia trójfazowego, charakteryzują się one m.in. niższymi tętnieniami napięcia wyjściowego – dla prostowników jednofazowych wacha się ono (pomijając spadki na diodach) od $0$ do $V_{LN}\sqrt{2}$, a dla pełnookresowego trójfazowego od $V_{LL}\sqrt{2}\sin{60}$ do $V_{LL}\sqrt{2}$ (gdzie $V_{LN}$ to napięcie skuteczne pomiędzy fazą a przewodem neutralnym, a $V_{LL} = V_{LN}\sqrt{3}$ to napięcie skuteczne międzyfazowe). Zobacz symulację: \url{http://ln.opcode.eu.org/prost3}.
	% https://www.falstad.com/circuit/circuitjs.html?cct=%24+1+0.000005+2.008553692318767+45+5+50%0Av+144+144+224+144+0+1+40+5+0+0+0.5%0Ad+352+144+352+80+2+default%0Ad+352+336+352+144+2+default%0Ad+400+208+400+80+2+default%0Ad+400+336+400+208+2+default%0Ar+512+160+512+256+0+100%0Ax+559+216+653+219+4+20+load%0Av+144+208+224+208+0+1+40+5+0+2.0943951023931953+0.5%0Av+144+272+224+272+0+1+40+5+0+4.1887902047863905+0.5%0Aw+144+144+144+208+0%0Aw+144+208+144+272+0%0Aw+224+208+400+208+0%0Aw+224+144+352+144+0%0Ad+304+272+304+80+2+default%0Aw+224+272+304+272+0%0Ad+304+336+304+272+2+default%0Aw+304+336+352+336+0%0Aw+352+336+400+336+0%0Aw+304+80+352+80+0%0Aw+352+80+400+80+0%0Aw+400+80+512+80+0%0Aw+512+80+512+160+0%0Aw+400+336+512+336+0%0Aw+512+336+512+256+0%0Ao+5+64+0+4098+10+0.1+0+8+5+3+0+0+0+3+7+0+7+3+8+0+8+3%0A

\subsection{dzielnik napięcia z diodą Zenera}

W rozdziale \ref{dzielnik} omawialiśmy rezystancyjny dzielnik napięcia złożony z dwóch rezystorów. Wadą takiego układu była duża zależność napięcia wyjściowego od obciążenia. Zjawisko to można ograniczyć zastępując jeden z rezystorów (ten równolegle połączony z obciążeniem) diodą Zenera w polaryzacji zaporowej, która charakteryzuje się dość stałym spadkiem napięcia. Zobacz symulację \url{http://ln.opcode.eu.org/zener}, zauważ że nadal nie jest to rozwiązanie idealne, ale znacznie bardziej stabilne od poprzedniego.
	%https://www.falstad.com/circuit/circuitjs.html?cct=%24+1+0.000005+10.20027730826997+54+5+50%0Az+336+288+336+160+3+default-zener%0Ag+336+288+336+304+0%0Aw+336+160+416+160+0%0Ar+416+224+416+288+0+1000%0Ag+416+288+416+304+0%0Ar+272+160+336+160+0+500%0As+416+160+416+224+0+1+false%0As+480+160+480+224+0+1+false%0Ag+480+288+480+304+0%0Ar+480+224+480+288+0+100%0Aw+416+160+480+160+0%0AR+272+160+192+160+0+0+40+11.68+0+0+0.5%0Aw+480+160+560+160+0%0Ap+560+160+560+288+1+0%0Ag+560+288+560+304+0%0A38+11+0+2+24+Voltage%0A
% END: Elektronika - Dioda

\begin{teacherOnly}
\noindent\begin{minipage}[t]{0.6\textwidth}
\strong{POKAZ: LED jako detektor światła}\\
Wcześniej można pokazać poniższy schemat (bez podpisu "detektor" prz D1) i poprosić aby spróbować odgadnąć co on może robić.
Następnie omówić działanie układu, pokazać na pojedynczym LED że oświetlenie powoduje pojawienie się napięcia i pokazać że ten układ działa.
\end{minipage}
\hfill
\begin{minipage}[t]{0.35\textwidth}
\vspace{-10pt}
\includegraphics[width=\textwidth]{img/elektronika/zad-detektor_na_LED-pokaz}
\end{minipage}
\end{teacherOnly}
