% Copyright (c) 2017-2020 Matematyka dla Ciekawych Świata (http://ciekawi.icm.edu.pl/)
% Copyright (c) 2017-2020 Robert Ryszard Paciorek <rrp@opcode.eu.org>
% 
% MIT License
% 
% Permission is hereby granted, free of charge, to any person obtaining a copy
% of this software and associated documentation files (the "Software"), to deal
% in the Software without restriction, including without limitation the rights
% to use, copy, modify, merge, publish, distribute, sublicense, and/or sell
% copies of the Software, and to permit persons to whom the Software is
% furnished to do so, subject to the following conditions:
% 
% The above copyright notice and this permission notice shall be included in all
% copies or substantial portions of the Software.
% 
% THE SOFTWARE IS PROVIDED "AS IS", WITHOUT WARRANTY OF ANY KIND, EXPRESS OR
% IMPLIED, INCLUDING BUT NOT LIMITED TO THE WARRANTIES OF MERCHANTABILITY,
% FITNESS FOR A PARTICULAR PURPOSE AND NONINFRINGEMENT. IN NO EVENT SHALL THE
% AUTHORS OR COPYRIGHT HOLDERS BE LIABLE FOR ANY CLAIM, DAMAGES OR OTHER
% LIABILITY, WHETHER IN AN ACTION OF CONTRACT, TORT OR OTHERWISE, ARISING FROM,
% OUT OF OR IN CONNECTION WITH THE SOFTWARE OR THE USE OR OTHER DEALINGS IN THE
% SOFTWARE.

\IfStrEq{\dbEntryID}{}{
	\begin{center}
		\includegraphics[width=0.55\textwidth]{img/elektronika/zadania_teoretyczne-szeregowe_rownolegle}
	\end{center}
	\insertZadanie{\currfilepath}{polaczenie_szeregowe}{}
	\insertZadanie{\currfilepath}{polaczenie_rownolegle}{}
	\begin{center}
		\includegraphics[width=0.75\textwidth]{img/elektronika/zadania_teoretyczne-A}
	\end{center}
	\insertZadanie{\currfilepath}{prad_R1}{}
	\insertZadanie{\currfilepath}{napiecie_T1}{}
	\insertZadanie{\currfilepath}{napiecie_T234}{}
}

\IfStrEq{\dbEntryID}{praktyczne}{
	\insertZadanie{\currfilepath}{zbuduj_ladowanie_kondensatora}{}
	\insertZadanie{\currfilepath}{zbuduj_spadek_napiecia_na_led}{}
	\insertZadanie{\currfilepath}{zbuduj_stabilizator_zener1}{}
}

\IfStrEq{\dbEntryID}{rozwiazania}{
	\insertRozwiazanie{\currfilepath}{prad_R1}{}
	\insertRozwiazanie{\currfilepath}{napiecie_T1}{}
	\insertRozwiazanie{\currfilepath}{napiecie_T234}{}
}


%
% zadania teoretyczne
%

\dbEntryBegin{prad_R1}\if1\dbEntryCheckResults
Oszacuj wartość prądu płynącego przez R1. Odpowiedź krótko uzasadnij.
\fi
\dbEntryBegin{prad_R1-rozwiazanie}\if1\dbEntryCheckResults
Należy uwzględnić spadek napięcia na diodzie około 1.7V, w efekcie czego mamy napięcie na rezystorze 3.3V.
Napięcie to wraz z wartością tego rezystora określa prąd płynący w obwodzie i wynoszący 3.3mA.
\fi

\dbEntryBegin{napiecie_T1}\if1\dbEntryCheckResults
Podaj wartość napięcia (względem GND) w punkcie T1 w sytuacji gdy S1 jest wciśnięty (zwarty) oraz w sytuacji gdy jest rozwarty (nie przewodzi). Odpowiedź krótko uzasadnij.
\fi
\dbEntryBegin{napiecie_T1-rozwiazanie}\if1\dbEntryCheckResults
Przycisk wciśnięty – punkt T1 połączony z masą, czyli napięcie w T1 wynosi 0V.\\
Przycisk rozwarty – prąd nie płynie, spadek napięcia na rezystorze wynosi 0V (brak prądu), czyli napięcie w T1 wynosi 5V.
\fi

\dbEntryBegin{napiecie_T234}\if1\dbEntryCheckResults
Podaj wartość napięcia (względem GND) w punkach T2, T3, T4. Odpowiedź krótko uzasadnij.
\fi
\dbEntryBegin{napiecie_T234-rozwiazanie}\if1\dbEntryCheckResults
T2 – typowy dzielnik rezystancyjny, napięcie w T2 wynosi 3V.\\
T3 – dzielnik z diodą Zenera, dioda Zenera przewodzi, napięcie w T3 wynosi 3.3V.\\
T4 – dzielnik z diodą Zenera, dioda Zenera nie przewodzi (przyłożone napięcie mniejsze od napięcia przebicia), napięcie w T4 wynosi 0V.
\fi

\dbEntryBegin{polaczenie_szeregowe}\if1\dbEntryCheckResults
  Rezystory $R_1$ i $R_2$ połączone zostały szeregowo i podłączone do źródła napięcia U, tak jak pokazano na powyższym schemacie.
  \begin{enumerate}[label=\alph*)]
    \item Zapisz zależność (wzór) określający spadek napięcia na rezystorze $R_1$.
    \item Zapisz zależność (wzór) określający sumaryczną moc, która wydzieli się na obu tych rezystorach.
  \end{enumerate}
\fi

\dbEntryBegin{polaczenie_rownolegle}\if1\dbEntryCheckResults
  Rezystory $R_3$ i $R_4$ połączone zostały równolegle i podłączone do źródła napięcia U, tak jak pokazano na powyższym schemacie.
  \begin{enumerate}[label=\alph*)]
    \item Zapisz zależność (wzór) określający prąd płynący przez rezystor $R_3$.
    \item Zapisz zależność (wzór) określający łączną rezystancję obu rezystorów (czyli ich rezystancję zastępczą).
  \end{enumerate}
\fi

%
% zadania praktyczne
%

\dbEntryBegin{zbuduj_ladowanie_kondensatora}\if1\dbEntryCheckResults
\noindent\begin{minipage}[b]{0.55\textwidth}
Zbuduj układ przedstawiony na schemacie i zaobserwuj zmianę napięcia na kondensatorze w momencie załączania, wyłączania zasilania.

\vspace{13pt}

Zobacz jak zmieni się działanie układu gdy zmienisz wartości elementów (np. wartość rezystora R2).

\ifdefined\ladowanieKondensatoraWartosci\vspace{13pt}\ladowanieKondensatoraWartosci\fi
\end{minipage}
\hfill
\begin{minipage}[b]{0.4\textwidth}
\includegraphics[width=\textwidth]{img/elektronika/zad-kondensator}
\end{minipage}
\fi


\dbEntryBegin{zbuduj_spadek_napiecia_na_led}\if1\dbEntryCheckResults
\noindent\begin{minipage}[b]{0.7\textwidth}
Zbuduj układ przedstawiony na schemacie i zaobserwować że dla różnych napięć wejściowych (z zakresu 5-13V) na diodzie świecącej występuje stały spadek napięcia.
Zaobserwuj że zmianie ulega wartość prądu płynącego w takim obwodzie oraz że wynika ona z napięcia odłożonego na rezystorze i wartości jego rezystancji.
\end{minipage}
\hfill
\begin{minipage}[b]{0.25\textwidth}
\includegraphics[width=\textwidth]{img/elektronika/zad-led}\vspace{0.5cm}
\end{minipage}

\hfill \rule{0.8\textwidth}{.3pt}\hfill 

\noindent\begin{minipage}[b]{0.27\textwidth}
\includegraphics[width=\textwidth]{img/elektronika/zad-led_2}
\end{minipage}
\hfill
\begin{minipage}[b]{0.6\textwidth}
Jeżeli nie posiadasz regulowanego źródła napięcia możesz w jego roli użyć dzielnik z rezystorem nastawnym, tak jak pokazano na schemacie po lewej.
\vspace{1.3cm}
\end{minipage}
\hfill 
\fi


\dbEntryBegin{zbuduj_stabilizator_zener1}\if1\dbEntryCheckResults
\noindent\begin{minipage}[b]{0.77\textwidth}
Zbuduj układ stabilizacji napięcia w oparciu o diodę Zenera przedstawiony na schemacie obok.

Zastanów się nad sposobem działania tego układu – w tym celu dokonaj pomiarów napięcia wyjściowego w zależności od napięcia wejściowego.

Zobacz jak na napięcie wyjściowe wpływa wielkość obciążenia symulowanego przez R2.
\vspace{13pt}
\end{minipage}
\hfill
\begin{minipage}[b]{0.17\textwidth}
\includegraphics[width=\textwidth]{img/elektronika/zad-stabilizator_zener1}
\end{minipage}
\fi
