% BEGIN: Elektronika - Transmisja - sterowanie linią
\section{Transmisja - sterowanie linią}
\subsection{bufory}

Bufor jest to układ przekazujący sygnał logiczny z wejścia na wyjście. Bufor może służyć do:
\begin{itemize}
\item regeneracji sygnału,
\item uniemożliwieniu wprowadzenia sygnału z jego strony wyjściowej na wejściową,
\item decydowania o jego przepuszczeniu lub nie (trój-stanowy),
\item decydowania o kierunku przepuszczenia sygnału (dwa trój-stanowe albo trój-stanowy dwukierunkowy),
\item konwersji na linię open-collector / open-drain,
\item negacji sygnału (niektóre bufory realizują funkcję NOT).
\end{itemize}

\subsection{enkodery}

Enkoder "n to m" jest to układ o n wejściach, który na swoim m bitowym wyjściu wystawia numer (typowo) wejścia o najwyższym numerze, które znajduje się w stanie niskim. Możliwe są też warianty wystawiające numer pierwszego (a nie ostatniego) w kolejności wejścia lub wybierające wejście ze stanem wysokim.

Jako że wejścia numerowane są zazwyczaj od zera do 2m to układ najczęściej posiada dodatkowe wyjście informujące że którekolwiek z wejść jest w stanie aktywnym. Typowo numer wystawiany jest w postaci NKB, ale możliwe są inne kodowania.

Układ pozwala na redukcję ilości wejść potrzebnych do obsługi n-bitowego sygnału w którym tylko jeden bit może być ustawiony lub w którym można pozwolić sobie na obsługę kolejnych linii z kasowaniem ich bitu (np. wektor przerwań z priorytetyzacją).

\subsection{dekodery}

Dekoder "m to n" jest układem o działaniu przeciwnym do enkodera. Aktywuje on wyjście o numerze odpowiadającym wartości na m-bitowym wejściu adresowym. Typowo posiada także wejście zezwolenia na aktywację wyjść, które może zostać użyte do podłączenia informacji że którekolwiek z wejść enkodera było w stanie aktywnym lub do podłączenia sygnału danych z multipleksowanej linii celem jej demultipleksacji.

Przykład użycia enkodera i dekodera do obsługi matrycy przełączników (klawiatury) można zobaczyć na symulacji: \url{http://ln.opcode.eu.org/matrix}.

\subsection{(de)multipleksery cyfrowy}

Multiplekser cyfrowy (jednokierunkowy) na wyjście kopiuje stan wskazanego (poprzez adres podany na wejście adresowe) wejścia. W przypadku braku sygnału "enable" w zależności od rozwiązania wyjście pozostanie w stanie niskim lub wysokiej impedancji.

Demultiplekser cyfrowy (jednokierunkowy) to zazwyczaj układ dekodera w którym na wejście enabled podany jest sygnał z multipleksowanej linii. Nie wybrane wyjścia pozostają w stanie niskim lub wysokim (zależnie od użycia nieodwracającego lub odwracającego dekodera). Cyfrowe demultipleksery z 3 stanowym wyjściem są rzadkością. Demultipleksację można rozwiązać także przy pomocy odpowiednio sterowanych (np. z dekodera adresu) buforów trój-stanowych lub dwu-wejściowych multiplekserów.

\subsection{(de)multipleksery analogowy}

Multiplekser analogowy (dwukierunkowy) działa na zasadzie przełącznika łączącego wskazane wejście z wyjściem, dzięki elektrycznemu "zwarciu" (na ogół rezystancja takiego zwarcia to kilkadziesiąt omów) wejścia z wyjściem transmisja może odbywać się w obu kierunkach. Pozwala to także na wykorzystanie tego samego układu w roli multipleksera i demultipleksera.
% END: Elektronika - Transmisja - sterowanie linią


% BEGIN: Elektronika - Typy transmisji
\section{Topologie i typy transmisji}

W zależności od układu fizycznych połączeń komunikujących się urządzeń wyróżnia się różne topologie sieci.
Na schemacie poniżej przedstawione zostały główne topologie połączeń:

\begin{itemize}
\item \strong{magistrala} (linear bus) -- wszystkie urządzenia są podłączone do jednej linii (wspólnego medium transmisyjnego), okablowanie nie wyróżnia punktu centralnego
\item \strong{łańcuch} (daisy chain) -- struktura okablowania podobna jak w magistrali, ale medium transmisyjne jest podzielone (połączenie n urządzeń składa się z n-1 łączy punkt-punkt pomiędzy urządzeniami)
\item \strong{pierścień} (ring) -- topologia daisy chain w której końce są połączone, uodparnia to na pojedyncze uszkodzenie
\item \strong{gwiazda} (star) -- wszystkie podłączenia biorą początek w węźle centralnym, w zależności od konstrukcji węzła centralnego może być realizowana w oparciu o współdzielone medium lub połączenia punkt-punkt
\item \strong{krata} (mesh) -- każde urządzenie ma bezpośrednie połączenie punkt-punkt do każdego innego urządzenia (połączenie n urządzeń wymaga n(n-1)/2 połączeń punkt punkt)
\end{itemize}

\begin{center}
    \includegraphics[width=0.7\textwidth]{img/elektronika/topologie}
\end{center}

Ponadto występują topologie mieszane złożone z opisanych powyżej: gwiazda wielokrotna (tzn. taka gdzie niektóre z węzłów stanowią punkty centralne kolejnych gwiazd), magistrala lub ring pomiędzy punktami centralnymi gwiazd, gwiazda w której w węzłach występują magistrale lub pierścienie, itd.

\vspace{7pt}

Wyróżnić można także typy transmisji:
\begin{itemize}
\item \strong{simplex} -- umożliwia tylko transmisję jednokierunkową
\item \strong{half-duplex} -- umożliwia transmisję dwukierunkową, ale tylko w jedną stronę równocześnie
\item \strong{full-duplex} -- umożliwia pełną transmisję dwukierunkową (równoczesne nadawanie i odbiór)
\end{itemize}

\section{Magistrala równoległa}

\begin{center}
    \includegraphics[width=0.85\textwidth]{img/elektronika/magistrala_rownolegla}
\end{center}
Magistrala równoległa jest zespołem linii, wraz z układami nimi sterującymi, umożliwiającym równoległe przesyłanie danych (w jednym czasie / takcie zegara na magistrali wystawiane / przesyłane jest całe n-bitowe słowo).
Można wyróżnić szyny sterującą (kierunek przypływu, żądania obsługi, etc), adresową (adres układu który ma prawo nadawać) i danych (przesyłane dane). Często szyna adresowa współdzieli linie transmisyjne z szyną danych.
Do realizacji magistrali (celem umożliwiania podłączenia wielu układów) stosuje się zazwyczaj bufory trój-stanowe, a do zapewnienia współdzielonej szyny żądania obsługi (interrupt request) często układy typu open-collector.

Typowy układ realizacji magistrali half-duplex ze współdzielonymi liniami danych i adresu przestawiony został na schemacie zamieszczonym obok.
Zadaniem dekodera adresu jest ustalenie czy wystawiony na magistrali adres (w trakcie wysokiego stanu linii "Adres / Not Dane") jest adresem danego urządzenia i zapamiętanie tej informacji do czasu wystawienia nowego adresu. Informacja ta jest wykorzystywana do sterowania dwukierunkowym buforem trój-stanowym (jako sygnał enable).
O kierunku działania bufora decyduje sygnał "Read / Not Write". Przy magistralach o ustalonym protokole transmisyjnym sterowanie kierunkiem może być zależne od wykonywanej komendy (po ustawieniu adresu urządzenie odczytuje z magistrali polecenie i w zależności od niego steruje kierunkiem bufora - odczytuje lub zapisuje dane na magistralę).
Zastosowanie kilku linii typu OC do odbierania żądań obsługi pozwala (na podstawie tego które z tych linii znalazły się w stanie niskim na identyfikację urządzenia lub grupy urządzeń, z której niektóre zgłaszają żądanie obsługi.

W przypadku prostych urządzeń wejścia / wyjścia zamiast buforu dwukierunkowego może być umieszczony np.
jednokierunkowy bufor (lub n-bitowy rejestr) z wyjściami trój-stanowymi, który wystawia dane na magistralę w oparciu o sygnał zapisu na magistralę (WR) oraz zegar (clock) albo
n-bitowy rejestr do którego zapisywane są dane z magistrali w oparciu o sygnał RD i Clock.

\section{Magistrala szeregowa}

\begin{center}
    \includegraphics[width=0.85\textwidth]{img/elektronika/magistrala_szeregowa}
\end{center}
W magistrali szeregowej dane przesyłane są bit po bicie w pojedynczej linii. Podobnie jak w magistrali równoległej oprócz linii danych mogą występować także linie sterujące. Prostą realizację magistrali szeregowej zapewniają rejestry przesuwne.

Przykładowy układ realizacji magistrali simplex (jednokierunkowej) z rozdzielonymi szynami danych i adresową został na schemacie zamieszczonym obok.
W prezentowanym przykładzie oprócz adresu master wystawia trzy sygnały - dane, zegar i strobe. Z każdym taktem zegara na linii danych wystawiany jest kolejny bit który jest wczytywany do zespołu rejestrów. Sygnał strobe służy do przepisania wartości z rejestrów przesuwnych do rejestrów wyjściowych, takie rozwiązanie zapobiega zmianom wyjść w trakcie przesyłania nowych danych poprzez szynę szeregową, jest ono jednak opcjonalne.

W zależności od konstrukcji dekodera adresu szyna adresowa może być równoległa (w najprostszym przypadku - przez całą transmisję do danego urządzenia jego adres musi być wystawiony na szynie a dekoder jest układem bramek NOT i wielowejściowej bramki AND) lub szeregowa (w takim wypadku powinna posiadać własny zegar lub sygnał informujący o nadawaniu adresu z taktami zegara głównego, a dekoder powinien być wyposażony w rejestr przesuwny do odebrania i przechowywania aktualnego adresu z magistrali). Natomiast jeżeli magistrala byłaby oparta tylko na połączonych szeregowo rejestrach (wyjście serial-out do wejścia serial-in) to szyna adresowa nie jest potrzebna, ale konieczne może być każdorazowe wpisanie wszystkich wartości na szynę (czas zapisu rośnie z ilością podłączonych rejestrów).

\section{Standardowe interfejsy}

Istnieje wiele zestandaryzowanych interfejsów zarówno szeregowych jak i równoległych, wśród najważniejszych należy wymienić:

\begin{center} \includegraphics[width=0.47\textwidth]{img/elektronika/spi}    \includegraphics[width=0.47\textwidth]{img/elektronika/twi} \end{center}

\subsection{SPI (Serial Peripheral Interface)}
    jest to szeregowa magistrala full-duplex działająca w układzie master-slave złożona z linii zegarowej (SCLK), nadawania przez mastera (MOSI), odbioru przez mastera (MISO) oraz linii służących do aktywacji urządzenia slave (SS / CS). 

\begin{wrapfigure}{r}{0.48\textwidth} \begin{center} \vspace{-20pt} \includegraphics[width=0.43\textwidth]{img/elektronika/onewire} \vspace{-20pt} \end{center} \end{wrapfigure}

\subsection{I2C (TWI)}
    jest to szeregowa magistrala half-duplex złożona z linii sygnałowej (SDA) i zegara (SCL) posiadająca zdefiniowany format ramki wraz z adresowaniem. Z wyjątkiem bitu startu i stopu stan linii danych może zmieniać się tylko przy niskim stanie linii zegarowej.
    Nadajniki są typu open-drain przez co realizowany jest iloczyn na drucie, co pozwala na wykrywanie kolizji (jeżeli dany nadajnik nie nadaje zera a linia jest w stanie zera to nadaje także ktoś inny). Pozwala to także na uzyskanie układów multimaster, pomimo iż typowo na magistrali takiej występuje tylko jeden układ master (nadający sygnał zegara i inicjujący transmisję). 

\subsection{1 wire (one-wire)}
    jest to szeregowa magistrala half-duplex złożona z jedynie z linii sygnałowej (która może także służyć do zasilania urządzeń) posiadająca zdefiniowany format ramki wraz z adresowaniem. Standardowe nadawanie jest realizowane jako open-drain (wyjątkiem jest nadawanie tzw power-byte). 

\subsection{USART}
    jest to uniwersalny synchroniczny i asynchroniczny nadajnik i odbiornik, umożliwia realizację szeregowej transmisji synchronicznej (zgodnie z zegarem) lub asynchronicznej (wykrywanie początku ramki na podstawie linii danych). Interfejs korzysta z rozdzielonych linii nadajnika i odbiornika (wyjście danych TxD oraz wejście danych RxD, co umożliwia realizację transmisji full-duplex) oraz może korzystać z dodatkowych sygnałów sterujących (wyjście RTS informujące o gotowości do odbioru oraz wejście CTS informacji o gotowości odbioru / zezwolenia na nadawanie). Niekiedy dostępne jest także wyjście załączenia nadajnika używane do pracy w trybie half-duplex (linie TxD i RxD połączone buforem trój-stanowym nadajnika).

    \begin{center} \includegraphics[width=0.94\textwidth]{img/elektronika/uart1} \end{center}
    Interfejs ten najczęściej wykorzystywany jest w trybie asynchronicznym jako UART. W połączeniach UART zarówno nadajnik jak i odbiornik muszą mieć ustawione takie same parametry transmisji (szybkość, znaczenie 9 bitu (typowo bit parzystości, ale może także oznaczać np. pole adresowe), itp).
    Głównymi standardami elektrycznymi dla UART są: poziomy napięć układów elektronicznych używających tych portów (3.3V, 5V), RS-232 (w pełnym wariancie używa sygnałów kontroli przepływu, poziom logiczny 1 wynosi od -15V do -3V, a poziom logiczny 0 od +3V do +15V), RS-422 (transmisja różnicowa full-duplex pomiędzy dwoma urządzeniami) i RS-485 (transmisja różnicowa half-duplex w oparciu o szynę łączącą wiele urządzeń, kompatybilny elektrycznie z RS-422), możliwia jest też transmisja światłowodowa i bezprzewodowa.

    \begin{center} \resizebox{0.87\textwidth}{!}{\includegraphics[trim={0 0 0 9cm},clip]{img/elektronika/uart2}} \end{center}
    \begin{center} \resizebox{0.87\textwidth}{!}{\includegraphics[trim={0 6cm 0 0},clip]{img/elektronika/uart2}} \end{center}
% END: Elektronika - Typy transmisji

\begin{ProTip}[breakable]{Rezystory terminujące \zaawansowane{30}}
Niektóre  ze standardów interfejsów komunikacyjnych przewidują kończenie swoich magistral rezystorem terminującym.
Zastosowanie takiego rezystora ma na celu eliminację odbić sygnału, które mogłyby powstać na końcu linii transmisyjnej.

\vspace{7pt}

Zjawisko to występuje w przypadku \textit{linii długich}, czyli takich których długość jest zbliżona lub większa od długości fali związanej z przesyłanym sygnałem.
Jeżeli rozważymy np. impuls o czasie trwania 1$\mu \rm s$ to zajmie on na kablu odcinek o długości około 200m (zależy to od prędkości rozchodzenia się fali elektromagnetycznej w ośrodku który stanowi dany przewód).
Zatem dla sygnału 1MHz (czyli takiego gdzie pojedyncze impulsy są właśnie długości 1$\mu \rm s$) przewód o długości kilkuset metrów będzie linią długą.

\vspace{7pt}

Odbicia te wynikają z faktu, iż przemieszczanie się sygnału (np. naszego impulsu 5V o czasie trwania 1$\mu \rm s$) wzdłuż przewodu związane jest z
	ładowaniem kolejnych pojemności pasożytniczych, związanych z odcinkiem przewodu do którego dociera sygnał.
Dzieje się to kosztem rozładowania pojemności odcinka przewodu który sygnał już opuścił.

W momencie gdy sygnał trafia na koniec przewodnika nie ma możliwości rozładowania tej pojemności na kolejny odcinek przewodu, więc ładunek z nią związany „rozpływa się po kablu” powodując powstanie odbicia.
Odbicie takie (biegnące od końca przewodu w stronę nadajnika) nakłada się na kolejne impulsy naszego sygnału (biegnące od nadajnika) i powoduje zakłócenia w ich odbiorze (interpretacji).

Zastosowanie odpowiedniego rezystora na końcu linii pozwala na rozładowanie tej pojemności (tak jakby był tam kolejny nieskończenie długi odcinek przewodu) i eliminację odbicia.
Wartość tej rezystancji jest charakterystyczna dla danego przewodu i określana przez parametr nazywany \textit{impedancją falową}.

Rezystor terminujący stanowi obciążenie dla nadajnika i powinien on być stosowany tylko na końcach magistrali, czyli np. na ostatnich urządzeniach podłączonych do magistrali (a nie przy każdym urządzeniu do niej włączonym).

\vspace{7pt}

Jeżeli długość linii jest dużo mniejsza (dla sygnałów prostokątnych przyjmuje się że około 20-40 razy) od długości odcinka jaką zajmuje pojedynczy impuls (np. linia długości 3m, dla przykładowego sygnału 1MHz)
to nie ma potrzeby stosowania rezystorów terminujących (często nawet gdy w ogólności dany standard je przewiduje), gdyż stan całej linii jest spójny i wymuszany przez nadajnik (nie jest to przypadek linii długiej).

\vspace{7pt}

Standard I2C nie przewiduje rezystorów terminujących (i nie powinny być w nim używane, zwłaszcza że są to linie open drain i powstawałby dzielnik z rezystorem pull-up).
Wynika to z tego iż przy maksymalnej prędkości tego interfejsu za linie długie należałoby uznać odcinki co najmniej kilkunastometrowe, a z innych względów standard ten posiada ograniczenie do kilku metrów.

Standard RS-485 przewiduje stosowanie rezystorów terminujących 120 Ω, jednak w przypadku krótkich połączeń i/lub małych prędkości transmisji mogą one być pominięte.
\end{ProTip}
