% Copyright (c) 2017-2020 Matematyka dla Ciekawych Świata (http://ciekawi.icm.edu.pl/)
% Copyright (c) 2017-2020 Robert Ryszard Paciorek <rrp@opcode.eu.org>
% 
% MIT License
% 
% Permission is hereby granted, free of charge, to any person obtaining a copy
% of this software and associated documentation files (the "Software"), to deal
% in the Software without restriction, including without limitation the rights
% to use, copy, modify, merge, publish, distribute, sublicense, and/or sell
% copies of the Software, and to permit persons to whom the Software is
% furnished to do so, subject to the following conditions:
% 
% The above copyright notice and this permission notice shall be included in all
% copies or substantial portions of the Software.
% 
% THE SOFTWARE IS PROVIDED "AS IS", WITHOUT WARRANTY OF ANY KIND, EXPRESS OR
% IMPLIED, INCLUDING BUT NOT LIMITED TO THE WARRANTIES OF MERCHANTABILITY,
% FITNESS FOR A PARTICULAR PURPOSE AND NONINFRINGEMENT. IN NO EVENT SHALL THE
% AUTHORS OR COPYRIGHT HOLDERS BE LIABLE FOR ANY CLAIM, DAMAGES OR OTHER
% LIABILITY, WHETHER IN AN ACTION OF CONTRACT, TORT OR OTHERWISE, ARISING FROM,
% OUT OF OR IN CONNECTION WITH THE SOFTWARE OR THE USE OR OTHER DEALINGS IN THE
% SOFTWARE.

% BEGIN: Elektronika - Transmisja - sterowanie linią
\section{Transmisja - sterowanie linią}
\subsection{bufory}

Bufor jest to układ przekazujący sygnał logiczny z wejścia na wyjście. Bufor może służyć do:
\begin{itemize}
\item regeneracji sygnału,
\item uniemożliwieniu wprowadzenia sygnału z jego strony wyjściowej na wejściową,
\item decydowania o jego przepuszczeniu lub nie (trój-stanowy),
\item decydowania o kierunku przepuszczenia sygnału (dwa trój-stanowe albo trój-stanowy dwukierunkowy),
\item konwersji na linię open-collector / open-drain,
\item negacji sygnału (niektóre bufory realizują funkcję NOT).
\end{itemize}

\subsection{enkodery}

Enkoder "n to m" jest to układ o n wejściach, który na swoim m bitowym wyjściu wystawia numer (typowo) wejścia o najwyższym numerze, które znajduje się w stanie niskim. Możliwe są też warianty wystawiające numer pierwszego (a nie ostatniego) w kolejności wejścia lub wybierające wejście ze stanem wysokim.

Jako że wejścia numerowane są zazwyczaj od zera do 2m to układ najczęściej posiada dodatkowe wyjście informujące że którekolwiek z wejść jest w stanie aktywnym. Typowo numer wystawiany jest w postaci NKB, ale możliwe są inne kodowania.

Układ pozwala na redukcję ilości wejść potrzebnych do obsługi n-bitowego sygnału w którym tylko jeden bit może być ustawiony lub w którym można pozwolić sobie na obsługę kolejnych linii z kasowaniem ich bitu (np. wektor przerwań z priorytetyzacją).

\subsection{dekodery}

Dekoder "m to n" jest układem o działaniu przeciwnym do enkodera. Aktywuje on wyjście o numerze odpowiadającym wartości na m-bitowym wejściu adresowym. Typowo posiada także wejście zezwolenia na aktywację wyjść, które może zostać użyte do podłączenia informacji że którekolwiek z wejść enkodera było w stanie aktywnym lub do podłączenia sygnału danych z multipleksowanej linii celem jej demultipleksacji.

Przykład użycia enkodera i dekodera do obsługi matrycy przełączników (klawiatury) można zobaczyć na symulacji: \url{http://ln.opcode.eu.org/matrix}.

\subsection{(de)multipleksery cyfrowy}

Multiplekser cyfrowy (jednokierunkowy) na wyjście kopiuje stan wskazanego (poprzez adres podany na wejście adresowe) wejścia. W przypadku braku sygnału "enable" w zależności od rozwiązania wyjście pozostanie w stanie niskim lub wysokiej impedancji.

Demultiplekser cyfrowy (jednokierunkowy) to zazwyczaj układ dekodera w którym na wejście enabled podany jest sygnał z multipleksowanej linii. Nie wybrane wyjścia pozostają w stanie niskim lub wysokim (zależnie od użycia nieodwracającego lub odwracającego dekodera). Cyfrowe demultipleksery z 3 stanowym wyjściem są rzadkością. Demultipleksację można rozwiązać także przy pomocy odpowiednio sterowanych (np. z dekodera adresu) buforów trój-stanowych lub dwu-wejściowych multiplekserów.

\subsection{(de)multipleksery analogowy}

Multiplekser analogowy (dwukierunkowy) działa na zasadzie przełącznika łączącego wskazane wejście z wyjściem, dzięki elektrycznemu "zwarciu" (na ogół rezystancja takiego zwarcia to kilkadziesiąt omów) wejścia z wyjściem transmisja może odbywać się w obu kierunkach. Pozwala to także na wykorzystanie tego samego układu w roli multipleksera i demultipleksera.
% END: Elektronika - Transmisja - sterowanie linią


% BEGIN: Elektronika - Typy transmisji
\section{Topologie i typy transmisji}

W zależności od układu fizycznych połączeń komunikujących się urządzeń wyróżnia się różne topologie sieci.
Na schemacie poniżej przedstawione zostały główne topologie połączeń:

\begin{itemize}
\item \strong{magistrala} (linear bus) -- wszystkie urządzenia są podłączone do jednej linii (wspólnego medium transmisyjnego), okablowanie nie wyróżnia punktu centralnego
\item \strong{łańcuch} (daisy chain) -- struktura okablowania podobna jak w magistrali, ale medium transmisyjne jest podzielone (połączenie n urządzeń składa się z n-1 łączy punkt-punkt pomiędzy urządzeniami)
\item \strong{pierścień} (ring) -- topologia daisy chain w której końce są połączone, uodparnia to na pojedyncze uszkodzenie
\item \strong{gwiazda} (star) -- wszystkie podłączenia biorą początek w węźle centralnym, w zależności od konstrukcji węzła centralnego może być realizowana w oparciu o współdzielone medium lub połączenia punkt-punkt
\item \strong{krata} (mesh) -- każde urządzenie ma bezpośrednie połączenie punkt-punkt do każdego innego urządzenia (połączenie n urządzeń wymaga n(n-1)/2 połączeń punkt punkt)
\end{itemize}

\begin{center}
    \includegraphics[width=0.7\textwidth]{img/elektronika/topologie}
\end{center}

Ponadto występują topologie mieszane złożone z opisanych powyżej: gwiazda wielokrotna (tzn. taka gdzie niektóre z węzłów stanowią punkty centralne kolejnych gwiazd), magistrala lub ring pomiędzy punktami centralnymi gwiazd, gwiazda w której w węzłach występują magistrale lub pierścienie, itd.

\vspace{7pt}

Wyróżnić można także typy transmisji:
\begin{itemize}
\item \strong{simplex} -- umożliwia tylko transmisję jednokierunkową
\item \strong{half-duplex} -- umożliwia transmisję dwukierunkową, ale tylko w jedną stronę równocześnie
\item \strong{full-duplex} -- umożliwia pełną transmisję dwukierunkową (równoczesne nadawanie i odbiór)
\end{itemize}

\subsection{Magistrala równoległa}

\begin{center}
    \includegraphics[width=0.85\textwidth]{img/elektronika/magistrala_rownolegla}
\end{center}
Magistrala równoległa jest zespołem linii, wraz z układami nimi sterującymi, umożliwiającym równoległe przesyłanie danych (w jednym czasie / takcie zegara na magistrali wystawiane / przesyłane jest całe n-bitowe słowo).
Można wyróżnić szyny sterującą (kierunek przypływu, żądania obsługi, etc), adresową (adres układu który ma prawo nadawać) i danych (przesyłane dane). Często szyna adresowa współdzieli linie transmisyjne z szyną danych.
Do realizacji magistrali (celem umożliwiania podłączenia wielu układów) stosuje się zazwyczaj bufory trój-stanowe, a do zapewnienia współdzielonej szyny żądania obsługi (interrupt request) często układy typu open-collector.

Typowy układ realizacji magistrali half-duplex ze współdzielonymi liniami danych i adresu przestawiony został na schemacie zamieszczonym obok.
Zadaniem dekodera adresu jest ustalenie czy wystawiony na magistrali adres (w trakcie wysokiego stanu linii "Adres / Not Dane") jest adresem danego urządzenia i zapamiętanie tej informacji do czasu wystawienia nowego adresu. Informacja ta jest wykorzystywana do sterowania dwukierunkowym buforem trój-stanowym (jako sygnał enable).
O kierunku działania bufora decyduje sygnał "Read / Not Write". Przy magistralach o ustalonym protokole transmisyjnym sterowanie kierunkiem może być zależne od wykonywanej komendy (po ustawieniu adresu urządzenie odczytuje z magistrali polecenie i w zależności od niego steruje kierunkiem bufora - odczytuje lub zapisuje dane na magistralę).
Zastosowanie kilku linii typu OC do odbierania żądań obsługi pozwala (na podstawie tego które z tych linii znalazły się w stanie niskim na identyfikację urządzenia lub grupy urządzeń, z której niektóre zgłaszają żądanie obsługi.

W przypadku prostych urządzeń wejścia / wyjścia zamiast buforu dwukierunkowego może być umieszczony np.
jednokierunkowy bufor (lub n-bitowy rejestr) z wyjściami trój-stanowymi, który wystawia dane na magistralę w oparciu o sygnał zapisu na magistralę (WR) oraz zegar (clock) albo
n-bitowy rejestr do którego zapisywane są dane z magistrali w oparciu o sygnał RD i Clock.

\subsection{Magistrala szeregowa}

\begin{center}
    \includegraphics[width=0.85\textwidth]{img/elektronika/magistrala_szeregowa}
\end{center}
W magistrali szeregowej dane przesyłane są bit po bicie w pojedynczej linii. Podobnie jak w magistrali równoległej oprócz linii danych mogą występować także linie sterujące. Prostą realizację magistrali szeregowej zapewniają rejestry przesuwne.

Przykładowy układ realizacji magistrali simplex (jednokierunkowej) z rozdzielonymi szynami danych i adresową został na schemacie zamieszczonym obok.
W prezentowanym przykładzie oprócz adresu master wystawia trzy sygnały - dane, zegar i strobe. Z każdym taktem zegara na linii danych wystawiany jest kolejny bit który jest wczytywany do zespołu rejestrów. Sygnał strobe służy do przepisania wartości z rejestrów przesuwnych do rejestrów wyjściowych, takie rozwiązanie zapobiega zmianom wyjść w trakcie przesyłania nowych danych poprzez szynę szeregową, jest ono jednak opcjonalne.

W zależności od konstrukcji dekodera adresu szyna adresowa może być równoległa (w najprostszym przypadku - przez całą transmisję do danego urządzenia jego adres musi być wystawiony na szynie a dekoder jest układem bramek NOT i wielowejściowej bramki AND) lub szeregowa (w takim wypadku powinna posiadać własny zegar lub sygnał informujący o nadawaniu adresu z taktami zegara głównego, a dekoder powinien być wyposażony w rejestr przesuwny do odebrania i przechowywania aktualnego adresu z magistrali). Natomiast jeżeli magistrala byłaby oparta tylko na połączonych szeregowo rejestrach (wyjście serial-out do wejścia serial-in) to szyna adresowa nie jest potrzebna, ale konieczne może być każdorazowe wpisanie wszystkich wartości na szynę (czas zapisu rośnie z ilością podłączonych rejestrów).

\insertZadanie{booklets-sections/electronics/zadania-20-30-cyfrowa.tex}{hc574}{}
\insertZadanie{booklets-sections/electronics/zadania-20-30-cyfrowa.tex}{hc595}{}
