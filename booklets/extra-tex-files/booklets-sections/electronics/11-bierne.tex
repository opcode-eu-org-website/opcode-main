% Copyright (c) 2017-2020 Matematyka dla Ciekawych Świata (http://ciekawi.icm.edu.pl/)
% Copyright (c) 2017-2020 Robert Ryszard Paciorek <rrp@opcode.eu.org>
% 
% MIT License
% 
% Permission is hereby granted, free of charge, to any person obtaining a copy
% of this software and associated documentation files (the "Software"), to deal
% in the Software without restriction, including without limitation the rights
% to use, copy, modify, merge, publish, distribute, sublicense, and/or sell
% copies of the Software, and to permit persons to whom the Software is
% furnished to do so, subject to the following conditions:
% 
% The above copyright notice and this permission notice shall be included in all
% copies or substantial portions of the Software.
% 
% THE SOFTWARE IS PROVIDED "AS IS", WITHOUT WARRANTY OF ANY KIND, EXPRESS OR
% IMPLIED, INCLUDING BUT NOT LIMITED TO THE WARRANTIES OF MERCHANTABILITY,
% FITNESS FOR A PARTICULAR PURPOSE AND NONINFRINGEMENT. IN NO EVENT SHALL THE
% AUTHORS OR COPYRIGHT HOLDERS BE LIABLE FOR ANY CLAIM, DAMAGES OR OTHER
% LIABILITY, WHETHER IN AN ACTION OF CONTRACT, TORT OR OTHERWISE, ARISING FROM,
% OUT OF OR IN CONNECTION WITH THE SOFTWARE OR THE USE OR OTHER DEALINGS IN THE
% SOFTWARE.

% BEGIN: Elektronika - bierne
\section{Elementy bierne}

\begin{wrapfigure}{r}{0.35\textwidth}
  \begin{center}
    \vspace{-40pt}
    \includegraphics[width=0.3\textwidth]{img/elektronika/symbole}
    \vspace{-20pt}
  \end{center}
\end{wrapfigure}

\subsection{Rezystor}
\begin{teacherOnly}
	\begin{easylist}[itemize]
		& czy opór to jedyny parametr rezystora?
			&&& mamy też dokładność, stabilność temperaturową, napięciową, ...
			&&& ... ale przede wszystkim moc
		& prawo Ohma i rezystor
		& rezystancyjny dzielnik napięcia (\textbf{symulacja})
	\end{easylist}
\end{teacherOnly}

Rezystor (opornik) wprowadza do układu rezystancję związaną z swoją wartością nominalną. Typowo służy do ograniczania wartości prądu przez niego przepływającego.

Powoduje wydzielanie się energii (cieplnej) związanej z stratami na rezystancji - moc wydzielana dana jest zależnościami: $P = UI = \frac{U^2}{R} = I^2R$, czyli przy stałym napięciu przyłożonym do rezystora im większy jego opór tym mniejsza moc się wydzieli (gdyż popłynie mniejszy prąd), ale przy stałym prądzie płynącym przez rezystor moc rośnie wraz ze wzrostem oporu.

Rezystor jest elementem spełniającym prawo Ohma\footnote{Jest to zasadniczo jedyny element elektroniczny, który podlega temu prawu. Niektóre z elementów (jak kondensatory i cewki) podlegają rozszerzeniu prawa Ohma dla prądu przemiennego. Wiele innych elementów (jak np. diody i tranzystory) nie podlegają prawu Ohma.}.

\subsubsection{inne parametry rezystora}
Rzeczywisty rezystor oprócz samej wartości oporu elektrycznego charakteryzują też inne parametry, m.in. takie jak:
\begin{itemize}
\item maksymalna moc która może zostać wydzielona na danym elemencie,
\item dokładność, czyli to jak bardzo opór danego elementu może być odległy od wartości nominalnej,
\item stabilność oporu w funkcji w funkcji temperatury oraz w funkcji napięcia przyłożonego do elementu.
\end{itemize}

\subsubsection{rezystancyjny dzielnik napięcia}\label{dzielnik}

Jednym z najprostszych, użytecznych obwodów są dwa rezystory połączone szeregowo z źródłem napięcia. Układ taki nazywamy rezystancyjnym dzielnikiem napięcia. Pozwala on na uzyskanie napięcia niższego od napięcia źródła zgodnie z proporcją użytych rezystorów. Zobacz symulację: \url{http://ln.opcode.eu.org/dzielnik}.
	% https://www.falstad.com/circuit/circuitjs.html?cct=%24+1+0.000005+10.20027730826997+54+5+50%0Ag+336+320+336+336+0%0Aw+336+192+416+192+0%0Ar+416+256+416+320+0+1000%0Ag+416+320+416+336+0%0Ar+336+112+336+192+0+500%0As+416+192+416+256+0+1+false%0As+480+192+480+256+0+1+false%0Ag+480+320+480+336+0%0Ar+480+256+480+320+0+100%0Aw+416+192+480+192+0%0AR+336+112+336+80+0+0+40+12+0+0+0.5%0Aw+480+192+560+192+0%0Ap+560+192+560+320+1+0%0Ag+560+320+560+336+0%0Ar+336+192+336+320+0+500%0A38+10+0+2+24+Voltage%0A
Zwróć uwagę że napięcie wyjściowe z takiego układu jest bardzo zależne od pobieranego prądu / wielkości dołączonego obciążenia (w tym celu możesz użyć przełączników umieszczonych w symulowanym układzie), z tego powodu dzielnik rezystancyjny stosowany jest głównie w przypadkach gdy wiemy że obciążenie będzie pobierało niewielki prąd.

Rezystancyjny dzielnik napięcia jest bardzo często stosowany w celu proporcjonalnego podziału (obniżenia) napięcia wejściowego nieznanej (zmiennej) wielkości (np. celem jego pomiaru, przy użyciu miernika o ograniczonej skali),
a nie w celu uzyskania napięcia wyjściowego o konkretnej wartości (co można uzyskać w lepszy - bardziej stabilny sposób).

\subsection{Kondensator}
\begin{teacherOnly}
	\begin{easylist}[itemize]
		& "magazynuje napięcie"
		& zastosowania:
		&& stabilizacja napięcia
		&& opóźnienie zmiany jakiegoś sygnału (stała czasowa) (\textbf{symulacja})
		&& odcinanie składowej sygnały zmiennego (rozwarcie dla prądu stałego, ale przewodzi zmienny) (\textbf{symulacja})
	\end{easylist}
\end{teacherOnly}

Kondensator wprowadza do układu pojemność związaną z swoją wartością nominalną.
Pojemność wyraża zdolność do gromadzenia ładunku przez dany element - im większa pojemność tym więcej ładunku (przy takim samym przyłożonym napięciu) zgromadzi element. $C = \frac{q}{U}$

Kondensator typowo służy do ograniczania zmian napięcia (poprzez gromadzenie energii w polu elektrycznym) lub wprowadzenia opóźnienia (stałej czasowej) związanej z jego ładowaniem / rozładowywaniem.
Czas potrzebny do zmiany napięcia na kondensatorze dany jest zależnością: $\Delta T = \frac{C \cdot \Delta U}{I}$.

Zobacz symulację procesu ładowania / rozładowywania kondensatora: \url{http://ln.opcode.eu.org/cap}
	% https://www.falstad.com/circuit/circuitjs.html?cct=%24+1+0.000005+16.13108636308289+50+5+50%0Av+96+336+96+64+0+0+40+5+0+0+0.5%0AS+256+144+256+64+0+1+false+0+3%0Aw+96+64+240+64+0%0Aw+272+64+400+64+0%0Ac+256+144+256+256+0+0.00019999999999999998+0.001%0Ar+256+256+256+336+0+100%0Aw+96+336+256+336+0%0Aw+256+336+400+336+0%0Ar+400+64+400+336+0+1000%0Ao+4+128+0+4102+0.009765625+0.00009765625+0+2+4+3%0A38+4+0+0.000009999999999999999+0.00101+Capacitance%0A
(klikając na przełącznik w górnej części schematu  można wybierać pomiędzy rozładowywaniem a ładowaniem kondensatora, zwróć uwagę na różną wartość oporu użytego do tych operacji).

Innym częstym zastosowaniem jest kondensatora jest odcinanie składowej stałej – kondensator stanowi rozwarcie dla prądu stałego, ale przewodzi prąd zmienny (ze względu na prąd związany z jego ładowanie / rozładowywaniem).
Zobacz symulację: \url{http://ln.opcode.eu.org/cap_ac}
	% https://www.falstad.com/circuit/circuitjs.html?cct=%24+1+0.000005+16.13108636308289+50+5+50%0Aw+112+144+256+144+1%0Ac+256+144+256+256+0+0.00019999999999999998+-0.8719567578156338%0Ar+256+256+256+336+0+100%0Aw+112+336+256+336+0%0Av+112+336+112+144+0+1+20.000500000000002+5+0+0+0.5%0Ao+1+128+0+4102+5+0.00009765625+0+2+1+3%0A38+1+0+0.000009999999999999999+0.00101+Capacitance%0A38+4+3+0.001+40+Frequency%0A

Najistotniejszym parametrem rzeczywistych kondensatorów oprócz pojemności nominalnej jest maksymalne napięcie przy którym może pracować oprócz tego istotne mogą być parametry takie jak rezystancja wewnętrzna, maksymalna temperatura w której kondensator może pracować, żywotność tego elementu, itd.

\subsection{Cewka}
\begin{teacherOnly}
	\begin{easylist}[itemize]
		& mówiliśmy o magazynowaniu ... cewka "magazynuje prąd"
		& gdzie występuje (w jakich bardziej złożonych elementach) - transformator, przekaźnik
		& sterowanie przez tranzystor => dioda zabezpieczająca (\textbf{symulacja})
	\end{easylist}
\end{teacherOnly}

Cewka (dławik) wprowadza do układu indukcyjność związaną z swoją wartością nominalną. Samodzielnie występująca cewka typowo służy do ograniczania zmian prądu (poprzez gromadzenie energii w polu magnetycznym). Czas potrzebny zmiany prądu płynącego przez cewkę (dławik stawia opór takiej zmianie tak jak kondensator zmianie napięcia) dany jest zależnością: $\Delta T = \frac{L \cdot \Delta I}{U}$.

Głównym (ale nie jedynym) parametrem rzeczywistej cewki oprócz indukcyjności jest maksymalny prąd który może przewodzić.

\subsubsection{Przekaźniki, styczniki i transformatory}

Cewki możemy spotkać w urządzeniach takich jak przekaźniki, czy styczniki\footnote{Zasadniczo przekaźnik i stycznik jest to to samo urządzenie. Przyjmuje się rozróżnienie w nazewnictwie - przekaźniki przełączają mniejsze prądy niż styczniki.}.
Nawinięte na odpowiednim rdzeniu pełnią one tam funkcję elektromagnesu odpowiedzialnego za zmianę fizycznej pozycji styków prowadzącą do ich zwarcia lub rozwarcia (przełączania).

Innym urządzeniem opartym o cewki są transformatory - wykorzystują one kilka cewek na wspólnym rdzeniu do przekazywania energii poprzez pole magnetyczne (jedna z cewek dzięki przepływowi zmiennego prądu elektrycznego wytwarza zmienne pole magnetyczne, inna dzięki zmiennemu polu magnetycznemu wytwarza przemienny prąd elektryczny). Transformator typowo służy do zmiany napięcia lub separacji galwanicznej obwodów.

\subsubsection{Rozłączanie cewki}

Jako że cewka jest elementem który dąży do zachowania płynącego przez niego prądu, to w przypadku rozwarcia obwodu zawierającego cewkę napięcie na niej będzie rosło i bez problemów może wielokrotnie przekroczyć napięcie zasilania.
Zobacz symulację: \url{http://ln.opcode.eu.org/cewka} (rozłącz przełącznik i zaobserwuj co dzieje się z napięciem na cewce).
	% https://www.falstad.com/circuit/circuitjs.html?cct=%24+1+0.000005+12.050203812241895+46+5+50%0Av+96+336+96+144+0+0+40+5+0+0+0.5%0Ac+256+144+256+256+0+2e-8+2.185811780586783%0Ar+256+256+256+336+0+100%0Al+336+144+336+336+0+1+0.2814236307368567%0Aw+256+144+336+144+0%0Aw+336+336+256+336+0%0As+96+144+256+144+0+0+false%0Ar+96+336+256+336+0+10%0Ao+1+16+0+4354+640+0.00009765625+0+2+3+0%0A38+1+0+0.000009999999999999999+0.00101+Capacitance%0A
Zjawisko to bywa użyteczne i jest wykorzystywane w niektórych układach (np. przetwornicach podnoszących napięcie), ale często bywa też niepożądane, a nawet bardzo szkodliwe – może prowadzić do uszkadzania innych elementów w obwodzie (w szczególności elementu przełączającego).

Aby przeciwdziałać temu zjawisku można dołączyć równolegle do cewki odpowiednio mały opór, który pozwoli na rozładowanie się cewki.
Wadą takiego rozwiązania są straty związane z przewodzeniem przez ten rezystor w momencie gdy cewka jest zasilona.
Warto zauważyć że pojawiające się na cewce napięcie ma odwrotny znak (kierunek) niż spadek napięcia na tym elemencie w trakcie pracy.
Pozwala to na podłączenie równolegle z cewką elementu który przewodzi tylko w jednym kierunku\footnote{
	Nawet jeżeli element ten fizycznie jest obok elementu przełączającego powinien być podłączany równolegle do cewki a nie do elementu przełączającego.
}, w taki sposób aby w normalnym stanie nie przewodził, a po odłączeniu zasilania cewki pozwalał na jej rozładowanie.
Takim elementem jest dioda.
% END: Elektronika - bierne
