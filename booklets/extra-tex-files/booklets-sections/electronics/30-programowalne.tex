% Copyright (c) 2017-2020 Matematyka dla Ciekawych Świata (http://ciekawi.icm.edu.pl/)
% Copyright (c) 2017-2020 Robert Ryszard Paciorek <rrp@opcode.eu.org>
% 
% MIT License
% 
% Permission is hereby granted, free of charge, to any person obtaining a copy
% of this software and associated documentation files (the "Software"), to deal
% in the Software without restriction, including without limitation the rights
% to use, copy, modify, merge, publish, distribute, sublicense, and/or sell
% copies of the Software, and to permit persons to whom the Software is
% furnished to do so, subject to the following conditions:
% 
% The above copyright notice and this permission notice shall be included in all
% copies or substantial portions of the Software.
% 
% THE SOFTWARE IS PROVIDED "AS IS", WITHOUT WARRANTY OF ANY KIND, EXPRESS OR
% IMPLIED, INCLUDING BUT NOT LIMITED TO THE WARRANTIES OF MERCHANTABILITY,
% FITNESS FOR A PARTICULAR PURPOSE AND NONINFRINGEMENT. IN NO EVENT SHALL THE
% AUTHORS OR COPYRIGHT HOLDERS BE LIABLE FOR ANY CLAIM, DAMAGES OR OTHER
% LIABILITY, WHETHER IN AN ACTION OF CONTRACT, TORT OR OTHERWISE, ARISING FROM,
% OUT OF OR IN CONNECTION WITH THE SOFTWARE OR THE USE OR OTHER DEALINGS IN THE
% SOFTWARE.

% BEGIN: Elektronika - Układy programowalne
\section{Układy programowalne}
\begin{teacherOnly}
	\begin{easylist}[itemize]
	& układy programowalne
		&& programowalna logika
			&& komórka pamięci może realizować dowolną funkcję logiczną
		&& procesory i mikro-kontrolery
	\end{easylist}
\end{teacherOnly}

\subsection{układy o programowalnej strukturze (PLD)}

Są to układy w których programowany jest układ bramek, przerzutników, itp. "umieszczanych" w kości oraz ich połączeń.

Program dla takich układów tworzony jest w Hardware Description Language (najczęściej VHDL lub Verilog) i zamiast wykonywanego kodu opisuje strukturę układu logicznego (połączenia bramek, tablice prawdy, etc), która następnie jest programowana w fizycznej kości.

Najprostszym przykładem układu o programowalnej strukturze logicznej jest układ pamięci $2^n$ bitowej z n-bitową szyną adresową adresującą pojedyncze bity - pozwala on na realizację dowolnej funkcji logicznej o n wejściach i pojedynczym wyjściu.

Do kategorii tej zaliczają się układy typu:
\begin{itemize}
	\item SPLD
	\begin{itemize}
		\item PLE - programowalna matryca bramek OR
		\item PAL i GAL - programowalna matryca AND z dodatkowymi bramkami OR (często także obudowana rejestrami i multiplekserami na wyjściach)
		\item PLA - programowalne matryce AND i OR
	\end{itemize}
	\item CPLD
	\item FPGA - programowalny element pamięciowy (możliwość zdefiniowania dowolnej - na ogół 4 wejściowej - funkcji w każdym elemencie logicznym, programowalne połączenia między elementami logicznymi i pinami, itd)
\end{itemize}

\subsection{systemy procesorowe}
Są to systemy realizujące ciąg instrukcji pobieranych z jakiejś pamięci.

System taki składa się z procesora odpowiedzialnego za interpretację i wykonywanie kolejnych instrukcji oraz pamięci z której pobierane są instrukcje i dane (może to być jedna pamięć, mogą to być rozdzielone pamięci). Do kategorii tej zaliczają się zarówno typowe systemy komputerowe, systemy obliczeniowe jak i różnego rodzaju programowalne mikrokontrolery.

Procesor pracuje w cyklach rozkazowych, w ramach których przetwarza pojedynczą instrukcję. Cykl taki może trwać od 1 do kilku lub więcej cykli zegarowych i typowo składa się z następujących kroków:
\begin{enumerate}
	\item pobranie instrukcji z pamięci - realizowane jest poprzez wystawienie na szynę adresową zawartości \emph{licznika programu} (zawierające adres instrukcji do wykonania) oraz wygenerowanie cyklu odczytu z pamięci, po wykonaniu odczytu danych następuje ich zapamiętanie w \emph{rejestrze instrukcji} oraz zwiększenie wartości \emph{licznika programu} o jeden;\\
		(zawartość rejestru \emph{licznika programu} po resecie procesora określa skąd pobierana będzie pierwsza instrukcja, pod takim adresem zazwyczaj umieszczana jest jakaś pamięć typu ROM lub flash)
	\item dekodowanie instrukcji - układ dekodera (np. oparty o PLA) dokonuje zdekodowania instrukcji znajdującej się w \emph{rejestrze instrukcji} i konfiguracji procesora w zależności od jej kodu i (opcjonalnie) jej argumentów; może to być np.:
	\begin{itemize}
		\item odpowiednie ustawienie multiplekserów pomiędzy rejestrami a jednostką ALU oraz wystawienie odpowiedniego kod operacji dla ALU (celem wykonania operacji arytmetycznej na wartościach rejestrów)
		\item wystawienie zawartości wskazanego rejestru na szynę adresową, podłączenie wskazanego rejestru do szyny danych oraz skonfigurowanie operacji odczytu/zapisu (celem wykonania odczytu lub zapisu wartości rejestru z/do pamięci)
	\end{itemize}
	\item wykonanie instrukcji - realizacja wcześniej zdekodowanej instrukcji zgodnie z ustawioną konfiguracją procesora
\end{enumerate}

Instrukcje skoku polegają na załadowaniu nowej wartości do \emph{licznika programu}, w przypadku skoków warunkowych jest to uzależnione od stanu \emph{rejestru flag}, które ustawiane są w oparciu o wynik ostatniej operacji wykonywanej przez ALU.

Przedstawiony model działania jest przykładowym i w rzeczywistym procesorze może to wyglądać odmiennie - np. długość instrukcji może być większa niż długość słowa używanego przez procesor / szerokość szyny danych co rozbudowuje fazę pobierania instrukcji z pamięci, mogą występować instrukcje bardziej złożone (np. operacje wykonywane z argumentem pobieranym z pamięci a nie rejestru), może także występować więcej faz (np. poprzez wydzielenie faz dostępu do pamięci, czy zapisywania wyników działania instrukcji).
Procesor może także działać potokowo, czyli nakładać na siebie kolejne fazy wykonywania różnych instrukcji (np. w czasie wykonywania jednej instrukcji realizować pobieranie kolejnej).

\subsubsection{Mikrokontrolery}

Mikrokontroler jest układem typu "System on Chip" zawierającym w jednym układzie procesor, pamięć RAM, układy wejścia-wyjścia (np. GPIO, porty szeregowe typu USART, SPI, I2C, przetworniki ADC), pamięć typu Flash (dla programu).
% END: Elektronika - Układy programowalne
