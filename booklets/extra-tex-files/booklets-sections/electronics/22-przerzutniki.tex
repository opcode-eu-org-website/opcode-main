% BEGIN: Elektronika - Przerzutniki i rejestry
\section{Przerzutniki i rejestry}

\begin{teacherOnly}
	\begin{easylist}[itemize]
	& elektronika cyfrowa - rejestry i magistrale
		&& przerzutnik jako podstawowa jednostka pamięci
			&&& zatrzask RS => konstrukcję z 2 NAND (\textbf{symulacja})
			&&& chcemy zapamiętywać sygnał w określonym momencie a nie tylko mieć osobne sygnały ,,zapamiętaj 1'', ,,zapamiętaj '' => wejścia D i CLK
				&&&& zatrzask D (\textbf{symulacja})
				&&&& zatrzask vs przerzutnik
				&&&& przerzutnik z dwóch zatrzasków (\textbf{symulacja})
		&& rejestr równoległy, szeregowy, ... (\textbf{symulacja})
		&& sterowanie linią => bufory, trójstanowość, open-colektor
		&& topologie połączeń
		&& magistrale => transmisja równoległa i szeregowa, typowe sygnały sterujące magistralą
		&& przykłady interfejsów w elektronicy: SPI, I2C (zwrócić uwagę na open-drain), UART (zrócić uwagę na brak zegara), 1wire
	\end{easylist}
\end{teacherOnly}

\subsection{przerzutniki i ich budowa}

RS Flip-flop (RS Latch) jest podstawowym układem służącym do zapamiętania jednego bitu informacji. Posiada on dwa wejścia (set i reset) i dwa wyjścia (Q i NOT Q), wejścia mogą reagować na stan wysoki (oznaczane jako S i R) lub niski (oznaczane jako wejścia zanegowane ~S i ~R), jedno z wyjść może być jedynie wewnętrzne (nie wyprowadzone na zewnątrz układu). Podanie stanu wysokiego na wejście S (niskiego na ~S) powoduje wystawienie stanu wysokiego na wyjściu Q, a podanie stanu wysokiego na wejście R (niskiego na ~R) powoduje wystawienie stanu niskiego na wyjściu Q. Stan na wyjściu Q nie zmienia się po zmianie wejść S i R na stan niski (zostaje zapamiętany).

Zobacz i przeanalizuj symulację działania zatrzasku RS: \url{http://ln.opcode.eu.org/rs} z wejściami zanegowanymi.
	% https://www.falstad.com/circuit/circuitjs.html?cct=%24+1+0.000005+1.500424758475255+50+5+50%0A151+256+160+368+160+0+2+0+5%0A151+256+288+368+288+0+2+5+5%0Aw+368+160+368+192+0%0Aw+368+192+256+256+0%0Aw+368+288+368+256+0%0Aw+368+256+256+192+0%0Aw+256+192+256+176+0%0Aw+256+256+256+272+0%0AL+256+304+176+304+0+1+true+5+0%0AL+256+144+176+144+0+1+true+5+0%0AM+368+160+448+160+0+2.5%0AM+368+288+448+288+0+2.5%0Ax+159+120+231+123+4+24+set%0Ax+438+138+457+141+4+24+Q%0Ax+147+281+207+284+4+24+reset%0Ax+438+266+457+269+6+24+Q%0A

\subsection{zatrzask a przerzutnik}

Zatrzask jest elementem reagującym na poziom sygnału na wejściu "enable" (E). W przypadku nie zanegowanego wejścia E, jeżeli jest ono w stanie wysokim sygnał na wyjściach (Q i NOT Q) jest funkcją sygnałów wejściowych, natomiast stan niski wejścia E blokuje zmianę sygnału wyjściowego (zostaje on zapamiętany).

Przerzutnik jest elementem reagującym na zbocze sygnału na wejściu "clock" (CLK). W zależności od konstrukcji może reagować na zbocze narastające, opadające albo na oba (wtedy na jednym realizuje odczyt wejść a na drugim zmianę stanu wyjść).

\subsection{zatrzask i przerzutnik D}

\begin{wrapfigure}{r}{0.7\textwidth}
  \begin{center}
    \vspace{-25pt}
    \includegraphics[width=0.65\textwidth]{img/elektronika/przerzutnikD}
    \vspace{-25pt}
  \end{center}
\end{wrapfigure}

Posiada jedno wejścia sygnałowe "data" (D) oraz wejście "enable" (E) w przypadku zatrzasku lub wejście "clock" (CLK) w przypadku przerzutnika. Może także posiadać asynchroniczne (niezależne od stanu wejścia E / CLK) wejścia reset i set (zanegowane lub proste). Obniżenie sygnału E lub zbocze sygnału CLK powodują zapamiętanie (i wystawienie na wyjściu Q) stanu wejścia D.

Zobacz symulację zatrzasku typu D zbudowanego z bramek NAND: \url{http://ln.opcode.eu.org/zatrzask} (możesz zmieniać stan wejścia D klikając na nie, zegar zmienia się automatycznie).
	% https://www.falstad.com/circuit/circuitjs.html?cct=%24+1+0.000005+10.20027730826997+50+5+50%0A151+192+160+288+160+0+2+5+5%0A151+192+272+288+272+0+2+0+5%0Aw+192+176+192+192+0%0Aw+192+256+192+240+0%0Aw+288+240+288+272+0%0Aw+288+240+192+192+0%0Aw+288+192+288+160+0%0Aw+288+192+192+240+0%0A151+80+144+192+144+0+2+0+5%0A151+80+288+192+288+0+2+5+5%0AM+288+160+352+160+0+2.5%0AM+288+272+352+272+0+2.5%0Aw+80+160+80+304+0%0AI+48+128+48+272+0+0.5+5%0Aw+48+128+80+128+0%0Aw+48+272+80+272+0%0AL+48+128+48+96+0+1+false+5+0%0Ax+39+62+57+65+4+24+D%0AR+80+304+48+304+1+2+10+2.5+2.5+0+0.5%0Ao+16+64+0+4102+5+0.00009765625+0+2+16+3+D%0Ao+10+64+0+4102+5+0.00009765625+0+1+Q%0Ao+18+64+0+4102+5+0.00009765625+0+2+18+3+clk%0A
Zwróć uwagę iż przy wysokim stanie sygnału zegara (enable) stan wyjścia odpowiada stanowi wejścia (zatrzask jest przeźroczysty),
	natomiast przy niskim stanie zegara wyjście nie reaguje na zmiany stanu wejścia i znajduje się w takim stanie w jakim było wejście w chwili opadającego zbocza sygnału zegarowego.

Zobacz symulację przerzutnika D złożonego z dwóch zatrzasków: \url{http://ln.opcode.eu.org/przerzutnik}.
	% https://www.falstad.com/circuit/circuitjs.html?cct=%24+1+0.000005+10.20027730826997+50+5+50%0A151+448+160+544+160+0+2+5+5%0A151+448+272+544+272+0+2+0+5%0Aw+448+240+448+256+0%0Aw+448+176+448+192+0%0Aw+544+240+544+272+0%0Aw+544+240+448+192+0%0Aw+544+160+544+192+0%0Aw+544+192+448+240+0%0A151+336+144+448+144+0+2+0+5%0A151+336+288+448+288+0+2+5+5%0A151+192+160+288+160+0+2+5+5%0A151+192+272+288+272+0+2+0+5%0Aw+336+128+288+128+0%0Aw+288+128+288+160+0%0Aw+288+272+336+272+0%0Aw+192+176+192+192+0%0Aw+192+256+192+240+0%0Aw+288+240+288+272+0%0Aw+288+240+192+192+0%0Aw+288+192+288+160+0%0Aw+288+192+192+240+0%0A151+80+144+192+144+0+2+5+5%0A151+80+288+192+288+0+2+5+5%0AI+80+352+320+352+0+0.5+5%0AM+544+160+608+160+0+2.5%0AM+544+272+608+272+0+2.5%0Ax+536+130+555+133+4+24+Q%0Ax+536+321+555+324+6+24+Q%0Aw+80+160+80+304+0%0Aw+80+304+80+352+0%0AI+48+128+48+272+0+0.5+5%0Aw+48+128+80+128+0%0Aw+48+272+80+272+0%0AL+48+128+48+96+0+1+false+5+0%0Ax+39+62+57+65+4+24+D%0AR+80+352+48+352+1+2+20+2.5+2.5+0+0.5%0Ax+291+113+307+116+4+24+X%0Aw+320+352+320+304+0%0Aw+320+304+336+304+0%0Aw+320+304+320+160+0%0Aw+320+160+336+160+0%0Ao+33+64+0+4102+5+0.00009765625+0+2+33+3+D%0Ao+12+64+0+4102+5+0.00009765625+0+2+12+3+X%0Ao+24+64+0+4102+5+0.00009765625+0+1+Q%0Ao+35+64+0+4102+5+0.00009765625+0+2+35+3+clk%0A
Zauważ że w żadnej fazie sygnału zegarowego nie jest on przeźroczysty (wyjście Q nie zależy od obecnego stanu wejścia D).
Zwróć uwagę że sygnał wejściowy zostanie zapamiętany i wystawiony na wyjście Q w momencie opadającego zbocza sygnału zegarowego.

\subsection{rejestry}
Mianem rejestru n-bitowego określa się zespół n przerzutników (rzadziej zatrzasków), często z uwspólnionym sterowaniem (sygnały clock, set, reset, etc) służący do zapamiętania n-bitowej wartości (liczby). W zależności od sposobu zapisu i odczytu można wyróżnić:

\subsubsection{rejestry równoległe}
Posiadają taką samą liczbę wejść jak i wyjść, sygnał na i-tym wyjściu jest bezpośrednio powiązany z sygnałem z i-tego wejścia (jest sygnałem zapamiętanym z tego wejścia).

Zobacz symulację rejestru równoległego zbudowanego z przerzutników typu D: \url{http://ln.opcode.eu.org/rejestr1} (stan wszystkich wejść zostanie zapamiętany i przepisany na wyjścia w chwili narastającego zbocza zegara).
	% https://www.falstad.com/circuit/circuitjs.html?cct=%24+1+0.000005+10.20027730826997+50+5+50%0A155+64+-16+112+-16+0+0%0A155+64+96+96+96+0+0%0A155+64+320+96+320+0+0%0A155+64+208+112+208+0+0%0AL+64+-16+-80+-16+0+0+false+5+0%0AL+64+96+-80+96+0+0+false+5+0%0AL+64+208+-80+208+0+0+false+5+0%0AL+64+320+-80+320+0+0+false+5+0%0Aw+64+16+32+16+0%0Aw+32+16+32+128+0%0Aw+32+128+64+128+0%0Aw+32+128+32+240+0%0Aw+32+240+64+240+0%0Aw+32+240+32+352+0%0Aw+32+352+64+352+0%0Aw+32+352+32+400+0%0AL+32+400+-80+400+0+0+false+5+0%0AM+160+-16+240+-16+0+2.5%0AM+160+96+240+96+0+2.5%0AM+160+208+240+208+0+2.5%0AM+160+320+240+320+0+2.5%0A;

\subsubsection{rejestry szeregowe serial-input}
Z kolejnymi sygnałami zegarowymi odczytywany jest stan wejścia szeregowego, a stan poprzedni przenoszony jest do kolejnego przerzutnika w ramach rejestru. W ten sposób po n cyklach zegara n-bitowy rejestr ma zapisaną nową zawartość. Często posiada zespolony z nim rejestr równoległy zapobiegający zmianie stanu wyjść w trakcie ładowania danych z wejścia szeregowego przepisanie danych z rejestru przesuwnego do rejestru odpowiedzialnego za sterowanie wyjściami sterowane jest osobnym sygnałem zegarowym.

Zobacz symulację prostego rejestru z wejściem szeregowym (bez zatrzasku/rejestru wyjściowego): \url{http://ln.opcode.eu.org/rejestr2}.
	% https://www.falstad.com/circuit/circuitjs.html?cct=%24+1+0.000005+10.20027730826997+50+5+50%0A155+-224+-16+-176+-16+0+0%0A155+-96+96+-64+96+0+0%0A155+160+320+192+320+0+0%0A155+32+208+80+208+0+0%0AL+-224+-16+-368+-16+0+0+false+5+0%0Aw+-224+16+-256+16+0%0Aw+-256+16+-256+128+0%0Aw+-256+128+-96+128+0%0Aw+-256+128+-256+240+0%0Aw+-256+240+32+240+0%0Aw+-256+240+-256+352+0%0Aw+-256+352+160+352+0%0Aw+-256+352+-256+400+0%0AL+-256+400+-368+400+0+0+false+5+0%0AM+-96+-16+-48+-16+0+2.5%0AM+32+96+80+96+0+2.5%0AM+160+208+208+208+0+2.5%0AM+256+320+304+320+0+2.5%0Aw+-128+-16+-112+-16+0%0Aw+-112+-16+-112+96+0%0Aw+-112+96+-96+96+0%0Aw+-112+-16+-96+-16+0%0Aw+16+96+32+96+0%0Aw+16+208+32+208+0%0Aw+16+96+16+208+0%0Aw+0+96+16+96+0%0Aw+144+208+160+208+0%0Aw+144+320+160+320+0%0Aw+144+208+144+320+0%0Aw+128+208+144+208+0%0A;
Zauważ że stan wyjść zmienia się na bieżąco w trakcie szeregowego wpisywania wartości do rejestru.

Zobacz symulację rejestru z wejściem szeregowym i rejestrem wyjściowym: \url{http://ln.opcode.eu.org/rejestr3}.
	% https://www.falstad.com/circuit/circuitjs.html?cct=%24+1+0.000005+10.20027730826997+50+5+50%0A155+-448+-48+-400+-48+0+0%0A155+-320+64+-288+64+0+0%0A155+-64+288+-32+288+0+0%0A155+-192+176+-144+176+0+0%0AL+-448+-48+-592+-48+0+0+false+5+0%0Aw+-448+-16+-480+-16+0%0Aw+-480+-16+-480+96+0%0Aw+-480+96+-320+96+0%0Aw+-480+96+-480+208+0%0Aw+-480+208+-192+208+0%0Aw+-480+208+-480+320+0%0Aw+-480+320+-64+320+0%0Aw+-480+320+-480+368+0%0AL+-480+368+-592+368+0+0+false+5+0%0AM+192+-48+240+-48+0+2.5%0Aw+-352+-48+-336+-48+0%0Aw+-336+-48+-336+64+0%0Aw+-336+64+-320+64+0%0Aw+-208+176+-192+176+0%0Aw+-208+64+-208+176+0%0Aw+-224+64+-208+64+0%0Aw+-80+288+-64+288+0%0Aw+-80+176+-80+288+0%0Aw+-96+176+-80+176+0%0A155+96+-48+144+-48+0+0%0A155+96+64+144+64+0+0%0AM+192+64+240+64+0+2.5%0A155+96+176+144+176+0+0%0AM+192+176+240+176+0+2.5%0A155+96+288+144+288+0+0%0AM+192+288+240+288+0+2.5%0Aw+32+288+96+288+0%0Aw+-64+176+96+176+0%0Aw+96+64+-208+64+0%0Aw+96+-48+-336+-48+0%0Aw+-80+176+-64+176+0%0Aw+96+-16+80+-16+0%0Aw+80+-16+80+96+0%0Aw+80+96+96+96+0%0Aw+80+96+80+208+0%0Aw+80+208+96+208+0%0Aw+80+208+80+320+0%0Aw+80+320+96+320+0%0Aw+80+320+80+400+0%0Aw+80+400+-480+400+0%0AL+-480+400+-592+400+0+0+false+5+0%0A;
Zauważ, że stan wyjść zmienia się na skutek osobnego sygnału, który może zostać wygenerowany po zakończeniu szeregowego zapisu do rejestru.

\subsubsection{rejestry szeregowe paraller-input serial-output}
Z kolejnymi sygnałami zegarowymi na wyjście szeregowe wystawiany jest stan kolejnego z rejestrów wejściowych. Wariant asynchroniczny posiada osobny sygnał powodujący odczyt wejść do rejestru (sygnał działa jak "enable" w zatrzaskach). Wariant synchroniczny posiada sygnał decydujący o tym czy na zboczu zegara dokonywany jest odczyt wejść czy też przesuwanie zawartości rejestru umożliwiający odczyt z wyjścia szeregowego.

\subsubsection{liczniki}
Z kolejnymi sygnałami zegarowymi zwiększana jest o jeden wartość rejestru. Prostszy w budowie licznik asynchroniczny ma większe (i w dodatku rosnące wraz z bitowością licznika) ograniczenia dotyczące szybkości zliczania od licznika synchronicznego, ze względu na opóźnienie z jakim dochodzi zliczany sygnał (CLK) do kolejnych stopni licznika.
% END: Elektronika - Przerzutniki i rejestry
