% Copyright (c) 2017-2020 Matematyka dla Ciekawych Świata (http://ciekawi.icm.edu.pl/)
% Copyright (c) 2017-2020 Robert Ryszard Paciorek <rrp@opcode.eu.org>
% 
% MIT License
% 
% Permission is hereby granted, free of charge, to any person obtaining a copy
% of this software and associated documentation files (the "Software"), to deal
% in the Software without restriction, including without limitation the rights
% to use, copy, modify, merge, publish, distribute, sublicense, and/or sell
% copies of the Software, and to permit persons to whom the Software is
% furnished to do so, subject to the following conditions:
% 
% The above copyright notice and this permission notice shall be included in all
% copies or substantial portions of the Software.
% 
% THE SOFTWARE IS PROVIDED "AS IS", WITHOUT WARRANTY OF ANY KIND, EXPRESS OR
% IMPLIED, INCLUDING BUT NOT LIMITED TO THE WARRANTIES OF MERCHANTABILITY,
% FITNESS FOR A PARTICULAR PURPOSE AND NONINFRINGEMENT. IN NO EVENT SHALL THE
% AUTHORS OR COPYRIGHT HOLDERS BE LIABLE FOR ANY CLAIM, DAMAGES OR OTHER
% LIABILITY, WHETHER IN AN ACTION OF CONTRACT, TORT OR OTHERWISE, ARISING FROM,
% OUT OF OR IN CONNECTION WITH THE SOFTWARE OR THE USE OR OTHER DEALINGS IN THE
% SOFTWARE.

% BEGIN: Elektronika - Przerzutniki i rejestry
\section{Przerzutniki i rejestry}

\subsection{przerzutniki i ich budowa}

RS Flip-flop (RS Latch) jest podstawowym układem służącym do zapamiętania jednego bitu informacji. Posiada on dwa wejścia (set i reset) i dwa wyjścia (Q i NOT Q), wejścia mogą reagować na stan wysoki (oznaczane jako S i R) lub niski (oznaczane jako wejścia zanegowane ~S i ~R), jedno z wyjść może być jedynie wewnętrzne (nie wyprowadzone na zewnątrz układu). Podanie stanu wysokiego na wejście S (niskiego na ~S) powoduje wystawienie stanu wysokiego na wyjściu Q, a podanie stanu wysokiego na wejście R (niskiego na ~R) powoduje wystawienie stanu niskiego na wyjściu Q. Stan na wyjściu Q nie zmienia się po zmianie wejść S i R na stan niski (zostaje zapamiętany).

Zobacz i przeanalizuj symulację działania zatrzasku RS: \url{http://ln.opcode.eu.org/rs} z wejściami zanegowanymi.

\subsection{zatrzask a przerzutnik}

Zatrzask jest elementem reagującym na poziom sygnału na wejściu "enable" (E). W przypadku nie zanegowanego wejścia E, jeżeli jest ono w stanie wysokim sygnał na wyjściach (Q i NOT Q) jest funkcją sygnałów wejściowych, natomiast stan niski wejścia E blokuje zmianę sygnału wyjściowego (zostaje on zapamiętany).

Przerzutnik jest elementem reagującym na zbocze sygnału na wejściu "clock" (CLK). W zależności od konstrukcji może reagować na zbocze narastające, opadające albo na oba (wtedy na jednym realizuje odczyt wejść a na drugim zmianę stanu wyjść).

\subsection{zatrzask i przerzutnik D}

\begin{wrapfigure}{r}{0.7\textwidth}
  \begin{center}
    \vspace{-25pt}
    \includegraphics[width=0.65\textwidth]{img/elektronika/przerzutnikD}
    \vspace{-25pt}
  \end{center}
\end{wrapfigure}

Posiada jedno wejścia sygnałowe "data" (D) oraz wejście "enable" (E) w przypadku zatrzasku lub wejście "clock" (CLK) w przypadku przerzutnika. Może także posiadać asynchroniczne (niezależne od stanu wejścia E / CLK) wejścia reset i set (zanegowane lub proste). Obniżenie sygnału E lub zbocze sygnału CLK powodują zapamiętanie (i wystawienie na wyjściu Q) stanu wejścia D.

Zobacz symulację zatrzasku typu D zbudowanego z bramek NAND: \url{http://ln.opcode.eu.org/zatrzask} (możesz zmieniać stan wejścia D klikając na nie, zegar zmienia się automatycznie).
Zwróć uwagę iż przy wysokim stanie sygnału zegara (enable) stan wyjścia odpowiada stanowi wejścia (zatrzask jest przeźroczysty),
	natomiast przy niskim stanie zegara wyjście nie reaguje na zmiany stanu wejścia i znajduje się w takim stanie w jakim było wejście w chwili opadającego zbocza sygnału zegarowego.

Zobacz symulację przerzutnika D złożonego z dwóch zatrzasków: \url{http://ln.opcode.eu.org/przerzutnik}.
Zauważ że w żadnej fazie sygnału zegarowego nie jest on przeźroczysty (wyjście Q nie zależy od obecnego stanu wejścia D).
Zwróć uwagę że sygnał wejściowy zostanie zapamiętany i wystawiony na wyjście Q w momencie opadającego zbocza sygnału zegarowego.

\subsection{rejestry}
Mianem rejestru n-bitowego określa się zespół n przerzutników (rzadziej zatrzasków), często z uwspólnionym sterowaniem (sygnały clock, set, reset, etc) służący do zapamiętania n-bitowej wartości (liczby). W zależności od sposobu zapisu i odczytu można wyróżnić:

\subsubsection{rejestry równoległe}
Posiadają taką samą liczbę wejść jak i wyjść, sygnał na i-tym wyjściu jest bezpośrednio powiązany z sygnałem z i-tego wejścia (jest sygnałem zapamiętanym z tego wejścia).

Zobacz symulację rejestru równoległego zbudowanego z przerzutników typu D: \url{http://ln.opcode.eu.org/rejestr1} (stan wszystkich wejść zostanie zapamiętany i przepisany na wyjścia w chwili narastającego zbocza zegara).

\subsubsection{rejestry szeregowe serial-input}
Z kolejnymi sygnałami zegarowymi odczytywany jest stan wejścia szeregowego, a stan poprzedni przenoszony jest do kolejnego przerzutnika w ramach rejestru. W ten sposób po n cyklach zegara n-bitowy rejestr ma zapisaną nową zawartość. Często posiada zespolony z nim rejestr równoległy zapobiegający zmianie stanu wyjść w trakcie ładowania danych z wejścia szeregowego przepisanie danych z rejestru przesuwnego do rejestru odpowiedzialnego za sterowanie wyjściami sterowane jest osobnym sygnałem zegarowym.

Zobacz symulację prostego rejestru z wejściem szeregowym (bez zatrzasku/rejestru wyjściowego): \url{http://ln.opcode.eu.org/rejestr2}.
Zauważ że stan wyjść zmienia się na bieżąco w trakcie szeregowego wpisywania wartości do rejestru.

Zobacz symulację rejestru z wejściem szeregowym i rejestrem wyjściowym: \url{http://ln.opcode.eu.org/rejestr3}.
Zauważ, że stan wyjść zmienia się na skutek osobnego sygnału, który może zostać wygenerowany po zakończeniu szeregowego zapisu do rejestru.

\subsubsection{rejestry szeregowe paraller-input serial-output}
Z kolejnymi sygnałami zegarowymi na wyjście szeregowe wystawiany jest stan kolejnego z rejestrów wejściowych. Wariant asynchroniczny posiada osobny sygnał powodujący odczyt wejść do rejestru (sygnał działa jak "enable" w zatrzaskach). Wariant synchroniczny posiada sygnał decydujący o tym czy na zboczu zegara dokonywany jest odczyt wejść czy też przesuwanie zawartości rejestru umożliwiający odczyt z wyjścia szeregowego.

\subsubsection{liczniki}
Z kolejnymi sygnałami zegarowymi zwiększana jest o jeden wartość rejestru. Prostszy w budowie licznik asynchroniczny ma większe (i w dodatku rosnące wraz z bitowością licznika) ograniczenia dotyczące szybkości zliczania od licznika synchronicznego, ze względu na opóźnienie z jakim dochodzi zliczany sygnał (CLK) do kolejnych stopni licznika.
% END: Elektronika - Przerzutniki i rejestry

\setcounter{subsection}{0}
\insertZadanie{booklets-sections/electronics/zadania-20-30-cyfrowa.tex}{dane_rejestru}{}
