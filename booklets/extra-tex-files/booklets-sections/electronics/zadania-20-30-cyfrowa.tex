% Copyright (c) 2017-2020 Matematyka dla Ciekawych Świata (http://ciekawi.icm.edu.pl/)
% Copyright (c) 2017-2020 Robert Ryszard Paciorek <rrp@opcode.eu.org>
% 
% MIT License
% 
% Permission is hereby granted, free of charge, to any person obtaining a copy
% of this software and associated documentation files (the "Software"), to deal
% in the Software without restriction, including without limitation the rights
% to use, copy, modify, merge, publish, distribute, sublicense, and/or sell
% copies of the Software, and to permit persons to whom the Software is
% furnished to do so, subject to the following conditions:
% 
% The above copyright notice and this permission notice shall be included in all
% copies or substantial portions of the Software.
% 
% THE SOFTWARE IS PROVIDED "AS IS", WITHOUT WARRANTY OF ANY KIND, EXPRESS OR
% IMPLIED, INCLUDING BUT NOT LIMITED TO THE WARRANTIES OF MERCHANTABILITY,
% FITNESS FOR A PARTICULAR PURPOSE AND NONINFRINGEMENT. IN NO EVENT SHALL THE
% AUTHORS OR COPYRIGHT HOLDERS BE LIABLE FOR ANY CLAIM, DAMAGES OR OTHER
% LIABILITY, WHETHER IN AN ACTION OF CONTRACT, TORT OR OTHERWISE, ARISING FROM,
% OUT OF OR IN CONNECTION WITH THE SOFTWARE OR THE USE OR OTHER DEALINGS IN THE
% SOFTWARE.

\IfStrEq{\dbEntryID}{}{
	\insertZadanie{\currfilepath}{napiecie_T5}{}
	\insertZadanie{\currfilepath}{dane_rejestru}{}
	\insertZadanie{\currfilepath}{hc574}{}
	\insertZadanie{\currfilepath}{hc595}{}
}

\IfStrEq{\dbEntryID}{praktyczne}{
	\insertZadanie{\currfilepath}{zbuduj_przerzutnik_z_bramek}{}
	\insertZadanie{\currfilepath}{zbuduj_rejestr_przesuwny}{}
	\insertZadanie{\currfilepath}{zbuduj_bramka_z_tranzystorow}{}
}

\IfStrEq{\dbEntryID}{rozwiazania}{
% 	\insertRozwiazanie{\currfilepath}{}{}
}

%
% zadania teoretyczne
%

\dbEntryBegin{napiecie_T5}\if1\dbEntryCheckResults
Podaj wartość napięcia (względem GND) w punkcie T5. Przyjmujemy iż użyte bramki działają na poziomie napięć 5V (prawda) / 0V (fałsz). Odpowiedź krótko uzasadnij.
	\begin{center}
		\includegraphics[width=0.48\textwidth]{img/elektronika/zadania_teoretyczne-C}
	\end{center}
\fi

\dbEntryBegin{dane_rejestru}\if1\dbEntryCheckResults
  Znajdź dokumentację (kartę katalogową) do rejestru przesuwnego, który kupiłeś/aś (CD4094 lub 74HC595).
  Odczytaj, jakie jest największe napięcie zasilania tego układu (\textit{supply voltage}, w sekcji
  \textit{Absolute maximum ratings}), oraz do których pinów należy podłączać napięcie zasilania (w sekcji \textit{pin configuration}).
\fi

\dbEntryBegin{hc574}\if1\dbEntryCheckResults
Zapoznaj się z dokumentacją układu 74HC574 i opisz sposób jego użycia (wraz z sposobem sterowania) jako modułu podłączonego do 8 bitowej magistrali równoległej w roli układu wejściowego oraz w roli układu wyjściowego.
\fi

\dbEntryBegin{hc595}\if1\dbEntryCheckResults
Zapoznaj się z dokumentacją układu 74HC595 i opisz sposób jego użycia (wraz z sposobem sterowania) w roli układu wyjściowego podłączonego do magistrali szeregowej.
\fi


%
% zadania praktyczne
%

\dbEntryBegin{zbuduj_przerzutnik_z_bramek}\if1\dbEntryCheckResults
\noindent\begin{minipage}[b]{0.6\textwidth}
Na schemacie przedstawiono dwubramkową budowę przerzutnika RS w wariancie z wejściami nie zanegowanymi (zastosowanie bramek NAND w miejsce NOR spowodujwe zanegowanie wejść). Zbuduj taki układ i sprawdź jego działanie.
\end{minipage}
\hfill
\begin{minipage}[b]{0.35\textwidth}
\includegraphics[width=\textwidth]{img/elektronika/RS}
\end{minipage}
\fi

\dbEntryBegin{zbuduj_rejestr_przesuwny}\if1\dbEntryCheckResults
Podłącz do kolejnych wyjść układu rejestru przesuwnego z buforem wyjściowym (np. CD4094 lub 74HC595) 4 diody LED (pamiętaj o rezystorach).
Zapisz do rejestru i ustaw na wyjściach taką wartość aby świeciły się dwie pierwsze i ostatnia dioda, użyj w tym celu ręcznego manipulowania sygnałami:
\begin{itemize}
\item wejścia szeregowego (SERIAL IN), służącego do wprowadzania danych
\item zegara danych (CLOCK, CLK), determinującego chwilę odczytu kolejnego bitu z wejścia szeregowego
\item zegara wyjść (STROBE), determinującego chwilę przepisania danych z rejestru przesuwnego do rejestru wyjściowego
\end{itemize}

\textit{Wskazówka: zapoznaj się z dokumentacją posiadanego układu, ustal nazewnictwo używane do określania poszczególnych sygnałów (może się różnić nawet w zależności od producenta układu) oraz numery nóżek układu z nimi związane (mogą się różnić w zależności od modelu / wariantu obudowy).}
\fi

\dbEntryBegin{zbuduj_bramka_z_tranzystorow}\if1\dbEntryCheckResults
Spróbuj zbudować własną bramkę logiczną w oparciu o tranzystory NPN i PNP. Pamiętaj że w odróżnieniu od pokazanych na powyższym schemacie tranzystorów NMOS i PMOS wymagane jest stosowanie rezystora na bramce.

\textit{Wskazówka: zacznij od zbudowania bramki NOT, gdyż ona jest najprostsza – to po prostu półmostek H.}
\fi
