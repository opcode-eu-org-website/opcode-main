% Copyright (c) 2021 Matematyka dla Ciekawych Świata (http://ciekawi.icm.edu.pl/)
% Copyright (c) 2021 Robert Ryszard Paciorek <rrp@opcode.eu.org>
% 
% MIT License
% 
% Permission is hereby granted, free of charge, to any person obtaining a copy
% of this software and associated documentation files (the "Software"), to deal
% in the Software without restriction, including without limitation the rights
% to use, copy, modify, merge, publish, distribute, sublicense, and/or sell
% copies of the Software, and to permit persons to whom the Software is
% furnished to do so, subject to the following conditions:
% 
% The above copyright notice and this permission notice shall be included in all
% copies or substantial portions of the Software.
% 
% THE SOFTWARE IS PROVIDED "AS IS", WITHOUT WARRANTY OF ANY KIND, EXPRESS OR
% IMPLIED, INCLUDING BUT NOT LIMITED TO THE WARRANTIES OF MERCHANTABILITY,
% FITNESS FOR A PARTICULAR PURPOSE AND NONINFRINGEMENT. IN NO EVENT SHALL THE
% AUTHORS OR COPYRIGHT HOLDERS BE LIABLE FOR ANY CLAIM, DAMAGES OR OTHER
% LIABILITY, WHETHER IN AN ACTION OF CONTRACT, TORT OR OTHERWISE, ARISING FROM,
% OUT OF OR IN CONNECTION WITH THE SOFTWARE OR THE USE OR OTHER DEALINGS IN THE
% SOFTWARE.

\begin{itemize}
\item \emph{Vademecum informatyki praktycznej} (\url{http://vip.opcode.eu.org/}) - zbiór materiałów na temat elektroniki i programowania, zawierający także dość rozbudowaną \href{http://vip.opcode.eu.org/#Literatura}{listę literatury dodatkowej}.
\item \emph{Linux i Python w Elektronicznej Sieci} (\url{https://ciekawi.icm.edu.pl/lpes}) - strona domowa kursu \textit{LPES}, zawierająca nagrania i skrypty do innych wykładów, skrypty ćwiczeniowe, itd.
\item \emph{OpCode.eu.org} (\url{http://vip.opcode.eu.org/}) - strona internetowa autora kursu \textit{LPES}, zawierająca różne materiały z szeroko rozumianej inżynierii komputerowej i elektronicznej (część materiałów pokrywa się z zawartością skryptów z tego kursu, ale nie wszystkie)
\end{itemize}
