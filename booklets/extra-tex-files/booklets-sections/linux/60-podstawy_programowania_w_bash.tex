% Copyright (c) 2017-2020 Matematyka dla Ciekawych Świata (http://ciekawi.icm.edu.pl/)
% Copyright (c) 2017-2020 Robert Ryszard Paciorek <rrp@opcode.eu.org>
% 
% MIT License
% 
% Permission is hereby granted, free of charge, to any person obtaining a copy
% of this software and associated documentation files (the "Software"), to deal
% in the Software without restriction, including without limitation the rights
% to use, copy, modify, merge, publish, distribute, sublicense, and/or sell
% copies of the Software, and to permit persons to whom the Software is
% furnished to do so, subject to the following conditions:
% 
% The above copyright notice and this permission notice shall be included in all
% copies or substantial portions of the Software.
% 
% THE SOFTWARE IS PROVIDED "AS IS", WITHOUT WARRANTY OF ANY KIND, EXPRESS OR
% IMPLIED, INCLUDING BUT NOT LIMITED TO THE WARRANTIES OF MERCHANTABILITY,
% FITNESS FOR A PARTICULAR PURPOSE AND NONINFRINGEMENT. IN NO EVENT SHALL THE
% AUTHORS OR COPYRIGHT HOLDERS BE LIABLE FOR ANY CLAIM, DAMAGES OR OTHER
% LIABILITY, WHETHER IN AN ACTION OF CONTRACT, TORT OR OTHERWISE, ARISING FROM,
% OUT OF OR IN CONNECTION WITH THE SOFTWARE OR THE USE OR OTHER DEALINGS IN THE
% SOFTWARE.

% BEGIN: podstawy programowania w bashu

\subsection{Zmienne}

Określanie typów zmiennych w bashu odbywa się na podstawie wartości znajdującej się w zmiennej.
Zasadniczo wszystkie zmienne są napisami, a interpretacja typu ma miejsce przy ich użyciu (a nie przy tworzeniu).
\teacher{Porównując do Pythona - jest to wyraźnie mniej silne typowanie. }
Obsługiwane są liczby całkowite oraz napisy, bash nie posiada wbudowanej obsługi liczb zmiennoprzecinkowych.

\begin{CodeFrame*}[bash]{}
zmiennaA=-91
zmiennaB="qa   z"
zmiennaC=98.6 # to będzie traktowane jako napis a nie liczba
\end{CodeFrame*}
Zwróć uwagę na brak spacji pomiędzy nazwą zmiennej a znakiem równości w operacji przypisania - jest to wymóg składniowy.

Odwołanie do zmiennej odbywa się z użyciem znaku dolara, po którym występuje nazwa zmiennej. Nazwa może być ujęta w klamry, ale nie musi (jest to przydatne gdy nie chcemy dawać spacji pomiędzy nazwą zmiennej a np. fragmentem napisu). Rozwijaniu ulegają nazwy zmiennych znajdujące się w napisach umieszczonych w podwójnych cudzysłowach. Umieszczenie odwołania do zmiennej w podwójnych cudzysłowach zabezpiecza białe znaki (spacje nowe linie) przy przekazywaniu do funkcji i programów (w tym przy przekazywaniu do echo, celem wypisywania).

\begin{CodeFrame*}[bash]{}
echo  $zmiennaA ${zmiennaA}AA
echo "$zmiennaA ${zmiennaA}AA"
echo '$zmiennaA ${zmiennaA}AA'
\end{CodeFrame*}

Jeżeli chcemy aby zmienna była widoczna przez programy uruchamiane z naszej powłoki (w tym przez kolejne instancje bash'a, odpowiedzialne np. za wykonywanie kodu skryptu uruchamianego z pliku) należy ją wyeksportować za pomocą polecenia \Verb#export zmienna# (zwróć uwagę na brak dolara w tym miejscu).

\subsection{Podstawowe operacje}
Aby wykonać działania arytmetyczne należy umieścić je wewnątrz \Verb{$((} i \Verb{))}

Dodawanie, mnożenie, odejmowanie zapisuje się i działają one tak jak w normalnej matematyce, dzielenie zapisuje się przy pomocy ukośnika i jest ono zawsze dzieleniem całkowitym:

\begin{CodeFrame*}[bash]{}
a=12; b=3; x=5; y=6

e=$(( ($a + $b) * 4 - $y ))
c=$((  $x / $y ))

echo $e $c $z
\end{CodeFrame*}
Zauważ zachowanie przy odwołaniu do niezdefiniowanej zmiennej \Verb{z}.

Do operacji arytmetycznych może być też jest wykorzystywane polecenie let.
Najczęściej jest stosowane do inkrementacji podanej zmiennej, tak jak w poniższym przykładzie.

\begin{CodeFrame*}[bash]{}
echo $a
let a++
echo $a
\end{CodeFrame*}

Zarówno operator podwójnych nawiasów okrągłych jak i komenda let mogą obsługiwać wyrażenia logiczne.
Mimo to operacje logiczne najczęściej obsługiwane są komendą \Verb{test} lub operatorem \Verb{[ ]} wynik zwracany jest jako kod powrotu.
Należy zwrócić uwagę na escapowanie odwrotnym ukośnikiem nawiasów i na to że spacje mają znaczenie.
Negację realizuje \Verb{!}, należy pamiętać jednak że wynikiem negacji dowolnej liczby jest FALSE.

\begin{CodeFrame*}[bash]{}
[ \( $a -ge 0 -a $b -lt 2 \) -o $z -eq 5 ]; z=$?

echo $z
\end{CodeFrame*}
Wartość zmiennej \Verb{z} jest wynikiem warunku: \texttt{((a większe równe od zera) AND (b mniejsze od dwóch)) OR (z równe 5)}.
Bash stosuje logikę odwróconą 0 oznacza prawdę, coś nie zerowego to fałsz.

Jako operacje podstawowe powinniśmy patrzyć także na wykonanie innych programów i pobieranie ich standardowego wyjścia i/lub kodu powrotu.
Pobieranie standardowego wyjścia możemy realizować za pomocą ujęcia polecenia w \emph{backquotes} (\Verb{`}) lub operatora \Verb{$( )} (pozwala on na zagnieżdżanie takich operacji).
Natomiast kod powrotu ostatniej komendy znajduje się w zmiennej \Verb{$?} (używaliśmy tego już przy obliczaniu wyrażeń logicznych).

\begin{CodeFrame*}[bash]{}
a=`cat /etc/issuse`
b=$(cat /etc/issuse; cat /etc/resolv.conf)

echo  $a
echo  $b
echo "$b"
\end{CodeFrame*}
Zwróć uwagę na różnicę w wypisaniu zmiennej zawierającej znaki nowej linii objętej cudzysłowami i nie objętej nimi.

Bash nie obsługuje liczb zmiennoprzecinkowych, nieobsługiwane operacje można wykonać za pomocą innego programu np:

\begin{CodeFrame*}[bash]{}
a=`echo 'print(3/2)' | python3`
b=$(echo '3/2' | bc -l)
echo $a $b
\end{CodeFrame*}

\begin{ProTip}{inne polecenia}
Programowanie w bashu w dużej mierze polega na wywoływaniu innych programów (np. takich jak sed, grep, find, awk).
Sam bash oferuje jedynie podstawowe konstrukcje składniowe, obsługę zmiennych i pewnych podstawowych operacji na nich.

Na te zewnętrzne polecenia można patrzeć trochę jak na biblioteki w innych językach programowania –
	komendy gwarantowane przez standard stanowią „bibliotekę standardową” basha,
	a inne (np. użyty w powyższym przykładzie arytmetyki zmiennoprzecinkowej python) stanowią dodatkowe opcjonalne „biblioteki”, które pozwalają na łatwiejsze i szybsze rozwiązywanie problemów.
W zasadzie podobnie można patrzeć na wywołania zewnętrznych programów w ramach kodu Pythona, C czy innych języków (niekiedy łatwiej jest zrobić np. \Verb#system("mv plik nowyplik")# niż zakodować to bezpośrednio w Pythonie czy w C).
\end{ProTip}

\subsection{Uruchamianie kodu z pliku}

Dłuższe fragmenty kodu bashowego często wygodniej jest pisać w pliku tekstowym niż bezpośrednio w linii poleceń.
Plik taki może zostać wykonany przy pomocy polecenia: \Verb{./nazwa_pliku} pod warunkiem że ma prawo wykonalności
(powinien także zawierać w pierwszą linii komentarz określający program używany do interpretacji tekstowego pliku wykonywalnego,
w postaci: \Verb{#!/bin/bash}).
Może też być wykonany za pomocą wywołania: \Verb{bash nazwa_pliku}.

Przydatną alternatywą dla powyższych metod wykonania kodu zawartego w pliku jest włączenie go do aktualnej sesji basha przy pomocy \Verb{. ./nazwa_pliku}.
W odróżnieniu od poprzednich metod pozwala to na korzystanie z funkcji i zmiennych zdefiniowanych w tym pliku w kolejnych poleceniach.

\subsection{Pętle i warunki}

\subsubsection{Pętla for}

W bashu możemy korzystać z kilku wariantów pętli for. Jednym z najczęściej używanych jest przypadek iterowania po plikach\footnote{
	Dokładniej: iteracja odbywa się po liście napisów rozdzielanej spacjami - zobacz rezultat \Verb{echo /tmp/*}
}:

\begin{CodeFrame*}[bash]{}
for nazwa in /tmp/* ; do
	echo $nazwa;
done
\end{CodeFrame*}

Możliwe jest też iterowanie po wartościach całkowitych zarówno w stylu ,,shellowym'' \teacher{(zwróć uwagę na podobieństwo do Pythona) } jak i w stylu C

\begin{CodeFrame*}[bash]{}
for i in `seq 0 20`; do
	echo $i;
done

for (( i=0 ; $i<=20 ; i++ )) ; do
	echo $i;
done
\end{CodeFrame*}

\subsubsection{Pętla while}

Często używana jest pętla while w połączeniu z instrukcją \Verb$read$\footnote{
	Polecenie \Verb$read$ można także wykorzystać do wczytania danych podawanych przez użytkownika do jakiejś zmiennej – np. \Verb$read -p "wpisz coś >> " xyz$ wczyta tekst do zmiennej \Verb$xyz$.
	\Verb$read$ z opcją \Verb$-e$ potrafi korzystać z biblioteki readline, jednak np. współdzieli historię z historią basha. Dlatego często wygodniejsze może być zainstalowanie i użycie \Verb$rlwrap$, np:
	\Verb$xyz=`rlwrap -H historia.txt -S "wpisz coś >> " head -n1`$.
} co umożliwia przetwarzanie jakiegoś wejścia (wyniku komendy lub pliku) linia po linii (także z podziałem linii na słowa):

\begin{CodeFrame*}[bash]{}
cat /etc/fstab | while read slowo reszta; do
	echo $reszta;
done
\end{CodeFrame*}
Powyższa pętla wypisze po kolei wszystkie wiersze pliku \texttt{/etc/fstab} przekazanego przez stdin (przy pomocy komendy \texttt{cat})\footnote{
	Takie rozwiązanie nazywane jest \emph{martwym kotem} i powinno go się unikać.
	Lepszym rozwiązaniem jest przekazywanie pliku przez przekierowanie strumienia wejściowego przy pomocy \texttt{< plik},
	który w tym przypadku powinien znaleźć się za kończącym pętle słowem kluczowym \texttt{done}.
} z pominięciem pierwszego słowa (które wczytywane było do zmiennej slowo).

\begin{ProTip}[breakable]{przekierowania strumieni a zmienne}
Przekierowanie standardowego wyjścia na standardowe wejście odbywa się między dwoma różnymi procesami.
Zatem w konstrukcjach typu \texttt{while read} pętla while uruchamiana może być w procesie potomnym obecnej powłoki.
Efektem tego jest iż w niektórych przypadkach wykonywane modyfikacje zmiennych wewnątrz takiej pętli nie będą widoczne poza nią.

Przykładem takiej sytuacji jest poniższy kod (polecenie ps dodano aby pokazać utworzenie procesu potomnego powłoki):

\begin{CodeFrame*}[bash]{}
zm=0; ps -f
cat /etc/fstab | while read x; do
	[ $zm -lt 1 ] && ps -f
	zm=13
done
echo $zm
\end{CodeFrame*}

Jednak analogiczny kod w którym następuje przekierowanie z pliku zadziała poprawnie:

\begin{CodeFrame*}[bash]{}
zm=0; ps -f
while read x; do
	[ $zm -lt 1 ] && ps -f
	zm=13
done < /etc/fstab
echo $zm
\end{CodeFrame*}

Jeżeli do pętli ma trafić wyjście jakiegoś polecenia możemy użyć składni bash'a pozwalającej na podstawienie wyniku polecenia jako pliku w postaci \Verb$<(polecenie)$ wraz z przekierowaniem z pliku, na przykład:
\begin{CodeFrame*}[bash]{}
zm=0; ps -f
while read x; do
	[ $zm -lt 1 ] && ps -f
	zm=13
done < <$(cat /etc/fstab)
echo $zm
\end{CodeFrame*}

Zauważ spację pomiędzy dwoma znakami \Verb$<$ i brak spacji pomiędzy drugim \Verb$<$ i znakiem dolara).

Inną możliwością jest użycie kodu powrotu do odebrania wartości z wnętrza pętli:
\begin{CodeFrame*}[bash]{}
zm=0; ps -f
my_code() {
	while read x; do
		[ $zm -lt 1 ] && ps -f
		zm=13
	done;
	return $zm;
}
cat /etc/fstab | my_code
zm=$?
echo $zm
\end{CodeFrame*}

Została tu zdefiniowana funkcja bash'owa (\Verb$my_code$), o których więcej informacji znajdziesz w odpowiednim rozdziale poniżej.
\end{ProTip}

Słowa domyślnie rozdzielane są przy pomocy dowolnego ciągu spacji lub tabulatorów. Separator ten można zmienić za pomocą zmiennej \texttt{IFS}, np:

\begin{CodeFrame*}[bash]{}
IFS=":"
while read a b c; do echo "$a -- $c"; done < /etc/passwd
unset IFS # przywracamy domyślne zachowanie read poprzez usunięcie zmiennej IFS
\end{CodeFrame*}

Należy mieć na uwadze, że cudzysłów wokół wypisania zmiennej c są istotne – bez nich znak dwukropka mógłby być zmieniony na spacje.

% \setcounter{subsubsection}{0}
\insertZadanie{booklets-sections/linux/zadania-60-bash.tex}{petla_linki_html}{}

\subsubsection{Instrukcja if}

Poznane wcześniej obliczanie wartości wyrażeń logicznych najczęściej stosowane jest w instrukcji warunkowej \Verb{if}\footnote{
	Może być też stosowane np. w pokazanej wcześniej pętli \Verb{while}
}.

\begin{CodeFrame*}[bash]{}
# instruikcja if - else
if [ "$xx" = "kot" -o "$xx" = "pies" ]; then
	echo  "kot lub pies";
elif [ "$xx" = "ryba" ];  then
	echo  "ryba"
else
	echo  "coś innego"
fi
\end{CodeFrame*}

Zauważ że spacje wokół i wewnątrz nawiasów kwadratowych przy warunku są istotne składniowo,
zawartość nawiasów kwadratowych to tak naprawdę argumenty dla komendy \Verb{test}.
Oprócz typowych warunków logicznych możemy sprawdzać np. istnienie plików, czy też ich typ (link, katalog, etc).
Szczegółowy opis dostępnych warunków które mogą być użyte w tej konstrukcji znajduje się w \Verb{man test}.
\teacher{Zwrócić szczególną uwagę na tą dokumentację.}

Jako warunek może wystąpić dowolne polecenie wtedy sprawdzany jest jego kod powrotu 0 oznacza prawdę / zakończenie sukcesem,
a wartość nie zerowa fałsz / błąd

\begin{CodeFrame*}[bash]{}
if grep '^root:' /etc/passwd > /dev/null; then
	echo /etc/passwd zawiera root-a;
fi
\end{CodeFrame*}

Istnieje możliwość skróconego zapisu warunków z użyciem łączenia instrukcji
przy pomocy \Verb{&&} (wykonaj gdy poprzednia zwróciła zero -- true)
lub \Verb{||} (wykonaj gdy poprzednia zwróciła nie zero -- false):

\begin{CodeFrame*}[bash]{}
[ -f /etc/issuse ] && echo "jest plik /etc/issuse"

grep '^root:' /etc/passwd > /dev/null && echo /etc/passwd zawiera root-a;
\end{CodeFrame*}

\subsubsection{Instrukcja case}
Instrukcja \Verb{case} służy do rozważania wielu przypadków opartych na równości zmiennej z podanymi napisami.
\teacher{Zwróć uwagę na podobieństwo do switch z C/C++ oraz różnicę - operuje na napisach a nie liczbach}

\begin{CodeFrame*}[bash]{}
case $xx in
	kot | pies)
		echo  "kot lub pies"
		;;
	ryba)
		echo  "ryba"
		;;
	*)
		echo  "cos innego"
		;;
esac
\end{CodeFrame*}

% \setcounter{subsubsection}{0}
\insertZadanie{booklets-sections/linux/zadania-60-bash.tex}{warunek_istnienie_pliku}{}

\subsection{Definiowanie funkcji}

W bashu każda funkcja może przyjmować dowolną ilość parametrów pozycyjnych
(w identyczny sposób obsługiwane są argumenty linii poleceń dla całego skryptu).
Ilość parametrów znajduje się w zmiennej \Verb{$#}, lista wszystkich parametrów w \Verb{$@},
a do kolejnych parametrów możemy odwoływać się z użyciem \Verb{$1}, \Verb#$2#, itd.

\begin{CodeFrame*}[bash]{}
f1() {
	echo "wywołano z $# parametrami, parametry to: $@"
	
	[ $# -lt 2 ] && return;
	
	echo -e "drugi: $2\npierwszy: $1"
	
	# albo kolejnych w pętli
	for a in "$@"; do  echo $a;  done
	
	# lub z użyciem polecenia shift
	for i in `seq 1 $#`; do
		echo $1
		shift # powoduje zapomnienie $1
		      # i przenumerowanie argumentów pozycyjnych o 1
		      # wpływa na wartości $@ $# itp
	done
	
	# funkcja może zwracać tylko wartość numeryczną -- tzw kod powrotu
	return 83
}
\end{CodeFrame*}

Zwróć uwagę że w nawiasach po nazwie funkcji nie podajemy przyjmowanych argumentów, natomiast puste nawiasy te są elementem składniowym i muszą wystąpić. Jeżeli zapisujesz definicję funkcji w jednej linii, np. \Verb@abc() { echo "abc"; }@ to pamiętaj, że spacja po otwierającym nawiasie klamrowym jest obowiązkowa, podobnie jak średniki występujące po każdej (także ostatniej) instrukcji w ciele funkcji.

Wywołanie funkcji nie różni się niczym od wywołania programów czy instrukcji wbudowanych
(możemy używać przekierowań strumieni wejścia, wyjścia, czy też przechwycić wyjście do zmiennej). Powyższą funkcję możemy wywołać np. w następujący sposób: \Verb{f1 a "b c"   d}

\subsection{Grupowanie poleceń}

Funkcje są przykładem grupowania poleceń – funkcja stanowi nazwany blok kodu, czyli nazwaną grupę poleceń.
Polecenia w bashu możemy też grupować bez definiowania funkcji.
W tym celu możemy zastosować nawiasy klamrowe (tak jak w definicji funkcji) lub nawiasy okrągłe.

Stosując nawiasy klamrowe musimy pamiętać (tak samo jak było to w przypadku funkcji)
	o spacji po otwierającym nawiasie klamrowym i średniku (lub nowej linii) przed zamykającym.
Instrukcje podane w nawiasach klamrowych będą wykonane w bieżącej powłoce basha, czyli mogą modyfikować zmienne.

Polecenia podane w nawiasach okrągłych będą wykonane w podpowłoce, czyli ustawione lub zmodyfikowane w nich zmienne nie będą widoczne po zakończeniu bloku. Nawiasy okrągłe nie wymagają spacji i ostatniego średnika.

\begin{CodeFrame*}[bash]{}
a=0;
{ echo abc; a=1; }
echo $a
(echo abc; a=2)
echo $a
\end{CodeFrame*}

Grupowanie poleceń jest przydatne na przykład w celu grupowania ich przy używaniu operatorów łączenia poleceń w oparciu o kod powrotu (\shell{&&} i \shell{||}), a także w celu przekazania standardowego wyjścia wielu poleceń w ramach jednego strumienia.

\begin{CodeFrame*}[bash]{}
a=0;
{ echo AbC; echo abc; echo XyZ; a=1; } | grep b
echo $a
\end{CodeFrame*}

Zauważ że w tym wypadku nawiasy klamrowe zachowały się jak nawiasy okrągłe – modyfikacja zmiennej a nie jest widoczna po zakończeniu bloku. Wynika to z użycia przekierowania strumienia podobnie jak w sytuacji omawianej przy pętli while.
% END: podstawy programowania w bashu

\insertZadanie{booklets-sections/linux/zadania-60-bash.tex}{funkcja_n_razy_napis}{}
\insertZadanie{booklets-sections/linux/zadania-60-bash.tex}{funkcja_liczba_kotow}{}
