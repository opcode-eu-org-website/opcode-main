% Copyright (c) 2017-2020 Matematyka dla Ciekawych Świata (http://ciekawi.icm.edu.pl/)
% Copyright (c) 2017-2020 Robert Ryszard Paciorek <rrp@opcode.eu.org>
% 
% MIT License
% 
% Permission is hereby granted, free of charge, to any person obtaining a copy
% of this software and associated documentation files (the "Software"), to deal
% in the Software without restriction, including without limitation the rights
% to use, copy, modify, merge, publish, distribute, sublicense, and/or sell
% copies of the Software, and to permit persons to whom the Software is
% furnished to do so, subject to the following conditions:
% 
% The above copyright notice and this permission notice shall be included in all
% copies or substantial portions of the Software.
% 
% THE SOFTWARE IS PROVIDED "AS IS", WITHOUT WARRANTY OF ANY KIND, EXPRESS OR
% IMPLIED, INCLUDING BUT NOT LIMITED TO THE WARRANTIES OF MERCHANTABILITY,
% FITNESS FOR A PARTICULAR PURPOSE AND NONINFRINGEMENT. IN NO EVENT SHALL THE
% AUTHORS OR COPYRIGHT HOLDERS BE LIABLE FOR ANY CLAIM, DAMAGES OR OTHER
% LIABILITY, WHETHER IN AN ACTION OF CONTRACT, TORT OR OTHERWISE, ARISING FROM,
% OUT OF OR IN CONNECTION WITH THE SOFTWARE OR THE USE OR OTHER DEALINGS IN THE
% SOFTWARE.

% BEGIN: Napis jako kod bashowy
\subsection{Napis jako kod bashowy}

\subsubsection{nazwa polecenia w zmiennej}

Jeżeli mamy w zmiennej nazwę polecenia do wykonania to możemy je wykonać uruchamiając po prostu tą zmienną:

\begin{CodeFrame}[bash]{0.5\textwidth}
a="echo"
$a ABC
\end{CodeFrame}
\begin{CodeFrame}{auto}
ABC
\end{CodeFrame}

Możemy nawet uruchamiać polecenia które zapisane mamy w zmiennej wraz z opcjami i argumentami:

\begin{CodeFrame}[bash]{0.5\textwidth}
a="echo -e abc\ndef ..."
$a ABC
\end{CodeFrame}
\begin{CodeFrame}{auto}
abc
def ... ABC
\end{CodeFrame}

Metoda ta nie pozwala jednak na podstawianie wartości zmiennych podanych w napisie w momencie uruchamiania go, itp.

\subsubsection{eval}

Polecenie wbudowane \shell{eval} pozwala na wykonanie przekazanych do niego argumentów jako polecenie shellowe.
Zapewnia ono podstawienie występujących w tym napisie zmiennych, itd.

Polecenia tego możemy użyć np. do z wartości zmiennej, której nazwę mamy w innej zmiennej (zamiast \shell@${!x}@, które nie jest dostępne w czystym sh):

\begin{CodeFrame}[bash]{0.5\textwidth}
A='tekst do wypisania, $SHELL, `ls -d`'
B="A"

C=$( eval "echo \$$B" )
echo $C
\end{CodeFrame}
\begin{CodeFrame}{auto}
tekst do wypisania, $SHELL, `ls -d`
\end{CodeFrame}

Jeżeli chcielibyśmy aby kod znajdujący się w zmiennej A także został przetworzony (podstawiona wartość zmiennej oraz output komendy ls), możemy użyć eval dwukrotnie:

\begin{CodeFrame}[bash]{0.5\textwidth}
A='tekst do wypisania, $SHELL, `ls -d`'
B="A"

C=$( eval eval "echo \$$B" )
echo $C
\end{CodeFrame}
\begin{CodeFrame}{auto}
tekst do wypisania, /bin/bash, .
\end{CodeFrame}

Dzięki użyciu eval możemy też budować kod bashowy (np. fragmenty konstrukcji case) w oparciu o zawartość zmiennych:

\begin{CodeFrame}[bash]{0.5\textwidth}
x=a

LISTA_WYBORU="a) echo AA;; b) echo BB;;"
INNE="e|f"

eval "case $x in
	$LISTA_WYBORU
	c) echo "CCC";;
	$INNE) echo inne;;
esac"
\end{CodeFrame}
\begin{CodeFrame}{auto}
AA
\end{CodeFrame}

\subsubsection{envsubst}

Innym poleceniem umożliwiającym podstawienie wartości zmiennych występujących w jakimś napisie jest envsubst.
W odróżnieniu od eval nie wykonuje on podanego kodu, a jedynie przetwarza napis podstawiając wartości zmiennych
(zatem jest bezpieczniejsza przy przetwarzaniu danych niewiadomego pochodzenia).
Jako że jest to zewnętrzny program używane zmienne muszą być dla niego dostępne, czyli muszą być wyeksportowane.

\begin{CodeFrame}[bash]{0.5\textwidth}
x=12
A='tekst, ${x}, $SHELL, $x, `ls -d`'

C=`export x; echo $A | envsubst`
echo $C
\end{CodeFrame}
\begin{CodeFrame}{auto}
tekst, 12, /bin/bash, 12, `ls -d`
\end{CodeFrame}
% END: Napis jako kod bashowy


% BEGIN: Przekierowania, procesy, ...
\subsection{Przekierowania, procesy, ...}

Poznaliśmy już podstawy przekierowywania standardowych strumieni pomiędzy procesami oraz z użyciem plików.
Mówiliśmy też o pewnych ograniczeniach jakie wiąża się z takimi przekierowaniami i programowaniem w bashu.
Należy jednak wspomnieć jeszcze o kilku istotnych aspektach takich przekierowań.

\subsubsection{cat, tee i \shell{<<}}

Przekierowania takie są często używane do tworzenia zewnętrznych plików z poziomu kodu skryptu bashowego.
Wykonywanie takiej operacji z użyciem wielu instrukcji echo niekoniecznie jest wygodne, dlatego często stosowane jest następujące podejście:

\begin{CodeFrame*}[bash]{}
cat > plik << EOF
wieloliniowa
treść zapisywana
do wskazanego pliku
EOF
\end{CodeFrame*}

Widzimy wywołanie komendy cat, z którym mogliśmy się już spotkać.
Polecenie to wywołane z argumentami będącymi nazwami plików wypisze te pliki na standardowe wyjście.
Wywołane bez argumentów przeczyta swoje standardowe wejście i wypisze je na standardowe wyjście
	(co, dzięki użyciu pzrzekierowania strumieni, tylko pozornie może wydawać się bezsensowne – na przykład wywołanie \shell{cat > plik} pozwala nam wprowadzić treść do wskazanego pliku, przypomnij sobie także przesyłanie strumienowe plików z użyciem ssh).

Wyjście cat przekierowane jest do pliku, jednak zwraca tutaj uwagę inny operator - dwa znaki mniejszości (\shell{<<}).
Powoduje on że tekst podawany w kolejnych liniach będzie kierowany na standardowe wejście komendy po której wystąpił,
	aż do momentu napotkania linii zawierającej jedynie słowo podane po nim (w tym wypadku klasyczny \texttt{EOF}, ale może to być dowolne inne słowo - np. \texttt{KONIEC}, lub nawet ciąg znaków ze spacjami ujęty w cudzysłowa).

Innym poleceniem przydatnym przy manipulacji strumieniami jest tee.
Podobnie jak cat kopiuje on swoje standardowe wejście na standardowe wyjście.
Natomiast jeżeli poda mu się ścieżkę do pliku będzie on zapisywał te dane także do wskazanego pliku.

\subsubsection{standardowe wejście/wyjście w miejscu ścieżki do pliku}

Wiele programów jeżeli w miejscu w którym oczekuje ściezki do pliku otrzyma myślnik zinterpretuje to jako użycie w tym miejscu standardowego wejścia / wyjścia. Nawet jeżeli program nie wspiera tej konwencji możemy użyć specjalnych urządzeń reprezentujących te strumienie: \texttt{/dev/stdin}, \texttt{/dev/stdout}, \texttt{/dev/stderr}. Na przykład:

\begin{CodeFrame*}[bash]{}
echo "ABC" > /dev/stderr
\end{CodeFrame*}

\noindent
spowoduje wypisanie komunikatu ABC na standardowym wyjściu błędu.

Bash pozwala także na podstawienie standardowego wyjścia różnych komend w miejsce kilku plików bez potrzeby jawnego tworzenia plików tymaczasowych. Na przykład:

\begin{CodeFrame*}[bash]{}
diff <(cat /etc/passwd) <(cat /etc/passwd-)
\end{CodeFrame*}

\noindent
które poleceniu diff jako jeden plik postawia standardowe wyjście pierwszego cat, a jako drugi plik drugiego cat.
Oczywiście zaprezentowane zastosowanie tego z poleceniami \texttt{cat} jest bezsensowne (prościej podać ścieżki do plików),
	ale gdyby występował tam już np. jakiś grep mogłoby to być użyteczne.
Jest to rozszerzenie bashowe nie występujące w czystym sh.

\subsubsection{polecenia wykonywane na zakończenie}

Jak wiemy działający program może zostać zakończony w sposób niespodziewany, np. na skutek działania sygnałów.
Dotyczy to również basha wykonującego jakiś skrypt powłoki.
Możemy nawet samodzielnie zażądać przerwania wykonywania skryptu przy pierwszym błędzie (pierwszym poleceniu które zwróci nieobsłużony niezerowy kod powrotu) wydając polecenie \shell{set -e}.
Niekiedy chcemy móc w takiej sytuacji wykonać jakieś działania.
Możliwe jest to z użyciem instrukcji \shell{trap}, która pozwala na zdefiniowanie procedury obsługi sygnałów (oczywiście tych które mamy prawo obsłużyć) i innych zdarzeń. Na przykład:

\begin{CodeFrame*}[bash]{}
trap '{ echo "to koniec"; }' EXIT
\end{CodeFrame*}

\noindent
spowoduje wypisanie "to koniec" w momencie kończenia pracy powłoki.
Umieszczenie takiej instrukcji w kodzie skryptu spowoduje wypisanie tego komunikatu niezależnie od przyczyny zakończenia (dojście do końca skryptu wywołanie polecenia exit, przerwanie Control C, czy też otrzymanie innego przechwytywalnego sygnału).

% \subsubsection{zarządzanie procesami w tle}
% 
% Jak wiemy koncząc polecenie przy pomocy \shell{&} możemy uruchomić je w tle.
% Z użyciem polecenia \shell{bg} możemy też przeniesć w tło polecenie zawieszone za pomocą Control Z.
% Powłoka jako rodzic procesów związanych z tak uruchomionymi poleceniami posiada ich listę i pozwala na łatwe zarządzanie nimi.
% Listę tą możemy wyświetlic z użyciem polecenia \shell{jobs}, jezeli dodamy opcję \texttt{-l} to zobaczymy także identyfikatory procesów.
% 
% Korzystając z polecenia \shell{fg} któremu jako argument podamy numer jobu (nie PID) możemy przenieść takie zadanie na pierwszy plan. Ponowne przeniesienie w tło możliwe będzie na przykład dzięki Control Z i \shell{bg}. Możemy też przesłać sygnały do takich zadań z użyciem polecenia \shell{kill} podając albo PID albo numer jobu poprzedzony znakiem procenta.
% 
% Zadanie możemy także usunąć z listy jobów pozostawiajac je jednak działającym z użyciem \shell{disown}.

% END: Przekierowania, procesy, ...
