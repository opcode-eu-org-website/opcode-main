% Copyright (c) 2017-2020 Matematyka dla Ciekawych Świata (http://ciekawi.icm.edu.pl/)
% Copyright (c) 2017-2020 Robert Ryszard Paciorek <rrp@opcode.eu.org>
% 
% MIT License
% 
% Permission is hereby granted, free of charge, to any person obtaining a copy
% of this software and associated documentation files (the "Software"), to deal
% in the Software without restriction, including without limitation the rights
% to use, copy, modify, merge, publish, distribute, sublicense, and/or sell
% copies of the Software, and to permit persons to whom the Software is
% furnished to do so, subject to the following conditions:
% 
% The above copyright notice and this permission notice shall be included in all
% copies or substantial portions of the Software.
% 
% THE SOFTWARE IS PROVIDED "AS IS", WITHOUT WARRANTY OF ANY KIND, EXPRESS OR
% IMPLIED, INCLUDING BUT NOT LIMITED TO THE WARRANTIES OF MERCHANTABILITY,
% FITNESS FOR A PARTICULAR PURPOSE AND NONINFRINGEMENT. IN NO EVENT SHALL THE
% AUTHORS OR COPYRIGHT HOLDERS BE LIABLE FOR ANY CLAIM, DAMAGES OR OTHER
% LIABILITY, WHETHER IN AN ACTION OF CONTRACT, TORT OR OTHERWISE, ARISING FROM,
% OUT OF OR IN CONNECTION WITH THE SOFTWARE OR THE USE OR OTHER DEALINGS IN THE
% SOFTWARE.

% BEGIN: Użytkownicy, uprawnienia i procesy
\section{Użytkownicy, uprawnienia i procesy}

%Systemy ,,unixowe'' tworzone były jako wielo-użytkownikowe i wielo-procesowe, w wyniku tego posiadają 

\subsection{Uprawnienia do plików}
Podstawowe unixowe uprawnienia do plików składają się z trzech członów: uprawnienia dla właściciela (u), grupy (g) i pozostałych użytkowników (o).
W każdym z członów mogą być przyznane uprawnienia do czytania (r), pisania (w) i wykonywania (x); w odniesieniu do plików jest to intuicyjne (uprawnienie do wykonywania jest potrzebne do uruchomienia programów), natomiast w stosunku do katalogów wygląda to następująco: uprawnienia do czytania pozwalają na listowanie zawartości, do wykonania pozwalają na dostęp, do zawartości katalogu (wejścia do niego) do pisania na tworzenie nowych obiektów wewnątrz niego i zmienianie nazw istniejących.

Rozszerzeniem podstawowych uprawnień opisanych powyżej jest mechanizm Filesystem Access Control List (ACL, fACL).\\
Jest on opcjonalnym mechanizmem który (na wspierających go systemach plików) pozwala na definiowanie indywidualnych uprawnień do pliku dla poszczególnych użytkowników i grup – plik ma nadal swojego właściciela, grupę i wszystkich pozostałych, ale przed prawami dla "others" wchodzą prawa użytkowników i grup definiowanych w ACL. Wypadkowe prawa obliczane są jako suma wynikła z praw użytkownika i grup do których należy.\\
ACL pozwala ponadto definiować uprawnienia domyślne dla nowo powstałych plików w katalogu (są one opcją katalogu).

Wszystkie poniższe komendy przyjmują opcję \texttt{-R} powodującą rekursywne wykonywanie zmian na drzewku katalogów/plików rozpoczynającym się w podanej ścieżce.
\begin{itemize}
	\item \Verb{chown [opcje] właściciel ścieżka}
		zmiana właściciela pliku
	\item \Verb{chgrp [opcje] grupa ścieżka}
		zmiana grupy do której należy plik pliku
	\item \Verb{chmod [opcje] uprawnienia ścieżka}
		zmiana prawa dostępu do pliku(ów)
	
	\item \Verb{getfacl [opcje] [ścieżka]}
		odczyt uprawnień związanych z listami kontroli dostępu fACL
	\item \Verb{setfacl [opcje] [ścieżka]}
		ustawianie uprawnień związanych z listami kontroli dostępu fACL
\end{itemize}

\noindent\zaawansowane{**} Dodatkowo należy wspomnieć też o poleceniach takich jak:
\begin{itemize}
	\item \Verb{lsattr} / \Verb{chattr}
		wyświetla / modyfikuje atrybuty plików związanych z systemem plików (np. zabrania jakiejkolwiek modyfikacji pliku)
	\item \Verb{getcap} / \Verb{setcap}
		wyświetla / modyfikuje atrybuty plików związanych z właściwościami jądra (zasadniczo zwiększonymi uprawnieniami programów je posiadających, ale bardziej ograniczonymi niż wykonanie na prawach root przez SUID)
\end{itemize}

\subsection{Użytkownicy}
\begin{itemize}
	\item \Verb{id [użytkownik]}
		informacja o użytkowniku (m.in. grupy do których należy)
	\item \Verb{whoami}
		informacja o aktualnym użytkowniku
	\item \Verb{w lub who}
		informacja o zalogowanych użytkownikach
	
	\item \Verb{passwd [użytkownik]}
		zmiana hasła
		
	\item \Verb{su [użytkownik]}
		przełącza użytkownika (aby przełączony użytkownik miał dostęp do "naszego" x serwera wcześniej wydajemy \texttt{xhost LOCAL:użytkownik})
	\item \Verb{sudo}
		program pozwalający na wykonywanie uprzywilejowanych komend przez wyznaczonych użytkowników
\end{itemize}

\subsection{Procesy i zasoby}
\begin{itemize}
	\item \Verb{ps [opcje]}
		wyświetla aktualnie działające procesy i informacje o nich\\
			np. kombinacja opcji \texttt{-Af} powoduje wyświetlenie wszystkich procesów w rozszerzonym formacie wypisywania
		
	\item \Verb{top}
		monitorowanie procesów obciążających CPU, pamięć, itd
	\item \Verb{iotop}
		monitorowanie procesów obciążających I/O
	
	\item \Verb{kill [opcje] pid}
		przesyła sygnał do procesów o podanych PID
	\item \Verb{killall [opcje] nazwa}
		przesyła sygnał do procesów o pasującej nazwie
\end{itemize}

\begin{ProTip}[breakable]{kill i zabijanie procesu}
Polecenie kill domyślnie wysyła sygnał SIGTERM, który jest prośbą o zakończenie procesu (proces może ją uszanować lub nie, np. zignorować). Więc sam kill nie zabija procesu.

Wiele sygnałów może zostać przechwyconych i obsłużonych (zignorowanych) przez proces do którego są adresowane. Istnieją także sygnały, które nie mogą zostać obsłużone bądź zignorowane są to m.in.:
	SIGKILL (zakończenie procesu bez dania mu jakiejkolwiek szansy zrobienia czegoś na „do widzenia”, wysyłany przez \Verb#kill -9#),
	SIGSTOP (wstrzymanie procesu).
\end{ProTip}

\begin{ProTip}[breakable]{Ctrl+C / Ctrl+Z / Ctrl+D}
\textbf{Ctrl+C} wysyła sygnał SIGINT do procesu zajmującego terminal na którym został on wprowadzony. Sygnał ten jest prośbą o zakończenie procesu, którą proces może uszanować lub nie (np. może całkiem zignorować lub poprosić o potwierdzenie). Jest on podobny do SIGTERM, jednak jest innym sygnałem i może być inaczej obsłużony (np. w SIGTERM nie ma większego sensu pytać o potwierdzenie).

\vspace{6pt}\textbf{Ctrl+Z} wysyła sygnał SIGTSTP do procesu zajmującego terminal na którym został on wprowadzony. Sygnał ten jest prośbą o wstrzymanie procesu i oddanie terminala, prośba ta może być zignorowana przez proces. Proces przerwany w ten sposób może być wznowiony poleceniem \Verb#fg# (które wznowi go jako pierwszoplanowy – okupujący terminal) lub \Verb#bg# (które wznowi go jako jako proces w tle – oddając terminal, przodkowi który go posiadał wcześniej).

\vspace{6pt}\textbf{Ctrl+D} nie wysyła żadnego sygnału, działa tylko gdy proces czyta dane z terminala (podłączonego zazwyczaj do jego standardowego wejścia). Wysyła on do terminala znak EOT (End-of-Transmission), w efekcie czego:
\begin{itemize}
	\item (jeżeli bufor wejściowy jest niepusty) terminal wypycha bufor wejściowy do programu (tak jak po wprowadzeniu nowej linii), albo
	\item (jeżeli nie ma znaków w buforze) terminal zamyka strumień wprowadzanych danych do programu
\end{itemize}
Program nie otrzymuje w strumieniu znaku EOT (jest on przechwycony przez terminal).
Zamknięcie strumienia wejściowego na ogół prowadzi także do zakończenia działania programu, jednak (w odróżnieniu od Ctrl-C) pozwala programowi na normalne przetworzenie wprowadzonych danych.
\end{ProTip}

\begin{ProTip}[breakable]{Ctrl+S / Ctrl+Q}
\textbf{Ctrl+S} wstrzymuje przewijanie (odświeżanie) terminala, aby wznowić należy użyć \textbf{Ctrl+Q}.
\end{ProTip}
% END: Użytkownicy, uprawnienia i procesy

% BEGIN: Planowanie zadań
\subsection{Planowanie zadań}
Typowo system zapewnia usługę uruchamiania zadań o zadanym czasie. Z usługi tej można skorzystać przy pomocy poleceń:
\begin{itemize}
	\item \Verb{crontab} pozwala oglądać i edytować tablice zaplanowanych zadań cyklicznych (dla cron'a)
	\item \Verb{at} pozwala jednorazowo zaplanować zadanie
\end{itemize}
	Pliki konfiguracyjne crona / obsługiwane crontab-em mają postać: \texttt{minuty godzina  dzienMiesiaca miesiac dzienTygodnia polecenie}.
	Wpis oznacza że polecenie ma zostać wykonane jeżeli wszystkie warunki będą spełnione, jeżeli jakiś warunek nie jest nam potrzebny można użyć gwiazdki \texttt{*},
	z kolei \texttt{*/n} oznacza wykonywanie jeżeli dana wartość jest podzielna przez n. Np.:
		\texttt{*/20 3  * * 1 ls} oznacza wykonanie komendy ls w każdy poniedziałek o godzinie 3:00 3:20 i 3:40
	
	Standardowe wyjście, wyjście błędu oraz powiadomienie o niezerowym kodzie powrotu domyślnie są wysyłane na lokalny adres mailowy użytkownika będącego właścicielem danego contaba.
	Niekiedy dostępny jest także \texttt{anacron} pozwalający na mniej precyzyjne planowanie zadań.
% END: Planowanie zadań
