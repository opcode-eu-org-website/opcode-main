% Copyright (c) 2017-2020 Matematyka dla Ciekawych Świata (http://ciekawi.icm.edu.pl/)
% Copyright (c) 2017-2020 Robert Ryszard Paciorek <rrp@opcode.eu.org>
% 
% MIT License
% 
% Permission is hereby granted, free of charge, to any person obtaining a copy
% of this software and associated documentation files (the "Software"), to deal
% in the Software without restriction, including without limitation the rights
% to use, copy, modify, merge, publish, distribute, sublicense, and/or sell
% copies of the Software, and to permit persons to whom the Software is
% furnished to do so, subject to the following conditions:
% 
% The above copyright notice and this permission notice shall be included in all
% copies or substantial portions of the Software.
% 
% THE SOFTWARE IS PROVIDED "AS IS", WITHOUT WARRANTY OF ANY KIND, EXPRESS OR
% IMPLIED, INCLUDING BUT NOT LIMITED TO THE WARRANTIES OF MERCHANTABILITY,
% FITNESS FOR A PARTICULAR PURPOSE AND NONINFRINGEMENT. IN NO EVENT SHALL THE
% AUTHORS OR COPYRIGHT HOLDERS BE LIABLE FOR ANY CLAIM, DAMAGES OR OTHER
% LIABILITY, WHETHER IN AN ACTION OF CONTRACT, TORT OR OTHERWISE, ARISING FROM,
% OUT OF OR IN CONNECTION WITH THE SOFTWARE OR THE USE OR OTHER DEALINGS IN THE
% SOFTWARE.

% BEGIN: vi i vim
\subsection{vi i vim}
\Verb{vi} jest chyba najbardziej zaawansowanym edytorem, którego obecność gwarantuje standard POSIX\footnote{IEEE Std 1003.1-2017 (The Open Group Base Specifications Issue 7, 2018 edition), XCU part \url{https://pubs.opengroup.org/onlinepubs/9699919799/}}.
\Verb{vim} jest mocno rozbudowanym jego klonem, oferującym bardzo zaawansowane funkcjonalności, powszechnie stosowanym jako zamiennik oryginalnego \Verb{vi}.
\Verb{vim} obsługuje 3 podstawowe tryby pracy: komend (służący do wydawania opisanych niżej poleceń), wizualny (służący do zaznaczania i wydawania niektórych komend), edycji (wstawiania/nadpisywania - służący do wprowadzania tekstu).
Podstawowa klawiszologia:
\begin{itemize}
	\item przełączanie pomiędzy trybami:
	\begin{itemize}
		\item \Verb{Esc} powrót do trybu komend
		\item \Verb{i} tryb wstawiania; \Verb{A} tryb wstawiania ze skokiem na koniec linii, \Verb{o} / \Verb{O} tryb wstawiania ze wstawieniem nowej linii po / przed bierzącą
		\item \Verb{R} tryb zastępowania
		\item \Verb{Insert} zmiana trybu wstawiania i zastępowania
		\item \Verb{v} tryb wizualny (umożliwia zaznaczenie przy pomocy strzałek); \Verb{ctrl+v} tryb wizualny blokowy, \Verb{V} tryb wizualny liniowy
		
		\item \Verb{:set paste} włącza \Verb{:set nopaste} wyłącza tryb wklejania (nie będzie działać automatyczne formatowanie itp.)
		\item \Verb{gv} ponawia ostatnie zaznaczenie trybu wizualnego
	\end{itemize}

	\item wycinanie i kopiowanie:
	\begin{itemize}
		\item \Verb{y} skopiuj; \Verb{d} - wytnij (skopiuj i usuń)
			po \Verb{y}, \Verb{d} można podać np. \Verb{20l} lub \Verb{20[strzałka w prawo]} co oznacza 20 kolejnych znaków, \Verb{2w} oznacza dwa słowa
			(więcej o takich punktach skoku poniżej)
		\item \Verb{x} wytnij (skopiuj i usuń) znak (może być poprzedzone ilością znaków do wycięcia); wielkie \Verb{X} działa analogicznie, tyle że w tył
		\item \Verb{yy} skopiuj linię; \Verb{dd} - wytnij (skopiuj i usuń)
			w obu wypadkach może być poprzedzone ilością linii do skopiowania/wycięcia
		
		\item \Verb{p} wkleja po; \Verb{P} - wkleja przed
		
		\item komendy kopiowania i wklejania mogą być poprzedzone jedno-znakową nazwą rejestru w którym umieszczane są dane (poprzedzamy ją znakiem \Verb{"} i podajemy przed licznikiem, np. \Verb{"a3dd} wytnie do rejestru a 3 linie), część rejestrów jest używana automatycznie, a niektóre są tylko do odczytu, podgląd aktualnej zawartości rejestrów możliwy jest przy pomocy komendy \Verb{:registers}
	\end{itemize}
	
	\item wyszukiwanie, zastępowanie, skok do linii:
	\begin{itemize}
		\item \Verb{/} szukanie w przód, \Verb{?} szukanie w tył; \Verb{*} szukanie w przód słowa pod kursorem, \Verb{#} szukanie w tył słowa pod kursorem
		\item \Verb{n} wyszukanie następnego wystąpienie; \Verb{N} wyszukanie poprzedniego wystąpienie
		
		\item \Verb{G} przejście do wskazanej linii, numer podajemy przed G, 0 oznacza ostatnią linię w pliku, więc \Verb{0G} spowoduje przejście do niej
		
		\item \Verb{:[zakres]s@regexp@napis@[g]} wyszukaj i zastąp wyrażenie regularne regexp przez napis;
			zakres może być:
			\begin{itemize}
				\item numerem linii,
				\item przedziałem z numerami linii postaci \Verb{pierwsza,ostatnia}, gdzie:
					\Verb{.} oznacza bieżącą linię,
					\Verb{$} oznacza ostatnią linię w pliku,
					wartość numeryczna poprzedzona \Verb{+} oznacza tyle kolejnych linii od bieżącej, a poprzedzona \Verb{-} przed bieżącą,
				\item znakiem \Verb{%} (co oznacza cały plik),
				\item zakresem zaznaczonym w trybie wizualnym;
			\end{itemize}
			podanie opcji g powoduje zastępowanie wszystkich wystąpień a nie tylko pierwszego;
			znak \Verb{@} pełni rolę separatora i może zostać zamiast niego użyty inny znak
	\end{itemize}
	
	\item otwieranie, zapisywanie, zamykanie plików:
	\begin{itemize}
		\item \Verb{:e ścieżka} otwarcie wskazanego pliku
		\item \Verb{:w} zapis (można także podać ścieżkę pod jaka ma zostać zapisany plik)
		\item \Verb{:q} wyjście
		\item \Verb{:q!} wyjście bez zapisywania
		\item \Verb{:wq} zapis i wyjście
	\end{itemize}
	
	\item przełączanie się między otwartymi plikami i oknami:
	\begin{itemize}
		\item \Verb{:n} następny plik; \Verb{:N} poprzedni plik
		\item \Verb{:split} poziomy podział okna; \Verb{:vs} pionowy podział okna; \Verb{Ctrl}+\Verb{W} a następnie strzałka - przełączanie między oknami
	\end{itemize}
	
	\item cofanie i ponawianie edycji:
	\begin{itemize}
		\item \Verb{u}, \Verb{:undo} cofa ostatnią operację
		\item \Verb{Ctrl+r}, \Verb{:redo} ponawia cofniętą operację
	\end{itemize}
	
	\item punkty skoku (mogą być używane jako polecenia do poruszania się lub jako adresy w poleceniach takich jak \Verb{d}, \Verb{y}):
	\begin{itemize}
		\item \Verb{l} / \Verb{h} / \Verb{k} / \Verb{j} jeden znak/linię w prawo / lewo / górę / dół (działa tak jak strzałki)
		\item \Verb{0} / \Verb{^} / \Verb{$} początek linii / początek tekstu w linii, koniec linii
		
		\item \Verb{w} / \Verb{b} / \Verb{e} następne słowo / poprzednie słowo / koniec słowa; wielkie \Verb{W} / \Verb{B} / \Verb{E} działa analogicznie, różni się traktowaniem spacji przy słowie
		\item \Verb{f} / \Verb{F} następny / poprzedni znak podany po tej komendzie, włącznie z nim (np. \Verb{dfX} usunie wszystko do najbliższego wystąpienia X wraz z tym X); \Verb{t} / \Verb{T} działa analogicznie, tyle wyłącza podany znak
		
		\item poprzedzenie powyższych komend liczbą powoduje powtórzenie ich tyle razy - np. \Verb{10l} - 10 znaków w prawo, \Verb{3F:} - trzeci dwukropek w lewo
		\item punktem skoku jest też wyżej opisane polecenie \Verb{G} poprzedzane numerem linii do której ma się odbyć skok
		
		\item punktem skoku mogą być także swobodnie umieszczane z dokumencie zakładki identyfikowane pojedynczym znakiem:
		\begin{itemize}
				\item \Verb{m} i następie znak ją identyfikujący - utworzenie zakładki w miejscu kursora (np. \Verb{ma} - utworzy zakładkę a)
				\item \Verb{`} (backtick) / \Verb{'} (apostrof) skok do zakładki / linii z zakładką podaną po tej komendzie
				\item \Verb{:marks} - lista zakładek; \Verb{:delmarks} / \Verb{:delmarks!} - usunięcie zakładki / usunięcie wszystkich nie automatycznych zakładek
		\end{itemize}
	\end{itemize}

	\item inne:
	\begin{itemize}
		\item \Verb{:r plik}, wstawienie zawartości pliku
		\item \Verb{:%!xxd} pokazanie wartości numerycznych i umożliwienie edycji pliku jako binarnego;\\ \Verb{:%!xxd -r} powrót do normalnej edycji
		
		\item \Verb{>} / \Verb{<} zwiększanie / zmniejszanie wcięcia zaznaczonego (w trybie wizualnym) tekstu
		\item \Verb{zc} zwija bieżący blok, \Verb{zC} zwija bieżący blok aż do najwyższego poziomu, \Verb{zo} rozwija bieżące zwinięcie, \Verb{zO} rozwija rekurencyjnie bieżące zwinięcie, \Verb{zR} rozwija wszystkie zwinięcia w dokumencie
		
		\item \Verb{:set wrap} włącza \Verb{:set nowrap} wyłącza zawijania linii w podglądzie
	\end{itemize}
\end{itemize}
% END: vi i vim
