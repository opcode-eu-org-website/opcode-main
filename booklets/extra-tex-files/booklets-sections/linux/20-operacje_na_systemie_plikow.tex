% Copyright (c) 2017-2020 Matematyka dla Ciekawych Świata (http://ciekawi.icm.edu.pl/)
% Copyright (c) 2017-2020 Robert Ryszard Paciorek <rrp@opcode.eu.org>
% 
% MIT License
% 
% Permission is hereby granted, free of charge, to any person obtaining a copy
% of this software and associated documentation files (the "Software"), to deal
% in the Software without restriction, including without limitation the rights
% to use, copy, modify, merge, publish, distribute, sublicense, and/or sell
% copies of the Software, and to permit persons to whom the Software is
% furnished to do so, subject to the following conditions:
% 
% The above copyright notice and this permission notice shall be included in all
% copies or substantial portions of the Software.
% 
% THE SOFTWARE IS PROVIDED "AS IS", WITHOUT WARRANTY OF ANY KIND, EXPRESS OR
% IMPLIED, INCLUDING BUT NOT LIMITED TO THE WARRANTIES OF MERCHANTABILITY,
% FITNESS FOR A PARTICULAR PURPOSE AND NONINFRINGEMENT. IN NO EVENT SHALL THE
% AUTHORS OR COPYRIGHT HOLDERS BE LIABLE FOR ANY CLAIM, DAMAGES OR OTHER
% LIABILITY, WHETHER IN AN ACTION OF CONTRACT, TORT OR OTHERWISE, ARISING FROM,
% OUT OF OR IN CONNECTION WITH THE SOFTWARE OR THE USE OR OTHER DEALINGS IN THE
% SOFTWARE.

\section{Operacje na systemie plików}

\subsection{Podstawowe komendy}

% BEGIN: Operacje na systemie plików
\subsubsection{echo i znaki uogólniające}

Polecenie echo służy do wypisania przekazanych do niego argumentów na ekran – np. \Verb$echo abc xyz$ wypisze \textit{abc xyz}.
Polecenie to może zostać użyte także do wypisania plików pasujących do jakiegoś wzorca np. \Verb$echo a*$ wypisze pliki zaczynające się literą \textit{a}.

Dzieje się to dzięki obsłudze przez powłokę \textit{znaków uogólniających}, które mogą być użyte w napisach i zostaną rozwinięte przez powłokę do listy pasujących ścieżek lub nazw plików.
Podstawowymi znakami uogólniającymi powłoki są:
\begin{itemize}
	\item \Verb{?} oznaczający dowolny znak
	\item \Verb{*} oznaczający dowolny (także pusty) ciąg znaków
	\item \Verb{[a-z AD]} oznaczający dowolny znak z wymienionych w zbiorze ujętym w nawiasach kwadratowych, zbiór może być definiowany z użyciem zakresów, np. a-z AD oznacza dowolną małą literę od a do z włącznie, spację, dużą literą A lub D
	\item \Verb{[!a-z]} oznaczający dowolny znak z wyjątkiem znaków wymienionych w podanym zbiorze, zbiór może być definiowany z użyciem zakresów, np. a-z oznacza dowolną małą literę od a do z włącznie
\end{itemize}
Warto zwrócić uwagę że sama gwiazdka nie dopasowuje plików ukrytych (zaczynających się od kropki), czyli np. \Verb$echo *$ wypisze wszystkie pliki w bieżącym katalogu, z wyjątkiem tych których pierwszym znakiem w nazwie jest kropka.

Napisy (a więc także ścieżki i nazwy plików) mogą być ujęte w cudzysłowie pojedynczym (\texttt{'}, np. \texttt{'aaa bbb'}) lub podwójnym (\texttt{"}, np. \texttt{"aaa bbb"}) celem np. ochrony spacji w nich występujących.
Oba typy cudzysłowów zabezpieczają przed rozwijaniem znaków uogólniających (zastępowaniem napisu ze znakami listą pasujących nazw / ścieżek).
Cudzysłów pojedynczy (w odróżnieniu od podwójnego) zabezpiecza także przed interpretacją umieszczonych wewnątrz innych znaków specjalnych takich jak odwołania do zmiennych.

Listowanie plików jest operacją na tyle istotną i użyteczną że istnieje do tego dedykowane polecenie – \Verb$ls$. Pozwala ono na m.in. wypisywanie szczegółowych informacji o plikach, listowanie całej zawartości katalogu, sortowanie wyników itd.
Należy jednak pamiętać że rozwijaniem znaków uogólniających (czyli zamianą napisu je zawierającego na listę plików) dla polecenia \Verb$ls$ zajmuje się powłoka (czyli np. \Verb$bash$) – tak samo jak dla polecenia \Verb$echo$ – gdyż sama komenda \Verb$ls$ nie rozumie znaków uogólniających.
%
Kilka przykładów:
\begin{itemize}
	\item pliki z jednoznakową nazwą: \Verb{ls ?}
	\item pliki zaczynające się od znaku zapytania: \Verb{ls \?*} lub  \Verb{ls '?'*}
	\item pliki nie zaczynające się od a: \Verb{ls [!a]*}
	\item pliki mające w nazwie literę b: \Verb{ls *b*}
	\item pliki mające w nazwie literę a lub b: \Verb{ls *[ab]*}
	\item pliki zaczynające się od liter a,b,c,d: \Verb{ls [a-d]*}
	\item pliki ukryte (wliczając bieżący i nadrzędny katalog): \Verb{echo .*}
\end{itemize}

\subsubsection{listowanie i  wyszukiwanie plików}

\begin{itemize}
	\item \Verb{ls [opcje] [ścieżka]}
		listowanie zawartości katalogu, do ważniejszych opcji należy zaliczyć:\\
			\texttt{-a} wyświetlaj pliki ukryte (zaczynających się od kropki)\\
			\texttt{-l} wyświetlaj pliki w formie listy z szczegółowymi informacjami (uprawnienia, rozmiar, data modyfikacji, właściciel, grupa, rozmiar)\\
			\texttt{-1} wyświetlaj pliki w formie 1 plik w jednej linii (bez dodatkowych informacji; stosowane domyślne gdy wynik komendy przekazywany jest strumieniem do innej komendy lub pliku)\\
			\texttt{-h} stosuj jednostki typu k, M, G zamiast podawać rozmiar w bajtach\\
			\texttt{-t} sortuj wg daty modyfikacji\\
			\texttt{-S} sortuj wg rozmiaru\\
			\texttt{-r} odwróć kolejność sortowania\\
			\texttt{-c} użyj daty utworzenia zamiast daty modyfikacji (stosowane w połączeniu z \texttt{-l} i/lub \texttt{-t})\\
			\texttt{-d} wyświetlaj informacje o katalogu zamiast jego zawartości
		
	\item \Verb{find [opcje] [katalog startowy] [wyrażenie]}
		wyszukiwanie w systemie plików w oparciu o nazwę/ścieżkę lub właściwości pliku, do ważniejszych opcji należy zaliczyć:\\
			\texttt{-P} wypisuj informacje o linkach symbolicznych a nie plikach przez nie wskazywanych (domyślne)\\
			\texttt{-L} wypisuj informacje o wskazywanych przez linki symboliczne plikach\\
		do ważniejszych elementów wyrażenia należy zaliczyć:\\
			\texttt{-name "wyrażenie"} pliki których nazwa pasuje do wyrażenia korzystającego z shellowych znaków uogólniających\\
				komenda find (w odróżnieniu np. od \texttt{ls}) samodzielnie interpretują wyrażenia zawierające shellowe znaki uogólniające, w związku z czym konieczne może się okazać zabezpieczenie ich przed interpretacją przez powłokę np. przy pomocy umieszczenia wewnątrz pojedynczych cudzysłowów\\
			\texttt{-iname "wyrażenie"} jak \texttt{-name}, tyle że nie rozróżnia wielkości liter\\
			\texttt{-path "wyrażenie"} pliki których ścieżka pasuje do wyrażenia korzystającego z shellowych znaków uogólniających\\
			\texttt{-ipath "wyrażenie"} jak \texttt{-path}, tyle że nie rozróżnia wielkości liter\\
			\texttt{-regex "wyrażenie"} pliki których ścieżka pasuje do wyrażenia regularnego\\
			\texttt{-iregex "wyrażenie"} jak \texttt{-regexp}, tyle że nie rozróżnia wielkości liter\vspace{6pt}\\
			%
			\texttt{warunek -o warunek} łączy warunki sumą logiczną „OR” (zamiast domyślnego iloczynu logicznego „AND”)\\
			\texttt{! warunek} negacja warunku\vspace{6pt}\\
			%
			\texttt{-mtime [+|-]n} pliki których modyfikacja odbyła się \texttt{n}*24 godziny temu\\
			\texttt{-mmin [+|-]n} pliki których modyfikacja odbyła się \texttt{n} minut temu\\
			\texttt{-ctime [+|-]n} pliki które zostały utworzone \texttt{n}*24 godziny temu\\
			\texttt{-cmin [+|-]n} pliki które zostały utworzone \texttt{n} minut temu\\
			\texttt{-size [+|-]n[c|k|M|G]} pliki których rozmiar wynosi \texttt{n} (c - bajtów, k - kilobajtów, M - Megabajtów, G - gigabajtów)\\
			\emph{w powyższych testach \texttt{+} oznacza więcej niż, \texttt{-} oznacza mniej niż, uwaga: porównywaniu podlegają liczby całkowite, np. +1 oznacza $>1$ w liczbach całkowitych tzn. $\ge2$}\vspace{6pt}\\
			%
			\texttt{-exec}\Verb@ polecenie \{\} \;@ dla każdego znalezionego pliku wykonaj \texttt{polecenie} podstawiając ścieżkę do tego pliku pod \Verb@\{\}@ (zastosowane odwrotne ukośniki służą zabezpieczeniu nawiasów klamrowych i średnika przed zinterpretowaniem ich przez powłokę)\\
			\Verb@-execdir polecenie \{\} \;@, podobnie jak \texttt{-exec} tyle że polecenie zostanie uruchomione w katalogu w którym znajduje się wyszukany plik
		
	\item \Verb{du [opcje] ścieżka1 [ścieżka2 [...]]}
		wyświetlanie informacji o zajętej przestrzeni dyskowej przez wskazane pliki / katalogi, do ważniejszych opcji należy zaliczyć:\\
			\texttt{-s} podaje łączną ilość zajętego miejsca dla każdego argumentów (zamiast wypisywać rozmiar każdego pliku)\\
			\texttt{-c} podaje łączną ilość zajętego miejsca dla wszystkich argumentów\\
			\texttt{-h} stosuje jednostki typu k, M, G\\
		podawany rozmiar może się różnić (w obie strony) od wyniku ls: ls podaje rozmiar pliku (ile zawiera informacji lub ile zostało zadeklarowane że może jej zawierać), a du to ile zajmuje na dysku
		
	\item \Verb{df [opcje]}
		wyświetlanie informacji o zajętości miejsca na poszczególnych systemach plików
\end{itemize}

Należy zwrócić uwagę iż komenda \texttt{find} potrafi sama rozwijać znaki uogólniające\footnote{
	W przypadku argumentów niektórych z jej opcji. W przypadku określania katalogu startowego \texttt{find} zachowuje się jak inne komendy (np. \texttt{ls}) dla których znaki uogólniające musi rozwinąć powłoka.
	Na przykład jeżeli chemy przeszukać wszystkie katalogi zaczynające się na \textit{a} w poszukiwaniu plików zaczynających się na \textit{b} to należy wykonać: \Verb$find a* -name "b*"$,
	a nie \Verb$find "a*" -name "b*"$ czy \Verb$find a* -name b*$, itd.
}
i w przypadku argumentów opcji takich jak np. \texttt{-name} na ogół chcemy aby znaki uogólniające nie były rozwijane przez powłokę, a interpretowane przez samą komendę \texttt{find}
– w tym celu powinniśmy je zabezpieczyć przed rozwinięciem przy pomocy cudzysłowów.

Warto zauważyć także, że jeżeli komanda \texttt{ls} w wyniku rozwinięcia znaków uogólniających dostanie jako argument ścieżkę do katalogu to wylistuje jego zawartość (zachowanie to zmienia opcja \texttt{-d}).

\subsubsection{kopiowanie, przenoszenie, usuwanie, ...}
\begin{itemize}
	\item \Verb{cp [opcje] źródło1 [źródło2 [...]] cel}
		kopiuje wskazany plik (lub pliki) do wskazanej lokalizacji, w przypadku kopiowania wielu plików cel powinien być katalogiem, do ważniejszych opcji należy zaliczyć:\\
			\texttt{-r} pozwala na (rekursywne) kopiowanie katalogów\\
			\texttt{-a} podobnie jak \texttt{-r}, dodatkowo zachowując atrybuty plików\\
			\texttt{-l} zamiast kopiować tworzy twarde dowiązania (hard links)\\
			\texttt{-s} zamiast kopiować tworzy linki symboliczne do plików\\
			\texttt{-f} nadpisywanie bez pytania\\
			\texttt{-i} zawsze pytaj przed nadpisaniem
		
	\item \Verb{ln źródło1 [źródło2 [...]] cel}
		tworzy link (domyślnie „twardy”) do wskazanego pliku (lub plików) w wskazanej lokalizacji, w przypadku wskazania wielu plików źródłowych cel powinien być katalogiem, do ważniejszych opcji należy zaliczyć:\\
			\texttt{-s} tworzy dowiązania symboliczne (wskazujące na ścieżkę do oryginalnego pliku) zamiast twardych (wskazujących na te same dane co oryginalny plik)\\
			\texttt{-r} używa ścieżki względnej zamiast bezwzględnej przy tworzeniu dowiązań symbolicznych
			\vspace{5pt}\\\hspace*{1cm}\includegraphics[width=0.75\textwidth]{img/linux/link_twardy_i_symboliczny}
			\\
			Link twardy jest innym uchwytem do tych samych danych i może być używany także po skasowaniu oryginalnego pliku. Liczbę dowiązań do danego pliku pokazuje m.in. komenda \Verb{ls} z opcją \Verb{-l}.
			Nie można utworzyć linków twardych do katalogów, ani do plików na innym zasobie dyskowym (innym systemie plików).
			\\
			Link symboliczny wskazuje na konkretną ścieżkę (względną lub bezwzględną – co może mieć znaczenie przy przenoszeniu takiego linku) do dowolnego (nawet nie istniejącego – wtedy mówimy o zerwanym linku) pliku lub katalogu.
		
	\item \Verb{mv [opcje] źródło1 [źródło2 [...]] cel}
		przenosi wskazane pliki / katalogi do wskazanej lokalizacji, w przypadku przenoszenia wielu plików cel powinien być katalogiem, do ważniejszych opcji należy zaliczyć:\\
			\texttt{-f} nadpisywanie bez pytania\\
			\texttt{-i} zawsze pytaj przed nadpisaniem
		
	\item \Verb{rm [opcje] ścieżka1 [ścieżka2 [...]]}
		usuwa wskazane pliki, do ważniejszych opcji należy zaliczyć:\\
			\texttt{-r} pozwala na na (rekursywne) kasowanie katalogów wraz z zawartością\\
			\texttt{-f} usuwanie bez pytania\\
			\texttt{-i} zawsze pytaj przed usunięciem
			
	\item \Verb{mkdir [opcje] ścieżka1 [ścieżka2 [...]]}
		tworzy wskazane katalogi, do ważniejszych opcji należy zaliczyć:\\
			\texttt{-p} pozwala na tworzenie całej ścieżki a nie tylko ostatniego elementu, nie zgłasza błędu gdy wskazany katalog istnieje
\end{itemize}
% END: Operacje na systemie plików
