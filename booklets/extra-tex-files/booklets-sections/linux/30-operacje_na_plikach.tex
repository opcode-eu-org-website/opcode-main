% Copyright (c) 2017-2020 Matematyka dla Ciekawych Świata (http://ciekawi.icm.edu.pl/)
% Copyright (c) 2017-2020 Robert Ryszard Paciorek <rrp@opcode.eu.org>
% 
% MIT License
% 
% Permission is hereby granted, free of charge, to any person obtaining a copy
% of this software and associated documentation files (the "Software"), to deal
% in the Software without restriction, including without limitation the rights
% to use, copy, modify, merge, publish, distribute, sublicense, and/or sell
% copies of the Software, and to permit persons to whom the Software is
% furnished to do so, subject to the following conditions:
% 
% The above copyright notice and this permission notice shall be included in all
% copies or substantial portions of the Software.
% 
% THE SOFTWARE IS PROVIDED "AS IS", WITHOUT WARRANTY OF ANY KIND, EXPRESS OR
% IMPLIED, INCLUDING BUT NOT LIMITED TO THE WARRANTIES OF MERCHANTABILITY,
% FITNESS FOR A PARTICULAR PURPOSE AND NONINFRINGEMENT. IN NO EVENT SHALL THE
% AUTHORS OR COPYRIGHT HOLDERS BE LIABLE FOR ANY CLAIM, DAMAGES OR OTHER
% LIABILITY, WHETHER IN AN ACTION OF CONTRACT, TORT OR OTHERWISE, ARISING FROM,
% OUT OF OR IN CONNECTION WITH THE SOFTWARE OR THE USE OR OTHER DEALINGS IN THE
% SOFTWARE.

\section{Operacje na zawartości plików}

% BEGIN: grep
\subsection{grep i wyrażenia regularne}

Polecenie
\texttt{grep [opcje] wyrażenie [plik1 [plik2 [...]]]}
wyszukuje pasujące do wyrażenia regularnego wyrażenie linie w plikach, przydatne opcje:\\
	\texttt{-v} zamiast pasujących wypisz nie pasujące\\
	\texttt{-i} ignoruj wielkość liter\\
	\texttt{-a} przetwarzaj pliki binarne jak tekstowe\\
	\texttt{-E} korzystaj z ,,\emph{Extended Regular Expressions}'' (ERE) zamiast ,,\emph{Basic Regular Expressions}'' (BRE)\vspace{6pt}\\
	\texttt{-P} korzystaj z ,,\emph{Perl-compatible Regular Expressions}'' (PCRE) zamiast ,,\emph{Basic Regular Expressions}'' (BRE)\vspace{6pt}\\
	%
	\texttt{-r} rekursywnie przetwarzaj podane katalogi wyszukując w wszystkich znalezionych plikach\\
	\texttt{-R} jak -r, ale zawsze podąża za linkami symbolicznymi\\
	\texttt{--exclude="wyrażenie"} pomiń pliki pasujące do wyrażenie (może zawierać znaki uogólniające powłoki)\vspace{6pt}\\
	%
	\texttt{-l} wypisuje pliki z pasującymi liniami\\
	\texttt{-L} wypisuje pliki z bez pasujących linii

\vspace{13pt}\noindent Wyrażenia regularne\footnote{
	Podana składnia dotyczy ,,\emph{Extended Regular Expressions}'', przy BRE niektóre z znaków sterujących wymagają zabezpieczenia odwrotnym ukośnikiem.
} konstruuje się w oparciu o następujące znaki specjalne:
\vspace{-6pt}\begin{Verbatim}
.      - dowolny znak
[a-z]  - znak z zakresu
[^a-z] - znak z poza zakresu (aby mieć zakres z ^ należy dać go nie na początku)
^      - początek napisu/linii
$      - koniec napisu/linii

*      - dowolna ilość powtórzeń
?      - 0 lub jedno powtórzenie
+      - jedno lub więcej powtórzeń
{n,m}  - od n do m powtórzeń

()     - pod-wyrażenie (może być używane dla powtórzeń, a także referencji wstecznych)
\end{Verbatim}

Występowanie przełączników \texttt{-E} i \texttt{-P} wiąże  się z ewolucją składni wyrażeń regularnych przy jednoczesnym zachowywaniu kompatybilności z poprzednimi wersjami polecenia grep.
Jeżeli  coś było traktowane jako zwykły znak nie mogło się tak po prostu stać znakiem specjalnym i należało zastosować zabezpieczanie przy pomocy odwrotnym ukośnikiem lub wybór innego wariantu składni przy pomocy odpowiedniej opcji.
W efekcie \Verb#grep '^.\?$'# to to samo co \Verb#grep -E '^.?$'#, a \Verb#grep '^.?$'# to to samo co \Verb#grep -E '^.\?$'#.
% END: grep

\subsection{sed i inne narzędzia przetwarzania tekstów}
\begin{itemize}
	% BEGIN: sed
	\item \Verb{sed [opcje] [pliki]}
		edytuje plik zgodnie z podanymi poleceniami, przydatne opcje:\\
			\texttt{-e "polecenie"} - wykonuj na pliku polecenie (może wystąpić wielokrotnie celem podania wielu poleceń)\\
			\texttt{-f "plik"} - wczytaj polecenia z pliku plik\\
			\texttt{-E} - używaj rozszerzonych wyrażeń regularnych\\
			\texttt{-i} - modyfikuj podany plik zamiast wypisywać zmieniony na stdout\\
			\texttt{-n} - wyłącza domyślne wypisywanie linii, wypisanie musi być wykonane jawnie poleceniem \texttt{p}\\
		przydatne polecenia:\\
			\texttt{s@regexp@napis@[g]} - wyszukaj dopasowania do wyrażenia regularnego regexp i zastąp je przez napis, podanie opcji g powoduje zastępowanie wszystkich wystąpień a nie tylko pierwszego, znak \texttt{@} pełni rolę separatora i może zostać zamiast niego użyty inny znak\\
			\texttt{y@zbiór1@zbiór2@} - zastąp znaki z zbiór1 znakami odpowiadającymi im pod względem kolejności znakami z zbiór2, znak \texttt{@} pełni rolę separatora i może zostać zamiast niego użyty inny znak\\
		możliwe jest też m.in. adresowanie linii na których ma być wykonywana operacja, np: \texttt{0,/regexp/ s@regexp@napis@} wykona polecenie s na liniach od początku pliku do linii pasującej do wyrażenia regularnego regexp, czyli zastąpi tylko pierwsze wystąpienie w pliku\\
		Sed jest dość rozbudowanym narzędziem stanowiącym praktycznie coś na kształt interpretera języka programowania (o trochę dziwnej składni)
		i nie ogranicza się jedynie do prostych przypadków zaprezentowanych w tym skrypcie.
		Wiele przykładów użytecznych sktyptów sed'owych można znaleźć w spisie dostępnym pod adresem: \url{http://sed.sf.net/sed1line.txt}.
	% END: sed
	
	\vspace{6pt}
	
	% BEGIN: tail head
	\item \Verb{tail [opcje] [plik]}
		wyświetla ostatnie linie pliku, przydatne opcje:\\
			\texttt{-n x} określa że ma zostać wyświetlone x ostatnich linii\\
			\texttt{-f} uruchamia dopisywania (gdy do pliku zostaną dopisane nowe linie tail je wyświetli)
	\item \Verb{head [opcje] [plik]}
		wyświetla początkowe linie pliku, przydatne opcje:\\
			\texttt{-n x} określa że ma zostać wyświetlone x pierwszych linii
	% END: tail head
	
	\vspace{6pt}
	
	% BEGIN: diff patch
	\item \Verb{diff ścieżka1 ścieżka2}
		porównuje pliki lub katalogi (w przypadku tych drugich porównuje ze sobą pliki o takich samych nazwach oraz zgłasza fakt występowania pliku tylko w jednym z katalogów), przydatne opcje:\\
			\texttt{-r} rekursywnie przetwarzaj podane katalogi\\
			\texttt{-u} wypisuje różnice w formacie "unified"\\
			\texttt{-c} wypisuje różnice w formacie "context"
	\item \Verb{vimdiff ścieżka1 ścieżka2}
		porównuje pliki wyświetlając je jeden obok drugiego (podobnie jak \texttt{diff} z opcją \texttt{-y}), pozwalając jednak na edycję tych plików
	\item \Verb{patch}
	stosuje plik łaty (wynik diff'a) w celu zmodyfikowania plików, typowo:\\
		\texttt{patch -pn < plik.diff} co powoduje zastosowanie zmian opisanych w plik.diff na plikach w bieżącym katalogu,
		n określa ilość poziomów ścieżek podanych w pliku łaty które mają zostać zignorowane
	% END: diff patch
	% BEGIN: sort
	\item \Verb{sort [plik]}
		sortuje linie w wskazanym pliku, przydatne opcje:\\
			\texttt{-n} traktuj liczby jako wartości numeryczne a nie napisy\\
			\texttt{-i} ignoruj wielkość liter\\
			\texttt{-r} odwróć kolejność sortowania\\
			\texttt{-k n} sortuj wg kolumny n\\
			\texttt{-t sep} kolumny rozdzielane są przy pomocy separatora sep
	% END: sort
	
	\vspace{6pt}
	
	% BEGIN: cut
	\item \Verb{cut [opcje] [pliki]}
		wybiera z pliku zadany zestaw kolumn, przydatne opcje:\\
			\texttt{-f nnn} wypierz kolumny określone przez nnn (np. 1,3-4,6- oznacza kolumnę 1, kolumny od 3 do 4 i od 6, a -3 oznacza 3 pierwsze kolumny)\\
			\texttt{-d sep} kolumny rozdzielane są przy pomocy separatora sep (musi być pojedynczym jedno bajtowym znakiem, aby ominąć to ograniczenie należy skorzystać z awk)
	% END: cut
	% BEGIN: paste join comm uniq
	\item \Verb{paste}
		łączy (odpowiadające sobie pod względem numerów) linie z dwóch plików
	\item \Verb{join}
		łączy linie  z dwóch plików w oparciu o porównanie wskazanego pola
	\item \Verb{comm}
		porównuje dwa posortowane pliki pod względem unikalności linii (może wypisać wspólne lub występujące tylko w jednym z plików)
	\item \Verb{uniq}
		usuwa powtarzające się linie z posortowanego pliku, przydatne opcje:\\
			\texttt{-c} wypisz liczbę powtórzeń\\
			\texttt{-d} wypisz tylko linie z 2 lub więcej wystąpieniami\\
			\texttt{-u} wypisz tylko linie z 1 wystąpieniem
	% END: paste join comm uniq
\end{itemize}
