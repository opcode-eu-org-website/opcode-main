% Copyright (c) 2017-2020 Matematyka dla Ciekawych Świata (http://ciekawi.icm.edu.pl/)
% Copyright (c) 2017-2020 Robert Ryszard Paciorek <rrp@opcode.eu.org>
% 
% MIT License
% 
% Permission is hereby granted, free of charge, to any person obtaining a copy
% of this software and associated documentation files (the "Software"), to deal
% in the Software without restriction, including without limitation the rights
% to use, copy, modify, merge, publish, distribute, sublicense, and/or sell
% copies of the Software, and to permit persons to whom the Software is
% furnished to do so, subject to the following conditions:
% 
% The above copyright notice and this permission notice shall be included in all
% copies or substantial portions of the Software.
% 
% THE SOFTWARE IS PROVIDED "AS IS", WITHOUT WARRANTY OF ANY KIND, EXPRESS OR
% IMPLIED, INCLUDING BUT NOT LIMITED TO THE WARRANTIES OF MERCHANTABILITY,
% FITNESS FOR A PARTICULAR PURPOSE AND NONINFRINGEMENT. IN NO EVENT SHALL THE
% AUTHORS OR COPYRIGHT HOLDERS BE LIABLE FOR ANY CLAIM, DAMAGES OR OTHER
% LIABILITY, WHETHER IN AN ACTION OF CONTRACT, TORT OR OTHERWISE, ARISING FROM,
% OUT OF OR IN CONNECTION WITH THE SOFTWARE OR THE USE OR OTHER DEALINGS IN THE
% SOFTWARE.

\section{Inne polecenia}

Oprócz opisanych wcześniej najpopularniejszych / najistotniejszych poleceń istnieje wiele innych standardowych lub mniej standardowych (wymagających doinstalowania na wielu systemach) narzędzi linii poleceń. Poniżej wymienionych zostało kilka bardziej użytecznych przykładów.

Ponadto dowolny program w środowisku linuxowym (unixowym) może być uruchomiony z linii poleceń poprzez podanie jego nazwy (jeżeli jest w ścieżce wyszukiwania \Verb{$PARTH}) lub pełnej ścieżki do niego.
W bardzo wielu przypadkach takie uruchamianie pozwala przekazać do niego argument w postaci pliku do otwarcia lub inne opcje, czy nawet użycie programu normalnie pracującego z graficznym interfejsem użytkownika (takiego jak blender, inkscape, ...) w trybie nie interaktywnym (np. do automatycznej konwersji, itp.).

\begin{itemize}
	\item \Verb{date}
		data i czas, program ten potrafi także przeliczać datę i czas - np. \Verb$date -d @847103830 '+%Y-%m-%d %H:%M:%S'$, \Verb$date -d '1996-11-04 11:37:10' '+%s'$, \Verb$date -d '1996-11-04 11:37:10 +3week -2days'$
	\item \Verb{cal}
		kalendarz
	
	\item \Verb{wget} / \Verb{curl}
		pobieranie stron internetowych i plików
	
	\item \Verb{file}
		rozpoznaje typ pliku (w oparciu o zawartość)
	
	\item \Verb{convert}
		konwersje plików graficznych
	
	\vspace{6pt}
	
	\item \Verb{iconv}
		konwersje kodowań plików tekstowych
	\item \Verb{konwert}
		konwersje kodowań plików tekstowych – zarówno pomiędzy różnymi kodowaniami danego zbioru znaków, jak też pomiędzy kodowaniami nie pokrywającymi się czy też kodowaniami znaków 8 bitowych na mniejszej ilości bitów, na przykład://
			\Verb{konwert utf8-ascii} "inteligentnie" usunie znaki nie ascii z pliku kodowanego w utf-8 (np. znaczki z polskimi ogonkami zamieni na odpowiednie znaki ASCII bez tych ogonków);//
			\Verb{konwert qp-8bit} pozwoli zamienić kodowanie quoted printable na normalne 8 bitowe (rtf-8bit zrobi to z kodowaniem rtf'u)
		
	\item \Verb{mewencode} / \Verb{mewdecode}
		program (stanowiący część pakieu narzędzi dodatkowych dla kilenta pocztowego Mew) do obsługi kodowań mime (w tym Quoted-Printable, base64), m.in. zmienia kodowanie base64 na 8 bitowe
	\item \Verb{qprint}
		program do kodowania i dekodowania "Quoted-Printable"
	\item \Verb{base64}
		program do kodowania i dekodowania base64
	\item \Verb{strings}
		wypisuje sekwencje znaków drukowanych (określanie zawartości plików nietekstowych)
		
	\vspace{6pt}
	
	\item \Verb{command -v  komenda}
		zwraca wykonywaną ścieżkę / polecenie przy wykonywaniu \Verb{komenda}
\end{itemize}
