% Copyright (c) 2017-2020 Matematyka dla Ciekawych Świata (http://ciekawi.icm.edu.pl/)
% Copyright (c) 2017-2020 Robert Ryszard Paciorek <rrp@opcode.eu.org>
% 
% MIT License
% 
% Permission is hereby granted, free of charge, to any person obtaining a copy
% of this software and associated documentation files (the "Software"), to deal
% in the Software without restriction, including without limitation the rights
% to use, copy, modify, merge, publish, distribute, sublicense, and/or sell
% copies of the Software, and to permit persons to whom the Software is
% furnished to do so, subject to the following conditions:
% 
% The above copyright notice and this permission notice shall be included in all
% copies or substantial portions of the Software.
% 
% THE SOFTWARE IS PROVIDED "AS IS", WITHOUT WARRANTY OF ANY KIND, EXPRESS OR
% IMPLIED, INCLUDING BUT NOT LIMITED TO THE WARRANTIES OF MERCHANTABILITY,
% FITNESS FOR A PARTICULAR PURPOSE AND NONINFRINGEMENT. IN NO EVENT SHALL THE
% AUTHORS OR COPYRIGHT HOLDERS BE LIABLE FOR ANY CLAIM, DAMAGES OR OTHER
% LIABILITY, WHETHER IN AN ACTION OF CONTRACT, TORT OR OTHERWISE, ARISING FROM,
% OUT OF OR IN CONNECTION WITH THE SOFTWARE OR THE USE OR OTHER DEALINGS IN THE
% SOFTWARE.

% BEGIN: unix intro
\section{Praca w terminalu}

Komputer potrafi jedynie wykonywać jakiś wcześniej zaprogramowany ciąg instrukcji.
Każde wydane przez użytkownika polecenie wiąże się z uruchomieniem takiego ciągu instrukcji (programu komputerowego lub jakieś funkcji w ramach niego).

Podstawowym sposobem wydawania poleceń w systemach typu Unix jest wpisywanie ich w terminalu.
Terminal może pracować w trybie tekstowym lub może być uruchomiony (jako tzw. emulator terminala) w trybie graficznym.

\subsection{Powłoka}

Wprowadzane polecenia interpretowane są przez działający w terminalu program nazywany powłoką (interpreterem poleceń).
W terminalu mogą być uruchamiane kolejne (takie same lub różne) interpretery poleceń.
Różne interpretery korzystają z różnych składni oraz często różnią się znakiem zachęty (czyli wypisanym tekstem poprzedzającym wprowadzane polecenia).

\subsubsection{bash}
Bash jest chyba najpopularniejszą powłoką (interpreterem poleceń) systemową w środowiskach linuksowych.
Jest zgodny ze składnią z sh, zapewnia m.in. obsługę zmiennych (zasadniczo napisowych) oraz omówionych w dalszej części skryptu znaków uogólniających.

\begin{ProTip}[breakable]{Edycja i historia linii poleceń}
Bash, podobnie jak wiele innych interpreterów poleceń (np. Python) umożliwia edycję linii poleceń oraz korzystanie z jej historii.\footnote{
	Bash korzysta w tym celu z popularnej biblioteki \textit{GNU Readline} (więcej na jej temat: \url{https://en.wikipedia.org/wiki/GNU\_Readline}).
	Biblioteka ta może zostać użyta w kodzie programu do obsługi wejścia, ale może zostać dodana do także zewnętrznie przy pomocy np. \Verb$rlwrap$.
}
Poruszanie się po historii linii poleceń możliwe jest strzałkami góra/dół, wyszukiwanie w historii przy pomocy Ctrl+R.
Dostępne jest też polecenie wbudowane \texttt{history} umożliwiające wypisanie całej zapamiętanej historii oraz zarządzanie nią.

Bash pozwala także na dopełnianie nazw poleceń i ścieżek (a po odpowiedniej konfiguracji – pakiet \textit{bash-completion} – także innych argumentów poleceń) przy pomocy tabulatora.
\end{ProTip}

\subsubsection{screen i tmux}
\texttt{screen} i \texttt{tmux} są tzw. multiplexerami terminala - pozwala na uzyskanie wielu okien konsoli (także np. wyświetlanych jedno obok drugiego) na pojedynczym terminalu.
Ponadto pozwalają na odłączanie i podłączanie sesji, co pozwala na łatwe pozostawienie działającego programu po wylogowaniu i powrót do niego później.

\subsection{Komendy}

Unixowe komendy (czyli polecenia rozumiane przez bash lub inny interpreter zgodny z sh) składają się z nazwy polecenia oraz opcji i argumentów.
Nazwą polecenia może być nazwa funkcji wbudowanej, nazwa programu (znajdującego się w ścieżce wyszukiwania programów) lub pełna ścieżka do programu.
Po nazwie polecenia mogą występować opcje i/lub argumenty. Są one oddzielane od nazwy polecenia i od siebie przy pomocy spacji\footnote{
	zasadniczo dowolnego ciągu białych znaków: spacji, tabulatorów, ,,escapowanych'' nowych linii.
}.
Nie ma silnego rozróżnienia opcji od argumentów, typowo stosowaną konwencją jest rozpoczynanie opcji od pojedynczego myślnika (opcje krótkie - jednoliterowe) lub dwóch myślników (opcje długie).
	W przypadku stosowania tej konwencji po pojedynczym myślniku może występować kilka bezargumentowych opcji jednoliterowych.
	Typowo argumenty opcji oddzielane są od nich spacją (w przypadku opcji krótkich) lub znakiem równości (w przypadku opcji długich).
Jeżeli któryś z składników komendy (np. argument) zawiera spacje należy je zabezpieczyć przy pomocy odwrotnego ukośnika lub ujęcia zawierającego je napisu w apostrofy lub cudzysłowia.

\subsection{Uzyskiwanie pomocy}

Informację na temat działania danej komendy oraz jej opcji można uzyskać w wbudowanym systemie pomocy przy pomocy poleceń \Verb{man} lub \Verb{info} / \Verb{pinfo}.
Większość poleceń obsługuje także opcje \Verb{--help} lub \Verb{-h}, które wyświetlają informację na temat ich użycia.

\pagebreak[2]\begin{ProTip}{Notacja}
Zarówno w tekstach pomocy jak i w tym dokumencie stosowana jest konwencja polegająca na oznaczaniu opcjonalnych argumentów poprzez umieszczanie ich w nawiasach kwadartowych (jeżeli podajemy ten argument do komendy nie obejmujemy go już tymi nawiasami) oraz rozdzielaniu alternatywnych opcji przy pomocy~\Verb{|}. Np. \Verb{a [b] c|d} oznacza iż polecenie \Verb{a} wymaga argumentu postaci \Verb{c} albo \Verb{d}, który może być poprzedzony argumentem \Verb{b}.
\end{ProTip}

\subsection{more i less}

Jeżeli wynik jakiejś komendy nie mieści się na ekranie do jego obejrzenia możemy użyć poleceń \Verb{more} lub \Verb{less}. Są to programy umożliwiające przeglądanie tekstu ekran po ekranie.
\Verb{less} posiada większe możliwości od more (w szczególności posiada możliwość przeglądanie dokumentu w tył)\footnote{
wybrane przydatne opcje:
	\Verb{-X} nie czyści ekranu przy wychodzeniu z less'a (całość historii wyświetlania pliku pozostaje w historii terminala)
	\Verb{-F} automatycznie kończy gdy wyświetlany tekst mieści się na jednym ekranie
}. Programy te kończą się po wciśnięciu klawisza \Verb{q}. \Verb{less} umożliwia także wyszukiwanie -- klawisz \Verb{/} pozwala na wprowadzenie szukanej frazy, a \Verb{n} na wyszukanie kolejnego wystąpienia.
Programy te umożliwiają też wyświetlanie wskazanych jako argumenty plików.

\subsection{Przekierowania}

Typowo program posiada trzy strumienie danych: jeden wejściowy (stdin) i dwa wyjściowe (stdout i stderr). Standardowe wyjście możemy przekierować na standardowe wejście innego programu przy pomocy~\Verb{|}, np:
\begin{Verbatim}
ls --help | less
\end{Verbatim}
Konstrukcja ta przekieruje wynik komendy \Verb{ls} uruchomionej z opcją \Verb{--help} do komendy \Verb{less}.

Możemy także przekierować standardowe wyjście do pliku (przy pomocy~\Verb{>} lub~\Verb{>>}, gdy chcemy dopisywać do pliku) lub pobrać standardowe wejście z pliku (przy pomocy \Verb{<}).
\Verb{2>} pozwala na przekierowanie standardowego wyjścia błędu do pliku.

Jeżeli zachodzi potrzeba połączenia obu strumieni możemy użyć \Verb{2>&1} w celu przekierowania strumienia drugiego do pierwszego.
Następnie możemy użyć ~\Verb{|} aby przekierować połączony strumień do następnej komendy.
Jeżeli chcemy przekierować go do pliku połączenie strumieni powinno mieć miejsce po przekierowaniu pierwszego z nich do pliku, np.:
\\* \\* \vspace{-2\baselineskip} % nopagebreak here !
\begin{Verbatim}[]
ls . NieIstniejącyPlik >log.txt 2>&1
\end{Verbatim}
Bash pozwala użyć \Verb{>&} i \Verb{|&}, które przekierowują oba strumienie odpowiednio do pliku lub standardowego wejścia innego polecenia, ale jest to rozszerzenie wykraczające poza standardową składnię sh.
% END: unix intro

\subsection{Kod powrotu polecenia oraz łączenie poleceń}

Każde uruchamiane polecenie po zakończeniu działania zwraca liczbowy kod powrotu (w przypadku programów w C jest to wartość zwracana z funkcji \Verb$main$).
Zero oznacza że polecenie zakończyło się sukcesem (np. znaleziono szukane pliki), wartość nie zerowa że zakończyło się porażką (np. nie ma pasujących plików) lub błędem (np. składnia wprowadzonego polecenia była niepoprawna).

Polecenia mogą być łączone na różne sposoby – z wykorzystaniem tej informacji lub nie:
\begin{itemize}
	\item \Verb$a && b$ – polecenie b wykona się gdy a zakończyło się sukcesem (zwróciło kod 0)
	\item \Verb$a || b$ – polecenie b wykona się gdy a zakończyło się porażką lub błędem (zwróciło kod różny od~0)
	\item \Verb$a ; b$ – polecenie b po zakończeniu polecenia a (bez względu na jego kod powrotu)
	\item \Verb$a & b$ – polecenie b będzie wykonywane równocześnie z a (dokładniej polecenie a zostanie uruchomione w tle, a na terminal zajmie polecenie b)
\end{itemize}

Spacje w powyższych konstrukcjach są opcjonalne.
Średnik i pojedynczy \Verb$&$ mogą być dodane do polecenia także gdy nie ma kolejnego w ciągu:
\begin{itemize}
	\item \Verb$a&$ uruchomi polecenie a w tle i odda linię poleceń,
	\item \Verb$a;$ uruchomi polecenie a (dokładnie tak samo jakby nie było tego średnika).
\end{itemize}

% BEGIN: katalog roboczy
\subsection{Katalog roboczy}

System plików ma strukturę hierarchiczną (drzewiastą) i rozpoczyna się w korzeniu oznaczanym ukośnikiem: \texttt{/}.
Możliwe jest wyrażanie wszystkich ścieżek od korzenia, jednak nie jest to zbytnio wygodne.
Interpreter poleceń taki jak \textit{bash} potrafi znajdować się gdzieś w tej strukturze plików i miejsce to nazywane jest bieżącym katalogiem roboczym (\textit{Present Working Directory}).
Względem niego będą wyrażane ścieżki nie zaczynające się od korzenia, może być też oznaczony jawnie przy pomocy pojedynczej kropki.

\begin{itemize}
	\item \Verb{cd [ścieżka]}
		zmiana bieżącego katalogu,\\
			warto zauważyć, iż katalogi w ścieżce oddzielamy ukośnikami \texttt{/}, bieżący katalog oznaczamy kropką \texttt{.}, nadrzędny oznaczamy dwiema kropkami \texttt{..},\\
			ścieżki zaczynające się od ukośnika \texttt{/} oznaczają \emph{ścieżki bezwzględne} (od korzenia systemu plików), pozostałe oznaczają \emph{ścieżkę względną} (wyrażoną względem bierzącego katalogu),\\
			katalog domowy oznacza się tyldą\ \ \texttt{\~}
		
	\item \Verb{pwd}
		wyświetla ścieżkę do bieżącego katalogu
\end{itemize}
% END: katalog roboczy
