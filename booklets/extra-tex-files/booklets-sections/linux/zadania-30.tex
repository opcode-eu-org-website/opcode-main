% Copyright (c) 2017-2020 Matematyka dla Ciekawych Świata (http://ciekawi.icm.edu.pl/)
% Copyright (c) 2017-2020 Robert Ryszard Paciorek <rrp@opcode.eu.org>
% 
% MIT License
% 
% Permission is hereby granted, free of charge, to any person obtaining a copy
% of this software and associated documentation files (the "Software"), to deal
% in the Software without restriction, including without limitation the rights
% to use, copy, modify, merge, publish, distribute, sublicense, and/or sell
% copies of the Software, and to permit persons to whom the Software is
% furnished to do so, subject to the following conditions:
% 
% The above copyright notice and this permission notice shall be included in all
% copies or substantial portions of the Software.
% 
% THE SOFTWARE IS PROVIDED "AS IS", WITHOUT WARRANTY OF ANY KIND, EXPRESS OR
% IMPLIED, INCLUDING BUT NOT LIMITED TO THE WARRANTIES OF MERCHANTABILITY,
% FITNESS FOR A PARTICULAR PURPOSE AND NONINFRINGEMENT. IN NO EVENT SHALL THE
% AUTHORS OR COPYRIGHT HOLDERS BE LIABLE FOR ANY CLAIM, DAMAGES OR OTHER
% LIABILITY, WHETHER IN AN ACTION OF CONTRACT, TORT OR OTHERWISE, ARISING FROM,
% OUT OF OR IN CONNECTION WITH THE SOFTWARE OR THE USE OR OTHER DEALINGS IN THE
% SOFTWARE.

\IfStrEq{\dbEntryID}{}{
	\insertZadanie{\currfilepath}{upper_false}{}
	\insertZadanie{\currfilepath}{ls_nazwy}{}
	\insertZadanie{\currfilepath}{wyszukaj_napis}{}
}


% napisy nie wymagające awk ani programowania w bash

\dbEntryBegin{upper_false}\if1\dbEntryCheckResults
Wyświetl plik \texttt{/etc/passwd} z zastąpionym \texttt{false} przez \texttt{FALSE}.
\fi
\dbEntryBegin{upper_false-rozwiazanie}\if1\dbEntryCheckResults
\begin{CodeFrame*}[bash]{}
sed -e "s#false#FALSE#g" /etc/passwd
\end{CodeFrame*}
\fi


\dbEntryBegin{ls_nazwy}\if1\dbEntryCheckResults
Polecenie \Verb#ls -1d /etc/p*# wyświetla pełne ścieżki do plików i katalogów, znajdujących się (bezpośrednio) w \texttt{/etc/}, których nazwa zaczyna się literą \texttt{p}.
Przekieruj standardowe wyjście tej komendy do takiego ciągu poleceń aby uzyskać tylko nazwy tych plików (bez ścieżki).
\fi
\dbEntryBegin{ls_nazwy-rozwiazanie}\if1\dbEntryCheckResults
\begin{CodeFrame*}[bash]{}
ls -1d /etc/p* | cut -f3 -d/
\end{CodeFrame*}

\begin{CodeFrame*}[bash]{}
ls -1d /etc/p* | sed -e 's#^/etc/##'
\end{CodeFrame*}

\noindent Zwróć uwagę na:
\begin{itemize}
\item przekierowanie standardowego wyjścia polecenia ls do komendy modyfikującej strumieniowo napisy
\item uzycie poleceń cut lub sed w roli takiej komendy
\end{itemize}
\fi


\dbEntryBegin{wyszukaj_napis}\if1\dbEntryCheckResults
Napisz polecenie które wyszuka wszystkie wystąpienia napisu \Verb{nameserver} w plikach znajdujących się w katalogu \Verb{/etc} (wraz z jego podkatalogami).
\fi
\dbEntryBegin{wyszukaj_napis-rozwiazanie}\if1\dbEntryCheckResults
\begin{CodeFrame*}[bash]{}
grep -r nameserver /etc/
\end{CodeFrame*}

\noindent Zwróć uwagę że:
\begin{itemize}
\item do wyszukiwania po zawartości używamy polecenia grep z opcją -r, a nie find
\item można by użyć find z warunkiem exec wykonującym grep na danym pliku:\\
	\shell{find /etc/ -exec grep nameserver \{\} \;}\\
	ale gdy nie mamy potrzeby stosować innych warunków (dla których potrzebny byłby find) jest to przerostem formy (zarówno w zapisie jak i koszcie obliczeniowym wykonania – find dla każdego pliku musi zrobić fork i exec na odpowiedniego grep'a)
\end{itemize}
\fi
