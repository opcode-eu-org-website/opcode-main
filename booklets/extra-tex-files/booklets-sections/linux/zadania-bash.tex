% Copyright (c) 2017-2020 Matematyka dla Ciekawych Świata (http://ciekawi.icm.edu.pl/)
% Copyright (c) 2017-2020 Robert Ryszard Paciorek <rrp@opcode.eu.org>
% 
% MIT License
% 
% Permission is hereby granted, free of charge, to any person obtaining a copy
% of this software and associated documentation files (the "Software"), to deal
% in the Software without restriction, including without limitation the rights
% to use, copy, modify, merge, publish, distribute, sublicense, and/or sell
% copies of the Software, and to permit persons to whom the Software is
% furnished to do so, subject to the following conditions:
% 
% The above copyright notice and this permission notice shall be included in all
% copies or substantial portions of the Software.
% 
% THE SOFTWARE IS PROVIDED "AS IS", WITHOUT WARRANTY OF ANY KIND, EXPRESS OR
% IMPLIED, INCLUDING BUT NOT LIMITED TO THE WARRANTIES OF MERCHANTABILITY,
% FITNESS FOR A PARTICULAR PURPOSE AND NONINFRINGEMENT. IN NO EVENT SHALL THE
% AUTHORS OR COPYRIGHT HOLDERS BE LIABLE FOR ANY CLAIM, DAMAGES OR OTHER
% LIABILITY, WHETHER IN AN ACTION OF CONTRACT, TORT OR OTHERWISE, ARISING FROM,
% OUT OF OR IN CONNECTION WITH THE SOFTWARE OR THE USE OR OTHER DEALINGS IN THE
% SOFTWARE.

\IfStrEq{\dbEntryID}{}{
	\insertZadanie{\currfilepath}{petla_linki_html}{}
	\insertZadanie{\currfilepath}{warunek_istnienie_pliku}{}
	\insertZadanie{\currfilepath}{funkcja_n_razy_napis}{}
	\insertZadanie{\currfilepath}{funkcja_liczba_kotow}{}
}

% BEGIN: podstawy programowania w bashu - zadania
\dbEntryBegin{petla_linki_html}\if1\dbEntryCheckResults
Napisz pętle, która wypisze wszystkie pliki nieukryte z bieżącego katalogu w postaci linków HTML, czyli:
dla pliku o nazwie \Verb{ABC} powinna wypisać \Verb{<a href="ABC">ABC</a>}. Przedstaw zarówno rozwiązanie z użyciem pętli \Verb{for}, jak i pętli \Verb{while}.
\fi

\dbEntryBegin{warunek_istnienie_pliku}\if1\dbEntryCheckResults
Napisz warunek, który sprawdzi czy \Verb{/tmp/abc} istnieje i jest katalogiem.
\fi

\dbEntryBegin{funkcja_n_razy_napis}\if1\dbEntryCheckResults
Napisać funkcję przyjmującą dwa argumenty - liczbę i napis; funkcja ma wypisać napis tyle razy ile wynosi podana liczba.
\fi

\dbEntryBegin{funkcja_liczba_kotow}\if1\dbEntryCheckResults
Napisać funkcję przyjmującą jeden argument - liczbę kotów i wypisującą:
\begin{itemize}
	\item "Ala ma kota" dla ilości kotów równej 1
	\item "Ala ma x koty" lub "Ala ma x kotów" gdzie dobrana jest poprawna forma, a pod x podstawiona podana w argumencie ilość kotów.
\end{itemize}
Dla uproszczenia należy założyć że podana ilość kotów jest w zakresie od 1 do 9.
\fi
% END: podstawy programowania w bashu - zadania
