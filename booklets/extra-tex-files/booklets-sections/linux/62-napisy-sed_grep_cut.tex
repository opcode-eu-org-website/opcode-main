% Copyright (c) 2017-2020 Matematyka dla Ciekawych Świata (http://ciekawi.icm.edu.pl/)
% Copyright (c) 2017-2020 Robert Ryszard Paciorek <rrp@opcode.eu.org>
% 
% MIT License
% 
% Permission is hereby granted, free of charge, to any person obtaining a copy
% of this software and associated documentation files (the "Software"), to deal
% in the Software without restriction, including without limitation the rights
% to use, copy, modify, merge, publish, distribute, sublicense, and/or sell
% copies of the Software, and to permit persons to whom the Software is
% furnished to do so, subject to the following conditions:
% 
% The above copyright notice and this permission notice shall be included in all
% copies or substantial portions of the Software.
% 
% THE SOFTWARE IS PROVIDED "AS IS", WITHOUT WARRANTY OF ANY KIND, EXPRESS OR
% IMPLIED, INCLUDING BUT NOT LIMITED TO THE WARRANTIES OF MERCHANTABILITY,
% FITNESS FOR A PARTICULAR PURPOSE AND NONINFRINGEMENT. IN NO EVENT SHALL THE
% AUTHORS OR COPYRIGHT HOLDERS BE LIABLE FOR ANY CLAIM, DAMAGES OR OTHER
% LIABILITY, WHETHER IN AN ACTION OF CONTRACT, TORT OR OTHERWISE, ARISING FROM,
% OUT OF OR IN CONNECTION WITH THE SOFTWARE OR THE USE OR OTHER DEALINGS IN THE
% SOFTWARE.

% BEGIN: grep cut sed
\subsection{grep, cut, sed, ...}

Jako że większość operacji wykonywanych w powłoce takiej jak bash wiąże się z uruchamianiem zewnętrznych programów,
to także przetwarzanie napisów może być realizowane w ten sposób.
Opiera się na tym jedno z podejść do obsługi napisów w bashu,
którym jest korzystanie z standardowych komend POSIX, takich jak grep, cut, sed.

\begin{CodeFrame*}[bash]{}
# obliczanie długości napisu w znakach, w bajtach i ilości słów w napisie
echo -n "aąbcć 123" | wc -m
echo -n "aąbcć 123" | wc -c
echo -n "aąbcć 123" | wc -w

# obliczanie ilości linii (dokładniej ilości znaków nowej linii)
wc -l < /etc/passwd

# wypisanie 5 pola (rozdzielanego :) z pliku /etc/passwd  z eliminacją
# pustych linii oraz linii złożonych tylko ze spacji i przecinków
cut -f5 -d: /etc/passwd | grep -v '^[ ,]*$'
# komenda cut wybiera wskazane pola, opcja -d określa separator
\end{CodeFrame*}

Inną bardzo przydatną komendą jest sed pozwala ona m.in na zastępowanie
wyszukiwanego na podstawie wyrażenia regularnego tekstu innym:

\begin{CodeFrame*}[bash]{}
echo "aa bb cc bb dd bb ee" | sed -e 's@\([bc]\+\) \([bc]\+\)@X-\2-X@g'
\end{CodeFrame*}

Sedowe polecenie s przyjmuje 3 argumenty (oddzielane mogą być dowolnym znakiem który wystąpi za~\Verb@s@),
pierwszy to wyszukiwane wyrażenie, drugi tekst którym ma zostać zastąpione,
a trzeci gdy jest \Verb@g@ to powoduje zastępowanie wszystkich wystąpień a nie tylko pierwszego.

Należy zwrócić uwagę na różnicę w składni wyrażenia regularnego polegającą na poprzedzaniu
\Verb@(@, \Verb@)@ i \Verb@+@ odwrotnym ukośnikiem aby miały znaczenie specjalne
(jeżeli nie chcemy tego robić możemy włączyć obsługę ERE w sed poprzez opcję \Verb@-E@).

Innymi przydatnymi komendami przetwarzającymi (specyficznej postaci) napisy są polecaenia \Verb@basename@ i \Verb@dirname@.
Służą one do uzyskania nazwy najgłębszego elementu ścieżki oraz ścieżki bez tego najglębszego elementu. Zobacz wynik działania:

\begin{CodeFrame*}[bash]{}
basename /proc/sys/net/core/
dirname /proc/sys/net/core/
\end{CodeFrame*}
% END: grep cut sed

\insertZadanie{booklets-sections/linux/zadania-60-bash.tex}{passwd_warunek_na_uid_noawk}{}
\insertZadanie{booklets-sections/linux/zadania-60-bash.tex}{rekurecyjne_wyszukaj_i_zastap}{}
