% Copyright (c) 2017-2020 Matematyka dla Ciekawych Świata (http://ciekawi.icm.edu.pl/)
% Copyright (c) 2017-2020 Robert Ryszard Paciorek <rrp@opcode.eu.org>
% 
% MIT License
% 
% Permission is hereby granted, free of charge, to any person obtaining a copy
% of this software and associated documentation files (the "Software"), to deal
% in the Software without restriction, including without limitation the rights
% to use, copy, modify, merge, publish, distribute, sublicense, and/or sell
% copies of the Software, and to permit persons to whom the Software is
% furnished to do so, subject to the following conditions:
% 
% The above copyright notice and this permission notice shall be included in all
% copies or substantial portions of the Software.
% 
% THE SOFTWARE IS PROVIDED "AS IS", WITHOUT WARRANTY OF ANY KIND, EXPRESS OR
% IMPLIED, INCLUDING BUT NOT LIMITED TO THE WARRANTIES OF MERCHANTABILITY,
% FITNESS FOR A PARTICULAR PURPOSE AND NONINFRINGEMENT. IN NO EVENT SHALL THE
% AUTHORS OR COPYRIGHT HOLDERS BE LIABLE FOR ANY CLAIM, DAMAGES OR OTHER
% LIABILITY, WHETHER IN AN ACTION OF CONTRACT, TORT OR OTHERWISE, ARISING FROM,
% OUT OF OR IN CONNECTION WITH THE SOFTWARE OR THE USE OR OTHER DEALINGS IN THE
% SOFTWARE.

\IfStrEq{\dbEntryID}{}{
	\insertZadanie{\currfilepath}{petla_linki_html}{}
	\insertZadanie{\currfilepath}{warunek_istnienie_pliku}{}
	\insertZadanie{\currfilepath}{funkcja_n_razy_napis}{}
	\insertZadanie{\currfilepath}{funkcja_liczba_kotow}{}
	\insertZadanie{\currfilepath}{zmiana_rozszerzenia}{}
	\insertZadanie{\currfilepath}{passwd_warunek_na_uid_noawk}{}
	\insertZadanie{\currfilepath}{rekurecyjne_wyszukaj_i_zastap}{}
}


\dbEntryBegin{petla_linki_html}\if1\dbEntryCheckResults
Napisz pętle, która wypisze wszystkie pliki nieukryte z bieżącego katalogu w postaci linków HTML, czyli:
dla pliku o nazwie \Verb{ABC} powinna wypisać \Verb{<a href="ABC">ABC</a>}. Przedstaw zarówno rozwiązanie z użyciem pętli \Verb{for}, jak i pętli \Verb{while}.
\fi
\dbEntryBegin{petla_linki_html-rozwiazanie}\if1\dbEntryCheckResults
\begin{CodeFrame*}[bash]{}
for f in *; do echo "<a href=\"$f\">$f</a>"; done
ls | while read f; do echo "<a href=\"$f\">$f</a>"; done
\end{CodeFrame*}

\noindent Zwróć uwagę że:
\begin{itemize}
\item rozwiązania te różnią się jedynie sposobem uzyskania listy plików dla której mają wypisać linki
\item pętla for w każdym obiegu podstawia pod f kolejny element z listy nazw dopasowanych do gwiazdki (czyli wszystkich plików nieukrytych)
\item pętla while listę plików dostaje na standardowe wejście (jeden plik na linię) i wczytuje każdą kolejną linię (czyli kolejną nazwę pliku) stosując komendę read – jest to bardzo standardowe rozwiązanie do przetwarzania standardowego wejścia linia po linii
\item w obu wypadkach używamy takiego echo z napisem w podwójnych cudzysłowach (aby móc umieścić w nim zmienną), cudzysłowa które mają być wypisane zabezpieczamy odwrotnym ukośnikiem
\item wypisywanie można rozwiązać na kilka innych sposobów np.: \shell{echo '<a href="'"$f"'">'"$f"'</a>'} – zadziała tak samo, ale wydaje się być to mniej czytelne
\end{itemize}
\fi


\dbEntryBegin{warunek_istnienie_pliku}\if1\dbEntryCheckResults
Napisz warunek, który sprawdzi czy \Verb{/tmp/abc} istnieje i jest katalogiem.
\fi
\dbEntryBegin{warunek_istnienie_pliku-rozwiazanie}\if1\dbEntryCheckResults
\begin{CodeFrame*}[bash]{}
if [ -d /tmp/abc ]; then echo "jest katalogiem"; else echo "nie"; fi
\end{CodeFrame*}

lub krócej:

\begin{CodeFrame*}[bash]{}
[ -d /tmp/abc ] && echo "jest katalogiem";
\end{CodeFrame*}

\noindent Zwróć uwagę że:
\begin{itemize}
\item w celu warunkowego wypisania jakiejś informacji możemy użyć zarówno konstrukcji if, jak też łączenia poleceń,
	jednak w przypadku większej ilości poleceń objętych warunkiem konstrukcja z if jest bardziej czytelna
\item sprawdzenie czy podana ścieżka istnieje i jest katalogiem odbywa się przy pomocy opcji -d, informacja ta celowo nie była podana w treści skryptu – należało to sprawdzić w dokumentacji systemowej (\shell{man test}). \strong{Czytanie dokumentacji jest ważne!}
\end{itemize}
\fi


\dbEntryBegin{funkcja_n_razy_napis}\if1\dbEntryCheckResults
Napisać funkcję przyjmującą dwa argumenty - liczbę i napis; funkcja ma wypisać napis tyle razy ile wynosi podana liczba.
\fi
\dbEntryBegin{funkcja_n_razy_napis-rozwiazanie}\if1\dbEntryCheckResults
\begin{CodeFrame*}[bash]{}
f() { for i in `seq 1 $1`; do echo $2; done; }
\end{CodeFrame*}

\noindent Zwróć uwagę że:
\begin{itemize}
\item w nawiasach po nazwie funkcje nie piszemy nic na temat jej argumentów - one są puste
\item do argumentów odwołujemy się poprzez dolar numer argumentu
\item do n krotnego powtórzenia czynności używamy pętli for która iteruje po liście liczb zwracanej przez seq
\item seq objęta jest znakami ` oznaczającymi że należy wykonać podany w nich kod i podstawić w to miejsce jego standardowe wyjście, nie należy ich mylić z apostrofami używanymi do napisów (')
\item spacja po { oraz średnik (lub nowa linia) przed } są istotne składniowo
\end{itemize}
\fi


\dbEntryBegin{funkcja_liczba_kotow}\if1\dbEntryCheckResults
Napisać funkcję przyjmującą jeden argument - liczbę kotów i wypisującą:
\begin{itemize}
	\item "Ala ma kota" dla ilości kotów równej 1
	\item "Ala ma x koty" lub "Ala ma x kotów" gdzie dobrana jest poprawna forma, a pod x podstawiona podana w argumencie ilość kotów.
\end{itemize}
Dla uproszczenia należy założyć że podana ilość kotów jest w zakresie od 1 do 9.
\fi
\dbEntryBegin{funkcja_liczba_kotow-rozwiazanie}\if1\dbEntryCheckResults
\begin{CodeFrame*}[bash]{}
koty() {
	case $1 in
		1) echo "Ala ma kota";;
		2|3|4) echo "Ala ma $1 koty";;
		*) echo "Ala ma $1 kotów";;
	esac
}
\end{CodeFrame*}

\begin{CodeFrame*}[bash]{}
koty() {
 if [ $1 -eq 1 ]; then
  echo "Ala ma kota"
 elif [ $1 -lt 5 ]; then   # możnaby też elif [ $1 -eq 2 -o $1 -eq 3 -o $1 -eq 4 ]; then
  echo "Ala ma $1 koty"
 else
  echo "Ala ma $1 kotów"
 fi
}
\end{CodeFrame*}
\fi


\dbEntryBegin{zmiana_rozszerzenia}\if1\dbEntryCheckResults
Napisz polecenie które dla wszystkich plików z rozszerzeniem \Verb{.TXT}  w bierzącym katalogu (bez podkatalogów) dokona zmiany ich nazwy zmieniając rozszerzenie na \Verb{.txt}, zachowując podstawową część nazwy bez modyfikacji.
W rozwiązaniu nie korzystamy z polecenia \Verb{rename}.
\fi
\dbEntryBegin{zmiana_rozszerzenia-rozwiazanie}\if1\dbEntryCheckResults
\begin{CodeFrame*}[bash]{}
for f in *.TXT; do mv "$f" "${f%.TXT}.txt"; done

# albo:

for i in *.TXT; do mv "$i" "$(basename $i .TXT).txt"; done
\end{CodeFrame*}

Do iteracji po plikach używamy pętli for, możemy jej od razu podać wyrażenie określające po jakich plikach ma przebiegać.
Nie potrzebujemy tutaj konstrukcji typu \Verb#for f in `ls *.TXT`#, aczkolwiek w bardziej złożonych przypadkach może ona być użyteczna.
Na przykład gdy chcemy pominąć pliki w nazwie których występuje XYZ: \Verb#for f in `ls *.TXT | grep -v XYZ`#
(pod warunkiem że nie mamy plików ze spacjami w nazwie - wtedy lepiej zrobić \Verb#for f in *.TXT# i filtrować daną nazwę w każdym obiegu pętli lub użyć potoku \Verb#ls | grep | while read ...#).

W takim przypadku \Verb#${f%.TXT}.txt# jest lepsze od \Verb#${f/.TXT/.txt}# lub \Verb#${f//.TXT/.txt}# gdyż usuwa tylko ciąg \Verb#.TXT# wystepujący na końcu napisu.
\Verb#aa.TXT.xyz.TXT# jest poprawną nazwą pliku i zgodnie z warunkami zadania powinna być zmieniania na \Verb#aa.TXT.xyz.txt#, a nie na \Verb#aa.txt.xyz.TXT# lub \Verb#aa.txt.xyz.txt#.
\fi


\dbEntryBegin{passwd_warunek_na_uid_noawk}\if1\dbEntryCheckResults
Wyświetl z /etc/passwd linie w których UID (3 pole) ma warość >= 1000 nie korzystając z AWK.
Jeżeli masz pomysł przedstaw więcej niż jedno rozwiązanie.
\fi
\dbEntryBegin{passwd_warunek_na_uid_noawk-rozwiazanie}\if1\dbEntryCheckResults

\begin{CodeFrame*}[bash]{}
while read l; do
	x=$(echo $l| cut -f3 -d:);
	[ $x -ge 1000 ] && echo $l;
done < /etc/passwd
\end{CodeFrame*}

\begin{CodeFrame*}[bash]{}
IFS=":"; while read p1 p2 p3 pX; do
	if [ $p3 -ge 1000 ]; then echo "$p1:$p2:$p3:$pX"; fi
done < /etc/passwd; unset IFS
\end{CodeFrame*}

\begin{CodeFrame*}[bash]{}
# używając cut i paste tworzymy plik złozony z pole 3 tabulator cała linia
cut -f3 -d: /etc/passwd | paste /dev/stdin /etc/passwd | while read uid linia; do
	# następnie czytamy go w petli while-read jako dwa pola (przed i po tabulatorze)
	[ $uid -ge 1000 ] && echo $linia;
done
\end{CodeFrame*}

\begin{CodeFrame*}[bash]{}
egrep '^([^:]*:){2}[0-9]{4}' /etc/passwd
\end{CodeFrame*}

Warto zwrócić uwagę że rozwiązanie z awk (\Verb#awk -F: '$3>=1000 {print $0}' /etc/passwd#) jest znacznie prostsze.
\fi


\dbEntryBegin{rekurecyjne_wyszukaj_i_zastap}\if1\dbEntryCheckResults
Napisz funkcję która przyjmuje dwa argumenty - napis wyszukiwany i napis go zastępujący oraz dokonuje rekurencyjnego wyszukania i zamiany tych napisów w wszystkich plikach w bieżącym katalogu.

\textit{Wskazówka 1: polecenie \texttt{sed} z opcją \texttt{-i} i wskazaniem pliku modyfikuje zawartości tego pliku stosownie do poleceń wydanych sed'owi}\\
\textit{Wskazówka 2: dla uproszczenia możesz przyjąć że napisy te składają się jedynie z liter i cyfr.}
\fi
\dbEntryBegin{rekurecyjne_wyszukaj_i_zastap-rozwiazanie}\if1\dbEntryCheckResults
\begin{CodeFrame*}[bash]{}
rreplace() {
	grep -lR "$1" . | while read f; do
		sed -e "s@$1@$2@g" -i $f;
	done;
}
\end{CodeFrame*}

Zamiast \Verb#grep -lR "$1" .# można by napisać \Verb#grep -R "$1" . | cut -f 1 -d: | uniq#, w większości przypadków zadziała tak samo,
	ale jest bardziej skomplikowane, wymaga więcej zasobów i będzie miało problem z nazwami plików zawierającymi dwukropek.
\fi
