% Copyright (c) 2017-2020 Matematyka dla Ciekawych Świata (http://ciekawi.icm.edu.pl/)
% Copyright (c) 2017-2020 Robert Ryszard Paciorek <rrp@opcode.eu.org>
% 
% MIT License
% 
% Permission is hereby granted, free of charge, to any person obtaining a copy
% of this software and associated documentation files (the "Software"), to deal
% in the Software without restriction, including without limitation the rights
% to use, copy, modify, merge, publish, distribute, sublicense, and/or sell
% copies of the Software, and to permit persons to whom the Software is
% furnished to do so, subject to the following conditions:
% 
% The above copyright notice and this permission notice shall be included in all
% copies or substantial portions of the Software.
% 
% THE SOFTWARE IS PROVIDED "AS IS", WITHOUT WARRANTY OF ANY KIND, EXPRESS OR
% IMPLIED, INCLUDING BUT NOT LIMITED TO THE WARRANTIES OF MERCHANTABILITY,
% FITNESS FOR A PARTICULAR PURPOSE AND NONINFRINGEMENT. IN NO EVENT SHALL THE
% AUTHORS OR COPYRIGHT HOLDERS BE LIABLE FOR ANY CLAIM, DAMAGES OR OTHER
% LIABILITY, WHETHER IN AN ACTION OF CONTRACT, TORT OR OTHERWISE, ARISING FROM,
% OUT OF OR IN CONNECTION WITH THE SOFTWARE OR THE USE OR OTHER DEALINGS IN THE
% SOFTWARE.

\IfStrEq{\dbEntryID}{}{
	\insertZadanie{\currfilepath}{petla_linki_html}{}
	\insertZadanie{\currfilepath}{warunek_istnienie_pliku}{}
	\insertZadanie{\currfilepath}{funkcja_n_razy_napis}{}
	\insertZadanie{\currfilepath}{funkcja_liczba_kotow}{}
}

\IfStrEq{\dbEntryID}{rozwiazania}{
	\insertRozwiazanie{\currfilepath}{petla_linki_html}{}
	\insertRozwiazanie{\currfilepath}{warunek_istnienie_pliku}{}
	\insertRozwiazanie{\currfilepath}{funkcja_n_razy_napis}{}
	\insertRozwiazanie{\currfilepath}{funkcja_liczba_kotow}{}
}

% BEGIN: podstawy programowania w bashu - zadania
\dbEntryBegin{petla_linki_html}\if1\dbEntryCheckResults
Napisz pętle, która wypisze wszystkie pliki nieukryte z bieżącego katalogu w postaci linków HTML, czyli:
dla pliku o nazwie \Verb{ABC} powinna wypisać \Verb{<a href="ABC">ABC</a>}. Przedstaw zarówno rozwiązanie z użyciem pętli \Verb{for}, jak i pętli \Verb{while}.
\fi
\dbEntryBegin{petla_linki_html-rozwiazanie}\if1\dbEntryCheckResults
\begin{CodeFrame*}[bash]{}
for f in *; do echo "<a href=\"$f\">$f</a>"; done
ls | while read f; do echo "<a href=\"$f\">$f</a>"; done
\end{CodeFrame*}

\noindent Zwróć uwagę że:
\begin{itemize}
\item rozwiązania te różnią się jedynie sposobem uzyskania listy plików ald której mają wypisać linki
\item pętla for w każdym obiegu podstawia pod f kolejny element z listy nazw dopasowanych do gwiazdki (czyli wszystkich plików nieukrytych)
\item pętla while listę plików dostaje na standardowe wejście (jeden plik na linię) i wczytuje każdą kolejną linię (czyli kolejną nazwę pliku) stosując komendę read – jest to bardzo standardowe rozwiązanie do przetwarzania standardowego wejścia linia po linii
\item w obu wypadkach używamy takiego echo z napisem w podwójnych cudzysłowach (aby móc umieścić w nim zmienną), cudzysłowa które mają być wypisane zabezpieczamy odwrotnym ukośnikiem
\item wypisywanie można rozwiązać na kilka innych sposobów np.: \shell{echo '<a href="'"$f"'">'"$f"'</a>'} – zadziała tak samo, ale wydaje się być to mniej czytelne
\end{itemize}
\fi


\dbEntryBegin{warunek_istnienie_pliku}\if1\dbEntryCheckResults
Napisz warunek, który sprawdzi czy \Verb{/tmp/abc} istnieje i jest katalogiem.
\fi
\dbEntryBegin{warunek_istnienie_pliku-rozwiazanie}\if1\dbEntryCheckResults
\begin{CodeFrame*}[bash]{}
if [ -d /tmp/abc ]; then echo "jest katalogiem"; else echo "nie";
\end{CodeFrame*}

lub krócej:

\begin{CodeFrame*}[bash]{}
[ -d /tmp/abc ] && echo "jest katalogiem";
\end{CodeFrame*}

\noindent Zwróć uwagę że:
\begin{itemize}
\item w celu warunkowego wypisania jakiejś informacji możemy użyć zarówno konstrukcji if, jak też łączenia poleceń,
	jednak w przypadku większej ilości poleceń objętych warunkiem konstrukcja z if jest bardziej czytelna
\item sprawdzenie czy podana ścieżka istnieje i jest katalogiem odbywa się przy pomocy opcji -d, informacja ta celowo nie była podana w treści skryptu – należało to sprawdzić w dokumentacji systemowej (\shell{man test}). \textbf{Czytanie dokumentacji jest ważne!}
\end{itemize}
\fi


\dbEntryBegin{funkcja_n_razy_napis}\if1\dbEntryCheckResults
Napisać funkcję przyjmującą dwa argumenty - liczbę i napis; funkcja ma wypisać napis tyle razy ile wynosi podana liczba.
\fi
\dbEntryBegin{funkcja_n_razy_napis-rozwiazanie}\if1\dbEntryCheckResults
\begin{CodeFrame*}[bash]{}
f() { for i in `seq 1 $1`; do echo $2; done; }
\end{CodeFrame*}

\noindent Zwróć uwagę że:
\begin{itemize}
\item w nawiasach po nazwie funkcje nie piszemy nic na temat jej argumentów - one są puste
\item do argumentów odwołujemy się poprzez dolar numer argumentu
\item do n krotnego powtórzenia czynności używamy pętli for która iteruje po liście liczb zwracanej przez seq
\item seq objęta jest znakami ` oznaczającymi że należy wykonać podany w nich kod i podstawić w to miejsce jego standardowe wyjście, nie należy ich mylić z apostrofami używanymi do napisów (')
\item spacja po { oraz średnik (lub nowa linia) przed } są istotne składniowo
\end{itemize}
\fi


\dbEntryBegin{funkcja_liczba_kotow}\if1\dbEntryCheckResults
Napisać funkcję przyjmującą jeden argument - liczbę kotów i wypisującą:
\begin{itemize}
	\item "Ala ma kota" dla ilości kotów równej 1
	\item "Ala ma x koty" lub "Ala ma x kotów" gdzie dobrana jest poprawna forma, a pod x podstawiona podana w argumencie ilość kotów.
\end{itemize}
Dla uproszczenia należy założyć że podana ilość kotów jest w zakresie od 1 do 9.
\fi
\dbEntryBegin{funkcja_liczba_kotow-rozwiazanie}\if1\dbEntryCheckResults
\begin{CodeFrame*}[bash]{}
koty() {
	case $1 in
		1) echo "Ala ma kota";;
		2|3|4) echo "Ala ma $1 koty";;
		*) echo "Ala ma $1 kotów";;
	esac
}
\end{CodeFrame*}
\fi
% END: podstawy programowania w bashu - zadania
