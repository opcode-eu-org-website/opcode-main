% Copyright (c) 2017-2020 Matematyka dla Ciekawych Świata (http://ciekawi.icm.edu.pl/)
% Copyright (c) 2017-2020 Robert Ryszard Paciorek <rrp@opcode.eu.org>
% 
% MIT License
% 
% Permission is hereby granted, free of charge, to any person obtaining a copy
% of this software and associated documentation files (the "Software"), to deal
% in the Software without restriction, including without limitation the rights
% to use, copy, modify, merge, publish, distribute, sublicense, and/or sell
% copies of the Software, and to permit persons to whom the Software is
% furnished to do so, subject to the following conditions:
% 
% The above copyright notice and this permission notice shall be included in all
% copies or substantial portions of the Software.
% 
% THE SOFTWARE IS PROVIDED "AS IS", WITHOUT WARRANTY OF ANY KIND, EXPRESS OR
% IMPLIED, INCLUDING BUT NOT LIMITED TO THE WARRANTIES OF MERCHANTABILITY,
% FITNESS FOR A PARTICULAR PURPOSE AND NONINFRINGEMENT. IN NO EVENT SHALL THE
% AUTHORS OR COPYRIGHT HOLDERS BE LIABLE FOR ANY CLAIM, DAMAGES OR OTHER
% LIABILITY, WHETHER IN AN ACTION OF CONTRACT, TORT OR OTHERWISE, ARISING FROM,
% OUT OF OR IN CONNECTION WITH THE SOFTWARE OR THE USE OR OTHER DEALINGS IN THE
% SOFTWARE.

\IfStrEq{\dbEntryID}{}{
	\insertZadanie{\currfilepath}{vim}{}
	\insertZadanie{\currfilepath}{lsetc}{}
	\insertZadanie{\currfilepath}{findetc}{}
	\insertZadanie{\currfilepath}{linki_symboliczne}{}
	\insertZadanie{\currfilepath}{findetc2}{}
}


\dbEntryBegin{vim}\if1\dbEntryCheckResults
Przy pomocy edytora \textit{vim} otwórz plik \Verb#/etc/passwd# i zastąp wszystkie wystąpienia \Verb#bin# przez \Verb#XYZ#. Nie zapisuj pliku.
\fi
\dbEntryBegin{vim-rozwiazanie}\if1\dbEntryCheckResults
\begin{enumerate}
\item Uruchomić edytor za pomocą polecenia \Verb{vim /etc/passwd}.
\item W uruchomionym edytorze należy wydać polecenie \Verb{:%s@bin@XYZ@g}, które spowoduje zastąpienie wszystkich wystąpień \Verb{bin} przez \Verb{XYZ}.
\item Edytor należy opuścić bez zapisywania zmian przy pomocy polecenia \Verb{:q!}
\end{enumerate}

\noindent Zwróć uwagę że:
\begin{itemize}
\item Plik do otwarcia wskazujemy w linii poleceń przy pomocy ścieżki bezwzględnej zaczynającej się od \Verb{/}.
      Zamiast tego można by także otworzyć plik po uruchomieniu edytora lub użyć ścieżki względnej.
      Użycie ścieżki bezwzględnej zapewnia niezależność tej komendy od katalogu roboczego w którym zostanie wykonana.
\item Do zamiany i zamknięcia edytora używane są polecenia linii poleceń vim'a uruchamianej przy pomocy dwukropka (\Verb{:}) i dlatego są one poprzedzane dwukropkiem.
      Jest to inny zbiór komend niż komendy wprowadzane bezpośrednio (takie jak np. \Verb{dd} kasujące bieżącą linię).
      Ta linia poleceń też ma swoją historię.
\item Edytor opuszczamy z użyciem \Verb{:q!}, dodanie wykrzyknika jest potrzebne żeby zamknąć edytor bez zapisywania zmian.
\end{itemize}
\fi


\dbEntryBegin{lsetc}\if1\dbEntryCheckResults
Wyświetlić nazwy (mogą być wraz z pełną ścieżką) wszystkich plików i katalogów znajdujących się bezpośrednio w \texttt{/etc/} których druga litera to \texttt{a} natomiast trzecia to \texttt{p} lub \texttt{s}.

\emph{Wskazówka: to zadanie nie wymaga stosowania find.}
\fi
\dbEntryBegin{lsetc-rozwiazanie}\if1\dbEntryCheckResults
\begin{CodeFrame*}[bash]{}
ls -d /etc/?a[sp]*
\end{CodeFrame*}

albo:

\begin{CodeFrame*}[bash]{}
ls -d /etc/?as* /etc/?ap*
\end{CodeFrame*}

\noindent Zwróć uwagę że:
\begin{itemize}
\item Polecenie \Verb{ls} listuje wszystkie podane w linii poleceń argumenty oddzielane spacjami.
\item Jeżeli argumenty te zawierają znaki uogólniające powłoki to każdy taki argument zostanie zastapiony przez powłokę listą pasujących plików, lub zostanie przekazany w formie niezmienionej jeżeli nie ma pasujacych plików. Ma to kilka konsekwencji:
	\begin{itemize}
	\item W wyniku dopasowania nazw do polecenia ls mogą zostać przekazane ścieżki do katalogów, dla których ls domyślnie listuje ich zawartość.
	      Tutaj tego nie chcemy i dlatego używamy opcji \Verb{-d}
	\item W pierwszym wariancie wszystkie pliki opisaliśmy pojedynczym wyrażeniem uogólniającym basha, a w drugim podaliśmy dwa argumenty.
	      Polecenia te są prawie równoważne - wypiszą te ścieżki.
	      Jednak w przypadku braku dopasowań do jednego z wariantów (np. braku plików z trzecią literą p) – drugie polecenie wypisałoby dodatkowo komunikat o braku dopasowania do jednego z podanych argumentów (np. do \Verb{/etc/?ap*}). Wynika to z tego iż przy braku dopsowania do tego argumentu bash przekazałby go w formie niezmienionej do polecenia \Verb{ls}.
	\end{itemize}
\end{itemize}
\fi


\dbEntryBegin{findetc}\if1\dbEntryCheckResults
Wyszukaj (rekurencyjnie) wszystkie pliki w katalogu \texttt{/etc/} zmodyfikowane w przeciągu ostatnich 48 godzin.
\fi
\dbEntryBegin{findetc-rozwiazanie}\if1\dbEntryCheckResults
\begin{CodeFrame*}[bash]{}
cd /etc
find -mtime -2
\end{CodeFrame*}

lub

\begin{CodeFrame*}[bash]{}
find /etc -mtime -2
\end{CodeFrame*}

jeżeli nie uważamy katalogów za pliki, to możemy dodać stosowny warunek

\begin{CodeFrame*}[bash]{}
find -mtime -2 -type f
\end{CodeFrame*}

\noindent Zwróć uwagę że:
\begin{itemize}
\item polecenie find przeszukuje rekurencyjnie katalog podany przed warunkami. Jeżeli nie został on podany to przeszukuje bieżący katalog (\Verb{.}), który mozemy zmienić poleceniem \Verb{cd}.
\end{itemize}
\fi


\dbEntryBegin{linki_symboliczne}\if1\dbEntryCheckResults
Utworzyć w katalogu \texttt{/tmp} linki symboliczne \texttt{ll1} i \texttt{ll2} wskazujące na \texttt{/etc/passwd} odpowiednio poprzez ścieżkę bezwzględną i względną.
\fi
\dbEntryBegin{linki_symboliczne-rozwiazanie}\if1\dbEntryCheckResults
Podając pełne ścieżki do pliku docelowego:

\begin{CodeFrame*}[bash]{}
ln -s /etc/passwd /tmp/ll1
ln -s -r /etc/passwd /tmp/ll2
\end{CodeFrame*}

lub wchodząc do katalogu docelowego i podając tylko nazwy plików

\begin{CodeFrame*}[bash]{}
cd /tmp
ln -s /etc/passwd ll1
ln -s ../etc/passwd ll2
\end{CodeFrame*}

Link ze ścieżką względna można utworzyć także będąc w dowolnym katalogu poprzez wywołanie:

\begin{CodeFrame*}[bash]{}
ln -s ../etc/passwd ll2
\end{CodeFrame*}

\noindent Zwróć uwagę że:
\begin{itemize}
\item Pierwszy wariant możemy wykonać także będąc w \Verb{/tmp}
\item W przypadku podawania w \Verb{ln} ścieżki względnej (\Verb{ln -s ../etc/passwd ll2}) należy pamiętać iż zostanie ona literalnie zapisana w tworzonym linku.
	\begin{itemize}
	\item W związku z tym musi to być poprawna ścieżka z katalogu docelowego, a nie z bieżącego katalogu roboczego.
	\item Np. będąc w \Verb{/usr/local/bin/} ścieżka względna do \Verb{/etc/passwd} to \Verb{../../../etc/passwd}, ale gdy będąc w tamtym katalogu chemy utworzyć plik \Verb{/tmp/llX} linkujący ścieżką względną do \Verb{/etc/passwd} wykonamy polecenie \Verb{ln -s ../etc/passwd /tmp/llX}, bo ścieżka względem \Verb{/tmp} ma postać \Verb{ln -s ../etc/passwd}.
	\end{itemize}
\end{itemize}
\fi


\dbEntryBegin{findetc2}\if1\dbEntryCheckResults
Zmodyfikuj rozwiązanie zadania \ref{findetc} tak aby wyświetlać szczegóły (w tym datę modyfikacji) dla wyszukanych plików.
\fi
\dbEntryBegin{findetc2-rozwiazanie}\if1\dbEntryCheckResults
\begin{Verbatim}
find /etc -mtime -2 -exec ls -ld \{\} \;
\end{Verbatim}

\noindent Zwróć uwagę że:
\begin{itemize}
\item Wykorzystanie warunku \Verb{-exec}, który dodatkowo wyłącza standardowy output find'a
\item Zabezpieczamy klamerki (nie zawsze wymagane) i średnik (praktycznie zawsze wymagane) oraz spację pomiędzy klamerkami a średnikiem.
      Możemy je zabezpieczyć także cudzysłowami: \verb{find /etc -mtime -2 -exec ls -ld '{}' ';'},
      ale nie możemy umieścić klamerek i średnika w jednym napisie (np. \verb{find -mtime -2 -exec ls -ld '{\} ;'}).
\end{itemize}
\fi
