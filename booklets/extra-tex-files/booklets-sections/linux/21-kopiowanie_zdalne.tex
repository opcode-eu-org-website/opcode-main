% Copyright (c) 2017-2020 Matematyka dla Ciekawych Świata (http://ciekawi.icm.edu.pl/)
% Copyright (c) 2017-2020 Robert Ryszard Paciorek <rrp@opcode.eu.org>
% 
% MIT License
% 
% Permission is hereby granted, free of charge, to any person obtaining a copy
% of this software and associated documentation files (the "Software"), to deal
% in the Software without restriction, including without limitation the rights
% to use, copy, modify, merge, publish, distribute, sublicense, and/or sell
% copies of the Software, and to permit persons to whom the Software is
% furnished to do so, subject to the following conditions:
% 
% The above copyright notice and this permission notice shall be included in all
% copies or substantial portions of the Software.
% 
% THE SOFTWARE IS PROVIDED "AS IS", WITHOUT WARRANTY OF ANY KIND, EXPRESS OR
% IMPLIED, INCLUDING BUT NOT LIMITED TO THE WARRANTIES OF MERCHANTABILITY,
% FITNESS FOR A PARTICULAR PURPOSE AND NONINFRINGEMENT. IN NO EVENT SHALL THE
% AUTHORS OR COPYRIGHT HOLDERS BE LIABLE FOR ANY CLAIM, DAMAGES OR OTHER
% LIABILITY, WHETHER IN AN ACTION OF CONTRACT, TORT OR OTHERWISE, ARISING FROM,
% OUT OF OR IN CONNECTION WITH THE SOFTWARE OR THE USE OR OTHER DEALINGS IN THE
% SOFTWARE.

% BEGIN: zdalne kopiowanie
\subsection{zdalne kopiowanie}
Najprostszą metą kopiowania plików pomiędzy różnymi systemami jest wykorzystanie do tego ssh, typowo robi się to na jeden z kilku sposobów:
\begin{itemize}
	\item poleceniem
		\texttt{scp [opcje] źródło1 [źródło2 [...]] cel}
		, które
		kopiuje wskazany plik (lub pliki) do wskazanej lokalizacji, w przypadku kopiowania wielu plików cel powinien być katalogiem, do ważniejszych opcji należy zaliczyć:\\
		\texttt{-r} pozwala na (rekursywne) kopiowanie katalogów\\
		\texttt{-P port} określa port SSH\\
		W odróżnieniu od \texttt{cp} źródło lub cel w postaci \texttt{[user@]host:[ścieżka]} wskazują na zdalny system dostępny poprzez SSH.
	\item poleceniem
		\texttt{rsync [opcje] źródło cel}
		, które
		kopiuje (synchronizacjiuje) pliki i drzewa katalogów (zarówno lokalnie jak i zdalnie), do ważniejszych opcji należy zaliczyć:\\
			\texttt{-r} pozwala na (rekursywne) kopiowanie katalogów\\
			\texttt{-l} kopiuje linki symboliczne jako linki symboliczne (zamiast kopiowania zawartości pliku na który wskazują)\\
			\texttt{-t} zachowuje czas modyfikacji plików\\
			\texttt{-u} kopiuje tylko gdy plik źródłowy nowszy niż docelowy\\
			\texttt{-c} kopiuje tylko gdy plik źródłowy i docelowy mają inne sumy kontrolne\\
			\texttt{--delete} usuwa z docelowego drzewa katalogów elementy nie występujące w drzewie źródłowym\vspace{6pt}\\
			%
			\texttt{-e 'ssh'} pozwala na kopiowanie na/z zdalnych systemów za pośrednictwem ssh, źródło lub cel w postaci \texttt{[user@]host:[ścieżka]} wskazują na zdalny system
			\texttt{--partial --partial-dir=."-tmp-"} zachowuje skopiowane częściowo pliki w katalogu .-tmp- (pozwala na przerwanie i wznowienie transferu pliku)\\
			\texttt{--progress} pokazuje postęp kopiowania\\
			\texttt{--exclude="wzorzec"} pomija (w kopiowaniu i kasowaniu) pliki pasujące do wzorzec (wzorzec może zawierać znaki uogólniające powłoki)
			\texttt{-n} symuluje pracę (pokazuje co zostałoby skopiowane, ale nie kopiuje)
	\item stosując
		\texttt{sshfs [opcje] host:scieżka}
		, który
		montuje zdalny system plików z użyciem FUSE (filesystem in userspace) oraz SSH, do ważniejszych opcji należy zaliczyć:\\
			\texttt{-p port} określa inny niż domyślny port serwera SSH\\
			\texttt{-o workaround=rename}, który zapewnia poprawne \texttt{mv} na istniejący plik
	\item złożonego polecenia opartego na przekierowaniu wyjścia jakiejś komendy do ssh, które uruchamia po zdalnej stronie proces odbierający te dane na swoim standardowym wejściu, np.:
		\begin{itemize}
			\item \Verb{tar -czf - ścieżka1 [ścieżka2 [...]] | ssh [user@]host 'cat > plik.tgz'}\\
				archiwizuje wskazane pliki/katalogi bezpośrednio na zdalny system z użyciem tar i kompresji gzip do pliku \texttt{plik.tgz}
			\item \Verb{tar -cf - ścieżka1 [ścieżka2 [...]] | ssh [user@]host 'tar -xf - -C cel'}\\
				kopiuje wskazane pliki/katalogi na zdalny system z użyciem tar do katalogu \texttt{cel}
		\end{itemize}
\end{itemize}
% END: zdalne kopiowanie 
