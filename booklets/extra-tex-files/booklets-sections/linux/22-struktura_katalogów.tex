% Copyright (c) 2017-2020 Matematyka dla Ciekawych Świata (http://ciekawi.icm.edu.pl/)
% Copyright (c) 2017-2020 Robert Ryszard Paciorek <rrp@opcode.eu.org>
% 
% MIT License
% 
% Permission is hereby granted, free of charge, to any person obtaining a copy
% of this software and associated documentation files (the "Software"), to deal
% in the Software without restriction, including without limitation the rights
% to use, copy, modify, merge, publish, distribute, sublicense, and/or sell
% copies of the Software, and to permit persons to whom the Software is
% furnished to do so, subject to the following conditions:
% 
% The above copyright notice and this permission notice shall be included in all
% copies or substantial portions of the Software.
% 
% THE SOFTWARE IS PROVIDED "AS IS", WITHOUT WARRANTY OF ANY KIND, EXPRESS OR
% IMPLIED, INCLUDING BUT NOT LIMITED TO THE WARRANTIES OF MERCHANTABILITY,
% FITNESS FOR A PARTICULAR PURPOSE AND NONINFRINGEMENT. IN NO EVENT SHALL THE
% AUTHORS OR COPYRIGHT HOLDERS BE LIABLE FOR ANY CLAIM, DAMAGES OR OTHER
% LIABILITY, WHETHER IN AN ACTION OF CONTRACT, TORT OR OTHERWISE, ARISING FROM,
% OUT OF OR IN CONNECTION WITH THE SOFTWARE OR THE USE OR OTHER DEALINGS IN THE
% SOFTWARE.

% BEGIN: Struktura katalogów
\subsection{struktura katalogów}

Systemy unix'owe posiadają drzewiasty system plików zaczynający się w katalogu głównym oznaczanym przez ukośnik (\texttt{/}), w którym zamontowany jest główny system plików (rootfs), inne systemy plików mogą być montowane w kolejnych katalogach. Do najistotniejszych katalogów należy zaliczyć:
\begin{itemize}
	\item \Verb{/bin}
	      zawierający pliki wykonywalne podstawowych programów
	\item \Verb{/sbin}
	      zawierający pliki wykonywalne podstawowych programów administracyjnych
	\item \Verb{/lib}
	      zawierający pliki podstawowych bibliotek
	\item \Verb{/usr}
	      zawierający oprogramowanie dodatkowe (wewnętrznie ma podobną strukturę do głównego - tzn. katalogi \texttt{/usr/bin}, \texttt{/usr/sbin}, \texttt{/usr/lib}, itd)
	
	\vspace{6pt}
	
	\item \Verb{/etc}
	      zawierający konfiguracje ogólnosystemowe
	\item \Verb{/var}
	      zawierający dane programów i usług (takie jak kolejka poczty, harmonogramy zadań, bazy danych)
	\item \Verb{/home}
	      zawierający katalogi domowe użytkowników (często montowany z innego systemu plików, dlatego też root ma swój katalog domowy w \texttt{/root}, aby był dostępny nawet gdy takie montowanie nie doszło do skutku)
	\item \Verb{/tmp}
	      zawierający pliki tymczasowe (typowo czyszczony przy starcie systemu); w Linuxie występuje też \texttt{/run} przeznaczony do trzymania danych tymczasowych działających usług takich jak numery pid, blokady, itp
	
	\vspace{6pt}
	
	\item \Verb{/dev}
	      zawierający pliki reprezentujące urządzenia; w Linuxie występuje też \texttt{/sys} zawierający informacje i ustawienia dotyczące m.in. urządzeń
	\item \Verb{/proc}
	      zawierający informacje o działających procesach (w Linuxie także interfejs konfiguracyjny dla wielu parametrów jądra)
\end{itemize}
Pliki i katalogi których nazwa rozpoczyna się od kropki traktowane są jako pliki ukryte.

Z punktu widzenia programisty czy też użytkownika (prawie) wszystko jest plikiem, których istnieją różne rodzaje (zwykły plik, katalog, urządzenie znakowe, urządzenie blokowe, link symboliczny, kolejka FIFO, ...); pewnym wyjątkiem są urządzenia sieciowe (które nie mają reprezentacji w systemie plików (ale gniazda związane z nawiązanymi połączeniami obsługuje się zasadniczo tak jak pliki).
% END: Struktura katalogów
