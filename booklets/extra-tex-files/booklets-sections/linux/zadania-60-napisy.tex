% Copyright (c) 2017-2020 Matematyka dla Ciekawych Świata (http://ciekawi.icm.edu.pl/)
% Copyright (c) 2017-2020 Robert Ryszard Paciorek <rrp@opcode.eu.org>
% 
% MIT License
% 
% Permission is hereby granted, free of charge, to any person obtaining a copy
% of this software and associated documentation files (the "Software"), to deal
% in the Software without restriction, including without limitation the rights
% to use, copy, modify, merge, publish, distribute, sublicense, and/or sell
% copies of the Software, and to permit persons to whom the Software is
% furnished to do so, subject to the following conditions:
% 
% The above copyright notice and this permission notice shall be included in all
% copies or substantial portions of the Software.
% 
% THE SOFTWARE IS PROVIDED "AS IS", WITHOUT WARRANTY OF ANY KIND, EXPRESS OR
% IMPLIED, INCLUDING BUT NOT LIMITED TO THE WARRANTIES OF MERCHANTABILITY,
% FITNESS FOR A PARTICULAR PURPOSE AND NONINFRINGEMENT. IN NO EVENT SHALL THE
% AUTHORS OR COPYRIGHT HOLDERS BE LIABLE FOR ANY CLAIM, DAMAGES OR OTHER
% LIABILITY, WHETHER IN AN ACTION OF CONTRACT, TORT OR OTHERWISE, ARISING FROM,
% OUT OF OR IN CONNECTION WITH THE SOFTWARE OR THE USE OR OTHER DEALINGS IN THE
% SOFTWARE.

\IfStrEq{\dbEntryID}{}{
	\insertZadanie{\currfilepath}{passwd_warunek_na_uid_awk}{}
	\insertZadanie{\currfilepath}{awk_last}{}
}


% napisy z uzyciem awk

\dbEntryBegin{passwd_warunek_na_uid_awk}\if1\dbEntryCheckResults
Korzystając z AWK wyświetl z /etc/passwd linie w których UID (3 pole) ma wartość >= 1000.
\fi
\dbEntryBegin{passwd_warunek_na_uid_awk-rozwiazanie}\if1\dbEntryCheckResults
\begin{CodeFrame*}[bash]{}
awk -F: '$3>=1000 {print $0}' /etc/passwd
\end{CodeFrame*}

\noindent Zwróć uwagę na:
\begin{itemize}
\item ustawienie separatora pól przy pomocy opcji -F na dwukropek
\item wykorzystanie charakterystyki AWK polegającej na wykonywaniu bloku dla każdej pasującej linii
\item warunek na wartość 3 pola, awk sam zadba o konwersję napisu przeczytanego z pliku na liczbę celem wykonania tego porównania
\end{itemize}
\fi


\dbEntryBegin{awk_last}\if1\dbEntryCheckResults
Polecenie \Verb{last} wypisuje informację o ostatnich zalogowaniach w systemie. Napisz polecenie (wykorzystujące \Verb{last}), które wypisze informację jak często logowali się poszczególni użytkownicy.
\fi
\dbEntryBegin{awk_last-rozwiazanie}\if1\dbEntryCheckResults
\begin{CodeFrame*}[bash]{}
last | awk '/^wtmp begins/ {next} $1 !="" {x[$1]++} END{for (u in x) print u, x[u]}'
\end{CodeFrame*}

\noindent Zwróć uwagę na:
\begin{itemize}
\item zignorowanie linii zaczynającej się od "wtmp begins" z użyciem dopasowania wyrażenia regularnego i komendy next
\item zliczanie wystąpień nazwy użytkownika w tablicy x (indeksowanej tymi nazwami)
\item zastosowanie bloku END do wypisania zawartości tablicy, której używaliśmy do zliczania użytkowników
\end{itemize}
\fi
