% Copyright (c) 2017-2020 Matematyka dla Ciekawych Świata (http://ciekawi.icm.edu.pl/)
% Copyright (c) 2017-2020 Robert Ryszard Paciorek <rrp@opcode.eu.org>
% 
% MIT License
% 
% Permission is hereby granted, free of charge, to any person obtaining a copy
% of this software and associated documentation files (the "Software"), to deal
% in the Software without restriction, including without limitation the rights
% to use, copy, modify, merge, publish, distribute, sublicense, and/or sell
% copies of the Software, and to permit persons to whom the Software is
% furnished to do so, subject to the following conditions:
% 
% The above copyright notice and this permission notice shall be included in all
% copies or substantial portions of the Software.
% 
% THE SOFTWARE IS PROVIDED "AS IS", WITHOUT WARRANTY OF ANY KIND, EXPRESS OR
% IMPLIED, INCLUDING BUT NOT LIMITED TO THE WARRANTIES OF MERCHANTABILITY,
% FITNESS FOR A PARTICULAR PURPOSE AND NONINFRINGEMENT. IN NO EVENT SHALL THE
% AUTHORS OR COPYRIGHT HOLDERS BE LIABLE FOR ANY CLAIM, DAMAGES OR OTHER
% LIABILITY, WHETHER IN AN ACTION OF CONTRACT, TORT OR OTHERWISE, ARISING FROM,
% OUT OF OR IN CONNECTION WITH THE SOFTWARE OR THE USE OR OTHER DEALINGS IN THE
% SOFTWARE.

\IfStrEq{\dbEntryID}{zadania_komendy}{
	\insertZadanie{\currfilepath}{mkdir}{}
	\insertZadanie{\currfilepath}{copy_etc}{}
	\insertZadanie{\currfilepath}{find_etc}{}
	\insertZadanie{\currfilepath}{sort_passwd}{}
	\insertZadanie{\currfilepath}{z_bin_false}{}
}

\IfStrEq{\dbEntryID}{zadania_bash}{
	\insertZadanie{\currfilepath}{kopiowanie_tylko_plikow}{}
	\insertZadanie{\currfilepath}{wyszukaj_napis_kopiuj}{}
	
	\insertZadanie{\currfilepath}{zmiana_rozszerzenia}{}
	\insertZadanie{\currfilepath}{pliki_zawierajace_napis}{}
	\insertZadanie{\currfilepath}{parsowanie_cmdline}{}
	\insertZadanie{\currfilepath}{20ta_linia}{}
}



% bash

\dbEntryBegin{kopiowanie_tylko_plikow}\if1\dbEntryCheckResults
Napisz polecenie które skopiuje wszystkie pliki (nie katalogi ani linki symboliczne) z katalogu \Verb{/etc} do \Verb{/tmp}
% wyłączamy linki symboliczne aby uniknąć: cp /etc/* /tmp
\fi

\dbEntryBegin{wyszukaj_napis_kopiuj}\if1\dbEntryCheckResults
Napisz polecenie które przekopiuje wszystkie pliki zawierające słowo \Verb{hostname} z katalogu \Verb{/etc} (wraz z jego podkatalogami) do katalogu \Verb{/tmp/etc}
zachowując strukturę katalogów (czyli plik /etc/a/b kopiowany jest do katalogu /tmp/etc/b).
Przyjmij że katalog \Verb{/tmp/etc} nie istnieje.
\fi

% PwES domowe (?):


\dbEntryBegin{mkdir}\if1\dbEntryCheckResults
Napisz polecenie, które utworzy katalog \texttt{/tmp/a/b} (katalog \texttt{b} znajdujący się wewnątrz katalogu \texttt{a} wewnątrz \texttt{/tmp})
\fi

\dbEntryBegin{copy_etc}\if1\dbEntryCheckResults
Napisz polecenie, które skopiuje pliki i katalogi (wraz z zawartością), znajdujące się bezpośrednio w katalogu \texttt{/etc}, których nazwa rozpoczyna się małą literą \texttt{b} do katalogu \texttt{/tmp}.
\fi

\dbEntryBegin{find_etc}\if1\dbEntryCheckResults
Napisz polecenie, które wyświetli nazwy (mogą być ze ścieżką) plików (nie katalogów) znajdujących się bezpośrednio w katalogu \texttt{/etc} (nie w jego podkatalogach), których rozmiar jest większy niż 300 bajtów.

\emph{Wskazówka: W skrypcie podane są tylko wybrane opcje omawianych poleceń. Rozwiązanie tego zadania może ułatwić zapoznanie się z pełną dokumentacją poleceń takich jak \texttt{ls}, czy \texttt{find}.}
\fi


\dbEntryBegin{sort_passwd}\if1\dbEntryCheckResults
Wyświetl posortowaną numerycznie, według 3go pola zawartość pliku \Verb{/etc/passwd}.
\fi

\dbEntryBegin{z_bin_false}\if1\dbEntryCheckResults
Wyświetl z pliku \Verb{/etc/passwd} loginy użytkowników (pierwsze pole), którzy mają ustawioną powłokę (pole 7) na \Verb{/bin/false}
\fi

\dbEntryBegin{20ta_linia}\if1\dbEntryCheckResults
Wyświetl 20tą linię z pliku \Verb{/etc/passwd}. Przedstaw dwa różne (używające innych komend) rozwiązania.
\fi


\dbEntryBegin{zmiana_rozszerzenia}\if1\dbEntryCheckResults
Napisz polecenie które dla wszystkich plików z rozszerzeniem \Verb{.TXT}  w bierzącym katalogu (bez podkatalogów) dokona zmiany ich nazwy zmieniając rozszerzenie na \Verb{.txt}, zachowując podstawową część nazwy bez modyfikacji.
W rozwiązaniu nie korzystamy z polecenia \Verb{rename}.
\fi

\dbEntryBegin{pliki_zawierajace_napis}\if1\dbEntryCheckResults
Napisz polecenie które wyszuka i przekopiuje do katalogu \Verb{/tmp} pliki z katalogu \Verb{/etc} (wraz z jego podkatalogami), które (w swojej treści) zawierają napis \Verb{nameserver}.
\fi

\dbEntryBegin{parsowanie_cmdline}\if1\dbEntryCheckResults
Plik \Verb{/proc/cmdline} zawiera informację o opcjach przekazanych do jądra podczas startu. Kolejne opcje rozdzielane są spacją, a nazwę opcji od jej argumentu rozdziela znak rówwności.
Napisz polecenie które wypisze argument opcji root. Dla pliku \Verb{/proc/cmdline} postaci:\\
 \Verb{BOOT_IMAGE=/vmlinuz-4.9-amd64 root=UUID=cad866ab-aabd-4686-8376-e4b9f1c2ae9e rw}\\
polecenie powinno wypisać:\\
 \Verb{UUID=cad866ab-aabd-4686-8376-e4b9f1c2ae9e}

\textbf{Uwaga:} nie wolno zakładać że \Verb#root=# jest drugą opcją linii poleceń jądra, nie wolno zakładać że nie ma tam innej opcji kończącej się na \Verb#root=# (np. \Verb#nfsroot=#).
\fi


% bardziej dodatkowe

\dbEntryBegin{wyszukaj_napis2}\if1\dbEntryCheckResults
% tak jak na ogół to jest robione (po prostu opcja do grep) to raczej ma mały sens, dlatego w dodatkowych (ale w zasadzie jest to lepsze podejście niż cut | uniq ...)
Zmodyfikuj rozwiązanie zadania \ref{wyszukaj_napis} tak aby wyłącznie wypisywało nazwy plików (mogą być ze ścieżką) zawierających wyszukiwany napis.
\fi

\dbEntryBegin{lsetc2}\if1\dbEntryCheckResults
Wyświetlić same nazwy (bez ścieżki) wszystkich plików i katalogów znajdujących się bezpośrednio w \texttt{/etc/} których druga litera to \texttt{a} natomiast trzecia to \texttt{p} lub \texttt{s}.
Przedstaw rozwiązanie z użyciem i bez użycia komendy \texttt{cd}.
\fi
