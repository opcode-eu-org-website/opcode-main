% Copyright (c) 2017-2020 Matematyka dla Ciekawych Świata (http://ciekawi.icm.edu.pl/)
% Copyright (c) 2017-2020 Robert Ryszard Paciorek <rrp@opcode.eu.org>
% 
% MIT License
% 
% Permission is hereby granted, free of charge, to any person obtaining a copy
% of this software and associated documentation files (the "Software"), to deal
% in the Software without restriction, including without limitation the rights
% to use, copy, modify, merge, publish, distribute, sublicense, and/or sell
% copies of the Software, and to permit persons to whom the Software is
% furnished to do so, subject to the following conditions:
% 
% The above copyright notice and this permission notice shall be included in all
% copies or substantial portions of the Software.
% 
% THE SOFTWARE IS PROVIDED "AS IS", WITHOUT WARRANTY OF ANY KIND, EXPRESS OR
% IMPLIED, INCLUDING BUT NOT LIMITED TO THE WARRANTIES OF MERCHANTABILITY,
% FITNESS FOR A PARTICULAR PURPOSE AND NONINFRINGEMENT. IN NO EVENT SHALL THE
% AUTHORS OR COPYRIGHT HOLDERS BE LIABLE FOR ANY CLAIM, DAMAGES OR OTHER
% LIABILITY, WHETHER IN AN ACTION OF CONTRACT, TORT OR OTHERWISE, ARISING FROM,
% OUT OF OR IN CONNECTION WITH THE SOFTWARE OR THE USE OR OTHER DEALINGS IN THE
% SOFTWARE.

% BEGIN: Wbudowane przetwarzanie napisów w bashu
\subsection{Wbudowane przetwarzanie napisów w bash'u}

Wbudowane przetwarzanie napisów w bashu opiera się na odwołaniach do zmiennych w postaci \Verb@${}@:

\begin{itemize}
	\item \shell@${zmienna:-"napis"}@ zwróci napis gdy zmienna nie jest zdefiniowana lub jest pusta
	\item \shell@${zmienna:="napis"}@ zwróci napis oraz wykona podstawienie zmienna="napis" gdy zmienna nie jest zdefiniowana lub jest pusta
	\item \shell@${zmienna:+"napis"}@ zwróci napis gdy zmienna jest zdefiniowana i nie pusta
	
	\vspace{6pt}
	
	\item \shell@${#str}@    zwróci długość napisu w zmiennej str
	\item \shell@${str:n}@   zwróci pod-napis zmiennej str od n do końca
	\item \shell@${str:n:m}@ zwróci pod-napis zmiennej str od n o długości m
	
	\vspace{6pt}
	
	\item \shell@${str/"n1"/"n2"}@  zwróci wartość str z zastąpionym pierwszym wystąpieniem n1 przez n2
	\item \shell@${str//"n1"/"n2"}@  zwróci wartość str z zastąpionymi wszystkimi wystąpieniami n1 przez n2
	
	\vspace{6pt}
	
	\item \shell@${str#"ab"}@ zwróci wartość str z obciętym "ab" z początku
	\item \shell@${str%"fg"}@ zwróci wartość str z obciętym "fg" z końca
	
	\item \shell@${!x}@ zwróci wartość zmiennej, której nazwa jest w zmiennej x
\end{itemize}
W napisach do obcięcia możliwe jest stosowanie shellowych znaków uogólniających, czyli \Verb@*@, \Verb@?@, \Verb@[abc]@, itd operator \Verb@#@ i \Verb@%@ dopasowują minimalny napis do usunięcia, natomiast operatory \Verb@##@ i \Verb@%%@ dopasowują maksymalny napis do usunięcia.
Należy pamiętać że wiele z powyższych zapisów jest rozszerzeniami basha niedostepnymi w podstaowej składni sh.

Przykład:

\begin{CodeFrame}[bash]{.65\textwidth}
a=""; b=""; c=""
echo ${a:-"aa"} ${b:="bb"} ${c:+"cc"}
echo $a $b $c

a="x"; b="y"; c="z"
echo ${a:-"aa"} ${b:="bb"} ${c:+"cc"}
echo $a $b $c

x=abcdefg
echo ${#x} ${x:2} ${x:0:3} ${x:0:$((${#x}-2))}
echo ${x#"abc"} ${x%"efg"}
echo ${x#"ac"}  ${x%"eg"}

x=abcd.e.fg
echo ${x#*.} ${x##*.} ${x%.*} ${x%%.*}

y="aa bb cc bb dd bb ee"
echo ${y/"bb"/"XX"}
echo ${y//"bb"/"XX"}
\end{CodeFrame}
\begin{CodeFrame}{auto}
aa bb
bb
x y cc
x y z
7 cdefg abc abcde
defg abcd
abcdefg abcdefg
e.fg fg abcd.e abcd
aa XX cc bb dd bb ee
aa XX cc XX dd XX ee
\end{CodeFrame}

\subsubsection{Wyrażenia regularne}

Możliwe jest także korzystanie z wyrażeń regularnych.
Polecenie \Verb@expr match $x 'wr1\(wr2\)wr3'@ zwróci na stdout (wypisze) część \Verb@$x@ pasującą do wyrażenia regularnego \Verb@wr2@,
wyrażenia regularne \Verb@wr1@ i \Verb@wr2@ pozwalają na określanie części napisu do odrzucenia.
Alternatywną składnią jest \Verb@expr $x : 'wr1\(wr2\)wr3'@

Przykład użycia jednego i drugiego wariantu składni do podziału napisu na część przed i po znaku równości:

\begin{CodeFrame*}[bash]{}
z="ab=cd"
expr match $z '^\([^=]*\)='
expr $z : '^[^=]*=\(.*\)$'
\end{CodeFrame*}

Możliwe jest też sprawdzanie dopasowań wyrażeń regularnych poprzez (zwróć uwagę na brak cytowania wyrażenia regularnego):

\begin{CodeFrame*}[bash]{}
[[ "$z" =~ ^([^=]*)= ]] && echo "OK"
\end{CodeFrame*}

\subsubsection{printf}

Możliwe jest także zaawansowane formatowanie napisów, konwertowanie liczb na napisy,
w tym wypisywanie w różnych systemach liczbowych przy pomocy \Verb@printf@\footnote{
	Instrukcja \Verb@printf@ ma składnię opartą na tej funkcji z C, interpretuje ona także liczby zmiennoprzecinkowe.
}:

\begin{CodeFrame}[bash]{0.57\textwidth}
printf "0o%o %d 0x%x\n" 0xf 010 3
\end{CodeFrame}
\begin{CodeFrame}{auto}
0o17 8 0x3
\end{CodeFrame}
% END: Wbudowane przetwarzanie napisów w bashu
