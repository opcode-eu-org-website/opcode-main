% Copyright (c) 2016-2020 Matematyka dla Ciekawych Świata (http://ciekawi.icm.edu.pl/)
% Copyright (c) 2016-2017 Łukasz Mazurek
% Copyright (c) 2018-2020 Robert Ryszard Paciorek <rrp@opcode.eu.org>
% 
% MIT License
% 
% Permission is hereby granted, free of charge, to any person obtaining a copy
% of this software and associated documentation files (the "Software"), to deal
% in the Software without restriction, including without limitation the rights
% to use, copy, modify, merge, publish, distribute, sublicense, and/or sell
% copies of the Software, and to permit persons to whom the Software is
% furnished to do so, subject to the following conditions:
% 
% The above copyright notice and this permission notice shall be included in all
% copies or substantial portions of the Software.
% 
% THE SOFTWARE IS PROVIDED "AS IS", WITHOUT WARRANTY OF ANY KIND, EXPRESS OR
% IMPLIED, INCLUDING BUT NOT LIMITED TO THE WARRANTIES OF MERCHANTABILITY,
% FITNESS FOR A PARTICULAR PURPOSE AND NONINFRINGEMENT. IN NO EVENT SHALL THE
% AUTHORS OR COPYRIGHT HOLDERS BE LIABLE FOR ANY CLAIM, DAMAGES OR OTHER
% LIABILITY, WHETHER IN AN ACTION OF CONTRACT, TORT OR OTHERWISE, ARISING FROM,
% OUT OF OR IN CONNECTION WITH THE SOFTWARE OR THE USE OR OTHER DEALINGS IN THE
% SOFTWARE.

%  BEGIN: Pliki
\subsection{Pliki}
Do tej pory wszystkie dane, z których korzystały nasze programy, wprowadzaliśmy bezpośrednio do kodu programu.
W realnych zastosowaniach bardzo często użyteczniejsze jest korzystanie z danych zapisanych w osobnych plikach.

\subsubsection{Zapisywanie tekstu do pliku}

Zapis do pliku tekstowego możemy zrealizować w sposób następujący:
\begin{CodeFrame*}[python]{}
plik = open('dane.txt', 'wt', encoding='utf8')
plik.write("teskt1\n")
plik.write("teskt2\nteskt3")
plik.close()
\end{CodeFrame*}

\noindent Jak to działa?
\begin{itemize}
\item Polecenie z pierwszej linijki otwiera plik \Verb{dane.txt} i zapewnia dostęp do niego poprzez zmienną \Verb{plik}.
      Opcja \python{'w'} oznacza, że plik jest otwarty ,,do zapisu'' (od angielskiego \textit{write}).
      Opcja \python{'t'} oznacza, że plik traktowany jako plik tekstowy\footnote{
        Tekst możemy zapisywać także do plików otwieranych jako binarne,
        w takim wypadku argument funkcji write musi mieć typ \Verb{bytes} a nie \Verb{str}, czyli być już jawnie zakodowanym w jakimś standardzie.
      }.
      Argument \Verb{encoding} pozwala na określenie kodowania użytego do zapisu pliku tekstowego,
        jest on opcjonalny i gdy nie zostanie podany kodowanie pliku zależne jest od ustawień systemowych.
\item Druga i trzecia komenda zapisuje podany jako argument tekst do pliku \Verb{dane.txt}
      (zwróć uwagę na wstawianie nowej linii przy pomocy \python{'\n'})
\item Ostatnie polecenie zamyka dostęp do pliku \Verb{dane.txt}.
\end{itemize}

Po uruchomieniu powyższego kodu powinien zostać utworzony plik ,,dane.txt'', zawierający 3 linie tekstu. Jeżeli plik taki wcześniej istniał zostanie on nadpisany.

\subsubsection{Wczytywanie tekstu z pliku}

\begin{CodeFrame*}[python]{}
plik = open('dane.txt', 'rt', encoding='utf8')
for linia in plik:
  print(linia, end="")
plik.close()
\end{CodeFrame*}

Zauważ, że została używa opcja \python{'r'} do otwarcia pliki co oznacza otwarcie do odczytu (od angielskiego \textit{read}). Jeżeli chcemy wczytać cały plik do zmiennej napisowej możemy, zamiast pętli czytającej kolejne linie, użyć metody \python{read()}:
\begin{CodeFrame*}[python]{}
plik = open('dane.txt', 'rt', encoding='utf8')
napis = plik.read()
plik.close()
\end{CodeFrame*}

Po otwartym pliku możemy się przemieszczać metodą \python{seek}, na przykład \python{plik.seek(0)} przesunie punkt odczytu na początek pliku i umożliwi jego ponowne przeczytanie.

\subsubsection{Czekanie na dane}

Niekiedy nasz program musi poczekać na jakieś dane (np. wprowadzane z standardowego wejścia przez użytkownika).
Typowo funkcje odczytu (takie jak \python{sys.stdin.read()}, \python{sys.stdin.readline()}, \python{input()}) czekają na koniec wczytywanych danych lub na koniec linii.
Komplikacja pojawia się kiedy chcielibyśmy aby nasz program miał ograniczenie czasowe takiego oczekiwania lub czekał na pojawienie się danych w jednym z kilku źródeł.
W takich przypadkach przydatna jest funkcja systemowa \python{select()}, którą w Pythonie znajdziemy w module \textit{select}.

\begin{CodeFrame*}[python]{}
import sys, os, select

rdfd, _, _ = select.select([sys.stdin], [], [], 3.0)

if not rdfd:
	print("czas minął")

for fd in rdfd:
	print("czytam z:", fd)
	a = os.read(fd.fileno(), 1024)
	print("wczytałem:", a)
\end{CodeFrame*}

Funkcja \python{select()} przyjmuje 3 listy „deskryptorów plików” (czyli tego co zwraca np. funkcja \python{open()}) oraz ilość sekund, którą ma czekać na początek danych. Pierwsza lista związana jest z plikami z których chcemy czytać, druga pisać, a trzecia z plikami na których czekamy na wyjątkowe warunki. Funkcja ta zwraca również 3 takie listy, ale zawierające jedynie deskryptory plików na których pożądana operacja jest możliwa (np. są dane do wczytania, można zapisać dane).

Funkcja \python{select()} kończy działanie gdy pojawią się jakiekolwiek dane (nie czeka na koniec danych – EOF). Zauważ, że do odczytu zastosowana została funkcja \python{os.read()} a nie metoda \python{fd.read()}, wynika to z faktu, iż \python{fd.read()} czeka na EOF lub podaną ilość bajtów, a \python{os.read()} wczytuje to co jest dostępne i ogranicza jedynie maksymalną ilość wczytywanych danych (resztę możemy doczytać kolejnym wywołaniem).

\begin{teacherOnly} Można pokazać różnicę w zachowaniu poniższego kodu z os.read() i fd.read() dla różnej ilości wprowadzanych znaków – np. 1 i 5
\begin{Verbatim}
import sys, os, select, time

while True:
	time.sleep(3)
	rdfd, _, _ = select.select([sys.stdin], [], [], 3.0)
	if not rdfd:
		print("czas minął")
	for fd in rdfd:
		print("czytam z:", fd)
		#a = os.read(fd.fileno(), 3)
		a = fd.read(3)
		print("wczytałem:", a)
\end{Verbatim}
\end{teacherOnly}
%  END: Pliki

\insertZadanie{booklets-sections/python/zadania-50_60-fork_pliki.tex}{zadanie_select}{}
\insertZadanie{booklets-sections/python/zadania-50_60-fork_pliki.tex}{zapis_do_pliku}{}
