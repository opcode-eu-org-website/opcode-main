% Copyright (c) 2018-2020 Matematyka dla Ciekawych Świata (http://ciekawi.icm.edu.pl/)
% Copyright (c) 2018-2020 Robert Ryszard Paciorek <rrp@opcode.eu.org>
% 
% MIT License
% 
% Permission is hereby granted, free of charge, to any person obtaining a copy
% of this software and associated documentation files (the "Software"), to deal
% in the Software without restriction, including without limitation the rights
% to use, copy, modify, merge, publish, distribute, sublicense, and/or sell
% copies of the Software, and to permit persons to whom the Software is
% furnished to do so, subject to the following conditions:
% 
% The above copyright notice and this permission notice shall be included in all
% copies or substantial portions of the Software.
% 
% THE SOFTWARE IS PROVIDED "AS IS", WITHOUT WARRANTY OF ANY KIND, EXPRESS OR
% IMPLIED, INCLUDING BUT NOT LIMITED TO THE WARRANTIES OF MERCHANTABILITY,
% FITNESS FOR A PARTICULAR PURPOSE AND NONINFRINGEMENT. IN NO EVENT SHALL THE
% AUTHORS OR COPYRIGHT HOLDERS BE LIABLE FOR ANY CLAIM, DAMAGES OR OTHER
% LIABILITY, WHETHER IN AN ACTION OF CONTRACT, TORT OR OTHERWISE, ARISING FROM,
% OUT OF OR IN CONNECTION WITH THE SOFTWARE OR THE USE OR OTHER DEALINGS IN THE
% SOFTWARE.

\begin{ProTip}[breakable]{Reguły DRY i KISS}
\textbf{„Don't Repeat Yourself”} (\textit{nie powtarzaj się}) jest jedną z dwóch głównych reguł programistycznych (ale ma także pewne zastosowania w innych dziedzinach techniki).
Zaleca ona unikanie potarzania tych samych czynności, czy też tworzenia takich samych, a nawet analogicznych, podobnych fragmentów kodu.

Narzędziami ułatwiającymi realizację tego celu są m.in.:
\begin{itemize}
\item systemy i skrypty służące automatyzacji różnego rodzaju czynności (takich jak np. kompilacja, instalacja, aktualizacja, monitoring działania) –
      zarówno systemy takie jak make, cmake, doxygen ale również wszystkie drobne skrypty (np. shellowe czy pythonowe) tworzone w tym celu w codziennej pracy informatyka
\item elementy składniowe (m.in. takie jak pętle i funkcje) oraz mechanizmy (np. polimorfizm) dostępne w językach programowania pozwalające na eliminację powtórzeń kodu
\item biblioteki, moduły, itp pozwalające na współdzielenie tych samych rozwiązań, tego samego kodu, pomiędzy różnymi projektami
\item elementy biblioteki systemowej pozwalające na wywoływanie innych programów (np. exec) i komunikację z nimi (np. poprzez strumienie wejścia/wyjścia)
\end{itemize}

Unikanie powtórzeń takiego samego lub (co często nawet gorsze) tylko nieznacznie zmienionego kodu jest też szczególnie istotne ze względu na łatwość utrzymania kodu
– np. jakąś poprawkę wprowadza się tylko w odpowiednio sparametryzowanej funkcji, a nie kilkunastu podobnych (ale nie identycznych, ze względu na brak parametryzacji) fragmentach kodu.

W zastosowaniach nie programistycznych przejawia się często wydzielaniem modułów i dążeniem do ich powtarzalności, redukcji ilości ich typów (np. dzięki parametryzacji, czy konfigurowalności).

\vspace{7pt}

Drugą, nawet chyba ważniejszą, z tych dwóch reguł jest \textbf{„Keep It Simple, Stupid”} (niekiedy \textit{Keep It Small and Simple}), którą można streścić jako \textit{proste jest lepsze}.
Reguła KISS jest bardziej ogólna (można nawet powiedzieć że wynika z niej reguła DRY), posiada dużo szersze pole zastosowań (także nie technicznych) i może być uważana za implementację \textit{Brzytwy Ockhama} w inżynierii.
Zaleca ona m.in.:
\begin{itemize}
\item tworzenie przejrzystych, czytelnych i prostych rozwiązań (zarówno pod względem samego projektu, koncepcji, jak też ich implementacji, wykonania)
\item wybór rozwiązania prostszego spośród (równie) skutecznych rozwiązań jakiegoś problemu
\item myślenie o łatwości późniejszego utrzymania i serwisu tworzonego rozwiązania (czy to kodu programu, czy urządzenia elektronicznego, a nawet budynku)
\end{itemize}

W duchu prostoty nakazywanej regułą KISS należy także starać się trzymać powszechnie stosowanych standardów
(gdy tylko jest to możliwe i nie powoduje zbyt wielkiej komplikacji naszego projektu),
zamiast każdorazowo tworzyć nowe, własne standardy, protokoły czy interfejsy.
\end{ProTip}
