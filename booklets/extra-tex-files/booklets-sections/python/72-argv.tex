% Copyright (c) 2021 Matematyka dla Ciekawych Świata (http://ciekawi.icm.edu.pl/)
% Copyright (c) 2021 Robert Ryszard Paciorek <rrp@opcode.eu.org>
% 
% MIT License
% 
% Permission is hereby granted, free of charge, to any person obtaining a copy
% of this software and associated documentation files (the "Software"), to deal
% in the Software without restriction, including without limitation the rights
% to use, copy, modify, merge, publish, distribute, sublicense, and/or sell
% copies of the Software, and to permit persons to whom the Software is
% furnished to do so, subject to the following conditions:
% 
% The above copyright notice and this permission notice shall be included in all
% copies or substantial portions of the Software.
% 
% THE SOFTWARE IS PROVIDED "AS IS", WITHOUT WARRANTY OF ANY KIND, EXPRESS OR
% IMPLIED, INCLUDING BUT NOT LIMITED TO THE WARRANTIES OF MERCHANTABILITY,
% FITNESS FOR A PARTICULAR PURPOSE AND NONINFRINGEMENT. IN NO EVENT SHALL THE
% AUTHORS OR COPYRIGHT HOLDERS BE LIABLE FOR ANY CLAIM, DAMAGES OR OTHER
% LIABILITY, WHETHER IN AN ACTION OF CONTRACT, TORT OR OTHERWISE, ARISING FROM,
% OUT OF OR IN CONNECTION WITH THE SOFTWARE OR THE USE OR OTHER DEALINGS IN THE
% SOFTWARE.

\begin{ProTip}[breakable]{Argumenty linii poleceń \zaawansowane{20}}
Tworzone przez nas skrypty pythonowe mogą przyjmować (tak jak różne inne poznane programy) argumenty (i opcje) w linii poleceń.
Aby mieć dostęp do listy argumentów przekazanych w linii poleceń należy skorzystać z \Verb#sys.argv#:

\begin{CodeFrame}[python]{.5\textwidth}
import sys
print(sys.argv)
\end{CodeFrame}
\begin{CodeFrame}{auto}
$ python3 /tmp/skrypt.py aa tt
['/tmp/skrypt.py', 'aa', 'tt']
\end{CodeFrame}

Jak możemy zauważyć jest to standardowa pythonowa lista zawierająca w kolejnych swoich elementach nazwę z którą został wywołany nasz skrypt i kolejne przekazane do niego argumenty.

Moduł argparse dostarcza dość wygodny parser opcji w standardowej notacji (długie z dwoma myślnikami, krótkie z jednym, itd.):

\begin{CodeFrame}[python]{.5\textwidth}
import argparse
parser = argparse.ArgumentParser()
parser.add_argument(
  '-v', "--verbose", action="store_true",
  help='opcja typu przełącznik'
)
parser.add_argument(
  'ARG', nargs='?',
  help='argument pozycyjny (opcjonalny)'
)
args = parser.parse_args()
print(args)
\end{CodeFrame}
\begin{CodeFrame}{auto}
$ python3 /tmp/skrypt.py aa tt
usage: skrypt.py [-h] [-v] [ARG]
skrypt.py: error: unrecognized arguments: tt
$ python3 /tmp/skrypt.py -v aa
Namespace(ARG='aa', verbose=True)
\end{CodeFrame}

To tylko mały przykład możliwości argparse, a szczegóły (jak zwykle) – w dokumentacji.
\end{ProTip}
