% Copyright (c) 2018-2020 Matematyka dla Ciekawych Świata (http://ciekawi.icm.edu.pl/)
% Copyright (c) 2018-2020 Robert Ryszard Paciorek <rrp@opcode.eu.org>
% 
% MIT License
% 
% Permission is hereby granted, free of charge, to any person obtaining a copy
% of this software and associated documentation files (the "Software"), to deal
% in the Software without restriction, including without limitation the rights
% to use, copy, modify, merge, publish, distribute, sublicense, and/or sell
% copies of the Software, and to permit persons to whom the Software is
% furnished to do so, subject to the following conditions:
% 
% The above copyright notice and this permission notice shall be included in all
% copies or substantial portions of the Software.
% 
% THE SOFTWARE IS PROVIDED "AS IS", WITHOUT WARRANTY OF ANY KIND, EXPRESS OR
% IMPLIED, INCLUDING BUT NOT LIMITED TO THE WARRANTIES OF MERCHANTABILITY,
% FITNESS FOR A PARTICULAR PURPOSE AND NONINFRINGEMENT. IN NO EVENT SHALL THE
% AUTHORS OR COPYRIGHT HOLDERS BE LIABLE FOR ANY CLAIM, DAMAGES OR OTHER
% LIABILITY, WHETHER IN AN ACTION OF CONTRACT, TORT OR OTHERWISE, ARISING FROM,
% OUT OF OR IN CONNECTION WITH THE SOFTWARE OR THE USE OR OTHER DEALINGS IN THE
% SOFTWARE.

\begin{itemize}
\item \emph{The Python Tutorial} (\url{https://docs.python.org/3/tutorial/}) - oficjalny Tutorial Pythona.
\item \emph{Biblioteka Riklaunima: Podstawy Pythona} (\url{http://www.python.rk.edu.pl/w/p/podstawy/}).
\item \emph{A Byte of Python} (\url{https://python.swaroopch.com/}).
\item \emph{How to Think Like a Computer Scientist: Learning with Python 3} (\url{http://openbookproject.net/thinkcs/python/english3e/}).
\item \emph{Zanurkuj w Pythonie} (\url{https://pl.wikibooks.org/wiki/Zanurkuj_w_Pythonie}).
\end{itemize}
