% Copyright (c) 2016-2020 Matematyka dla Ciekawych Świata (http://ciekawi.icm.edu.pl/)
% Copyright (c) 2016-2017 Łukasz Mazurek
% Copyright (c) 2018-2020 Robert Ryszard Paciorek <rrp@opcode.eu.org>
% 
% MIT License
% 
% Permission is hereby granted, free of charge, to any person obtaining a copy
% of this software and associated documentation files (the "Software"), to deal
% in the Software without restriction, including without limitation the rights
% to use, copy, modify, merge, publish, distribute, sublicense, and/or sell
% copies of the Software, and to permit persons to whom the Software is
% furnished to do so, subject to the following conditions:
% 
% The above copyright notice and this permission notice shall be included in all
% copies or substantial portions of the Software.
% 
% THE SOFTWARE IS PROVIDED "AS IS", WITHOUT WARRANTY OF ANY KIND, EXPRESS OR
% IMPLIED, INCLUDING BUT NOT LIMITED TO THE WARRANTIES OF MERCHANTABILITY,
% FITNESS FOR A PARTICULAR PURPOSE AND NONINFRINGEMENT. IN NO EVENT SHALL THE
% AUTHORS OR COPYRIGHT HOLDERS BE LIABLE FOR ANY CLAIM, DAMAGES OR OTHER
% LIABILITY, WHETHER IN AN ACTION OF CONTRACT, TORT OR OTHERWISE, ARISING FROM,
% OUT OF OR IN CONNECTION WITH THE SOFTWARE OR THE USE OR OTHER DEALINGS IN THE
% SOFTWARE.

\IfStrEq{\dbEntryID}{}{
	\insertZadanie{\currfilepath}{obiektowo_string}{}
	\insertZadanie{\currfilepath}{licz_powtorzenia}{}
	\insertZadanie{\currfilepath}{funkcja_jako_argument}{}
}


\dbEntryBegin{obiektowo_string}\if1\dbEntryCheckResults
Zapoznaj się z dokumentacją klasy odpowiedzialnej za napisy (\Verb{str}),
zwróć szczególną uwagę na metody \Verb{split}, \Verb{find}, \Verb{replace}.
Korzystając z metod klasy \Verb{str} napisz funkcję \Verb{parse} która dla napisu będącego jej argumentem
	wykona zamianę wszystkich ciągów "XY" na spację oraz
	dokona rozbicia napisu złożonego z pól rozdzielanych dwukropkiem na listę napisów odpowiadających poszczególnym polom.
Funkcja powinna działać w następujący sposób:
\begin{Verbatim}
 > l = parse("Ala:maXYkota:i inne:zwierzeta")
 > print(l)
['Ala', 'ma kota', 'i inne', 'zwierzeta']
\end{Verbatim}
\fi

\dbEntryBegin{licz_powtorzenia}\if1\dbEntryCheckResults
Napisz funkcję \Verb{zlicz} która dla podanej listy policzy powtórzenia jej elementów. Przykład użycia:
\begin{Verbatim}
 > zlicz(["AX", "B", "AX"])
AX wystepuje 2 razy
B wystepuje 1 razy
\end{Verbatim}
Wskazówka: Użyj słownika, w którym element będzie stanowił klucz, a krotność jego wystąpień wartość.
Możesz użyć metody \python{get()} do pobierania wartości z słownika, jeżeli w nim jest lub wartości domyślnej w przeciwnym wypadku - szczegóły zobacz w dokumentacji
\fi

\dbEntryBegin{funkcja_jako_argument}\if1\dbEntryCheckResults
Napisz funkcję która przyjmuje dwa argumenty: listę oraz funkcję. Funkcja ma za zadanie wykonać przekazaną do niej funkcję na każdym elemencie listy. Przykład użycia:
\begin{Verbatim}
>>> wykonaj([1,2,3], print)
1
2
3
\end{Verbatim}
\fi
