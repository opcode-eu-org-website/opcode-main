% Copyright (c) 2018-2020 Matematyka dla Ciekawych Świata (http://ciekawi.icm.edu.pl/)
% Copyright (c) 2018-2020 Robert Ryszard Paciorek <rrp@opcode.eu.org>
% 
% MIT License
% 
% Permission is hereby granted, free of charge, to any person obtaining a copy
% of this software and associated documentation files (the "Software"), to deal
% in the Software without restriction, including without limitation the rights
% to use, copy, modify, merge, publish, distribute, sublicense, and/or sell
% copies of the Software, and to permit persons to whom the Software is
% furnished to do so, subject to the following conditions:
% 
% The above copyright notice and this permission notice shall be included in all
% copies or substantial portions of the Software.
% 
% THE SOFTWARE IS PROVIDED "AS IS", WITHOUT WARRANTY OF ANY KIND, EXPRESS OR
% IMPLIED, INCLUDING BUT NOT LIMITED TO THE WARRANTIES OF MERCHANTABILITY,
% FITNESS FOR A PARTICULAR PURPOSE AND NONINFRINGEMENT. IN NO EVENT SHALL THE
% AUTHORS OR COPYRIGHT HOLDERS BE LIABLE FOR ANY CLAIM, DAMAGES OR OTHER
% LIABILITY, WHETHER IN AN ACTION OF CONTRACT, TORT OR OTHERWISE, ARISING FROM,
% OUT OF OR IN CONNECTION WITH THE SOFTWARE OR THE USE OR OTHER DEALINGS IN THE
% SOFTWARE.

\IfStrEq{\dbEntryID}{}{
	\ifdefined\noExtraInfoMode\else
		\subsection{fork i pliki}
	\fi
	
	\insertZadanie{\currfilepath}{zadanie_select}{}
	\insertZadanie{\currfilepath}{zadanie_fork}{}
	\insertZadanie{\currfilepath}{zapis_do_pliku}{}
}


\dbEntryBegin{zadanie_select}\if1\dbEntryCheckResults
Napisz funkcję, który wczytuje dane z standardowego wejścia. Funkcja powinna przyjmować jeden argument określający maksymalny czas oczekiwania na kolejną porcję danych.
Każde pojawienie się danych wejściowych powinno resetować odliczanie timeoutu podanego w argumencie. Po skutecznym upływie tego timeoutu funkcja powinna zwrócić wszystkie wczytane dane.
\\\textit{Wskazówka: zmodyfikuj przykład użycia funkcji select() podany w skrypcie.}
\fi

\dbEntryBegin{zadanie_select-rozwiazanie}\if1\dbEntryCheckResults
\begin{minted}[frame=single]{python}
import sys, os, select

def czytaj(timeout):
  bufor = ""
  while True:
    rdfd, _, _ = select.select([sys.stdin], [], [], timeout)

    if not rdfd:
      return bufor

    for fd in rdfd:
      bufor += os.read(fd.fileno(), 1024).decode()

czytaj(13)
\end{minted}

\noindent Zwróć uwagę że:
\begin{itemize}
\item funkcja w nieskończonej pętli wykonuje \python{select} z ustawioną wartością timeoutu
\item w oparciu o wynik działania select funkcja rozróżnia przypadek timeoutu (rdfd jest \python{None}, czyli \python{not rdfd} jest prawdą) i zwraca w tej sytuacji wczytane wcześniej dane
\item jeżeli nie było timeoutu, a pojawiły się jakieś dane (rdfd nie jest \python{None}) to funkcja wczytuje je z użyciem funkcji read
\item jako że funkcja read wymaga określenia jakiegoś skończonej wielkości bufora, to do wczytywania danych użyta jest pętla for, co zapewnia wczytanie wszystkich dostępnych w danym momencie danych
\item na wczytanych danych użyta jest metoda \python{decode} w celu zamiany ciągu bitowego na napis i tak przekonwertowane dane dodawane są do bufora napisowego (o dynamicznie dostosowywanej przez Pythona długości)
\end{itemize}
\fi


\dbEntryBegin{zadanie_fork}\if1\dbEntryCheckResults
Napisz program który utworzy 1 potomka, rodzić powinien wypisać PID potomka i swój. Natomiast potomek powinien utworzyć kolejny proces w którym zostanie uruchomiona komenda \Verb#ps -f# w taki sposób aby potomek odebrał do zmiennej jej standardowe wyjście i wypisał je na ekran.
\fi

\dbEntryBegin{zadanie_fork-rozwiazanie}\if1\dbEntryCheckResults
\begin{minted}[frame=single]{python}
import os, subprocess

pid = os.fork()

if pid == 0:
   res = subprocess.run(["ps", "-f"], stdout=subprocess.PIPE)
   print("Standardowe wyjście z komendy to: " + res.stdout.decode())
else:
  print("Mój PID to ", os.getpid(), "PID mojego potomka to:", pid)
\end{minted}

\noindent Zwróć uwagę że:
\begin{itemize}
\item korzystamy z funkcji fork do rozdzielania procesu na dwa (rodzica i potomka)
\item każdy z tych procesów zaczyna wykonywanie od wyjścia z funkcji fork, przy czym różni się w nich wartość, którą ta funkcja zwróciła
\item wartości tej używamy do rozróżnienia rodzica od potomka
\item w rodzicu (który otrzymał niezerową wartość, będącą numerem PID utworzonego potomka) wypisujemy stosowny komunikat na temat numerów PID
\item w potomku używamy \python{subprocess.run} do uruchomienia wskazanego polecenia w kolejnym potomku i przechwycenia jego wyjścia (opcjonalny argument \python{stdout=subprocess.PIPE})
\item po zakończeniu działania subprocess wypisujemy dane otrzymane na standardowym wyjściu uruchomionego polecenia
\item dane te w ogólności są danymi binarnymi i Python używa typu ciągu bajtowego do ich przechowywania, my wiemy że ps wypisuje tekst na standardowym wyjściu zatem korzystamy z metody \python{decode} do jego konwersji na napis
\end{itemize}
\fi


\dbEntryBegin{zapis_do_pliku}\if1\dbEntryCheckResults
Napisz funkcję \Verb#zapisz# która przyjmuje dwa argumenty: słownik oraz nazwę pliku. Funkcja ma utworzyć plik o podanej nazwie i zapisać do niego otrzymany słownik, w taki sposób że każda linii odpowiada jednej parze klucz wartość, a separatorem pomiędzy kluczem a wartością jest znak tabulacji. Dla uproszczenia zakładamy że elementy słownika są napisami (zarówno klucze jak i wartości) i nie zawierają znaków nowej linii ani tabulacji.

Na przykład dla wywołania \python{zapisz({"a": "qwe", "d": "123"}, "xx")} funkcja powinna utworzyć plik z zawartością:
\vspace{-8pt}\begin{Verbatim}
a	qwe
d	123
\end{Verbatim}
\fi

\dbEntryBegin{zapis_do_pliku-rozwiazanie}\if1\dbEntryCheckResults
\begin{minted}[frame=single]{python}
def zapisz(slownik, nazwa):
  plik = open(nazwa, 'wt', encoding='utf8')
  for klucz in slownik:
    plik.write(klucz + "\t" + slownik[klucz] +"\n")
  plik.close()
\end{minted}

\noindent Zwróć uwagę że:
\begin{itemize}
\item funkcja używa open z argumentami wskazującymi otwarcie pliku tekstowego w trybie do zapisu, w tym przykładzie podaliśmy także kodowanie, ale jest to zbędne gdyż podany utf8 byłby i tak użyty domyślnie
\item następnie w ramach pętli przechodzącej po przekazanym do funkcji słowniku (dokładnie po kluczach w tym słowniku) wywołujemy funkcję write i podajemy do niej odpowiednio przygotowany napis (jako że plik otworzyliśmy w trybie tekstowym, write przyjmuje napisy)
\item po wpisaniu wszystkich danych dokonujemy zamknięcia pliku (bez wykonania tej operacji część danych mogłaby nie być fizycznie zapisana w pliku w momencie kończenia funkcji)
\end{itemize}
\fi
