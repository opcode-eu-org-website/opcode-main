% Copyright (c) 2016-2020 Matematyka dla Ciekawych Świata (http://ciekawi.icm.edu.pl/)
% Copyright (c) 2016-2017 Łukasz Mazurek
% Copyright (c) 2018-2020 Robert Ryszard Paciorek <rrp@opcode.eu.org>
% 
% MIT License
% 
% Permission is hereby granted, free of charge, to any person obtaining a copy
% of this software and associated documentation files (the "Software"), to deal
% in the Software without restriction, including without limitation the rights
% to use, copy, modify, merge, publish, distribute, sublicense, and/or sell
% copies of the Software, and to permit persons to whom the Software is
% furnished to do so, subject to the following conditions:
% 
% The above copyright notice and this permission notice shall be included in all
% copies or substantial portions of the Software.
% 
% THE SOFTWARE IS PROVIDED "AS IS", WITHOUT WARRANTY OF ANY KIND, EXPRESS OR
% IMPLIED, INCLUDING BUT NOT LIMITED TO THE WARRANTIES OF MERCHANTABILITY,
% FITNESS FOR A PARTICULAR PURPOSE AND NONINFRINGEMENT. IN NO EVENT SHALL THE
% AUTHORS OR COPYRIGHT HOLDERS BE LIABLE FOR ANY CLAIM, DAMAGES OR OTHER
% LIABILITY, WHETHER IN AN ACTION OF CONTRACT, TORT OR OTHERWISE, ARISING FROM,
% OUT OF OR IN CONNECTION WITH THE SOFTWARE OR THE USE OR OTHER DEALINGS IN THE
% SOFTWARE.

%  BEGIN: Słowniki
\subsection{Słowniki}

Kolejnym użytecznym typem zmiennych w Pythonie są słowniki (zwane niekiedy \emph{mapami} lub \emph{tablicami asocjacyjnymi}). Podobnie jak listy służą do do przechowywania innych zmiennych.
W odróżnieniu jednak od list w słownikach przechowywane są pary klucz - wartość, gdzie unikalny klucz służy do identyfikowania wartości.
Zwróć uwagę na analogię z normalnymi słownikami klucz to słowo które wyszukujemy, a wartość to jego opis.

\begin{CodeFrame}[python]{0.60\textwidth}
slownik = { "bd" : "xx", 5: True, "a" : 11 }
for klucz in slownik:
    print (klucz, "=>", slownik[klucz])
\end{CodeFrame}
\begin{CodeFrame}{auto}
a => 11
bd => xx
5 => True
\end{CodeFrame}

Zauważ że zarówno klucz, jak i wartość mogą być dowolnego typu oraz że słownik nie zachowuje kolejności dodawania elementów.

Możliwe jest także sprawdzanie istnienia jakiegoś elementu w słowniku, usuwanie, dodawanie i zmienianie elementów słowniku, itd
(zwróć także uwagę na inną metodę wypisywania słownika - poprzednio iterowaliśmy po kluczach, teraz po liście par klucz-wartość):

\begin{CodeFrame}[python]{0.60\textwidth}
if "bd" in slownik:
    print ("jest element o kluczu 'bd'")
    del slownik['bd']
slownik[15] = True
slownik["a"] = "yy"
for k,v in m.items():
    print (k, "=>", v)
\end{CodeFrame}
\begin{CodeFrame}{auto}
jest element o kluczu 'bd'
a => yy
15 => True
\end{CodeFrame}
%  END: Słowniki

%  BEGIN: Sortowanie słownika
\subsubsection{Sortowanie słownika}

Jak już wspomnieliśmy słownik nie zachowuje porządku elementów. Jeżeli chcemy uzyskać posortowaną listę kluczy,
wartości lub par klucz-wartość z słownika możemy skorzystać z funkcji \python{sorted()}.
W przypadku par wywołanie będzie wyglądać następująco:

\begin{CodeFrame}[python]{0.50\textwidth}
mapa = {'5': 3, 'bd': 20, 'a': 101}
lista = sorted( mapa.items() )
print(lista)
\end{CodeFrame}
\begin{CodeFrame}{auto}
[('5', 3), ('a', 101), ('bd', 20)]
\end{CodeFrame}

Zwróć uwagę, iż użyliśmy tej samej metody \python{items()}, z której korzystaliśmy do iterowania po parach klucz-wartość
(dla listy samych kluczy lub wartości należy użyć w tym miejscu innej metody klasy \Verb{dict}).
Zapewne zauważyłeś że sortowanie zostało przeprowadzone w oparciu o klucze, co jednak jeżeli chcielibyśmy posortować taką listę w oparciu o wartości?
W takim przypadku możemy skorzystać z opcjonalnego argumentu funkcji \python{sorted()} o nazwie \Verb{key},
który przyjmuje funkcję mającą za zadanie na podstawie otrzymanego elementu listy (w tym wypadku pary klucz - wartość) zwrócić klucz sortowania:

\begin{CodeFrame}[python]{0.50\textwidth}
mapa = {'5': 3, 'bd': 20, 'a': 101}
def k(x):
    return x[1]
lista = sorted( mapa.items(), key=k )
print(lista)
\end{CodeFrame}
\begin{CodeFrame}{auto}
[('5', 3), ('bd', 20), ('a', 101)]
\end{CodeFrame}
%  END: Sortowanie słownika

%  BEGIN: Funkcje jako argumenty
\subsection{Funkcje jako argumenty funkcji \zaawansowane{20}}\label{Funkcje_jako_argumenty}

W powyższym przykładzie jednym za argumentów funkcji \python{sorted()} jest inna funkcja.
Zauważ, że funkcja może być takim samym argumentem innej funkcji jak dowolna inna zmienna,
może być też wynikiem zwracanym przez funkcję oraz może być przechowywana w zmiennej.

\begin{CodeFrame}[python]{0.50\textwidth}
def dzialanie(operacja):
    if operacja == "dodaj":
        def f(a, b):
            return a+b
        return f
    elif operacja == "mnóż":
        def f(a, b):
            return a*b
        return f
def dwa(funkcja, argument):
    return funkcja(2, argument)

d = dzialanie("dodaj")
a = dwa(d, 11)
b = dzialanie("mnóż")(3,4)
print(a, b, d(3,4))
\end{CodeFrame}
\begin{minipage}[t]{0.46\textwidth}
\begin{Verbatim}[frame=single]
13 12 7
\end{Verbatim}

\vspace{6pt}\noindent Zauważ że:
\begin{itemize}[leftmargin=7mm]
\item wynikiem funkcji \python{dzialanie()} jest funkcja wykonująca wskazane działanie,
\item funkcja \python{dwa()} jako argumenty przyjmuje funkcję realizującą działanie dwuargumentowe i jeden argument przekazywany do niej,
\item zmienna d wskazuje na funkcję zwróconą przez funkcję \python{dzialanie()} i może być używana jako funkcja.
\end{itemize}
\end{minipage}
%  END: Funkcje jako argumenty
