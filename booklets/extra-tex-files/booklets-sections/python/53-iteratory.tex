% Copyright (c) 2018-2020 Matematyka dla Ciekawych Świata (http://ciekawi.icm.edu.pl/)
% Copyright (c) 2018-2020 Robert Ryszard Paciorek <rrp@opcode.eu.org>
% 
% MIT License
% 
% Permission is hereby granted, free of charge, to any person obtaining a copy
% of this software and associated documentation files (the "Software"), to deal
% in the Software without restriction, including without limitation the rights
% to use, copy, modify, merge, publish, distribute, sublicense, and/or sell
% copies of the Software, and to permit persons to whom the Software is
% furnished to do so, subject to the following conditions:
% 
% The above copyright notice and this permission notice shall be included in all
% copies or substantial portions of the Software.
% 
% THE SOFTWARE IS PROVIDED "AS IS", WITHOUT WARRANTY OF ANY KIND, EXPRESS OR
% IMPLIED, INCLUDING BUT NOT LIMITED TO THE WARRANTIES OF MERCHANTABILITY,
% FITNESS FOR A PARTICULAR PURPOSE AND NONINFRINGEMENT. IN NO EVENT SHALL THE
% AUTHORS OR COPYRIGHT HOLDERS BE LIABLE FOR ANY CLAIM, DAMAGES OR OTHER
% LIABILITY, WHETHER IN AN ACTION OF CONTRACT, TORT OR OTHERWISE, ARISING FROM,
% OUT OF OR IN CONNECTION WITH THE SOFTWARE OR THE USE OR OTHER DEALINGS IN THE
% SOFTWARE.

%  BEGIN: Iteratory i generatory
\subsection{Iteratory i generatory \zaawansowane{**}}

Iterator jest obiektem pozwalającym na dostęp do kolejnych elementów jakiejś kolekcji (np. listy).
Są one przydatne np. gdy chcemy uzyskiwać kolejne elementy kolekcji nie iterując po niej w ramach pętli \python{for}.
Jego użycie wygląda następująco:

\begin{CodeFrame*}[python]{}
l = [6, 7, 8, 9]
i = iter(l)  # zmienna i jest tutaj iteratorem
print( next(i) )
print( next(i) )
\end{CodeFrame*}

Niekiedy zamiast tworzenia listy lepsze może być uzyskiwanie jej kolejnych elementów "na żywo".
Funkcjonalność taką w pythonie zapewniają generatory.
Są to funkcje które zwracają kolejne elementy danej kolekcji używając słowa kluczowego \python{yield}, zamiast \python{return}.
Pamiętają one też swój stan wewnętrzny pomiędzy wywołaniami w ramach poszczególnych iteracji.

Generatory możemy używać np. do iterowania po nich w pętli \python{for},
możemy tez używać iteratorów do pobierania kolejnych wartości z generatora:

\begin{CodeFrame*}[python]{}
def f(l):
    a, b = 0, 1
    for i in range(l):
        yield a
        a, b = b, a + b

ii = iter( f(8) )
for i in f(16):
    print("i =", i)
    if i > 6:
        print("ii =", next(ii))
\end{CodeFrame*}

Można także tworzyć generatory nieskończone:

\begin{CodeFrame*}[python]{}
def ff():
    a, b = 0, 1
    while True:
      yield a
      a, b = b, a + b
\end{CodeFrame*}
%  END: Iteratory i generatory
