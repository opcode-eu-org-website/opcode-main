% Copyright (c) 2018-2020 Matematyka dla Ciekawych Świata (http://ciekawi.icm.edu.pl/)
% Copyright (c) 2018-2020 Robert Ryszard Paciorek <rrp@opcode.eu.org>
% 
% MIT License
% 
% Permission is hereby granted, free of charge, to any person obtaining a copy
% of this software and associated documentation files (the "Software"), to deal
% in the Software without restriction, including without limitation the rights
% to use, copy, modify, merge, publish, distribute, sublicense, and/or sell
% copies of the Software, and to permit persons to whom the Software is
% furnished to do so, subject to the following conditions:
% 
% The above copyright notice and this permission notice shall be included in all
% copies or substantial portions of the Software.
% 
% THE SOFTWARE IS PROVIDED "AS IS", WITHOUT WARRANTY OF ANY KIND, EXPRESS OR
% IMPLIED, INCLUDING BUT NOT LIMITED TO THE WARRANTIES OF MERCHANTABILITY,
% FITNESS FOR A PARTICULAR PURPOSE AND NONINFRINGEMENT. IN NO EVENT SHALL THE
% AUTHORS OR COPYRIGHT HOLDERS BE LIABLE FOR ANY CLAIM, DAMAGES OR OTHER
% LIABILITY, WHETHER IN AN ACTION OF CONTRACT, TORT OR OTHERWISE, ARISING FROM,
% OUT OF OR IN CONNECTION WITH THE SOFTWARE OR THE USE OR OTHER DEALINGS IN THE
% SOFTWARE.

%  BEGIN: Podstawy programowania równoległego
\section{Podstawy programowania równoległego}

\subsection{procesy i fork()}

Aby w systemie mógł działać więcej niż 1 proces konieczna jest możliwość utworzenia nowego procesu (potomka) z poziomu procesu aktualnie działającego (rodzica). Możliwe są dwa podejścia:
\begin{itemize}
	\item utworzenie "czystego" procesu uruchamiającego podany kod programu z podanymi argumentami (spawn)
	\item utworzenie kopii aktualnego procesu, która zacznie wykonywać się niezależnie od momentu rozgałęzienia (fork)
\end{itemize}
W przypadku zastosowania fork proces potomny otrzymuje kopię pamięci rodzica (ma dostęp do wszystkich jego zmiennych oraz zasobów uzyskanych przed fork(); dalsze operacje na zmiennych są niezależne). Po utworzeniu kopi procesu można (ale nie trzeba) zastąpić wykonywany w nim program innym poprzez funkcje z rodziny exec. Cechy te powodują że mechanizm fork jest bardziej elastyczny od spawn.

\begin{CodeFrame*}[python]{}
import os

print("nasz pid to:", os.getpid())

pid = os.fork()
if pid == 0:
	print("jesteśmy w procesie potomnym, nasz pid to:", os.getpid())
else:
	print("jesteśmy w rodzicu, pid naszego potomka to:", pid)
\end{CodeFrame*}

\begin{teacherOnly}
Bardziej rozbudowany przykład do demonstacji życia procesów:
\begin{Verbatim}
import os, time, signal

print("nasz pid to:", os.getpid())

pid = os.fork()
if pid == 0:
	print("jesteśmy w procesie potomnym, nasz pid to:", os.getpid())
	time.sleep(4);  print("potomek 1")
	time.sleep(7);  print("potomek 2")
else:
	print("jesteśmy w rodzicu, pid naszego potomka to:", pid)
	time.sleep(5);  print("rodzic 1")
	time.sleep(4);  print("zabijam potomka")
	os.kill(pid, signal.SIGTERM)
	time.sleep(5);  print("rodzic 2")
\end{Verbatim}
\end{teacherOnly}

\subsection{wywołanie zewnętrznej komendy}

Najprostszym sposobem uruchomienia innej komendy z poziomu Pythona jest użycie funkcji \Verb{system()} z modułu \Verb{os}:

\begin{CodeFrame*}[python]{}
import os

inStr = "Ala ma kota\nKot ma psa\n..."

os.system('echo -en "' + inStr + '" | grep -v A')
\end{CodeFrame*}

Jak widać przekazujemy do niej napis takiej samej postaci jak wyglądałby komenda uruchamiana w terminalu.
Mechanizm ten nie daje jednak zbyt dużej kontroli nad uruchamianiem tego polecenia
(nie pozwala na proste odebranie jego standardowego wyjścia, przekazanie wejścia również wymaga dodatkowego zabiegu w postaci dodania komendy echo, itd.).
Bardziej elastycznym rozwiązaniem jest pythonowy moduł \textit{subprocess}:

\begin{CodeFrame*}[python]{}
import subprocess

inStr = "Ala ma kota\nKot ma psa\n..."

# uruchamiamy subprocess z grep'em
res = subprocess.run(["grep", "-v", "A"], input=inStr.encode(), stdout=subprocess.PIPE)
print("Kod powrotu to: " + str(res.returncode))
print("Standardowe wyjście z komendy to: " + res.stdout.decode())
# warto zwrócić uwagę na kodowanie i dekodowanie napisów
# (przekazywanych/odbieranych przez stdin/stdout) do / z utf-8

# jeżeli chcemy korzystać np. z znaków uogólniających powłoki lub podać
# komendę jako pojedynczy napis (a nie listę argumentów) to można użyć
# opcji shell=True:
subprocess.run(["ls -ld /etc/pa*"], shell=True)
# jeżeli potrzebujemy tylko rozbicia napisu na listę argumentów można
# użyć shlex.split()

# run() pozwala także (obok subprocess.PIPE) na przekazywanie
# istniejących deskryptorów (lub subprocess.DEVNULL, co ignoruje wyjście)
# w ramach stdin, stdout, stderr

# moduł subprocess oferuje także funkcję Popen() dającą większą kontrolę
# nad uruchamianiem komendy
\end{CodeFrame*}


\subsection{komunikacja międzyprocesowa}

W systemie wieloprocesowym konieczne jest zapewnienie mechanizmów komunikacji pomiędzy procesami, zwłaszcza jeżeli grupa procesów ma realizować wspólne zadanie.

Jednym z takich mechanizmów (można powiedzieć że nawet podstawowym) jest poznane już wcześniej łącze nie nazwane
	(pipe, uzyskiwane np. w bashowej linii poleceń przy pomocy \Verb{|}) pozwalające na przekazywanie strumienia danych od jednego do kolejnego procesu.
Podobnie działa łącze nazwane z tym że nie jest uzyskiwane w wyniku funkcji \Verb{pipe()} a otwarcia specjalnego pliku (utworzonego \Verb{mkfifo()}) przez dwa procesy (jeden do czytania drugi do pisania).

Innymi mechanizmami komunikacji międzyprocesowej są m.in:
\begin{itemize}
	\item sygnały \teacher{o nich będziemy mówić na dalszych zajęciach}
	\item kolejki komunikatów
	\item pamięć współdzielona
\end{itemize}

Stosowanie pamięci współdzielonej wymaga często też stosowania mechanizmów ochrony dostępu do niej (wejścia do „krytycznych sekcji” kodu). Koncepcja takiej ochrony wygląda następująco:
\begin{Verbatim}
if !blokada:
	blokada = True
	# działania na pamięci wspólnej
	blokada = False
\end{Verbatim}
Jednak nie może być zrealizowana w tak prosty sposób, gdzyż przełączenie procesów może nastąpić pomiędzy sprawdzeniem warunku na zmiennej \Verb{blokada} a zmianą jej wartości
	(lub mogą one działać idealnie równolegle i w tym samym momencie sprawdzać wartość zmiennej \Verb{blokada}).
Dlatego do ochrony sekcji krytycznych stosuje się mechanizmy systemowe takie jak semafory i lock'i.

\subsection{wątki}

Oprócz możliwości pełnego rozgałęzienia procesu (utworzenia potomka), możliwe jest także tworzenie wątków (zwanych też lekkimi procesami) w ramach bieżącego procesu.
Wątek (w odróżnieniu od procesu potomnego) korzysta z tej samej pamięci (przestrzeni adresowej) co oryginalny proces i wszystkie inne jego wątki (czyli \textit{out of the box} mają pamięć współdzieloną).
Jednak każdy wątek posiada niezależny stos (umieszczany w innym fragmencie współdzielonej pamięci), który jest używany m.in. do przechowywania zmiennych lokalnych (w tym argumentów funkcji),
	czyli dopóki ograniczamy się do zmiennych lokalnych nie ma potrzeby stosowania ochrony sekcji krytycznych ze względu na dostęp do pamięci.

\subsection{„Python-way”}

Zaprezentowane powyżej podejście korzysta w dużej mierze z funkcji analogicznych do funkcji systemowych biblioteki standardowej C zgromadzonych w module \textit{os}.
Python oferuje obok wspomnianego modułu \textit{subprocess} także inne własne mechanizmy związane z tworzeniem wielu procesów poprzez moduł \textit{multiprocessing}
	oraz oferuje wsparcie dla wątków w module \textit{threading}\footnote{
		Należy mieć na uwadze iż pythonowe wątki są niepełnowartościowe - ze względu na konstrukcję interpretera CPython,
		jedynie jeden wątek w danej chwili może być aktywny - wykorzystywać CPU, pozostałe mogą jedynie czekać.
	}.
Jednak, jako że w ramach tego kursu nie będziemy zajmować się programowaniem równoległym jako takim, to modułów tych nie omówimy w tym skrypcie ani na zajęciach.
Zainteresowanym polecam zapoznanie się z \url{http://vip.opcode.eu.org/#Procesy_i_wątki}.
%  END: Podstawy programowania równoległego
