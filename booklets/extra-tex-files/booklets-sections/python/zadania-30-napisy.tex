% Copyright (c) 2016-2020 Matematyka dla Ciekawych Świata (http://ciekawi.icm.edu.pl/)
% Copyright (c) 2016-2017 Łukasz Mazurek
% Copyright (c) 2018-2020 Robert Ryszard Paciorek <rrp@opcode.eu.org>
% 
% MIT License
% 
% Permission is hereby granted, free of charge, to any person obtaining a copy
% of this software and associated documentation files (the "Software"), to deal
% in the Software without restriction, including without limitation the rights
% to use, copy, modify, merge, publish, distribute, sublicense, and/or sell
% copies of the Software, and to permit persons to whom the Software is
% furnished to do so, subject to the following conditions:
% 
% The above copyright notice and this permission notice shall be included in all
% copies or substantial portions of the Software.
% 
% THE SOFTWARE IS PROVIDED "AS IS", WITHOUT WARRANTY OF ANY KIND, EXPRESS OR
% IMPLIED, INCLUDING BUT NOT LIMITED TO THE WARRANTIES OF MERCHANTABILITY,
% FITNESS FOR A PARTICULAR PURPOSE AND NONINFRINGEMENT. IN NO EVENT SHALL THE
% AUTHORS OR COPYRIGHT HOLDERS BE LIABLE FOR ANY CLAIM, DAMAGES OR OTHER
% LIABILITY, WHETHER IN AN ACTION OF CONTRACT, TORT OR OTHERWISE, ARISING FROM,
% OUT OF OR IN CONNECTION WITH THE SOFTWARE OR THE USE OR OTHER DEALINGS IN THE
% SOFTWARE.

\IfStrEq{\dbEntryID}{}{
	\subsection{napisy}
	
	\insertZadanie{\currfilepath}{funkcja_wspak}{}
	\teacher{\insertRozwiazanie{\currfilepath}{funkcja_wspak}{}}
	\insertZadanie{\currfilepath}{funkcja_wyiksuj}{}
	\insertZadanie{\currfilepath}{dekodowanie_utf8_w_base64}{}
	
	\subsection{wyrażenia regularne}
	
	\insertZadanie{\currfilepath}{regex_czy_slowo}{}
	\insertZadanie{\currfilepath}{regex_czy_liczba}{}
}


%
% napisy
%

\dbEntryBegin{funkcja_wspak}\if1\dbEntryCheckResults
Napisz funkcję, która dla danej listy słów wypisze każde słowo z listy wspak. Np. dla listy \python{['Ala', 'ma', 'kota']} funkcja powinna wypisać:
\begin{Verbatim}
alA
am
atok
\end{Verbatim}
\fi

\dbEntryBegin{funkcja_wspak-rozwiazanie}\if1\dbEntryCheckResults
\begin{minted}[frame=single]{python}
def wskap(lista):
  for slowo in lista:
    # w pythonie zamiast poniższej pętli można prościej ...
    # ale warto poznać (także) takie rozwiązanie
    for i in range(len(slowo)):
      print(slowo[-1 - i], end = '')
    print()
\end{minted}
%
Prostszym rozwiązaniem (nie wymagającym jawnego pisania pętli w pętli) jest:
\begin{minted}[frame=single]{python}
def wskap(lista):
  for slowo in lista:
    print(slowo[::-1])
\end{minted}
\fi

\dbEntryBegin{funkcja_wyiksuj}\if1\dbEntryCheckResults
Napisz funkcję \python{wyiksuj(napis)}, która zwróci dany \Verb{napis}, zastępując każdą małą literę przez \Verb{x} i
każdą wielką literę przez \Verb{X}, natomiast resztę znaków pozostawi bez zmian.
Np. dla napisu \Verb{'Python 3.6.1 (default, Dec 2015, 13:05:11)'} funkcja powinna zwrócić napis: \Verb{Xxxxxx 3.6.1 (xxxxxxx, Xxx 2015, 13:05:11)}

\begin{teacherOnly}
\noindent Wskazówka:\\
Dla każdego znaku użyj konstrukcji \python{if}/\python{elif}/\python{else}, aby rozróżnić pomiędzy trzema przypadkami:
małe litery, wielkie litery, pozostałe znaki.
\end{teacherOnly}
\fi

\dbEntryBegin{dekodowanie_utf8_w_base64}\if1\dbEntryCheckResults
Napisz program dekodujący napis kodowany w UTF8 zakodowany przy pomocy base64 mający postać:
\python{b'UHl0aG9uIGplc3QgZmFqbnkg8J+Yjg==\n'}.\\
Wskazówka: dane wejściowe funkcji \python{decode()} muszą być typu "bytes", można to uzyskać poprzedzając napis prefiksem \Verb{b}, tak jak powyżej.
\fi


%
% wyrażenia regularne
%

\dbEntryBegin{regex_czy_liczba}\if1\dbEntryCheckResults
Napisz funkcję która sprawdzi z użyciem wyrażeń regularnych czy dany napis jest liczbą (tzn. jest złożony z cyfr i kropki, a na początku może wystąpić + albo -).
\fi

\dbEntryBegin{regex_czy_slowo}\if1\dbEntryCheckResults
Napisz funkcję która sprawdzi z użyciem wyrażeń regularnych czy dany napis jest słowem (tzn. nie zawiera spacji).
\fi
