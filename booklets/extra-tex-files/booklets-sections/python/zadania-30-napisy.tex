% Copyright (c) 2016-2020 Matematyka dla Ciekawych Świata (http://ciekawi.icm.edu.pl/)
% Copyright (c) 2016-2017 Łukasz Mazurek
% Copyright (c) 2018-2020 Robert Ryszard Paciorek <rrp@opcode.eu.org>
% 
% MIT License
% 
% Permission is hereby granted, free of charge, to any person obtaining a copy
% of this software and associated documentation files (the "Software"), to deal
% in the Software without restriction, including without limitation the rights
% to use, copy, modify, merge, publish, distribute, sublicense, and/or sell
% copies of the Software, and to permit persons to whom the Software is
% furnished to do so, subject to the following conditions:
% 
% The above copyright notice and this permission notice shall be included in all
% copies or substantial portions of the Software.
% 
% THE SOFTWARE IS PROVIDED "AS IS", WITHOUT WARRANTY OF ANY KIND, EXPRESS OR
% IMPLIED, INCLUDING BUT NOT LIMITED TO THE WARRANTIES OF MERCHANTABILITY,
% FITNESS FOR A PARTICULAR PURPOSE AND NONINFRINGEMENT. IN NO EVENT SHALL THE
% AUTHORS OR COPYRIGHT HOLDERS BE LIABLE FOR ANY CLAIM, DAMAGES OR OTHER
% LIABILITY, WHETHER IN AN ACTION OF CONTRACT, TORT OR OTHERWISE, ARISING FROM,
% OUT OF OR IN CONNECTION WITH THE SOFTWARE OR THE USE OR OTHER DEALINGS IN THE
% SOFTWARE.

\IfStrEq{\dbEntryID}{}{
	\subsection{napisy}
	
	\insertZadanie{\currfilepath}{funkcja_wspak}{}
	\insertZadanie{\currfilepath}{funkcja_wyiksuj}{}
	\insertZadanie{\currfilepath}{dekodowanie_utf8_w_base64}{}
	
	\subsection{wyrażenia regularne}
	
	\insertZadanie{\currfilepath}{regex_czy_slowo}{}
	\insertZadanie{\currfilepath}{regex_czy_liczba}{}
}

\IfStrEq{\dbEntryID}{rozwiazania}{
	\insertRozwiazanie{\currfilepath}{funkcja_wspak}{}
	\insertRozwiazanie{\currfilepath}{funkcja_wyiksuj}{}
	\insertRozwiazanie{\currfilepath}{dekodowanie_utf8_w_base64}{}
	\insertRozwiazanie{\currfilepath}{regex_czy_slowo}{}
	\insertRozwiazanie{\currfilepath}{regex_czy_liczba}{}
}

%
% napisy
%

\dbEntryBegin{funkcja_wspak}\if1\dbEntryCheckResults
Napisz funkcję, która dla danej listy słów wypisze każde słowo z listy wspak. Np. dla listy \python{['Ala', 'ma', 'kota']} funkcja powinna wypisać:
\begin{Verbatim}
alA
am
atok
\end{Verbatim}
\fi

\dbEntryBegin{funkcja_wspak-rozwiazanie}\if1\dbEntryCheckResults
\begin{minted}[frame=single]{python}
def wskap(lista):
  for slowo in lista:
    # w pythonie zamiast poniższej pętli można prościej ...
    # ale warto poznać (także) takie rozwiązanie
    for i in range(len(slowo)):
      print(slowo[-1 - i], end = '')
    print()
\end{minted}
%
Prostszym rozwiązaniem (nie wymagającym jawnego pisania pętli w pętli) jest:
\begin{minted}[frame=single]{python}
def wskap(lista):
  for slowo in lista:
    print(slowo[::-1])
\end{minted}
które korzysta z odwrócenia napisu przy pomocy pobrania wszystkich jego elemntów z krokiem -1 poprzez \python{slowo[::-1]}
\fi


\dbEntryBegin{funkcja_wyiksuj}\if1\dbEntryCheckResults
Napisz funkcję \python{wyiksuj(napis)}, która zwróci dany \Verb{napis}, zastępując każdą małą literę przez \Verb{x} i
każdą wielką literę przez \Verb{X}, natomiast resztę znaków pozostawi bez zmian.
Np. dla napisu \Verb{'Python 3.6.1 (default, Dec 2015, 13:05:11)'} funkcja powinna zwrócić napis: \Verb{Xxxxxx 3.6.1 (xxxxxxx, Xxx 2015, 13:05:11)}

\begin{teacherOnly}
\noindent Wskazówka:\\
Dla każdego znaku użyj konstrukcji \python{if}/\python{elif}/\python{else}, aby rozróżnić pomiędzy trzema przypadkami:
małe litery, wielkie litery, pozostałe znaki.
\end{teacherOnly}
\fi
\dbEntryBegin{funkcja_wyiksuj-rozwiazanie}\if1\dbEntryCheckResults
\begin{CodeFrame*}[python]{}
def wyiksuj(napis):
  duzy_alfabet = 'AĄBCĆDEĘFGHIJKLŁMNŃOÓPRSŚTUWYZŹŻ'
  maly_alfabet = 'aąbcćdeęfghijklłmnńoóprsśtuwyzźż'
  for c in napis:
    if c in duzy_alfabet:
      print('X', end = '')
    elif c in maly_alfabet:
      print('x', end = '')
    else:
      print(c, end = '')
\end{CodeFrame*}

inne rozwiązanie:

\begin{CodeFrame*}[python]{}
def wyiksuj(napis):
  for c in napis:
    if c.isupper():
      print('X', end = '')
    elif c.islower():
      print('x', end = '')
    else:
      print(c, end = '')
\end{CodeFrame*}

jeszcze inne rozwiązanie (w tej formie obsługuje tylko litery ASCII, ale aktualna wersja zadania to dopuszcza):

\begin{CodeFrame*}[python]{}
def wyiksuj(napis):
  import re
  napis = re.sub("[a-z]", "x", napis)
  return re.sub("[A-Z]", "X", napis)
\end{CodeFrame*}

\noindent Zwróć uwagę że:
\begin{itemize}
\item iterowanie po elementach napisu (znakach) z użyciem pętli \python{for}
\item zastosowanie konstrukcji \python{a in b} do sprawdzenia czy element a (w tym wypadku znak) należy do kolekcji b (w tym wypadku napisu, ale mogał by to być także np. lista znaków)
\item zastosowanie metod \python{isupper()} i \python{islower()} w drugim wariancie rozwiązania, podobne porównanie dla znaków ASCI można łatwo wykonać w oparciu o wartość numerycznego kodu tego znaku
\item zwięzłość rozwiązania z użyciem wyrażeń regularnych
\end{itemize}
\fi


\dbEntryBegin{dekodowanie_utf8_w_base64}\if1\dbEntryCheckResults
Napisz program dekodujący napis kodowany w UTF8 zakodowany przy pomocy base64 mający postać:
\python{b'UHl0aG9uIGplc3QgZmFqbnkg8J+Yjg==\n'}.\\
Wskazówka: dane wejściowe funkcji \python{decode()} muszą być typu "bytes", można to uzyskać poprzedzając napis prefiksem \Verb{b}, tak jak powyżej.
\fi
\dbEntryBegin{dekodowanie_utf8_w_base64-rozwiazanie}\if1\dbEntryCheckResults
\begin{CodeFrame*}[python]{}
import codecs
d = b'UHl0aG9uIGplc3QgZmFqbnkg8J+Yjg==\n'
d = codecs.decode(d, 'base64')
d = d.decode()
print(d)
\end{CodeFrame*}

Zakodowany tekst to: \textcolor{red}{Python jest fajny {\Symbola 😎}}

\noindent Zwróć uwagę że:
\begin{itemize}
\item zdejmowanie kolejnych kodowań w kolejnych krokach procedury – w odwrotnej kolejności niż były nakładane
\item funckcja \python{codecs.decode} wymaga jako danych wejściowych ciągu bajtowego, i taki ciąg zrwaca
\item metoda \python{decode} ciągu bajtowego zwraca napis powstały przez zdekodowanie tego ciągu z użyciem utf-8
\end{itemize}
\fi



%
% wyrażenia regularne
%

\dbEntryBegin{regex_czy_slowo}\if1\dbEntryCheckResults
Napisz funkcję która sprawdzi z użyciem wyrażeń regularnych czy dany napis jest słowem (tzn. nie zawiera spacji).
\fi
\dbEntryBegin{regex_czy_slowo-rozwiazanie}\if1\dbEntryCheckResults
\begin{CodeFrame*}[python]{}
import re
def spr(x):
  if re.match("[^ ]*$", x):
    print(x, "jest słowem")
  else:
    print(x, "NIE jest słowem")
\end{CodeFrame*}

Zadanie polega przede wszystkim na wymyśleniu odpowiedniego wyrażenia regularnego.
Ze względu że funkcja \python{match} dopasowuje zawsze od początku napisu (ale nie wymaga dojścia do końca napisu) nasze wyrażenie musi konczyć się dolarem,
aby wyrażenie było dopasowywane do całości sprawdzanego napisu.
Zastosowane wyrażenie wymaga aby napis nie zawierał spacji - wtedy uznajemy go za słowo.
\fi

\dbEntryBegin{regex_czy_liczba}\if1\dbEntryCheckResults
Napisz funkcję która sprawdzi z użyciem wyrażeń regularnych czy dany napis jest liczbą (tzn. jest złożony z cyfr i kropki, a na początku może wystąpić + albo -).
\fi
\dbEntryBegin{regex_czy_liczba-rozwiazanie}\if1\dbEntryCheckResults
\begin{CodeFrame*}[python]{}
import re
def spr(x):
  if re.match("[+-]?[0-9.]+$", x):
    print(x, "jest liczbą")
  else:
    print(x, "NIE jest liczbą")
\end{CodeFrame*}
\fi
