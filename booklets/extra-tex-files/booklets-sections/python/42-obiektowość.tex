% Copyright (c) 2018-2020 Matematyka dla Ciekawych Świata (http://ciekawi.icm.edu.pl/)
% Copyright (c) 2018-2020 Robert Ryszard Paciorek <rrp@opcode.eu.org>
% 
% MIT License
% 
% Permission is hereby granted, free of charge, to any person obtaining a copy
% of this software and associated documentation files (the "Software"), to deal
% in the Software without restriction, including without limitation the rights
% to use, copy, modify, merge, publish, distribute, sublicense, and/or sell
% copies of the Software, and to permit persons to whom the Software is
% furnished to do so, subject to the following conditions:
% 
% The above copyright notice and this permission notice shall be included in all
% copies or substantial portions of the Software.
% 
% THE SOFTWARE IS PROVIDED "AS IS", WITHOUT WARRANTY OF ANY KIND, EXPRESS OR
% IMPLIED, INCLUDING BUT NOT LIMITED TO THE WARRANTIES OF MERCHANTABILITY,
% FITNESS FOR A PARTICULAR PURPOSE AND NONINFRINGEMENT. IN NO EVENT SHALL THE
% AUTHORS OR COPYRIGHT HOLDERS BE LIABLE FOR ANY CLAIM, DAMAGES OR OTHER
% LIABILITY, WHETHER IN AN ACTION OF CONTRACT, TORT OR OTHERWISE, ARISING FROM,
% OUT OF OR IN CONNECTION WITH THE SOFTWARE OR THE USE OR OTHER DEALINGS IN THE
% SOFTWARE.

%  BEGIN: Obiektowość
\subsection{Obiektowość}

Jak mogliśmy zauważyć przy sprawdzaniu typów zmiennych są one klasami. Związane z tym jest m.in. to iż posiadają one metody służące do operowania na nich.
Opis danego typu wraz z dostępnymi metodami można obejrzeć przy pomocy polecenia \python{help()}, np. \python{help("list")}.
\teacher{
Chcemy nauczyć korzystania z dokumentacji, więc w poniższych zadaniach z nią związanych starajmy się aby uczniowie sami odnajdywali w niej odpowiedzi.
}

W przypadku list za pomocą metod tej klasy mamy możliwość wstawiania wartości na daną pozycję, sortowania i odwracania kolejności elementów:

\begin{CodeFrame}[python]{0.50\textwidth}
l = ["i", "m"]
l.insert(1, "c")
print(l)
l.reverse()
print(l)
l.sort()
print(l)
\end{CodeFrame}
\begin{CodeFrame}{auto}
['i', 'c', 'm']
['m', 'c', 'i']
['c', 'i', 'm']
\end{CodeFrame}

Zwróć uwagę że sortowanie i odwracanie modyfikuje istniejącą listę a nie tworzy kopii.
%  END: Obiektowość
