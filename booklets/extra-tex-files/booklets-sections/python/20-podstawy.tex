% Copyright (c) 2016-2020 Matematyka dla Ciekawych Świata (http://ciekawi.icm.edu.pl/)
% Copyright (c) 2016-2017 Łukasz Mazurek
% Copyright (c) 2018-2020 Robert Ryszard Paciorek <rrp@opcode.eu.org>
% 
% MIT License
% 
% Permission is hereby granted, free of charge, to any person obtaining a copy
% of this software and associated documentation files (the "Software"), to deal
% in the Software without restriction, including without limitation the rights
% to use, copy, modify, merge, publish, distribute, sublicense, and/or sell
% copies of the Software, and to permit persons to whom the Software is
% furnished to do so, subject to the following conditions:
% 
% The above copyright notice and this permission notice shall be included in all
% copies or substantial portions of the Software.
% 
% THE SOFTWARE IS PROVIDED "AS IS", WITHOUT WARRANTY OF ANY KIND, EXPRESS OR
% IMPLIED, INCLUDING BUT NOT LIMITED TO THE WARRANTIES OF MERCHANTABILITY,
% FITNESS FOR A PARTICULAR PURPOSE AND NONINFRINGEMENT. IN NO EVENT SHALL THE
% AUTHORS OR COPYRIGHT HOLDERS BE LIABLE FOR ANY CLAIM, DAMAGES OR OTHER
% LIABILITY, WHETHER IN AN ACTION OF CONTRACT, TORT OR OTHERWISE, ARISING FROM,
% OUT OF OR IN CONNECTION WITH THE SOFTWARE OR THE USE OR OTHER DEALINGS IN THE
% SOFTWARE.

\section{Podstawowe elementy składniowe}

\begin{teacherOnly}
\noindent Dla osób zaznajomionych z C/C++:
\begin{itemize}
\item kolejne instrukcje (zamiast średnika) kończy znak nowej linii
\item bloki rozpoczyna dwukropek, a wyznacza je wcięcie o danej ilości znaków (nie mieszamy tabulatorów z spacjami)
\item nie ma konstrukcji i++, czy też ++i, jest za to i += 1
\item średnik na końcu instrukcji (linii) nie jest błędem składniowym (jest ignorowany)
\item warunek if'a w nawiasach nie jest błędem składniowym (ale po nawiasach musi być dwukropek)
\item nie ma pętli for w stylu C (,,trójinstrukcyjnej''), w Pythonie pętla for zawsze iteruje po elementach jakiejś listy
\end{itemize}
\end{teacherOnly}


%  BEGIN: Funkcje
\subsection{Definiowanie własnych funkcji}

Bardzo często będziemy chcieli móc wielokrotnie wykorzystać raz napisany fragment kodu.
W tym celu będziemy tworzyć własne \emph{funkcje}. Definicja funkcji ma następującą postać:

\begin{CodeFrame*}[python]{}
def nazwa_funkcji(argumenty):
  pierwsze_polecenie
  drugie_polecenie
  ...
\end{CodeFrame*}

\noindent
Zwróć uwagę na kilka rzeczy:
\begin{itemize}
	\item Na końcu pierwszej linijki jest dwukropek.
	\item Druga linijka musi być \emph{wcięta}, tzn. rozpoczynać się od spacji, kilku spacji lub znaku tabulacji.
	\item Jeżeli w ramach funkcji chcemy wykonać kilka instrukcji muszą one mieć taki sam poziom wcięcia.
	\item ,,Wnętrze'' funkcji kończymy wracając do takiego samego poziomu wcięcia na jakim ją rozpoczęliśmy
	      (takiego wcięcia jakie miała linijka z słowem kluczowym \python{def}).
\end{itemize}\vspace{-4pt}
Jest to typowy sposób wyznaczania bloku kodu w Pythonie i będziemy go jeszcze spotykać w innych konstrukcjach (które poznamy już niedługo), dlatego szczególnie wart jest zapamiętania.

Gdy umieszczamy inną konstrukcję korzystającą z bloku kodu we wnętrzu jakiegoś innego bloku (np. funkcji), blok tej instrukcji musi być ,,bardziej'' wcięty od bloku w którym jest zawarty,
powrót do poziomu wcięcia zewnętrznego bloku oznacza zakończenie bloku tej instrukcji i kontynuowanie zewnętrznego bloku.

\begin{ProTip}{Porada}
Na funkcję można patrzeć jak na nazwany kawałek kodu, który możemy wywołać z innego miejsca ze odmiennymi wartościami zmiennych stanowiących jej argumenty.
\end{ProTip}

Polecenie wywołania funkcji ma postać \python{nazwa_funkcji(argumenty)} i możemy napisać je w tym samym pliku, poniżej definicji tej funkcji.
Typowo ilość i kolejność argumentów w definicji, jak i w wywołaniu powinny być takie same.
Jeżeli nasza funkcja nie potrzebuje przyjmować argumentów nawiasy okrągłe w jej definicji i wywołaniu pozostawiamy puste.
Jeżeli potrzebujemy więcej argumentów rozdzielamy je w obu przypadkach przecinkami (tak jak miało to miejsce w korzystaniu z funkcji \python{print}).

\paragraph{Przykład}
Napiszmy funkcję, która wypisuje swój argument podniesiony do kwadratu i wywołajmy ją:

\begin{CodeFrame}[python]{.5\textwidth}
def kwadrat(x):
  print(x * x)

kwadrat(7)
kwadrat(2 + 3)
\end{CodeFrame}
\begin{CodeFrame}{auto}
49
25
\end{CodeFrame}

\noindent
Zwróć uwagę, iż wywołania funkcji w powyższym przykładzie nie są wcięte --- są poza blokiem funkcji.

\begin{ProTip}{Polecenia wieloliniowe w konsoli interaktywnej 🤔}
Możliwe jest wprowadzanie poleceń wieloliniowych w konsoli interaktywnej.
W takim wypadku po wprowadzeniu pierwszej linii (rozpoczynającej blok, np. \python{def})
nastąpi zmiana znaku zachęty na \Verb{...}, co oznacza tryb wprowadzania bloku poleceń.
Następnie wprowadzamy kolejne instrukcje wykonywane w ramach tego bloku (np. funkcji) pamiętająć o wcięciach.
Wprowadzanie bloku kończymy pustą linią, po czym znak zachęty powróci do standardowego \Verb{>>>}.
\end{ProTip}

\subsubsection{Wartość zwracana z funkcji}

Często chcemy aby funkcja zamiast wypisać wynik swojego działania na ekran zwróciła go w taki sposób aby można było go zapisać do jakiejś zmiennej,
możliwe to jest poprzez zastosowanie instrukcji \python{return}. Przerywa ona działanie funkcji w miejscu w którym została wykonała,
powoduje powrót do miejsca gdzie wywołana została funkcja i zwraca podaną do niej wartość:

\begin{CodeFrame}[python]{.5\textwidth}
def kwadrat(x):
  return x * x

a = kwadrat(7)
print( a - 2, kwadrat(4) )
\end{CodeFrame}
\begin{CodeFrame}{auto}
47 16
\end{CodeFrame}
%  END: Funkcje

\vspace{-13pt}

%  BEGIN: Funkcje2
\subsubsection{Argumenty domyślne i nazwane {\Symbola 🤔}}

Możliwe jest podanie wartości domyślnych dla wybranych argumentów funkcji. Utworzy to z nich argumenty opcjonalne, które nie muszą być podawane przy wywołaniu funkcji.
Argumenty z wartościami domyślnymi muszą występować w definicji funkcji po argumentach bez takich wartości.
Przy wywołaniu funkcji można odwoływać się do jej argumentów z podaniem ich nazw, pozwala to na podawanie argumentów w innej kolejności niż podana w definicji funkcji,
co jest przydatne zwłaszcza przy funkcjach z wieloma argumentami opcjonalnymi.

\begin{CodeFrame}[python]{.65\textwidth}
def potega(a = 2, b = 2):
    return a ** b

print( potega(), potega(4), potega(4, 3) )
print( potega(b = 3), potega(b = 1, a = 4) )
\end{CodeFrame}
\begin{CodeFrame}{auto}
4 16 64
8 4
\end{CodeFrame}

\subsubsection{Zasięg zmiennej {\Symbola 🤔}}

W Pythonie wewnątrz funkcji widoczne są zmienne zdefiniowane poza nią, jednak aby móc modyfikować taką zmienną wewnątrz
funkcji należy ją tam zadeklarować jako globalną przy pomocy słowa kluczowego \python{global}:

\begin{CodeFrame}[python]{.5\textwidth}
def test():
  global b
  a, b = 5, 13
  print(a, b, c)

a, b, c = 1, 3, 7
test()
print(a, b, c)
\end{CodeFrame}
\begin{CodeFrame}{auto}
5 13 7
1 13 7
\end{CodeFrame}

\noindent
Analizując działanie powyższego kodu zwrócić uwagę na:
\begin{itemize}
\item zasłonięcie globalnego \Verb{a} poprzez lokalne a wewnątrz funkcji (nie można zmodyfikować globalnej zmiennej \Verb{a} w funkcji),
\item możliwość dostępu do globalnych zmiennych w funkcji dopóki ich nie zasłonimy zmienną lokalną (tak używamy zmiennej \Verb{c})
\item możliwość zmodyfikowania zmiennej globalnej gdy jest zadeklarowana w funkcji jako \python{global}
\end{itemize}
%  END: Funkcje2

%  BEGIN: Pętla for
\subsection{Pętla \python{for}}
Załóżmy, że chcemy obliczyć kwadraty wszystkich liczb od 1 do 4.
Zgodnie z dotychczasową wiedzą, w tym celu musimy wykonać 4 działania:

\begin{CodeFrame}[python]{.5\textwidth}
print(1 * 1)
print(2 * 2)
print(3 * 3)
print(4 * 4)
\end{CodeFrame}
\begin{CodeFrame}{auto}
1
4
9
16
\end{CodeFrame}

Widzimy jednak, że te działania są bardzo podobne i chciałoby się je wykonać ,,za jednym zamachem''.
Do wykonywania wielokrotnie tego samego (lub podobnego) kodu służą pętle.
Najprostszym rodzajem pętli jest pętla \python{for}, która dla danej \emph{listy} i operacji do wykonania
wykonuje tę operację po kolei na każdym elemencie listy.

Do wykonania powyższego zadania służy pętla \python{for} w następującej postaci:

\begin{CodeFrame}[python]{.5\textwidth}
for x in [1, 2, 3, 4]:
    print(x * x)
\end{CodeFrame}
\begin{CodeFrame}{auto}
1
4
9
16
\end{CodeFrame}

\noindent Spróbuj przepisać tę pętlę i uruchomić program.
Zauważ że wnętrze pętli jest wyznaczone w sposób analogiczny do wnętrza funkcji:
\begin{itemize}
	\item Rozpoczyna się od dwukropka kończącego pierwszą linię.
	\item Kolejne linijki są \emph{wcięte}, tzn. rozpoczynać się od spacji, kilku spacji lub znaku tabulacji.
	\item Jeżeli w ramach pętli chcielibyśmy wykonać kilka instrukcji muszą one mieć taki sam poziom wcięcia.
	\item ,,Wnętrze'' pętli kończymy wracając do takiego samego poziomu wcięcia na jakim ją rozpoczęliśmy
	      (takiego wcięcia jakie miała linijka z słowem kluczowym \python{for}).
	\item Pętle możemy zagnieżdżać jedna w drugiej --- blok wewnętrznej pętli musi być ,,bardziej'' wcięty.
	Powrót do poziomu wcięcia zewnętrznej pętli oznacza zakończenie pętli wewnętrznej i kontynuowanie zewnętrznej.
\end{itemize}
%  END: Pętla for

%  BEGIN: Lista kolejnych liczb naturalnych
\subsection{Lista kolejnych liczb naturalnych}
Często potrzebujemy, aby pętla przeszła po liście kilku kolejnych liczb naturalnych.
W tym celu możemy oczywiście podać wprost kolejne elementy listy (tak jak w powyższym przykładzie),
jednak istnieje wygodniejsze rozwiązanie, mianowicie polecenie \python{range()}:

\begin{CodeFrame}[python]{0.5\textwidth}
for x in range(7):
    print(x, end = ', ')
\end{CodeFrame}
\begin{CodeFrame}{auto}
0, 1, 2, 3, 4, 5, 6, 
\end{CodeFrame}

\begin{CodeFrame}[python]{0.5\textwidth}
for x in range(5, 10):
    print(x, end = ', ')
\end{CodeFrame}
\begin{CodeFrame}{auto}
5, 6, 7, 8, 9, 
\end{CodeFrame}

\begin{CodeFrame}[python]{0.5\textwidth}
for x in range(10, 20, 3):
    print(x, end = ', ')
\end{CodeFrame}
\begin{CodeFrame}{auto}
10, 13, 16, 19, 
\end{CodeFrame}

\noindent Na powyższych przykładach widzimy, że polecenie \python{range()} występuje w trzech wersjach:
\begin{itemize}
	\item \python{range(kon)} generuje listę kolejnych liczb od 0 (\strong{włącznie}) do \Verb{kon} (\strong{wyłącznie}).
	\item \python{range(pocz, kon)} generuje listę kolejnych liczb od \Verb{pocz} (\strong{włącznie}) do 
		\Verb{kon} (\strong{wyłącznie}).
	\item \python{range(pocz, kon, krok)} generuje listę liczb od \Verb{pocz} (\strong{włącznie}) 
		do \Verb{kon} (\strong{wyłącznie}), przeskakując w każdym kroku o \Verb{krok}.
\end{itemize}

\begin{ProTip}{\normalfont{\strong{Do zapamiętania}}}
\normalsize Wszystkie przedziały w Pythonie są domknięte z lewej strony i otwarte z prawej strony,
tzn. zawierają swój lewy koniec i nie zawierają swojego prawego końca.
\end{ProTip}
%  END: Lista kolejnych liczb naturalnych

%  BEGIN: Typ logiczny
\subsection{Typ logiczny}

Jak już się przekonaliśmy można używać Pythona jako kalkulatora. Możemy go także użyć do obliczania wartości wyrażeń logicznych. Służy do tego wbudowany dwuwartościowy typ logiczny z wartościami:
\begin{itemize}
\item \python{True} oznaczającą logiczną jedynkę / prawdę
\item \python{False} oznaczającą logiczne zero / fałsz
\end{itemize}
Operacje na tym typie wykonujemy z użyciem słów kluczowych: \python{and}, \python{or}, \python{not} oznaczających odpowiednio:
iloczyn logiczny (aby był prawdą oba warunki muszą być spełnione), sumę logiczną (aby wynik był prawdą co najmniej jednej z warunków musi być spełniony) oraz negację logiczną.
Podobnie jak w zwykłych operacjach arytmetycznych możemy grupować ich fragmenty (celem wymuszenia kolejności działań) przy pomocy nawiasów okrągłych.

Wartościom tego typu mogą odpowiadać wybrane wartości innych typów (np. liczba całkowita 0 odpowiada \python{False}, a pozostałe liczby całkowite \python{True}).
Wartościami tego typu są też wyniki różnego rodzaju porównań, takich jak: \python{<} (mniejsze), \python{>} (większe), \python{<=} (mniejsze równe),
\python{>=} (większe równe), \python{==} (równe), \python{!=} (nierówne).
%  END: Typ logiczny

%  BEGIN: Instrukcja warunkowa if
\subsection{Instrukcja warunkowa \python{if}}

Często chcemy, aby program zachowywał się w różny sposób w zależności od tego, czy jakiś warunek jest spełniony, czy nie.
W Pythonie (jak w większości języków programowania) służy do tego instrukcja warunkowa \python{if}.

Przypuśćmy, że chcemy napisać funkcję, która dla podanej wartości sprawdzi czy odpowiada ona logicznej prawdzie czy fałszowi i wypisuje odpowiedni komunikat.
Zatem kod będzie wyglądał następująco:

\begin{CodeFrame}[python]{0.50\textwidth}
def sprawdz(x):
    if x:
        print(x, '-- prawda')
    else:
        print(x, '-- nie prawda')
sprawdz(1)
sprawdz(0)
\end{CodeFrame}
\begin{CodeFrame}{auto}
1 -- prawda
0 -- nie prawda
\end{CodeFrame}

\noindent Zwróć uwagę na następujące rzeczy:
\begin{itemize}
	\item \python{if} to po polsku ,,jeśli'', \python{else} to po polsku ,,w przeciwnym przypadku''.
	\item Linijki rozpoczynające się od \python{if} i \python{else} (podobnie jak linijki rozpoczynające się np. od \python{def}) kończą się dwukropkiem.
	\item ,,Wnętrze'' \python{if}-a i \python{else}-a (linijki 3 i 5) jest wcięte (bardziej niż samo wnętrze definicji funkcji \Verb{sprawdz}).
	\item Linijka 3 zostanie wykonana, jeśli spełniony będzie warunek z linijki 2, czyli jeśli wartość zmiennej x będzie odpowiadała prawdzie.
	\item Linijka 5 zostanie wykonana, jeśli warunek z linijki 2 nie będzie spełniony.
\end{itemize}
W powyższym przykładzie użyliśmy konstrukcji \python{if}/\python{else} do rozróżnienia pomiędzy dwoma przypadkami.
Używając komendy \python{elif} (skrót od \python{else if}) możemy stworzyć bardziej skomplikowany kod do rozróżnienia pomiędzy kilkoma różnymi przypadkami:

\begin{CodeFrame}[python]{0.6\textwidth}
for x in range(0, 5):
    if x < 1 or x == 4:
        print('mniejsze od 1 lub równe 4')
    elif x in [0,2,3]:
        print('0 2 lub 3')
    else:
        print('nic ciekawego')
\end{CodeFrame}
\begin{CodeFrame}{auto}
mniejsze od 1 lub równe 4
nic ciekawego
0 2 lub 3
0 2 lub 3
mniejsze od 1 lub równe 4
\end{CodeFrame}

Ten kod składa się z trzech bloków, które są wykonywane w zależności od spełnienia poszczególnych warunków:
\python{if}, \python{elif}, \python{else}.
Mamy dużą dowolność w konstruowaniu tego typu fragmentów kodu: 
bloków \python{elif} może być dowolnie wiele, blok \python{else} może występować jako ostatni blok,
ale może też go nie być w ogóle.

W powyższym przykładzie widzimy również, że w roli warunków sprawdzanych w ramach \python{if}a mogą występować bardziej złożone wyrażenia.
Możemy tutaj użyć dowolnego wyrażenia którego wynik odpowiada wartości logicznej \python{True}/\python{False},
najczęściej spotkamy się z wyrażeniami złożonymi z poznanych już operatorów porównań (\python{<}, \python{>}, \python{<=},
\python{>=}, \python{==}, \python{!=}) i operacji logicznych (\python{and}, \python{or}, \python{not}).

Zwróć uwagę na warunek postaci ,,\python{A in B}''.
Taki warunek sprawdza, czy wartość reprezentowana przez \Verb{A} jest elementem \Verb{B}, a jego wynik oczywiście także jest wartością logiczną.
W naszym przykładzie sprawdzaliśmy, czy wartość zmiennej \Verb{x} występuje w podanej liście liczb, czyli czy jest 1, 2 lub 3.

Zauważ, że dla x wynoszącego 0 spełnione są dwa warunki (pierwszy i środkowy), w takim wypadku decydująca jest kolejność warunków i w konstrukcji
\python{if}/\python{elif} wykonany zostanie jedynie kod związany z pierwszym pasującym warunkiem.
%  END: Instrukcja warunkowa if

%  BEGIN: Pętla while
\subsection{Pętla \python{while}}

Do tej pory korzystaliśmy z pętli \python{for}, która pozwala na iterowanie po liście elementów. Innym istotnym rodzajem pętli
jest pętla \python{while}, która powoduje wykonywanie zawartego w niej kodu dopóki podany warunek jest spełniony.
\teacher{Zwrócić uwagę na możliwość / ryzyko zapętlenia.}

\begin{CodeFrame}[python]{0.50\textwidth}
a, b = 0, 1
while a <= 20:
    print(a, end=" ")
    a, b = b, a + b
\end{CodeFrame}
\begin{CodeFrame}{auto}
0 1 1 2 3 5 8 13
\end{CodeFrame}

Zwróć uwagę, że wewnątrz pętli \python{while} (tak samo jak innych konstrukcji używających wciętego bloku - takich jak \python{for}, czy \python{if})
może znajdować się więcej niż jedno polecenie. Trzeba tylko pamiętać, aby wszystkie były poprzedzone takim samym wcięciem.

Pętla \python{while} jest też naturalnym wyborem gdy w Pythonie chcemy przechodzić przez jakiś zakres liczb z krokiem nie całkowitym
	(wcześniej poznana instrukcja \python{range}, stosowana do iterowania po zakresie liczbowym w pętli \python{for}, wymaga aby krok był całkowity).
%  END: Pętla while

\subsection{\python{break} i \python{continue}}

W ramach zarówno pętli for jak i while możemy użyć instrukcji:
\begin{itemize}
	\item \python{break} powodującej przerwanie wykonywania pętli
	\item \python{continue} powodującej pominięcie pozostałych instrukcji w aktualnym obiegu pętli
\end{itemize}

Ich działanie może zobrazować poniższy kod:
\begin{CodeFrame}[python]{0.50\textwidth}
for x in [1, 2, 3, 4, 5, 6]:
  print("start", x)
  if x == 2:
    continue
  if x == 4:
    break
  print("...")
\end{CodeFrame}
\begin{CodeFrame}{auto}
start 1
...
start 2
start 3
...
start 4
\end{CodeFrame}


%  BEGIN: Wielokrotne przypisanie
\subsection{Wielokrotne przypisanie}

Zwróć uwagę w powyższym kodzie także na operację wielokrotnego przypisania postaci \python{a, b = x, y}.
Dokonuje ona przypisania wartości x do a i y do b, przy czym wartości x i y obliczane są przed zmodyfikowaniem a i b.
Pozwala to m.in. na zamianę wartości pomiędzy a i b bez stosowania zmiennej tymczasowej poprzez zapis: \python{a, b = b, a}.
Podobnie możemy zapisywać przypisania większej ilości wartości do większej ilości zmiennych np: \python{a, b, c = 1, 5, 9}.
Z notacji tej będziemy też często korzystać w dalszej części skryptu przy inicjalizacji zmiennych.
\teacher{Należy poświęcić chwilę uwagi tej notacji.}
%  END: Wielokrotne przypisanie
