% Copyright (c) 2018-2020 Matematyka dla Ciekawych Świata (http://ciekawi.icm.edu.pl/)
% Copyright (c) 2018-2020 Robert Ryszard Paciorek <rrp@opcode.eu.org>
% 
% MIT License
% 
% Permission is hereby granted, free of charge, to any person obtaining a copy
% of this software and associated documentation files (the "Software"), to deal
% in the Software without restriction, including without limitation the rights
% to use, copy, modify, merge, publish, distribute, sublicense, and/or sell
% copies of the Software, and to permit persons to whom the Software is
% furnished to do so, subject to the following conditions:
% 
% The above copyright notice and this permission notice shall be included in all
% copies or substantial portions of the Software.
% 
% THE SOFTWARE IS PROVIDED "AS IS", WITHOUT WARRANTY OF ANY KIND, EXPRESS OR
% IMPLIED, INCLUDING BUT NOT LIMITED TO THE WARRANTIES OF MERCHANTABILITY,
% FITNESS FOR A PARTICULAR PURPOSE AND NONINFRINGEMENT. IN NO EVENT SHALL THE
% AUTHORS OR COPYRIGHT HOLDERS BE LIABLE FOR ANY CLAIM, DAMAGES OR OTHER
% LIABILITY, WHETHER IN AN ACTION OF CONTRACT, TORT OR OTHERWISE, ARISING FROM,
% OUT OF OR IN CONNECTION WITH THE SOFTWARE OR THE USE OR OTHER DEALINGS IN THE
% SOFTWARE.

%  BEGIN: Zmienna, obiekt i referencja
\subsection{Zmienna, obiekt i referencja \zaawansowane{20}}

W Pythonie każda zmienna jest nazwą wskazującą na jakiś obiekt w pamięci. Podobnie każdy element listy czy słownika wskazuje na jakiś obiekt\footnote{
Zasadniczo wszystkie definiowane przez nas zmienne czy funkcje są elementem słownika związanego z danym kontekstem.
Do słowników tych można uzyskać dostęp poprzez funkcje \python{globals()} (słownik zawierający elementy zdeklarowane w kontekście globalnym) i
\python{locals()} (słownik zawierający elementy zadeklarowane w kontekście lokalnym).
}.
Na jeden obiekt może wskazywać wiele zmiennych i/lub elementów innych obiektów (takich jak listy czy słowniki).
Jeżeli zmienna nie ma na co wskazywać (np. został do niej przypisany wynik funkcji, która nie zwraca wartości) wskazuje na obiekt \python{None} (typu \python{NoneType}).
Zatem na wszystkie zmienne pythonowe możemy patrzeć jak na referencje do obiektów istniejących gdzieś w pamięci.

Do uzyskania identyfikatora obiektu związanego z daną nazwą, lub elementem innego obiektu służy funkcja \python{id} (w przypadku standardowej implementacji Pythona jest to po prostu adres w pamięci).

\begin{teacherOnly} Warto także pokazać działanie wspomnianych funkcji \python{globals()} i \python{locals()} (zwróć uwagę na różnicę między a i b oraz c i d):
\begin{CodeFrame*}[python]{}
a, b = 0, 0

def f1():
    def f2(x):
       a = 2
       print("f2", a, c)
       print(" locals  = ", locals())
       print(" globals = ", globals())
    
    a = 1
    c, d = 3, 4
    print("f1", a)
    print(" locals  = ", locals())
    print(" globals = ", globals())
    f2(7)

f1()
\end{CodeFrame*}
\end{teacherOnly}

\subsubsection{Usuwanie i czas życia zmiennych}

Instrukcja \python{del}, której używaliśmy już do usuwania elementów z listy lub słownika może być wykorzystana także do usuwania innych zmiennych.
Należy jednak pamiętać iż w Pythonie usunięcie zmiennej nie wiąże się z natychmiastowym zwolnieniem zajmowanej przez nią pamięci z kilku powodów:
\begin{itemize}
\item na pojedynczy obiekt może wskazywać kilka zmiennych
\item to Python decyduje o tym kiedy zwalniać / ponownie użyć pamięć pozostałą po obiektach na które nie wskazuje już żadna nazwa
\end{itemize}

\subsubsection{Kopiowanie obiektów}

Python w momencie przypisania wartości jednej zmiennej do innej nie tworzy kopii obiektu na który wskazuje zmienna, zamiast tego przypisuje referencję do istniejącego obiektu.
Jest to szczególnie zauważalne w obiektach, które mogą być wewnętrznie modyfikowalne (takich jak listy czy słowniki)\footnote{
	Zauważ że jedyną możliwością modyfikacji liczby czy napisu jest przypisanie wartości wyrażenia do zmiennej,
	a dla list czy słowników możemy je modyfikować bez operacji przypisania całej listy czy słownika do nowej czy tej samej zmiennej.
	Jest to podział na typy "immutable" i "mutable" - te pierwsze nie są wewnętrznie modyfikowalne
	(każda modyfikacja odbywa się przez przypisanie obiektu do zmiennej, w wyniku którego pod zmienną może zostać podpięty nowy obiekt).
}:

\begin{CodeFrame}[python]{0.55\textwidth}
a = [1, 2, 3]
b = a
print(a, b, "\n", hex(id(a)), hex(id(b)))
a[1] = 0
print(a, b, "\n", hex(id(a)), hex(id(b)))
del a
print(b,    "\n", hex(id(b)))
\end{CodeFrame}
\begin{CodeFrame}{auto}
[1, 2, 3] [1, 2, 3]
 0x7f50d76b2bc8 0x7f50d76b2bc8
[1, 0, 3] [1, 0, 3]
 0x7f50d76b2bc8 0x7f50d76b2bc8
[1, 0, 3]
 0x7f50d76b2bc8
\end{CodeFrame}

Jak widać \Verb{a} i \Verb{b} posiadają taki sam identyfikator obiektu zwracany przez funkcję \python{id},
modyfikacja \python{a[1]} wpłynęła na zawartość \Verb{b}, natomiast usunięcie \Verb{a} nie ma wpływu na \Verb{b}
(usunęliśmy tylko jedną z dwóch referencji na wspólny obiekt).
Jeżeli chcemy uzyskać kopię listy lub słownika musimy skorzystać z metody \python{copy()} odpowiedniego obiektu:

\begin{CodeFrame}[python]{0.55\textwidth}
a = [1, 2, 3]
b = a.copy()
b[1] = "X"
print(a, b, "\n", hex(id(a)), hex(id(b)))
\end{CodeFrame}
\begin{CodeFrame}{auto}
[1, 2, 3] [1, 'X', 3]
 0x7f50d76b2bc8 0x7f50d57a7088
\end{CodeFrame}

Zauważ że tak utworzone \Verb{b} ma inny identyfikator obiektu niż \Verb{a}.
Należy mieć także na uwadze że nawet argumenty funkcji przekazywane są jako referencje na obiekty a nie kopie obiektów,
natomiast dopiero operacja przypisania nowej wartości do zmiennej związanej z argumentem powoduje że zaczyna ona wskazywać
na nowo utworzony (w wyniku wyrażenia po prawej stronie znaku równości) obiekt.

\subsubsection{Dla jeszcze bardziej dociekliwych \zaawansowane{30}}

Osobom jeszcze bardziej dociekliwym w temacie wnętrzności Pythona możemy polecić lekturę artykułu omawiającego te zagadnienia \url{http://www.rwdev.eu/articles/objectthinking} oraz samodzielne eksperymenty.
%  END: Zmienna, obiekt i referencja
