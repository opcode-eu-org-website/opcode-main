% Copyright (c) 2016-2020 Matematyka dla Ciekawych Świata (http://ciekawi.icm.edu.pl/)
% Copyright (c) 2016-2017 Łukasz Mazurek
% Copyright (c) 2018-2020 Robert Ryszard Paciorek <rrp@opcode.eu.org>
% 
% MIT License
% 
% Permission is hereby granted, free of charge, to any person obtaining a copy
% of this software and associated documentation files (the "Software"), to deal
% in the Software without restriction, including without limitation the rights
% to use, copy, modify, merge, publish, distribute, sublicense, and/or sell
% copies of the Software, and to permit persons to whom the Software is
% furnished to do so, subject to the following conditions:
% 
% The above copyright notice and this permission notice shall be included in all
% copies or substantial portions of the Software.
% 
% THE SOFTWARE IS PROVIDED "AS IS", WITHOUT WARRANTY OF ANY KIND, EXPRESS OR
% IMPLIED, INCLUDING BUT NOT LIMITED TO THE WARRANTIES OF MERCHANTABILITY,
% FITNESS FOR A PARTICULAR PURPOSE AND NONINFRINGEMENT. IN NO EVENT SHALL THE
% AUTHORS OR COPYRIGHT HOLDERS BE LIABLE FOR ANY CLAIM, DAMAGES OR OTHER
% LIABILITY, WHETHER IN AN ACTION OF CONTRACT, TORT OR OTHERWISE, ARISING FROM,
% OUT OF OR IN CONNECTION WITH THE SOFTWARE OR THE USE OR OTHER DEALINGS IN THE
% SOFTWARE.

\IfStrEq{\dbEntryID}{}{
	\subsection{konstrukcje składniowe}
	
	\insertZadanie{\currfilepath}{funkcja_suma2}{}
	\insertZadanie{\currfilepath}{suma_poteg}{}
	\teacher{\insertRozwiazanie{\currfilepath}{suma_poteg}{}}
	\insertZadanie{\currfilepath}{funkcja_znak}{}
	\insertZadanie{\currfilepath}{suma_poteg2}{}
}


\dbEntryBegin{funkcja_suma2}\if1\dbEntryCheckResults
Napisz funkcję, która przyjmuje dwa argumenty i \ul[black]{zwraca} ich sumę. Użyj jej do obliczenia (oraz wypisania na konsolę) wartości kilku różnych sum.
\\\textit{Wskazówka: \python{print()} powinien być użyty na zewnątrz tej funkcji.}

\teacher{Warto zwrócić uwagę na to zadnie - jest ono dobrą okazją do wyjaśnienia różnicy między „funkcja wypisuje” a „funkcja zwraca”.}
\fi

\dbEntryBegin{suma_poteg}\if1\dbEntryCheckResults
Napisz program obliczający sumę $1^2 + 2^2 + 3^2 + \ldots + 99^2 + 100^2$. \teacher{(wynik: 338350)}
\fi

\dbEntryBegin{suma_poteg-rozwiazanie}\if1\dbEntryCheckResults
\begin{minted}[frame=single]{python}
sum = 0
for x in range(101):
    sum = sum + x**2
print(sum)
\end{minted}
%
Zwróć uwagę na użycie zewnętrznej w stosunku co do pętli zmiennej \python{sum},
służącej do przechowywania wartości modyfikowanej w każdym obiegu pętli (wyniku sumy)
– jest to typowy schemat rozwiązywania tego typu problemów programistycznych.
\fi

\dbEntryBegin{funkcja_znak}\if1\dbEntryCheckResults
Napisz funkcję \python{znak(liczba)} która wypiszę informację o znaku podanej liczby (wyróżniając zero) i zwróci jej wartość bezwzględną.
Wywołanie funkcji \Verb{znak} powinno wyglądać następująco:

\vspace{-7pt}
\begin{CodeFrame}[text]{.5\textwidth}
a = znak(7)
b = znak(-13)
c = znak(0)
print(a, b, c)
\end{CodeFrame}
\begin{CodeFrame}{auto}
7 jest dodatnia
-13 jest ujemna
0 to zero
7 13 0
\end{CodeFrame}
\fi

\dbEntryBegin{suma_poteg2}\if1\dbEntryCheckResults
Rozwiąż zadanie \ref{suma_poteg} stosując pętlę \python{while}.
\fi

