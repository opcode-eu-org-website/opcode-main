% Copyright (c) 2016-2020 Matematyka dla Ciekawych Świata (http://ciekawi.icm.edu.pl/)
% Copyright (c) 2016-2017 Łukasz Mazurek
% Copyright (c) 2018-2020 Robert Ryszard Paciorek <rrp@opcode.eu.org>
% 
% MIT License
% 
% Permission is hereby granted, free of charge, to any person obtaining a copy
% of this software and associated documentation files (the "Software"), to deal
% in the Software without restriction, including without limitation the rights
% to use, copy, modify, merge, publish, distribute, sublicense, and/or sell
% copies of the Software, and to permit persons to whom the Software is
% furnished to do so, subject to the following conditions:
% 
% The above copyright notice and this permission notice shall be included in all
% copies or substantial portions of the Software.
% 
% THE SOFTWARE IS PROVIDED "AS IS", WITHOUT WARRANTY OF ANY KIND, EXPRESS OR
% IMPLIED, INCLUDING BUT NOT LIMITED TO THE WARRANTIES OF MERCHANTABILITY,
% FITNESS FOR A PARTICULAR PURPOSE AND NONINFRINGEMENT. IN NO EVENT SHALL THE
% AUTHORS OR COPYRIGHT HOLDERS BE LIABLE FOR ANY CLAIM, DAMAGES OR OTHER
% LIABILITY, WHETHER IN AN ACTION OF CONTRACT, TORT OR OTHERWISE, ARISING FROM,
% OUT OF OR IN CONNECTION WITH THE SOFTWARE OR THE USE OR OTHER DEALINGS IN THE
% SOFTWARE.

\IfStrEq{\dbEntryID}{}{
	\ifdefined\noExtraInfoMode\else
		\subsection{konstrukcje składniowe}
	\fi
	
	\insertZadanie{\currfilepath}{funkcja_suma2}{}
	\insertZadanie{\currfilepath}{suma_poteg}{}
	\insertZadanie{\currfilepath}{funkcja_znak}{}
	\insertZadanie{\currfilepath}{suma_poteg2}{}
}


\dbEntryBegin{funkcja_suma2}\if1\dbEntryCheckResults
Napisz funkcję, która przyjmuje dwa argumenty i \ul[black]{zwraca} ich sumę. Użyj jej do obliczenia (oraz wypisania na konsolę) wartości kilku różnych sum.
\\\textit{Wskazówka: \python{print()} powinien być użyty na zewnątrz tej funkcji.}

\teacher{Warto zwrócić uwagę na to zadnie - jest ono dobrą okazją do wyjaśnienia różnicy między „funkcja wypisuje” a „funkcja zwraca”.}
\fi
\dbEntryBegin{funkcja_suma2-rozwiazanie}\if1\dbEntryCheckResults
\begin{CodeFrame*}[python]{}
def suma(a, b):
  return a + b

a = suma(17, 15)
print(a, suma(13, 16), suma(a, 11))
\end{CodeFrame*}
%
Zwróć uwagę na użycie słowa kluczowego \python{return} do zwrócenia wartości z funkcji.
W odróżnieniu od bezpośredniego wpisania wyniku na ekran z użyciem np. \python{print} pozwala to m.in. na przechowanie tego wyniku w zmiennej i użycie w dalszych obliczeniach,
co zostało zademonstrowane w drugiej części przykładu.
\fi


\dbEntryBegin{suma_poteg}\if1\dbEntryCheckResults
Napisz program obliczający sumę $1^2 + 2^2 + 3^2 + \ldots + 99^2 + 100^2$. \teacher{(wynik: 338350)}
\fi
\dbEntryBegin{suma_poteg-rozwiazanie}\if1\dbEntryCheckResults
\begin{CodeFrame*}[python]{}
sum = 0
for x in range(101):
    sum = sum + x**2
print(sum)
\end{CodeFrame*}
%
Zwróć uwagę na użycie zewnętrznej w stosunku co do pętli zmiennej \python{sum},
służącej do przechowywania wartości modyfikowanej w każdym obiegu pętli (wyniku sumy)
– jest to typowy schemat rozwiązywania tego typu problemów programistycznych.
\fi



\dbEntryBegin{funkcja_znak}\if1\dbEntryCheckResults
Napisz funkcję \python{znak(liczba)} która wypiszę informację o znaku podanej liczby (wyróżniając zero) i zwróci jej wartość bezwzględną.
Wywołanie funkcji \Verb{znak} powinno wyglądać następująco:

\vspace{-7pt}
\begin{CodeFrame}[text]{.5\textwidth}
a = znak(7)
b = znak(-13)
c = znak(0)
print(a, b, c)
\end{CodeFrame}
\begin{CodeFrame}{auto}
7 jest dodatnia
-13 jest ujemna
0 to zero
7 13 0
\end{CodeFrame}
\fi
\dbEntryBegin{funkcja_znak-rozwiazanie}\if1\dbEntryCheckResults
\begin{CodeFrame*}[python]{}
def znak(liczba): 
  if liczba > 0:
    print(liczba, "jest dodatnia")
    return liczba
  elif liczba < 0:
    print(liczba, "jest ujemna")
    return -liczba
  else:
    print(liczba, "to zero")
    return 0
\end{CodeFrame*}
%
Zadanie można by rozwiązać także używając funkcji obliczającej wartość bezwzględną (\python{abs()}), jednak ze względu na konstrukcję zadania musimy sami ustalić znak liczby, natomiast użycie tej (i tak już posiadanej) informacji do obliczenia wartości bezwzględnej jest wydajniejsze niż kolejne sprawdzanie znaku wewnątrz funkcji \python{abs()}.
\fi


\dbEntryBegin{suma_poteg2}\if1\dbEntryCheckResults
Rozwiąż zadanie \ref{suma_poteg} stosując pętlę \python{while}.
\fi
\dbEntryBegin{suma_poteg2-rozwiazanie}\if1\dbEntryCheckResults
\begin{CodeFrame*}[python]{}
sum = 0
x = 1
while x <= 100:
    sum = sum + x**2
    x = x + 1
print(sum)
\end{CodeFrame*}
Zauważ że w tym wariancie zadania potrzebujemy dwóch zmiennych zewnętrznych w stosunku co do pętli –
  jednej do przechowywania obliczanej sumy, a drugiej do przechowywania numeru kroku (wartości którą sumujemy).
Ta druga zmienna w rozwiązaniu z użyciem pętli \python{for} jest dostarczana przez samą konstrukcję tamtej pętli.
W przypadku pętli \python{while} to my jako twórcy kodu musimy ją zainicjalizować i zwiększać w każdym kroku.
Pozwala to jednak na stosowanie bardziej zaawansowanych mechanizmów modyfikowania tej zmiennej.
\fi
