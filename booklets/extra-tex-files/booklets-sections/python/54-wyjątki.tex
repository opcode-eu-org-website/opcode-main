% Copyright (c) 2018-2020 Matematyka dla Ciekawych Świata (http://ciekawi.icm.edu.pl/)
% Copyright (c) 2018-2020 Robert Ryszard Paciorek <rrp@opcode.eu.org>
% 
% MIT License
% 
% Permission is hereby granted, free of charge, to any person obtaining a copy
% of this software and associated documentation files (the "Software"), to deal
% in the Software without restriction, including without limitation the rights
% to use, copy, modify, merge, publish, distribute, sublicense, and/or sell
% copies of the Software, and to permit persons to whom the Software is
% furnished to do so, subject to the following conditions:
% 
% The above copyright notice and this permission notice shall be included in all
% copies or substantial portions of the Software.
% 
% THE SOFTWARE IS PROVIDED "AS IS", WITHOUT WARRANTY OF ANY KIND, EXPRESS OR
% IMPLIED, INCLUDING BUT NOT LIMITED TO THE WARRANTIES OF MERCHANTABILITY,
% FITNESS FOR A PARTICULAR PURPOSE AND NONINFRINGEMENT. IN NO EVENT SHALL THE
% AUTHORS OR COPYRIGHT HOLDERS BE LIABLE FOR ANY CLAIM, DAMAGES OR OTHER
% LIABILITY, WHETHER IN AN ACTION OF CONTRACT, TORT OR OTHERWISE, ARISING FROM,
% OUT OF OR IN CONNECTION WITH THE SOFTWARE OR THE USE OR OTHER DEALINGS IN THE
% SOFTWARE.

%  BEGIN: Obsługa błędów
\subsection{Obsługa błędów}
Wcześniej spotkaliśmy się już z komunikatem błędu. Błędy mogą wynikać z błędów składniowych w programie ale również nie przewidzianych zdarzeń w trakcie jego pracy.
Warto mieć na uwadze iż wszystkie błędy w Pythonie mają postać wyjątków które mogą zostać obsłużone blokiem \python{try}/\python{except}.

\begin{CodeFrame*}[python]{}
try:
  a = 5 / 0
except ZeroDivisionError:
  print("dzielenie przez zero")
except:
  print("inny błąd")
\end{CodeFrame*}

Przy obsłudze błędów może przydać się instrukcja pusta \python{pass}, która w tym przypadku pozwala na zignorowanie obsługi danego błędu.

\begin{CodeFrame*}[python]{}
try:
  slownik["a"] += 1
except:
  pass
\end{CodeFrame*}

Powyższy kod zwiększy wartość związaną z kluczem \python{"a"} w słowniku \python{slownik}, jednak gdy napotka błąd (np. słownik nie zawiera klucza \python{"a"}) zignoruje go.

Możemy także generować wyjątki z naszego kodu, służy do tego instrukcja \python{raise}, której należy przekazać obiektem dziedziczącym po \python{BaseException} np:

\begin{CodeFrame*}[python]{}
raise BaseException("jakiś błąd")
\end{CodeFrame*}
%  END: Obsługa błędów
