% Copyright (c) 2016-2020 Matematyka dla Ciekawych Świata (http://ciekawi.icm.edu.pl/)
% Copyright (c) 2016-2017 Łukasz Mazurek
% Copyright (c) 2018-2020 Robert Ryszard Paciorek <rrp@opcode.eu.org>
% 
% MIT License
% 
% Permission is hereby granted, free of charge, to any person obtaining a copy
% of this software and associated documentation files (the "Software"), to deal
% in the Software without restriction, including without limitation the rights
% to use, copy, modify, merge, publish, distribute, sublicense, and/or sell
% copies of the Software, and to permit persons to whom the Software is
% furnished to do so, subject to the following conditions:
% 
% The above copyright notice and this permission notice shall be included in all
% copies or substantial portions of the Software.
% 
% THE SOFTWARE IS PROVIDED "AS IS", WITHOUT WARRANTY OF ANY KIND, EXPRESS OR
% IMPLIED, INCLUDING BUT NOT LIMITED TO THE WARRANTIES OF MERCHANTABILITY,
% FITNESS FOR A PARTICULAR PURPOSE AND NONINFRINGEMENT. IN NO EVENT SHALL THE
% AUTHORS OR COPYRIGHT HOLDERS BE LIABLE FOR ANY CLAIM, DAMAGES OR OTHER
% LIABILITY, WHETHER IN AN ACTION OF CONTRACT, TORT OR OTHERWISE, ARISING FROM,
% OUT OF OR IN CONNECTION WITH THE SOFTWARE OR THE USE OR OTHER DEALINGS IN THE
% SOFTWARE.

%  BEGIN: Wprowadzenie
\subsection{Praca z konsolą interaktywną}

Pierwszym sposobem pracy z Pythonem jest praca w interaktywnej konsoli.
Uzyskujemy ją po uruchomieniu polecenia \Verb{python3}.
W konsoli tej początkowo wypisane są pewne informacje (m.in. używana wersja Pythona)
oraz znak zachęty (w Pythonie najczęściej~\Verb@>>>@)\footnote{
	Zauważ że jest on inny niż znak zachęty bash'a (zazwyczaj~\Verb@$@ poprzedzony dodatkowymi informacjami) –
	pozwala to na identyfikację interpretera poleceń w którym aktualnie pracujemy i wydawanie w odpowiedniej składni
	(bash nie rozumie poleceń w składni pythona, python nie rozumie poleceń w składni basha).
}.
Interpreter oczekuje, iż po tym znaku wpiszemy polecenie i naciśniemy Enter.
Wynik polecenia zostanie wypisany w kolejnym wierszu.

\teacher{Zwrócić uwagę na rozróżnianie konsoli pythonowej i bashowej - inny znak zachęty}

Najprostszym sposobem użycia konsoli Pythona jest użycie jej jako kalkulatora --- wpisujemy działanie
do obliczenia, naciskamy Enter i w kolejnym wierszu otrzymujemy wynik działania.
Przykład użycia konsoli Pythona jako kalkulatora znajduje się poniżej:
\begin{Verbatim}[frame=single]
>>> 2 + 2 * 2
6
>>> (2+2) * 2
8
>>> 2 ** 7
128
>>> 47 / 10
4.7
>>> 47 // 10
4
>>> 47 % 10
7
\end{Verbatim}
W powyższym przykładzie:
\begin{itemize}
\item Znak \Verb{**} oznacza podnoszenie do potęgi.
\item Znak \Verb{/} oznacza dzielenie.
\item Znak \Verb{//} oznacza dzielenie całkowite.
\item Znak \Verb{%} oznacza branie reszty z dzielenia.
\item Nawiasy okrągłe służą grupowaniu wyrażeń i wymuszaniu innej niż standardowa kolejności działań.
\item Spacje nie mają znaczenia (używamy ich jedynie dla zwiększenia czytelności).
\end{itemize}

\begin{ProTip}{Porada}
W konsoli interaktywnej przy pomocy strzałek góra/dół można przeglądać historię wydanych poleceń.
Polecenia te można także wykonać ponownie (naciskając enter), a przedtem także zmodyfikować
(poruszając się strzałkami prawo lewo).

Konsola ta posiada także mechanizm dopełniania wpisywanych poleceń przy pomocy tabulatora
(pojedyncze naciśnięcie dopełnia, gdy tylko jedna propozycja, podwójne wyświetla propozycje dopełnień).
\end{ProTip}

\subsubsection{Zmienne}

Podobnie jak w kalkulatorze możemy korzystać z \emph{pamięci}, w Pythonie możemy zapisywać wartości
w \emph{zmiennych}:
\begin{Verbatim}[frame=single]
>>> x = 3
>>> y = 4
>>> x
3
>>> x**2 + y**2
25
\end{Verbatim}
W pierwszych dwóch linijkach następuje \emph{przypisanie} wartości 3 do zmiennej~\Verb{x} oraz 
wartości 4 do zmiennej~\Verb{y}.
Od tej pory możemy korzystać z tych zmiennych, np. do obliczenia wartości wyrażenia $(x^2 + y^2)$.

\teacher{Pokazać, co się dzieje, jak odwołamy się do nieistniejącej zmiennej.\\Powiedzieć kilka słów na temat (czytania) komunikatów o błędach!}

\subsubsection{Moduły i zaawansowany kalkulator \zaawansowane{10}}
Python pozwala na wykonywanie bardziej zaawansowanych obliczeń.
Możliwe jest m.in. obliczenia wartości wyrażeń logicznych, konwertowanie systemów liczbowych, obliczanie wartości funkcji trygonometrycznych.
Duża część funkcji matematycznych w Pythonie zawarta jest w module ,,math'', który wymaga zaimportowania. Można to zrobić na przykład w sposób następujący:
\begin{Verbatim}[frame=single]
>>> import math
>>> math.sin(math.pi/2)
1.0
\end{Verbatim}
Zauważ, że odwołanie do elementów tak zaimportowanego modułu wymaga podania jego nazwy, następnie kropki i nazwy używanej funkcji z tego modułu.

\subsection{Pisanie i uruchamianie kodu programu}

Do tej pory korzystaliśmy z Pythona używając interaktywnej konsoli. 
Jest to całkiem wygodne narzędzie, jeśli wykonujemy tylko jednolinijkowe polecenia,
jednak pisanie dłuższych fragmentów kodu w tej konsoli staje się już bardzo niewygodne.
Drugą metodą korzystania z Pythona jest pisanie kodu programu (skryptu) w pliku tekstowym
i uruchamianie tego kodu w konsoli.

\begin{ProTip}{Moduły \zaawansowane{10}}
Nazwa pliku powinna być inna niż nazwy importowanych modułów, czyli jeżeli w kodzie mamy \Verb#import abc# to nasz plik nie powinien nazywać się \Verb#abc.py#,
w przeciwnym razie zamiast wskazanego modułu Python będzie próbował zaimportować nasz plik.
\end{ProTip}

Utwórz plik \Verb{mojProgram.py}\footnote{
	Pliki z skryptami Pythona tradycyjnie mają rozszerzenie \Verb{.py}.
	Nie jest ono jednak wymagane --- interpreter Pythona wykona kod z pliku o dowolnym rozszerzeniu a także z pliku bez rozszerzenia.
} z następującą zawartością:\noParBreak
\begin{CodeFrame*}[python]{}
x = 3
y = 4
print(x**2 + y**2)
\end{CodeFrame*}

W celu wykonania kodu zapisanego w pliku uruchom interpreter Pythona z jednym argumentem,
będącym nazwą tego pliku: \Verb{python3 mojProgram.py}.

\begin{ProTip}{Porada}
Zachowuj pliki z programami pisanymi w trakcie zajęć, używając nazw które pozwolą Ci łatwo zidentyfikować dany program.
Mogą one być pomocne w rozwiązywaniu kolejnych zadań oraz prac domowych.
\end{ProTip}

\subsubsection{funkcja \python{print}}
Zwróć uwagę, iż do wypisania wyniku działania na ekran została użyta funkcja \python{print}.
Nie korzystaliśmy z niej wcześniej, ponieważ bazowaliśmy na domyślnym zachowaniu interpretera przy
pracy interaktywnej powodującym wypisywanie na konsolę wyniku nie zapisywanego do zmiennej.
Jednak kiedy tworzymy program powinniśmy w jawny sposób określać co chcemy aby zostało wypisane na
konsolę właśnie np. za pomocą funkcji \python{print}.

Funkcja \python{print} wypisuje przekazane do niej (rozdzielane przecinkami) argumenty rozdzielając je spacjami.
Przechodzi ona domyślnie do następnej linii po każdym wywołaniu. Na przykład:

\begin{CodeFrame}[python]{.5\textwidth}
print("raz dwa", "trzy ...")
print(4, 5)
\end{CodeFrame}
\begin{CodeFrame}{auto}
raz dwa trzy ...
4 5
\end{CodeFrame}

\vspace{-14pt}

\begin{ProTip}{Informacja}
Ilekroć w niniejszych materiałach pojawią się dwie ramki, jedna obok drugiej, w lewej ramce znajdował
się będzie kod programu, a w prawej efekt jego działania wyświetlony w konsoli:
\end{ProTip}

Zachowanie funkcji \python{print} można zmienić, dodając do jej wywołania, na końcu listy argumentów argument postaci \python{end = X} i/lub \python{sep = Y},
gdzie \Verb{X} to otoczony apostrofami ciąg znaków, który chcemy wypisywać zamiast przejścia do nowej linii,
a \Verb{Y} to otoczony apostrofami ciąg znaków, który chcemy wypisywać zamiast spacji rozdzielającej wypisania kolejnych argumentów.
Na przykład:

\begin{CodeFrame}[python]{.5\textwidth}
x = 3
y = 4
print(x, '+ ', end='')
print(y, x + y, sep=' = ')
\end{CodeFrame}
\begin{CodeFrame}{auto}
3 + 4 = 7
\end{CodeFrame}

\vspace{-14pt}

\begin{ProTip}{Napisy}
Ciąg znaków ujęty w apostrofy lub cudzysłowy (w Pythonie nie ma znaczenia, której wersji użyjemy, ważne jest tylko aby znak rozpoczynający i kończący był taki sam) nazywamy napisem.
Możemy ich używać nie tylko w ramach funkcji print, ale też np. przypisywać do zmiennych. Więcej o napisach dowiemy się później.
\end{ProTip}

\vspace{-6pt}

\subsubsection{Komentarze}
Często chcemy móc umieścić w kodzie programu dodatkową informację, która ułatwi nam jego czytanie i zrozumienie w przyszłości.
Służą do tego tak zwane komentarze, które są ignorowane przez interpreter (bądź kompilator) danego języka.
W Pythonie podstawowym typem komentarza, jest komentarz jednoliniowy, rozpoczynający się od znaku \Verb{#} a kończący z końcem linii.
\teacher{W przypadku zdziwienia dlaczego akurat \texttt{\#} jest używany do komentarzy można wspomnieć że jest to bardzo często stosowane i związane ze standardem linii służącą do określenia programu używanego do zinterpretowania pliku ze skryptem postaci np. \texttt{\#!/bin/bash}}

\subsubsection{inne sposoby uruchamiania kodu z pliku \zaawansowane{20}}

Jeżeli do wywołania interpretera Pythona dodamy opcję \Verb{-i} (np. \Verb{python3 -i mojProgram.py})
po wykonaniu kodu z podanego pliku uruchomi on konsolę interaktywną w której będą dostępne elementy (m.in. zmienne) zdefiniowane w podanym pliku.

Możliwe jest także włączenie kodu z pliku do aktualnie uruchomionego interpretera (np. konsoli interaktywnej), w taki sposób jakbyśmy go wpisali
(czyli z wykonaniem wszystkich instrukcji i późniejszą możliwością dostępu do zdefiniowanych tam elementów).
Aby wczytać w ten sposób kod z pliku  \Verb{mojProgram.py} należy wykonać: \python{exec(open('mojProgram.py').read())}

\subsubsection{ipython \zaawansowane{20}}

\Verb{ipython3} jest wygodniejszym w pracy interaktywnej interpreterem Pythona w wersji 3. Pozwala on m.in. na lepsze przewijanie i edytowanie poleceń wieloliniowych w historii.
%  END: Wprowadzenie
