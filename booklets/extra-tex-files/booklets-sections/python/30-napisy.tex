% Copyright (c) 2016-2020 Matematyka dla Ciekawych Świata (http://ciekawi.icm.edu.pl/)
% Copyright (c) 2016-2017 Łukasz Mazurek
% Copyright (c) 2018-2020 Robert Ryszard Paciorek <rrp@opcode.eu.org>
% 
% MIT License
% 
% Permission is hereby granted, free of charge, to any person obtaining a copy
% of this software and associated documentation files (the "Software"), to deal
% in the Software without restriction, including without limitation the rights
% to use, copy, modify, merge, publish, distribute, sublicense, and/or sell
% copies of the Software, and to permit persons to whom the Software is
% furnished to do so, subject to the following conditions:
% 
% The above copyright notice and this permission notice shall be included in all
% copies or substantial portions of the Software.
% 
% THE SOFTWARE IS PROVIDED "AS IS", WITHOUT WARRANTY OF ANY KIND, EXPRESS OR
% IMPLIED, INCLUDING BUT NOT LIMITED TO THE WARRANTIES OF MERCHANTABILITY,
% FITNESS FOR A PARTICULAR PURPOSE AND NONINFRINGEMENT. IN NO EVENT SHALL THE
% AUTHORS OR COPYRIGHT HOLDERS BE LIABLE FOR ANY CLAIM, DAMAGES OR OTHER
% LIABILITY, WHETHER IN AN ACTION OF CONTRACT, TORT OR OTHERWISE, ARISING FROM,
% OUT OF OR IN CONNECTION WITH THE SOFTWARE OR THE USE OR OTHER DEALINGS IN THE
% SOFTWARE.

%  BEGIN: Napisy - wprowadzenie
\section{Napisy}
Do tej pory używaliśmy zmiennych do przechowywania liczb i operowania na nich. Zmienne mogą również jako wartości przyjmować litery, słowa, a nawet całe zdania:

\begin{CodeFrame}[python]{.5\textwidth}
x = 'A'
a, b, c = 'Ala', "ma", " kota i psa"
d = """ ... a co ma ...
 "kotek"?"""
print(x, a[2])
print(c[1], c[-1], c[-3])
print(a + b)
print(3 * a)
print(a + " " + b + c + d)
\end{CodeFrame}
\begin{CodeFrame}{auto}
A a
o a p
Alama
AlaAlaAla
Ala ma kota i psa ... a co ma ...
 "kotek"?
\end{CodeFrame}

\pagebreak[2]\noindent
Zwróć uwagę na następujące rzeczy:
\begin{itemize}
\item Napisy muszą być otoczone pojedynczymi apostrofami lub podwójnym cudzysłowami (nie ma znaczenia,
	którą wersję wybierzemy), w przypadku napisów wieloliniowych używamy trzykrotnie apostrofu lub cudzysłowowa na początku i końcu napisu.\\
	Nie przypisane do żadnej zmiennej napisy wieloliniowe mogą być stosowane jako komentarze wieloliniowe.
\item Przy użyciu liczby w nawiasie kwadratowym możemy poznać poszczególne litery napisu (\strong{\emph{numeracja rozpoczyna się od 0}}).
\item Ujemny indeks oznacza odliczanie liter od końca napisu: ostatnia litera napisu \python{c} to \python{c[-1]},
	przedostatnia to \python{c[-2]}, itd.
\item Przy użyciu znaku dodawania możemy sklejać (\emph{konkatenować}) napisy.
\item Przy użyciu znaku gwiazdki możemy mnożyć napisy (czyli sklejać same ze sobą).
\end{itemize}
Innymi przydatnymi operacjami na napisach jest sprawdzanie długości napisu poleceniem \python{len()}
oraz wycinanie podnapisu przy użyciu dwukropka:

\begin{CodeFrame}[python]{.7\textwidth}
tekst = 'Python'
dlugosc = len(tekst)
print(dlugosc, tekst[2:5], tekst[3:], tekst[:3])
\end{CodeFrame}
\begin{CodeFrame}{auto}
6 tho hon Pyt
\end{CodeFrame}

\pagebreak[2]\noindent
W powyższym przykładzie:
\begin{itemize}
\item komenda \python{tekst[2:5]} zwraca podnapis od znaku nr 2 (\strong{włącznie}) do znaku nr 5 
(\strong{wyłącznie}),
\item komenda \python{tekst[3:]} zwraca podnapis od znaku nr 3 (\strong{włącznie}) do końca, 
\item komenda \python{tekst[:3]} zwraca podnapis od początku do znaku nr 3 
(\strong{wyłącznie}).
\end{itemize}

Podobnie jak w \python{range()} możemy podać trzeci argument określający przedział czyli krok.
Pozwala to na wybieranie co n-tego znaku z napisu, zarówno zaczynając od początku jak i końca:

\begin{CodeFrame}[python]{.7\textwidth}
tekst = '123456789'
print(tekst[::2], tekst[1::2])
print(tekst[::-1], tekst[::-3])
print(tekst[::-1][::3], tekst[::3][::-1])
\end{CodeFrame}
\begin{CodeFrame}{auto}
13579 2468
987654321 963
963 741
\end{CodeFrame}

\pagebreak[2]\noindent
W powyższym przykładzie:
\begin{itemize}
\item komenda \python{tekst[::2]} zwraca co drugi znak,
\item komenda \python{tekst[1::2]} zwraca co drugi znak od znaku nr 1,
\item komenda \python{tekst[::-1]} zwraca napis od tyłu,
\item komenda \python{tekst[::-3]} zwraca co 3 znak z napisu od tyłu (warto zauważyć że nie zawsze jest to równoważne wypisaniu napisu złożonego z co 3 znaku od tyłu).
\end{itemize}

\subsection{Napis jako lista}

Wszystkie listy, których do tej pory używaliśmy w pętli \python{for} były listami liczb.
Okazuje się, że w Pythonie napisy mogą być traktowane jako lista, a dokładniej listą liter. 
Oznacza to, że po napisie można przejść przy użyciu pętli \python{for}, tak samo jak przechodziliśmy po liście liczb:

\begin{CodeFrame}[python]{0.50\textwidth}
for l in 'Abc':
    print('litera', end = ' ')
    print(l)
\end{CodeFrame}
\begin{CodeFrame}{auto}
litera A
litera b
litera c
\end{CodeFrame}

\subsubsection{Modyfikowalność napisów}

Python pozwala odwoływać się do poszczególnych znaków w napisie jak do elementów listy, jednak nie pozwala na ich modyfikowanie:

\begin{CodeFrame}[python]{0.25\textwidth}
s = "abcdefgh"
s[2] = "X"
print(s)
\end{CodeFrame}
\begin{CodeFrame}{auto}
Traceback (most recent call last):
  File "python", line 2, in <module>
TypeError: 'str' object does not support item assignment
\end{CodeFrame}

Zwróć uwagę na komunikat błędu, który został wyświetlony, podaje on informacji o tym co wywołało błąd (opis błędu) i w  której linii programu on wystąpił.
\strong{Czytanie ze zrozumieniem komunikatów o błędach ułatwia naprawianie niedziałającego programu.}

\teacher{
	\ \\ \strong{Warto pokazać kilka innych komunikatów o błędach i poprosić uczestników o ich wyjaśnienie.}\\\ 
}

Jeżeli zachodzi potrzeba modyfikowania napisu konkretnych znaków w napisie możemy użyć poznanej wcześniej metody uzyskiwania podnapisów:

\begin{CodeFrame}[python]{0.50\textwidth}
s = "abcdefgh"
s = s[:2] + "X" + s[3:5] + s[6:]
print(s)
\end{CodeFrame}
\begin{CodeFrame}{auto}
abXdegh
\end{CodeFrame}

Powyższy przykład w miejsce znaku nr 2 wstawia napis "X" oraz usuwa znak nr 5 z napisu.
Przy konieczności modyfikacji znak po znaku możemy użyć iteracji po napisie i budować nowy napis znak po znaku:

\begin{CodeFrame}[python]{0.50\textwidth}
s, ns = "abcdefgh", ""
for z in s:
    if z in "cf":
        ns = ns + "X"
    else:
        ns = ns + z
print(ns)
\end{CodeFrame}
\begin{CodeFrame}{auto}
abXdeXgh
\end{CodeFrame}
%  END: Napisy - wprowadzenie

%  BEGIN: Konwersje liczba -- napis
\subsection{Konwersje liczba -- napis}

Z punktu widzenia komputera liczba czy też element napisu, którym jest litera są pewną wartością numeryczną.
Natomiast my do zapisu liczb używamy różnych systemów (np. dziesiętnego, czy też szesnastkowego).
Domyślnie liczby wprowadzane do programu interpretowane są jako zapisane w systemie dziesiętnym,
podobnie liczby uzyskiwane poprzez konwersję napisu przy pomocy funkcji \python{int()}.
Możliwe jest jednak wprowadzanie liczb zapisanych w innych systemach liczbowych lub konwersja z napisu zawierającego liczbę ---
drugi, opcjonalny argument \python{int()} pozwala określić podstawę systemu z którego konwertujemy, zero oznacza automatyczne wykrycie w oparciu o prefix:

\begin{CodeFrame}[python]{0.7\textwidth}
# szesnastkowo
h1, h2, h3 = 0x1F, int("0x1F", 0), int("1F", 16)
# oktalnie
o1, o2, o3 = 0o17, int("0o17", 0), int("17", 8)
# binarnie
b1, b2, b3 = 0b101, int("0b101", 0), int("101", 2)

print("",h1,o1,b1, "\n",h2,o2,b2, "\n",h3,o3,b3)
\end{CodeFrame}
\begin{CodeFrame}{auto}
 31 15 5
 31 15 5
 31 15 5
\end{CodeFrame}
\teacher{
Zwrócić uwagę że w każdym wariancie konwersji dostajemy ten sam wynik. Jako ciekawostkę można zapytać się o konwertowanie z innych systemów liczbowych.
}

Możliwe jest także konwertowanie wartości liczbowej na napis w określonym systemie liczbowym:

\begin{CodeFrame*}[python]{}
a, b = 3, 13
c = (a + b) * b
s = "(" + bin(a) + " + " + oct(b) + ") * " + hex(b) + " = " + str(c)
print( s )
\end{CodeFrame*} 
% (0b11 + 0o15) * 0xd = 208
%  END: Konwersje liczba -- napis

%  BEGIN: Kodowania znaków
\subsection{Kodowania znaków}

Python używa Unicode dla obsługi napisów, jednak przed przekazaniem napisu do świata zewnętrznego konieczne może być zastosowanie konwersji do określonej postaci bytowej (zastosowanie odpowiedniego kodowania).
Służy do tego metoda \python{encode()} np.:

\begin{CodeFrame*}[python]{}
a = "aąbcć ... ←↓→"
inUTF7 = a.encode('utf7')
inUTF8 =  a.encode() # lub a.encode('utf8')
print("'" + a + "' w UTF7 to: " + str(inUTF7) + ", w UTF8: " + str(inUTF8))
\end{CodeFrame*}

Zmienne typu 'bytes' oprócz przekazania na zewnątrz (np. zapisu do pliku lub wysłania przez sieć) mogą zostać także m.in. zdekodowane do napisu z użyciem metody \python{decode()} lub poddane dalszej konwersji np. kodowaniu base64:

\begin{CodeFrame*}[python]{}
print("zdekodowany UTF7: " + inUTF7.decode('utf7'))

import codecs
b64 = codecs.encode(inUTF8, 'base64')
print("napis w UTF8 po zakodowaniu base64 to: " + str(b64))
\end{CodeFrame*}

W powyższym przykładzie należy zwrócić uwagę na instrukcję \python{import}, która służy do załączania bibliotek pythonowych do naszego programu.
W tym wypadku załączamy fragment standardowej biblioteki Pythona o nazwie \python{codecs}.

Base64 jest jednym z kodowań pozwalających na zapis danych binarnych w postaci ograniczonego zbioru znaków drukowalnych,
co pozwala m.in. na osadzanie danych binarnych (np. obrazki) w plikach tekstowych (np. dokumenty html, pliki źródłowe programów).
%  END: Kodowania znaków

%  BEGIN: Konwersja znak - numer unicode
\subsubsection{Konwersja pomiędzy znakiem a jego numerem}

Możliwe jest także konwertowanie pomiędzy liczbowym numerem znaku Unicode, a napisem go reprezentującym i w drugą stronę --- służą do tego odpowiednio funkcje \python{chr()} i \python{ord()}.
W ramach napisów można też użyć \python{\uNNNN}, gdzie \Verb{NNNN} jest (czterocyfrowym) numerem znaku lub po prostu umieścić dany znak w pliku kodowanym UTF8\footnote{Użyty w przykładzie symbol nieskończoności można uzyskać na standardowej polskiej klawiaturze pod Linuxem przy pomocy kombinacji AltGr + Shift + M}.

\begin{CodeFrame*}[python]{}
print(chr(0x221e) + " == \u221e == ∞")
print(hex(ord("∞")), hex(ord("\u221e")), hex(ord(chr(0x221e))) )
\end{CodeFrame*}

\teacher{
Warto skomentować że powyższy przykład demonstruje równoważność poszczególnych zapisów (drukują taki sam znak na konsoli, generują taki sam numer unicodowy).
}
%  END: Konwersja znak - numer unicode

%  BEGIN: Wyrażenia regularne 01
\subsection{Wyrażenia regularne {\Symbola 🤔}}

W przetwarzaniu napisów bardzo często stosowane są wyrażenia regularne służące do dopasowywania napisów do wzorca który opisują, wyszukiwaniu/zastępowaniu tego wzorca. Do typowej, podstawowej składni wyrażeń regularnych zalicza się m.in. następujące operatory:

\vspace{-6pt}\begin{Verbatim}
.      - dowolny znak
[a-z]  - znak z zakresu
[^a-z] - znak z poza zakresu (aby mieć zakres z ^ należy dać go nie na początku)
^      - początek napisu/linii
$      - koniec napisu/linii
\end{Verbatim}
\vspace{-8pt}\begin{Verbatim}
*      - dowolna ilość powtórzeń
?      - 0 lub jedno powtórzenie
+      - jedno lub więcej powtórzeń
{n,m}  - od n do m powtórzeń
\end{Verbatim}
\vspace{-8pt}\begin{Verbatim}
()     - pod-wyrażenie (może być używane dla operatorów powtórzeń,
         a także dla referencji wstecznych)
\end{Verbatim}

\pagebreak[2]\noindent
Python umożliwia korzystanie z wyrażeń regularnych za pomocą modułu \python{re}:

\begin{CodeFrame}[python]{0.57\textwidth}
import re

y = "aa bb cc bb ff bb ee"
x = "aa bb cc dd ff gg ee"

if re.match(".*[dz]", y):
  print("y zawiera d lub z")

if re.match(".*[dz]", x):
  print("x zawiera d lub z")

if re.match(".* ([a-z]{2}) .* \\1", y):
  print("y zawiera dwa razy to samo")

if re.match(".* ([a-z]{2}) .* \\1", x):
  print("x zawiera dwa razy to samo")

# zastępowanie
print (re.sub('[bc]+', "XX", y, 2))
print (re.sub('[bc]+', "XX", y))

# zachłanność
print (re.sub('bb (.*) bb', "X \\1 X", y))
print (re.sub('.*bb (.*) bb.*', "\\1", y))
print (re.sub('.*?bb (.*) bb.*', "\\1", y))
\end{CodeFrame}
\begin{CodeFrame}{auto}
x zawiera d lub z
y zawiera dwa razy to samo
aa XX XX bb ff bb ee
aa XX XX XX ff XX ee
aa X cc bb ff X ee
ff
cc bb ff
\end{CodeFrame}

Zwróć uwagę na:
\begin{itemize}
\item Działanie funkcji \Verb{match}, która dopasowuje wyrażenie do początku napisu (czyli tak jakby zaczynało się od \Verb{^}).
\item Odwołania wsteczne do pod-wyrażeń (fragmentów ujętych w nawiasy) postaci \Verb{\\x}, gdzie \Verb{x} jest numerem pod-wyrażenia.
\item ,,\emph{Zachłanność}'' (ang. \emph{greedy}) wyrażeń regularnych:
	\begin{itemize}
	\item w pierwszym wypadku \Verb{bb (.*) bb} dopasowało najdłuższy możliwy fragment, czyli \Verb{cc bb ff},
	\item w drugim przypadku gdy zostało poprzedzone \Verb{.*} dopasowało tylko \Verb{ff}, gdyż \Verb{.*} dopasowało najdłuższy możliwy fragment czyli \Verb{aa bb cc},
	\item w trzecim wypadku \Verb{bb (.*) bb} mogło i dopasowało najdłuższy możliwy fragment, czyli \Verb{cc bb ff}, gdyż było poprzedzone niezachłanną odmianą dopasowania dowolnego napisu, czyli: \Verb{.*?}.
	\end{itemize}
	Po każdym z operatorów powtórzeń (\Verb@. ? + {n,m}@) możemy dodać pytajnik (\Verb@.? ?? +? {n,m}?@) aby wskazać że ma on dopasowywać najmniejszy możliwy fragment, czyli ma działać nie zachłannie.
\end{itemize}
%  END: Wyrażenia regularne 01
