% Copyright (c) 2016-2020 Matematyka dla Ciekawych Świata (http://ciekawi.icm.edu.pl/)
% Copyright (c) 2016-2017 Łukasz Mazurek
% Copyright (c) 2018-2020 Robert Ryszard Paciorek <rrp@opcode.eu.org>
% 
% MIT License
% 
% Permission is hereby granted, free of charge, to any person obtaining a copy
% of this software and associated documentation files (the "Software"), to deal
% in the Software without restriction, including without limitation the rights
% to use, copy, modify, merge, publish, distribute, sublicense, and/or sell
% copies of the Software, and to permit persons to whom the Software is
% furnished to do so, subject to the following conditions:
% 
% The above copyright notice and this permission notice shall be included in all
% copies or substantial portions of the Software.
% 
% THE SOFTWARE IS PROVIDED "AS IS", WITHOUT WARRANTY OF ANY KIND, EXPRESS OR
% IMPLIED, INCLUDING BUT NOT LIMITED TO THE WARRANTIES OF MERCHANTABILITY,
% FITNESS FOR A PARTICULAR PURPOSE AND NONINFRINGEMENT. IN NO EVENT SHALL THE
% AUTHORS OR COPYRIGHT HOLDERS BE LIABLE FOR ANY CLAIM, DAMAGES OR OTHER
% LIABILITY, WHETHER IN AN ACTION OF CONTRACT, TORT OR OTHERWISE, ARISING FROM,
% OUT OF OR IN CONNECTION WITH THE SOFTWARE OR THE USE OR OTHER DEALINGS IN THE
% SOFTWARE.

%  BEGIN: Listy 01
\subsection{Listy}

Do tej pory listy traktowaliśmy głównie jako zbiór elementów po którym iterujemy. Zastosowanie list jest jednak znacznie szersze.
Lista stanowi pewnego rodzaju kontener do przechowywania innych zmiennych, w którym elementy zorganizowane są na zasadzie określenia ich (względnej) kolejności.
Lista może zawierać elementy różnych typów.

Na listach możemy wykonywać m.in. operacje modyfikowania, czy też usuwania jej elementów:

\begin{CodeFrame}[python]{0.50\textwidth}
l = ["i", "C", 0, "M"]
l[0] = "I"
del l[2]
print(l)
\end{CodeFrame}
\begin{CodeFrame}{auto}
['I', 'C', 'M']
\end{CodeFrame}

\noindent W powyższym przykładzie widzimy:
\begin{itemize}
\item Modyfikację pierwszego elementu listy (\python{l[0] = "I"}), z użyciem odwołania poprzez numer elementu.
      Elementy list numerujemy od zera. Ujemne wartości oznaczają numerowanie od końca listy, czyli -1 jest ostatnim elementem listy, -2 przedostatnim, itd.
\item Usunięcie trzeciego elementu listy (\python{del l[2]}). Powoduje to zmianę numeracji kolejnych elementów.
\end{itemize}
%  END: Listy 01

%  BEGIN: Listy 02
Jednak jeżeli chcemy modyfikować elementy listy iterując po niej, to konieczne jest iterowanie po indeksach (a nie jak dotychczas po wartościach):

\begin{CodeFrame}[python]{0.50\textwidth}
for i in range(len(l)):
    print(l[i])
    l[i] = "q"
print(l)
\end{CodeFrame}
\begin{CodeFrame}{auto}
I
C
M
['q', 'q', 'q']
\end{CodeFrame}

Dzieje się tak gdyż przypisanie do zmiennej \Verb{x} jakiejś wartości w ramach konstrukcji \python{for x in lista:}
modyfikuje tylko zmienną \Verb{x}, a nie element listy który został do niej pobrany.
%  END: Listy 02

%  BEGIN: Listy 03
\subsubsection{Wybór podlisty}

Możemy także tworzyć ,,podlisty'' przy pomocy operatora zakresów w identyczny sposób jak to zostało opisane przy napisach,
np. \python{ll[1::2]} zwróci listę złożoną z co drugiego elementu listy \Verb{ll} zaczynając od elementu o indeksie 1.
%  END: Listy 03

%  BEGIN: Lista jako modyfikowalny napis
\subsubsection{Lista jako modyfikowalny napis}

Listy mogą też służyć jako narzędzie do modyfikowania napisów.
W tym celu można skorzystać np. z listy złożonej z liter oryginalnego napisu:

\begin{CodeFrame}[python]{0.50\textwidth}
s = "abcdefgh"
l = list(s)
l[2] = "X"
del(l[5])
s = str.join("", l)
print(s)
\end{CodeFrame}
\begin{CodeFrame}{auto}
abXdegh
\end{CodeFrame}
%  END: Lista jako modyfikowalny napis

\subsubsection{Obiektowość list}

W przypadku list za pomocą metod tej klasy mamy możliwość wstawiania wartości na daną pozycję, sortowania i odwracania kolejności elementów:

\begin{CodeFrame}[python]{0.50\textwidth}
l = ["i", "m"]
l.insert(1, "c")
print(l)
l.reverse()
print(l)
l.sort()
print(l)
\end{CodeFrame}
\begin{CodeFrame}{auto}
['i', 'c', 'm']
['m', 'c', 'i']
['c', 'i', 'm']
\end{CodeFrame}

Zwróć uwagę że sortowanie i odwracanie modyfikuje istniejącą listę a nie tworzy kopii.
