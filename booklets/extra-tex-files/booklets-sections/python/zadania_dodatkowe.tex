% Copyright (c) 2016-2020 Matematyka dla Ciekawych Świata (http://ciekawi.icm.edu.pl/)
% Copyright (c) 2016-2017 Łukasz Mazurek
% Copyright (c) 2018-2020 Robert Ryszard Paciorek <rrp@opcode.eu.org>
% 
% MIT License
% 
% Permission is hereby granted, free of charge, to any person obtaining a copy
% of this software and associated documentation files (the "Software"), to deal
% in the Software without restriction, including without limitation the rights
% to use, copy, modify, merge, publish, distribute, sublicense, and/or sell
% copies of the Software, and to permit persons to whom the Software is
% furnished to do so, subject to the following conditions:
% 
% The above copyright notice and this permission notice shall be included in all
% copies or substantial portions of the Software.
% 
% THE SOFTWARE IS PROVIDED "AS IS", WITHOUT WARRANTY OF ANY KIND, EXPRESS OR
% IMPLIED, INCLUDING BUT NOT LIMITED TO THE WARRANTIES OF MERCHANTABILITY,
% FITNESS FOR A PARTICULAR PURPOSE AND NONINFRINGEMENT. IN NO EVENT SHALL THE
% AUTHORS OR COPYRIGHT HOLDERS BE LIABLE FOR ANY CLAIM, DAMAGES OR OTHER
% LIABILITY, WHETHER IN AN ACTION OF CONTRACT, TORT OR OTHERWISE, ARISING FROM,
% OUT OF OR IN CONNECTION WITH THE SOFTWARE OR THE USE OR OTHER DEALINGS IN THE
% SOFTWARE.

\IfStrEq{\dbEntryID}{}{
	\insertZadanie{\currfilepath}{dwucyfrowe_podzielne}{}
	\insertZadanie{\currfilepath}{niepodzielne}{}
	\insertZadanie{\currfilepath}{funkcja_xor}{}
	\insertZadanie{\currfilepath}{funkcja_rownowazne}{}
	\insertZadanie{\currfilepath}{funkcja_implikacja}{}
	\insertZadanie{\currfilepath}{zadanie_suma2}{}
	\insertZadanie{\currfilepath}{zadanie_suma3}{}
	\insertZadanie{\currfilepath}{funkcja_bezwzgledne}{}
	
	\insertZadanie{\currfilepath}{funkcja_skroty}{}
	\insertZadanie{\currfilepath}{funkcja_malePodwojnie}{}
	\insertZadanie{\currfilepath}{regex_czy_palindrom}{}
	\insertZadanie{\currfilepath}{regex_czy_konczy}{}
	\insertZadanie{\currfilepath}{funkcja_toStr}{}
	
	\insertZadanie{\currfilepath}{zadanie_trojkat_dwie_petle}{}
	\insertZadanie{\currfilepath}{zadanie_trojkat_while}{}
	\insertZadanie{\currfilepath}{zadanie_trojkat_jedna_petla}{}
	%\insertZadanie{\currfilepath}{zadanie_trojkat}{}
	\insertZadanie{\currfilepath}{zadanie_trojkat_rekurencja}{}
	
	\insertZadanie{\currfilepath}{obiektowo_sortuj}{}
	\insertZadanie{\currfilepath}{parsuj_klucz_wartosc}{}
	\insertZadanie{\currfilepath}{zadanie_fork2}{}
}

% specjalne

\dbEntryBegin{slownik_zamiast_ifelse}\if1\dbEntryCheckResults
Zastanów się czy konstrukcję if/elif w funkcji \python{dzialanie()} z rozdziału \ref{Funkcje_jako_argumenty} można by zastąpić słownikiem, jak to ewentualnie zrobić i jakie mogłoby mieć to zalety bądź wady?
\fi

\dbEntryBegin{slownik_zamiast_ifelse-rozwiazanie}\if1\dbEntryCheckResults
Możemy zdefiniować słownik, w którym kluczem jest nazwa działania a wartością funkcja je realizująca.
Zaleta takiego podejścia jest łatwe rozszerzania takiego kodu o nowe działania (poprzez wstawienie kolejnej pary do słownika, co może dziać się w trakcie pracy programu).
\fi

% PwES domowe (?):

\dbEntryBegin{zadanie_suma2}\if1\dbEntryCheckResults
Napisz funkcję, przyjmującą dwa argumenty $a$ i $b$, która obliczy i zwróci sumę liczb całkowitych większych od a i mniejszych od b.
\fi

\dbEntryBegin{zadanie_suma3}\if1\dbEntryCheckResults
Rozwiąż zadanie \ref{zadanie_suma2} używając pętli \python{while}
\fi

\dbEntryBegin{funkcja_bezwzgledne}\if1\dbEntryCheckResults
Napisz funkcję \python{bezwzgledne(lista)}, która dla danej listy liczb wypiszę listę wartości bezwzględnych
tych liczb, tj. liczby ujemne zamieni na przeciwne, a liczby nieujemne pozostawi bez zmian.
Poszczególne liczby powinny być oddzielone pojedynczymi spacjami.
Przykładowe użycie funkcji powinno wyglądać następująco:
\begin{Verbatim}
 > bezwzgledne([5, -10, 15, 0])
5 10 15 0
\end{Verbatim}
\fi

\dbEntryBegin{zadanie_trojkat}\if1\dbEntryCheckResults
Napisz program, który używając pętli, wypisze na ekranie \textit{trójkąt z iksów}, taki jak poniżej:
\begin{Verbatim}
X
XX
XXX
XXXX
XXXXX
XXXXXX
XXXXXXX
\end{Verbatim}
\fi

\dbEntryBegin{funkcja_toStr}\if1\dbEntryCheckResults
Napisz funkcję \python{toStr(liczba, podstawa)}, która konwertuje podaną liczbę do reprezentacji napisowej w systemie o podanej podstawie.\\
Wskazówka: do testowania poprawności działania możesz użyć funkcji \python{int(napis, podstawa)}, możemy przyjąć że podstawa jest mniejsza od 37 tak aby starczyło liter alfabetu łacińskiego.
\fi

\dbEntryBegin{obiektowo_sortuj}\if1\dbEntryCheckResults
Korzystając z metod klasy \Verb{list} i/lub funkcji \python{sorted()} napisz funkcję która sortuje podaną listę w kolejności malejącej.
Funkcja nie może zmodyfikować oryginalnej listy.
\fi

\dbEntryBegin{obiektowo_sortuj2}\if1\dbEntryCheckResults
Rozwiąż zadanie \ref{obiektowo_sortuj} w alternatywny sposób (czyli jeżeli użyłeś \python{sorted()} \Verb{list}, a jeżeli użyłeś metod użyj \python{sorted()}).
\fi

\dbEntryBegin{parsuj_klucz_wartosc}\if1\dbEntryCheckResults
Napisz funkcję która konwertuje listę napisów postaci \Verb#klucz=wartosc# na słownik.
Funkcja musi dokonywać podziału napisów z listy w oparciu o pierwsze wystąpienie znaku równości przy pomocy metody \python{find()} typu przechowującego napisy (\Verb{str}).
Funkcja musi dodawać kolejne napisy do słownika w taki sposób że część przed znakiem równości stanowi klucz, a część po znaku równości stanowi wartość.\\
Np. dla listy postaci: \python{["aa=13", "b=Ala=kot", "f=xyz"]} funkcja powinna zwrócić słownik:
\begin{Verbatim}
{'b': 'Ala=kot', 'aa': '13', 'f': 'xyz'}
\end{Verbatim}
\fi

\dbEntryBegin{zadanie_trojkat_rekurencja}\if1\dbEntryCheckResults
Napisz program, który wypisze na ekranie \textit{trójkąt z iksów}, taki jak poniżej:
\begin{Verbatim}
X
XX
XXX
XXXX
XXXXX
XXXXXX
XXXXXXX
\end{Verbatim}
Stosując zamiast co najmniej jednej pętli rekurencję.
\emph{Wskazówka: Funkcja rekurencyjna to funkcja, która wywołuje samą siebie (typowo ze zmodyfikowanymi argumentami), dopóki zachodzi jakiś ustalony warunek (typowo zależny od argumentów).}
\fi

\dbEntryBegin{zadanie_fork2}\if1\dbEntryCheckResults
Napisz program który utworzy 1 potomka, \ul[black]{potomek} powinien wypisać PID rodzica i swój. % następnie zabić rodzica, poczekać 5 sekund, wypisać "Koniec" i zakończyć swoje działanie.
\fi

% dodatkowe:

\dbEntryBegin{dwucyfrowe_podzielne}\if1\dbEntryCheckResults
Napisz pętlę, która wypisze wszystkie dwucyfrowe liczby podzielne przez 7.
Kolejne liczby powinny być wypisane w jednym wierszu i porozdzielane pojedynczymi spacjami.
\fi

\dbEntryBegin{niepodzielne}\if1\dbEntryCheckResults
Napisz funkcję który wypisze liczby od 0 do 20 z pominięciem liczb podzielnych przez wartość określoną w jej argumencie.
\fi

\dbEntryBegin{funkcja_xor}\if1\dbEntryCheckResults
Napisz funkcję, przyjmującą dwa argumenty $a$ i $b$, która będzie realizować funkcję xor (zwróci wartość $a$ XOR $b$).
\\\emph{Wskazówka: dla ułatwienia można przyjąć że argumenty są zawsze typu logiczne True/False.}
\fi

\dbEntryBegin{funkcja_rownowazne}\if1\dbEntryCheckResults
Napisz funkcję, przyjmującą dwa argumenty $a$ i $b$, która sprawdzi warunek równoważności $a \Leftrightarrow b$ (sprawdzi czy $a$ wtedy i tylko wtedy gdy $b$).
\\\emph{Wskazówka: dla ułatwienia można przyjąć że argumenty są zawsze typu logiczne True/False.}
\fi

\dbEntryBegin{funkcja_implikacja}\if1\dbEntryCheckResults
Napisz funkcję, przyjmującą dwa argumenty $a$ i $b$, która sprawdzi warunek implikacji $a \Rightarrow b$ (sprawdzi czy z a wynika b).
\\\emph{Wskazówka: dla ułatwienia można przyjąć że argumenty są zawsze typu logiczne True/False.}
\fi

\dbEntryBegin{funkcja_skroty}\if1\dbEntryCheckResults
Napisz funkcję, która dla danej listy słów wypisze w kolejnych wierszach ich skróty w postaci
\Verb{<pierwsza litera>-<ostatnia litera> (<dlugosc slowa>)}.\\
Np. dla listy \python{['Interdyscyplinarne', 'Centrum', 'Modelowania']} powinna wypisać:
\begin{Verbatim}
I-e (18)
C-m (7)
M-a (11)
\end{Verbatim}
Wskazówka: wynik funkcji \python{len()} mierzącej długość napisu jest liczbą.
Do rozwiązania tego zadania może Ci się przydać konwersja tej liczby na napis (aby dało się ją skleić z innymi napisami), z użyciem funkcji \python{str()}
\fi

\dbEntryBegin{funkcja_malePodwojnie}\if1\dbEntryCheckResults
Napisz funkcję, która dla danej listy słów wypisze każde słowo z listy powtarzając 
każdą małą literę dwukrotnie.
Np. dla \python{['Ala', 'ma', 'kota', 'i PSA']} funkcja powinna wypisać:
\begin{Verbatim}
Allaa
mmaa
kkoottaa
ii PSA
\end{Verbatim}
\fi

\dbEntryBegin{regex_czy_palindrom}\if1\dbEntryCheckResults
Jak wiemy język złożony ze słów postaci \Verb{aa..aabb..bbaa..bb} (gdzie ilość liter a przed ciągiem liter b jest równa ilości liter a po tym ciągu) nie jest regularny.
Jednak programistyczne wyrażenia regularne są rozszerzone w stosunku co do tych spotykanych w matematyce i umożliwiają opis takiego języka.
Napisz funkcję, korzystającą z dopasowywania wyrażeń regularnych, która będzie sprawdzała czy podane słowo należy do tego języka.
\fi

\dbEntryBegin{regex_czy_konczy}\if1\dbEntryCheckResults
Napisz funkcję która sprawdzi z użyciem wyrażeń regularnych czy dany napis kończy się \Verb{xyz}.
\fi

% trójkąt seria:

\dbEntryBegin{zadanie_trojkat_dwie_petle}\if1\dbEntryCheckResults
Używając dwóch pętli \python{for}, jedna wewnątrz drugiej, napisz program, który wypisze na ekranie
\textit{trójkąt z iksów}, taki jak poniżej:
\begin{Verbatim}
X
XX
XXX
XXXX
XXXXX
XXXXXX
XXXXXXX
\end{Verbatim}
\fi

\dbEntryBegin{zadanie_trojkat_while}\if1\dbEntryCheckResults
Zmodyfikuj rozwiązanie zadania \ref{zadanie_trojkat_dwie_petle} tak aby zamiast co najmniej jednej pętli \python{for} użyć pętli \python{while}.
\fi

\dbEntryBegin{zadanie_trojkat_jedna_petla}\if1\dbEntryCheckResults
Zmodyfikuj rozwiązanie zadania \ref{zadanie_trojkat_dwie_petle} tak aby korzystało tylko z jednej pętli.
\fi

% bardziej dodatkowe:

\dbEntryBegin{funkcja_suma}\if1\dbEntryCheckResults
Napisz funkcję, która przyjmuje dwa argumenty i wypisuje ich sumę. Użyj jej do obliczenia (wypisania na konsolę) wartości kilku różnych sum.
\fi

\dbEntryBegin{pole_kola}\if1\dbEntryCheckResults
Napisz funkcję, która oblicza i zwraca pole koła o podanym promieniu. Użyj jej do obliczenia powierzchni koła o promieniu 13.

\teacher{W razie potrzeby można wymyślić całą gamę zadań analogicznych do tego, np. obliczanie pola powierzchni trapezu, kuli, objętości stożka, kuli, etc.}
\fi

\dbEntryBegin{wypisywanie_reszta_z_dzielenia}\if1\dbEntryCheckResults
\teacher{Zadanie stworzone jako propozycja zadania domowego dla grup początkujących, które nie zrealizowały dalszej części skryptu}

Napisz funkcję która przyjmuje dwa argumenty, wypisuje je w jednej linii rozdzielając dwukropkiem.
W kolejnej linii powinna wypisać wartość reszty z dzielenia pierwszego argumentu przez drugi.
\fi

\dbEntryBegin{funkcja_kwadratowa}\if1\dbEntryCheckResults
\teacher{Zadanie stworzone jako propozycja zadania domowego dla grup początkujących, które nie zrealizowały dalszej części skryptu}

Napisz funkcję obliczającą i zwracającą kwadrat podanej liczby. Użyj jej w funkcji która ma obliczyć
i zwrócić wartość funkcji kwadratowej $ax^2 + bx + c$ w zadanym punkcie x, dla zadanych parametrów a, b i c.

\begin{teacherOnly}
Można dodać także wymóg wdotyczący wypisywania rozwiązania -- np:\\
Funkcja powinna wypisać rozwiązanie w postaci:\\
	\Verb{Ax^2 + Bx + C dla x = X wynosi XX}\\
gdzie A, B, C i X zastąpione są wartościami argumentów przekazanych do funkcji, a XX obliczoną wartością np:\\
	\Verb{2x^2 + 7x + -2 dla x = 3 wynosi 37}
\end{teacherOnly}
\fi
