% Copyright (c) 2018-2020 Matematyka dla Ciekawych Świata (http://ciekawi.icm.edu.pl/)
% Copyright (c) 2018-2020 Robert Ryszard Paciorek <rrp@opcode.eu.org>
% 
% MIT License
% 
% Permission is hereby granted, free of charge, to any person obtaining a copy
% of this software and associated documentation files (the "Software"), to deal
% in the Software without restriction, including without limitation the rights
% to use, copy, modify, merge, publish, distribute, sublicense, and/or sell
% copies of the Software, and to permit persons to whom the Software is
% furnished to do so, subject to the following conditions:
% 
% The above copyright notice and this permission notice shall be included in all
% copies or substantial portions of the Software.
% 
% THE SOFTWARE IS PROVIDED "AS IS", WITHOUT WARRANTY OF ANY KIND, EXPRESS OR
% IMPLIED, INCLUDING BUT NOT LIMITED TO THE WARRANTIES OF MERCHANTABILITY,
% FITNESS FOR A PARTICULAR PURPOSE AND NONINFRINGEMENT. IN NO EVENT SHALL THE
% AUTHORS OR COPYRIGHT HOLDERS BE LIABLE FOR ANY CLAIM, DAMAGES OR OTHER
% LIABILITY, WHETHER IN AN ACTION OF CONTRACT, TORT OR OTHERWISE, ARISING FROM,
% OUT OF OR IN CONNECTION WITH THE SOFTWARE OR THE USE OR OTHER DEALINGS IN THE
% SOFTWARE.

%  BEGIN: Typy zmiennych
\subsection{Określanie typu zmiennej}

Do tej pory poznaliśmy kilka typów zmiennych w Pythonie: liczby, napisy oraz listy.
Poznaliśmy także metody konwersji pomiędzy niektórymi z typów (np. instrukcje \python{str()}, \python{int()}).
Jeżeli chcemy dowiedzieć się jakiego typu jest dana zmienna możemy skorzystać z funkcji \python{type()}:

\begin{CodeFrame}[python]{0.50\textwidth}
a, b, c = 1, 3.14, "Python"
print(a, type(a))
print(b, type(b))
print(c, type(c))
c = (a == 1)
print(c, type(c))
\end{CodeFrame}
\begin{CodeFrame}{auto}
1 <class 'int'>
3.14 <class 'float'>
Python <class 'str'>
True <class 'bool'>
\end{CodeFrame}

Zauważ że inny typ związany jest z liczbami całkowitymi, inny z rzeczywistymi, a jeszcze inny z wartościami logicznymi (\python{True}/\python{False}). Zauważ także, że zmienna może zmienić swój typ.

\begin{teacherOnly}
\subsubsection{Typowanie w Pythonie a w innych językach \zaawansowane{***}}
W grupach gdzie część osób miała styczność z C++, zwłaszcza bardziej współczesnym ($\ge 11$) można porównać typowanie w Pythonie do typowania w C++ z użyciem słowa kluczowego auto. Python:
\begin{itemize}
\item określa zmiennej w momencie napotkania jej deklaracji na podstawie wartości do niej przypisywanej (tak samo jak C++ robi dla zmiennych auto)
\begin{CodeFrame*}[cpp]{}
#include <stdio.h>
int main() {
  auto a = 1;
  printf("%d", a);
}
\end{CodeFrame*}
\item nie pozwala odwołać się do zmiennej nie zadeklarowanej (np. PHP pozwala, generując jedynie "Notice")
\begin{CodeFrame*}[php]{}
<?php
$a = $b +1;
echo $a, $b;
?>
\end{CodeFrame*}
\item pozwala na zmianę typu zmiennej w trakcie działania (C++ nie pozwala nawet z typem auto)
\begin{CodeFrame*}[python]{}
a = "abc"
print(a, type(a))
a = 1
print(a, type(a))
\end{CodeFrame*}
\end{itemize}

\subsubsection{Wielkość zmiennej typu int \zaawansowane{***}}

(jest to bardziej ciekawostka, o której można wspomnieć gdy czas pozwala lub gdy padnie takie pytanie)\\ \\
Python nie posiada wbudowanego ograniczania wielkości liczb całkowitych, jednak wielkość wartości przechowywanej w tym typie może mieć wpływ na rozmiar zmiennej.

\begin{CodeFrame}[python]{0.45\textwidth}
x = 1
print(x, type(x), x.__sizeof__())
x = 12**10
print(x, type(x), x.__sizeof__())
x = 12**20
print(x, type(x), x.__sizeof__())
x = 13
print(x, type(x), x.__sizeof__())
\end{CodeFrame}
\begin{CodeFrame}{auto}
1 <class 'int'> 28
61917364224 <class 'int'> 32
3833759992447475122176 <class 'int'> 36
13 <class 'int'> 28
\end{CodeFrame}
\end{teacherOnly}
%  END: Typy zmiennych
