% Copyright (c) 2018-2020 Matematyka dla Ciekawych Świata (http://ciekawi.icm.edu.pl/)
% Copyright (c) 2018-2020 Robert Ryszard Paciorek <rrp@opcode.eu.org>
% 
% MIT License
% 
% Permission is hereby granted, free of charge, to any person obtaining a copy
% of this software and associated documentation files (the "Software"), to deal
% in the Software without restriction, including without limitation the rights
% to use, copy, modify, merge, publish, distribute, sublicense, and/or sell
% copies of the Software, and to permit persons to whom the Software is
% furnished to do so, subject to the following conditions:
% 
% The above copyright notice and this permission notice shall be included in all
% copies or substantial portions of the Software.
% 
% THE SOFTWARE IS PROVIDED "AS IS", WITHOUT WARRANTY OF ANY KIND, EXPRESS OR
% IMPLIED, INCLUDING BUT NOT LIMITED TO THE WARRANTIES OF MERCHANTABILITY,
% FITNESS FOR A PARTICULAR PURPOSE AND NONINFRINGEMENT. IN NO EVENT SHALL THE
% AUTHORS OR COPYRIGHT HOLDERS BE LIABLE FOR ANY CLAIM, DAMAGES OR OTHER
% LIABILITY, WHETHER IN AN ACTION OF CONTRACT, TORT OR OTHERWISE, ARISING FROM,
% OUT OF OR IN CONNECTION WITH THE SOFTWARE OR THE USE OR OTHER DEALINGS IN THE
% SOFTWARE.

\begin{ProTip}{Informacja}
Często w zadaniach programistycznych i zawsze w ramach tego kursu jeżeli jest powiedziane:
\begin{itemize}
\item „napisz funkcję” to znaczy że ma zostać napisana funkcja, a nie jedynie kod programu, który mógłby stanowić wnętrze (ciało) tej funkcji,
\item „napisz program” to znaczy że ma zostać napisany pełny kod programu realizujący podane czynności,
\item „napisz pętlę/warunek/...” to znaczy że wystarczy napisać sam kod pętli, warunku, innej konstrukcji (ale nie tylko jego wnętrze, lecz kod całej żądanej konstrukcji składniowej),

\item „napisz funkcję przyjmującą napis” to znaczy że funkcja ma mieć argument, który będzie traktowany jako napis
	(nie oznacza to że wymaga się wczytania tego napisu „z klawiatury”\footnote{
		Powszechnie używane w nauce programowania wczytywanie danych „z klawiatury” / odpytywanie użytkownika o kolejne parametry na ogół nie jest najlepszym rozwiązaniem programistycznym,
		o tym dlaczego dowiesz się w dalszych częściach tego kursu
	}),
\item „napisz funkcję zwracającą X” to znaczy że funkcja ma zwrócić (poprzez return) wartość określoną przez X (nie ma jej wypisywać na ekran),
\item „napisz funkcję wypisującą X” to znaczy że funkcja ma wypisać na ekran (standardowe wyjście) wartość określoną przez X,
\end{itemize}
\end{ProTip}
