% Copyright (c) 2018-2020 Matematyka dla Ciekawych Świata (http://ciekawi.icm.edu.pl/)
% Copyright (c) 2018-2020 Robert Ryszard Paciorek <rrp@opcode.eu.org>
% 
% MIT License
% 
% Permission is hereby granted, free of charge, to any person obtaining a copy
% of this software and associated documentation files (the "Software"), to deal
% in the Software without restriction, including without limitation the rights
% to use, copy, modify, merge, publish, distribute, sublicense, and/or sell
% copies of the Software, and to permit persons to whom the Software is
% furnished to do so, subject to the following conditions:
% 
% The above copyright notice and this permission notice shall be included in all
% copies or substantial portions of the Software.
% 
% THE SOFTWARE IS PROVIDED "AS IS", WITHOUT WARRANTY OF ANY KIND, EXPRESS OR
% IMPLIED, INCLUDING BUT NOT LIMITED TO THE WARRANTIES OF MERCHANTABILITY,
% FITNESS FOR A PARTICULAR PURPOSE AND NONINFRINGEMENT. IN NO EVENT SHALL THE
% AUTHORS OR COPYRIGHT HOLDERS BE LIABLE FOR ANY CLAIM, DAMAGES OR OTHER
% LIABILITY, WHETHER IN AN ACTION OF CONTRACT, TORT OR OTHERWISE, ARISING FROM,
% OUT OF OR IN CONNECTION WITH THE SOFTWARE OR THE USE OR OTHER DEALINGS IN THE
% SOFTWARE.

Python jest wysokopoziomowym językiem programowania ogólnego przeznaczenia.
Oznacza to że jego składnia została tak zbudowana aby maksymalizować czytelność kodu dla człowieka i być niezależna od sprzętowych i implementacyjnych detali
	oraz że nie ma pojedynczego dedykowanego obszaru zastosowań (z łatwością może być stosowany do różnych zastosowań).

Wsórd cech i zalet Pythona należy wymienić:
\begin{itemize}
\item jest językiem interpretowanym\footnote{może być i w niektórych sytuacjach podlega kompilacji do kodu pośredniego celem zwiększenia wydajności}, co daje łatwiejsze modyfikowanie kodu, eksperymentowanie z nim, itd
\item jest jednym z najpopularniejszych języków programowania (wg niektórych źródeł nawet najpopularniejszym), więc nie jest "dydaktyczną egzotyką", która potem do niczego się nie przyda
\item działa na wielu różnych platformach sprzętowych i na różnych systemach oparacyjnych
\item istnieje bardzo wiele bibliotek pythonowych (posiadających pythonowe API), a można korzystać także z "niedostosowanych" do Pythona bibliotek C (.so, .dll)
\item jest łatwo rozszerzalny przy pomocy (własnych) bibliotek/modułów tworzonych w C/C++ (podstawowy interpreter napisany jest C)
\item kod pythonowy może być łatwo wywoływany z poziomu C/C++, co pozwala na łatwe wykorzystanie Pythona jako języka skryptowego dla projektów tworzonych w C/C++
\end{itemize}

Ponadto Python jest wygodniejszy w uczeniu od wielu innych języków m.in. ze względu na to że
	kod pythonowy realizujący tą samą funkcjonalność, przy takim samym poziomie obsługi błędów, etc i podobnej czytelności,
	praktycznie zawsze jest krótszy od kodu C (a potrafi być krótszy kilkukrotnie, więc łatwiej go pokazać i omówić).

W ramach kursu zajmować się będziemy językiem Python w wersji 3 (czyli o pełnym numerze zaczynającym się od 3, np. 3.7.1).
Należy o tym pamiętać i zwracać na to uwagę, gdyż wersja ta różni się na tyle znacząco w stosunku co do starszej, lecz wciąż używanej wersji 2,
	że programy, prezentowane w tym skrypcie i te które będziemy pisać na zajęciach nie będą działać w drugiej wersji Pythona.

Skrypty poświęcone programowaniu w Pythonie, z których będziemy korzystać są dość kompletnym wprowadzeniem do programowania w Pythonie i omawiają prawie wszystkie najważniejsze elementy tego języka.
Zagadnienia bardziej zaawansowane, treści dodatkowe i ciekawostki zostały oznaczone ikonką {\Symbola 🤔} (Thinking Face Emoji).
