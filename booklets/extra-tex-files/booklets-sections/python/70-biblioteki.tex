% Copyright (c) 2018-2020 Matematyka dla Ciekawych Świata (http://ciekawi.icm.edu.pl/)
% Copyright (c) 2018-2020 Robert Ryszard Paciorek <rrp@opcode.eu.org>
% 
% MIT License
% 
% Permission is hereby granted, free of charge, to any person obtaining a copy
% of this software and associated documentation files (the "Software"), to deal
% in the Software without restriction, including without limitation the rights
% to use, copy, modify, merge, publish, distribute, sublicense, and/or sell
% copies of the Software, and to permit persons to whom the Software is
% furnished to do so, subject to the following conditions:
% 
% The above copyright notice and this permission notice shall be included in all
% copies or substantial portions of the Software.
% 
% THE SOFTWARE IS PROVIDED "AS IS", WITHOUT WARRANTY OF ANY KIND, EXPRESS OR
% IMPLIED, INCLUDING BUT NOT LIMITED TO THE WARRANTIES OF MERCHANTABILITY,
% FITNESS FOR A PARTICULAR PURPOSE AND NONINFRINGEMENT. IN NO EVENT SHALL THE
% AUTHORS OR COPYRIGHT HOLDERS BE LIABLE FOR ANY CLAIM, DAMAGES OR OTHER
% LIABILITY, WHETHER IN AN ACTION OF CONTRACT, TORT OR OTHERWISE, ARISING FROM,
% OUT OF OR IN CONNECTION WITH THE SOFTWARE OR THE USE OR OTHER DEALINGS IN THE
% SOFTWARE.

%  BEGIN: Biblioteki
\section{Biblioteki}

Ideą korzystania z funkcji w trakcie tworzenia programu jest zapewnienie jego większej czytelności oraz unikanie powtarzania kodu robiącego to samo w wielu miejscach programu – kod umieszczamy w funkcji którą tylko wywołujemy z odpowiednimi argumentami i odbieramy wynik działania (np. poprzez zwracaną wartość). Rozwinięciem tej ideii są biblioteki stanowiące zbiory funkcji oraz struktur danych (własnych typów zmiennych) służących do realizacji określonych zadań.

Do tej pory korzystaliśmy z elementów standardowej biblioteki dostarczanej z Pythonem. W rozdziale tym zaprezentujemy kilka różnych przykładowych bibliotek (w tym wchodzących w skład biblioteki standardowej Pythona), jednak żadnej z nich nie będziemy tutaj szczegółowo omawiać, gdyż nie miałoby to większego sensu. Istnieje ogromna liczba bibliotek dedykowanych różnym celom (obsługa formatów plików, standardów komunikacyjnych, tworzenie grafiki, ...) i nie ma sensu uczyć się ich bez realnej potrzeby zastosowania – programowanie w dużej mierze polega na wyszukiwaniu właściwych bibliotek, zapoznawaniu się z ich dokumentacją i wykorzystywaniu ich w własnych programach. W przypadku Pythona biblioteki najczęściej mają postać modułów pythonowych, które włączamy poprzez deklarację \python{include}.

\teacher{Poniższe przykłady mają zaprezentować potencjał możliwy do uzyskania dzięki dostępnym bibliotekom oraz to że są one dużym ułatwieniem dla programisty.}

\subsection{XML}

Extensible Markup Language (XML) jest tekstowym formatem wymiany danych. W odróżnieniu od formatu klasycznego formatu utożsamiającego linię z rekordem złożonym z pól oddzielanych wskazanym separatorem może on w łatwy sposób opisywać bardziej złożoną (drzewiastą a nie tabelkową) postać danych. Dokument XML składa się z zagnieżdżonych w sobie znaczników, każdy z nich może posiadać atrybuty oraz wartość, którą jest tekst zawierający lub nie kolejne znaczniki. Kolejność występowania elementów w dokumencie jest znacząca. Każdy znacznik otwierający posiada odpowiadający mu znacznik zamykający (np. \Verb{<b>aa</b>}), znaczniki bez wartości mogą być samo-zamykające (np. \Verb{<g />}). Dokumenty HTML mogą być zgodne z wymogami formalnymi XML tym samym stanowiąc dokumenty XML.

Do obsługi XML w Pythonie można skorzystać np. z modułu \textit{ElementTree} (ale nie jest on jedyną biblioteką której możemy użyć):
\begin{CodeFrame*}[python]{}
import xml.etree.ElementTree as xml

txt = """<a>
	<b>A<h>qwe ... rty</h></b> ABCD... &amp;&apos; HIJ...
	<c x="q" w="p p">EE FĄ</c> <g y="zz" />
	<c x="pp">123 <d rr="oo">456</d> 78 90.</c>
</a>"""

rootNode = xml.fromstring(txt)

print("nazwa głównego elementu to:", rootNode.tag)
print("jego potomkowie to:")
for subNode in rootNode:
	print(" ", subNode.tag, ":", xml.tostring(subNode, encoding="unicode"))

# możemy pobrać listę potomków o określonej nazwie
# albo od razu po nich iterować pętlą for subNode in rootNode.iter("c"):
cSubNodes = list( rootNode.iter("c") )
if cSubNodes:
	for subNode in cSubNodes:
		print('element "c" ma atrybuty': subNode.attrib
else
	print('nie ma elementów "c"')

# możemy też używać iteratorów bezpośrednio, np:
print("pierwszy węzeł c ma atrybuty:")
try:
	ci = rootNode.iter("c")
	print(next(ci).attrib)
except StopIteration:
	print(" [brak takiego węzła]")
\end{CodeFrame*}

\textit{ElementTree} pozwala też na modyfikowanie XMLa poprzez zmianę/dodawanie/usuwanie atrybutów, czy też całych tagów.

Innym sposobem zapisu ustrukturyzowanych danych w postaci tekstowej jest JSON. Przypomina on trochę output funkcji print z podanym do niej słownikiem lub listą. Do jego obsługi w Pythonie słuzy moduł \textit{json}:

\begin{CodeFrame*}[python]{}
import json, pprint

a='''{
	"info": "bbb",
	"ver": 31,
	"d": [
		{"a": 21, "b": {"x": 1, "y": 2}, "c": [9, 8, 7]},
		{"a": 17, "b": {"x": 6, "y": 7}, "c": [6, 5, 4]}
	]
}'''

# interpretacja napisu jako zbioru danych w formacie json
d = json.loads(a)

# wypisanie zbioru danych
pprint.pprint(d) # pprint ładnie formatuje złożone zbiory danych

# jak widać jest to zagnieżdżona struktura list i słowników
# odpowiadająca 1 do 1 temu co było w napisie

# dostęp do poszczególnych elementów: "po pythonowemu"
print(d["d"][1]["b"])
d["d"][1]["b"]["x"] = "XXX"

# wygenerowanie json'a w oparciu o zmienną pythonową
c = json.dumps(d, ensure_ascii=False)
print(c)
\end{CodeFrame*}

\subsection{SQL}

Innym sposobem przechowywania danych niż w postaci plików tekstowych są systemy baz danych.
Standardowym językiem używanym do komunikacji z systemami bazodanowymi jest SQL.
Pomimo jego standaryzacji istnieją różnice w składni zapytań dla poszczególnych silników bazodanowych (takich jak: MariaDB, PostgreSQL, SQLite, ...).

Typowo komunikacja z bazą danych odbywa się za pośrednictwem biblioteki odpowiedzialnej za nawiązanie połączenia z serwerem i przekazywanie do niego zapytań SQL.
Wymaga to działania osobnego procesu (często nawet na innej maszynie) obsługującego silnik bazodanowy, co jest pożądanym rozwiązaniem dla baz danych z których równocześnie może korzystać wielu klientów.
Typowym przykładem może być komunikacja skryptów jakiegoś serwisu interetowego z bazą danych.

Jednak takie podejście nie jest wygodne w rozwiązaniach nie wymagających współdzielenia bazy danych.
Do zastosowań takich można użyć biblioteki SQLite, która pozwala na łatwe stosowanie bazy SQLowej do wewnętrznych potrzeb aplikacji, bez konieczności uruchamiania osobnego systemu bazodanowego.
SQLite można wykorzystywać także bezpośrednio z poziomu Pythona, dzięki modułowi \textit{sqlite3}:

\begin{CodeFrame*}[python]{}

import sqlite3
import os.path

if os.path.isfile('example.db'):
	create = False
else:
	create = True

conn = sqlite3.connect('example.db')
c = conn.cursor()

if create:
	print("create new db")
	c.execute("CREATE TABLE users (uid INT PRIMARY KEY, name TEXT);")
	c.execute("CREATE TABLE posts (pid INT PRIMARY KEY, uid INT, text TEXT);")
	
	c.execute("INSERT INTO users VALUES (21, 'user A');")
	c.execute("INSERT INTO users VALUES (2671, 'user B');")
	
	c.execute("INSERT INTO posts VALUES (1, 21, 'abc ..');")
	c.execute("INSERT INTO posts VALUES (2, 21, 'qwe xyz');")
	c.execute("INSERT INTO posts VALUES (3, 2671, 'test');")

	conn.commit()

maxUid = 100
for r in c.execute("SELECT * FROM users WHERE uid < ?;", (maxUid,)):
	print(r)

for r in c.execute("SELECT u.name, p.text FROM users AS u JOIN posts AS p ON (u.uid = p.uid);"):
	print(r)
\end{CodeFrame*}

\subsection{GUI}

Przykłady użycia 3 różnych graficznych interfejsów użytkownika z poziomu Pythona można znaleźć na \url{http://vip.opcode.eu.org/#Graficzny_interfejs_użytkownika}.
W odróżnieniu od poprzednich przykładów, te biblioteki nie wchodzą w skład pythonowskiej biblioteki standardowej i mogą wymagać doinstalowania odpowiednich pakietów oprogramowania.
%  END: Biblioteki
