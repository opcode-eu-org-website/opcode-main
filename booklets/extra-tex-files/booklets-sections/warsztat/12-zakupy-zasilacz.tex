% Copyright (c) 2017-2020 Matematyka dla Ciekawych Świata (http://ciekawi.icm.edu.pl/)
% Copyright (c) 2017-2020 Robert Ryszard Paciorek <rrp@opcode.eu.org>
% 
% MIT License
% 
% Permission is hereby granted, free of charge, to any person obtaining a copy
% of this software and associated documentation files (the "Software"), to deal
% in the Software without restriction, including without limitation the rights
% to use, copy, modify, merge, publish, distribute, sublicense, and/or sell
% copies of the Software, and to permit persons to whom the Software is
% furnished to do so, subject to the following conditions:
% 
% The above copyright notice and this permission notice shall be included in all
% copies or substantial portions of the Software.
% 
% THE SOFTWARE IS PROVIDED "AS IS", WITHOUT WARRANTY OF ANY KIND, EXPRESS OR
% IMPLIED, INCLUDING BUT NOT LIMITED TO THE WARRANTIES OF MERCHANTABILITY,
% FITNESS FOR A PARTICULAR PURPOSE AND NONINFRINGEMENT. IN NO EVENT SHALL THE
% AUTHORS OR COPYRIGHT HOLDERS BE LIABLE FOR ANY CLAIM, DAMAGES OR OTHER
% LIABILITY, WHETHER IN AN ACTION OF CONTRACT, TORT OR OTHERWISE, ARISING FROM,
% OUT OF OR IN CONNECTION WITH THE SOFTWARE OR THE USE OR OTHER DEALINGS IN THE
% SOFTWARE.

\subsection{Zasilacz}

Jednym z najważniejszych elementów zestawu służącego do zabawy elektroniką jest źródło zasilania.
Może nim być nawet zwykła bateria, jednak dla wygody i bezpieczeństwa podłączanych układów warto posiadać podstawowy zasilacz regulowany z regulowanym ograniczeniem prądowym.
Powinien on zapewniać co najmniej:
\begin{itemize}
	\item regulację napięcia wyjściowego w zakresie od 2.5V do 7V
	\item regulowane ograniczenie prądowe\footnote{
		w przypadku próby pobrania większego prądu niż nastawiony zasilacz powinien
		przejść z trybu stałego napięcia (CV) do trybu stałego prądu (CV)
		i obniżyć podawane napięcie tak aby płynął nastawiony prąd.
	} w zakresie od 20mA do 500mA
	\item sygnalizacja trybu CV/CC (np. za pomocą diody LED)
\end{itemize}
Dla wygody jego używania warto aby był wyposażony także w:
\begin{itemize}
	\item woltomierz pokazujący wartość napięcia wyjściowego
	\item amperomierz pokazujący wartość prądu podawanego do obciążenia
\end{itemize}

Poniżej kilka propozycji przetwornic do wyboru.

\subsubsection{przetwornica DC/DC Step-Down XL4015}
	\begin{center}
		\includegraphics[height=3.3cm]{warsztat_elektroniczny/moduł_XL4015_1}
	\end{center}
	\begin{itemize}
		\wada brak zabezpieczenia przed odwrotną polaryzacją zasilania wejściowego (zamianą plusa z minusem na wejściu)
		\wada brak wskazań wartości napięcia i prądu
		\info od 8PLN (moduł "czerwony"), od 10PLN (moduł "niebieski")
		\uwaga należy zwrócić uwagę aby moduł posiadał dwa potencjomery - jeden do regulacji napięcia drugi prądu (występują moduły umożliwiające tylko regulację napięcia)
	\end{itemize}

\subsubsection{przetwornica DC/DC Step-Down XL4015 z woltomierzem i amperomierzem LED}
	\begin{center}
		\includegraphics[height=3.3cm]{warsztat_elektroniczny/moduł_XL4015_LED_1}
		\hspace{0.5cm}
		\includegraphics[height=3.3cm]{warsztat_elektroniczny/moduł_XL4015_LED_2}
	\end{center}

	\begin{itemize}
		\zaleta modularna konstrukcja (oparta na opisanym wcześniej module "czerwonym")
		\wada brak zabezpieczenia przed odwrotną polaryzacją zasilania wejściowego (zamianą plusa z minusem na wejściu)
		\wada mała dokładność pomiaru (wskazuje 0.01A gdy płynie 0.1A)
		\wada brak możliwości (oficjalnie udokumentowanej) kalibracji pomiarów
		\wada słaba czytelność wyświetlacza LED przy dobrym oświetleniu
		\info od 30PLN
	\end{itemize}
	
\subsubsection{przetwornica DC/DC Step-Down XL4015 z woltomierzem i amperomierzem LCD}
	\begin{center}
		\includegraphics[height=3.3cm]{warsztat_elektroniczny/moduł_XL4015_LCD_1}
		\hspace{0.5cm}
		\includegraphics[height=3.3cm]{warsztat_elektroniczny/moduł_XL4015_LCD_2}
		\hspace{0.5cm}
		\includegraphics[height=3.3cm]{warsztat_elektroniczny/moduł_XL4015_LCD_3}
	\end{center}
	\begin{itemize}
		\zaleta  zabezpieczenie przed odwrotną polaryzacją zasilania wejściowego (zamianą plusa z minusem na zaciskach wejściowych),
			\textbf{uwaga:} dotyczy wersji pokazanej na zdjęciu, podobna wersja z dwoma dodatkowymi LEDami najprawdopodobniej nie posiada tego zabezpieczenia
		\zaleta  dobra precyzja pomiaru (wskazuje 0.04A gdy płynie 0.03A)
		\zaleta  możliwość łatwej kalibracji pomiarów
		\wada niebezpieczne gniazdko USB (można podać na nie zbyt wysokie napięcia)
		\info od 40PLN
	\end{itemize}
	
\subsubsection{elementy dodatkowe}
Do wybranej przetwornicy sugerujemy dokupienie gniazda DC 2.5/5.5 wraz z przewodem oraz baterii 9V (ze złączem i wtykiem DC 2.5/5.5) lub zasilacza wtyczkowego np. 12V z prądem większym niż 0.9A (z wtykiem DC 2.5/5.5). Pozwoli to na wygodne podłączanie i odłączanie zasilania od przetwornicy poprzez rozpięcie wtyku DC.
\begin{center}
	\includegraphics[height=4.3cm]{warsztat_elektroniczny/przewody_zasilające}
\end{center}
