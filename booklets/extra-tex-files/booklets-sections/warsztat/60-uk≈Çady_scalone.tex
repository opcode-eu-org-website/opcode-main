% Copyright (c) 2017-2020 Matematyka dla Ciekawych Świata (http://ciekawi.icm.edu.pl/)
% Copyright (c) 2017-2020 Robert Ryszard Paciorek <rrp@opcode.eu.org>
% Copyright (c) 2020 Krzysztof Lasocki <krz.lasocki@gmail.com>
% 
% MIT License
% 
% Permission is hereby granted, free of charge, to any person obtaining a copy
% of this software and associated documentation files (the "Software"), to deal
% in the Software without restriction, including without limitation the rights
% to use, copy, modify, merge, publish, distribute, sublicense, and/or sell
% copies of the Software, and to permit persons to whom the Software is
% furnished to do so, subject to the following conditions:
% 
% The above copyright notice and this permission notice shall be included in all
% copies or substantial portions of the Software.
% 
% THE SOFTWARE IS PROVIDED "AS IS", WITHOUT WARRANTY OF ANY KIND, EXPRESS OR
% IMPLIED, INCLUDING BUT NOT LIMITED TO THE WARRANTIES OF MERCHANTABILITY,
% FITNESS FOR A PARTICULAR PURPOSE AND NONINFRINGEMENT. IN NO EVENT SHALL THE
% AUTHORS OR COPYRIGHT HOLDERS BE LIABLE FOR ANY CLAIM, DAMAGES OR OTHER
% LIABILITY, WHETHER IN AN ACTION OF CONTRACT, TORT OR OTHERWISE, ARISING FROM,
% OUT OF OR IN CONNECTION WITH THE SOFTWARE OR THE USE OR OTHER DEALINGS IN THE
% SOFTWARE.

\section{Układy scalone}
\textbf{Układy scalone} pozwalają szybko i sprawnie konstruować skomplikowane układy elektroniczne przy minimum nakładu i kosztu. Dzięki nim możemy
używać gotowych bloków cyfrowych (np. licznik, sumator, bramka logiczna) oraz analogowych (regulator napięcia, wzmacniacz operacyjny,
przełączniki anlogowe itp.) zamiast budować je od postaw z \textbf{elementów dyskretnych} (transyzstory, oporniki itp.). Produkowane są
układy o bardzo różnych stopniach integracji - od bardzo niskich, takich jak np. bramki logiczne, do bardzo wysokich, jak np. gotowe komputery
(tzw. SoC - system on a chip).
\\

\begin{wrapfigure}{r}{0.28\textwidth}
  \begin{center}
    \vspace{-28pt}
    \includegraphics[height=0.25\textwidth,angle=90,origin=c]{img/elektronika/DIP14}
    \vspace{-20pt}
    
    \small{14-pinowa obudowa DIP,\\ widok z~góry}
    \vspace{-19pt}
  \end{center}
\end{wrapfigure}

Układ scalony posiada od kilku do kilkudziesięciu wyprowadzeń które służą do łączenia go z innymi elementami układu elektronicznego.
Każde z tych wyprowadzeń pełni określoną funkcję (zasilanie, wejścia, wyjścia itp.). W karcie~katalogowej danego układu scalonego wszystkie
piny są ponumerowane, nazwane a ich funkcje opisane.
\\

Na razie skupimy się na prostszych układach, które możesz umieścić w swojej płytce stykowej - układach, które produkowane są w obudowach DIP.
Charakteryzuje się ona prostopadłymi, szeroko (w porównaniu do innych obudów) rozstawionymi pinami które można łatwo umieścić w płytce stykowej.

W obudowie typu DIP, oznaczenie pinu 1. ma formę wklęsłego wcięcia na jednym z końców układu, lub kropki na narożniku. W przypadku tego drugiego, pin przy którym znajduje się kropka to pin nr 1. W przypadku wcięcia, pin nr 1 to lewy dolny, patrząc na układ obrócony oznaczonym końcem w lewo, napisami na obudowie w Twoją stronę. Kolejne piny liczy się idąc przeciwnie do ruchu wskazówek zegara. Przedstawia to rysunek obok.
Przy podłączaniu układu scalonego ważne jest aby nie pomylić numerów pinów (nóżek). Każdy z nich ma z góry określoną funkcję.

\begin{ProTip}{\normalfont{\strong{Wskazówka}}}
  W elektronice do wyrażania wymiarów stosuje się m.in. jednostkę \textbf{mil}\footnote{zwaną też \textit{thou}, od \textit{thousands of an inch}},
  równą 1/1000 cala. Rozstaw nóżek w obudowie DIP to 100 mili (= 2.54 mm). Taki sam jest również rozstaw otworków na płytce stykowej.
\end{ProTip}
