% Copyright (c) 2017-2020 Matematyka dla Ciekawych Świata (http://ciekawi.icm.edu.pl/)
% Copyright (c) 2017-2020 Robert Ryszard Paciorek <rrp@opcode.eu.org>
% Copyright (c) 2020 Krzysztof Lasocki <krz.lasocki@gmail.com>
% 
% MIT License
% 
% Permission is hereby granted, free of charge, to any person obtaining a copy
% of this software and associated documentation files (the "Software"), to deal
% in the Software without restriction, including without limitation the rights
% to use, copy, modify, merge, publish, distribute, sublicense, and/or sell
% copies of the Software, and to permit persons to whom the Software is
% furnished to do so, subject to the following conditions:
% 
% The above copyright notice and this permission notice shall be included in all
% copies or substantial portions of the Software.
% 
% THE SOFTWARE IS PROVIDED "AS IS", WITHOUT WARRANTY OF ANY KIND, EXPRESS OR
% IMPLIED, INCLUDING BUT NOT LIMITED TO THE WARRANTIES OF MERCHANTABILITY,
% FITNESS FOR A PARTICULAR PURPOSE AND NONINFRINGEMENT. IN NO EVENT SHALL THE
% AUTHORS OR COPYRIGHT HOLDERS BE LIABLE FOR ANY CLAIM, DAMAGES OR OTHER
% LIABILITY, WHETHER IN AN ACTION OF CONTRACT, TORT OR OTHERWISE, ARISING FROM,
% OUT OF OR IN CONNECTION WITH THE SOFTWARE OR THE USE OR OTHER DEALINGS IN THE
% SOFTWARE.

\section{Przetwornica zasilająca}

Przetwornica z regulacją napięcia i ograniczenia prądowego będzie pełniła funkcję (stosunkowo taniego i przenośnego) zasilacza laboratoryjnego.

\subsection{Uruchomienie}
\subsubsection{Zasilanie przetwornicy}

\begin{wrapfigure}{r}{0.35\textwidth}
  \begin{center}
    \vspace{-40pt}
    \includegraphics[width=0.33\textwidth]{warsztat_elektroniczny/przewody_zasilające.jpg}
    \vspace{-40pt}
  \end{center}
\end{wrapfigure}


Nasza przetwornica może być zasilona z baterii 9V lub zasilacza wtyczkowego 12V DC.
Zarówno gniazdo baterii jak i zasilacz może być wyposażone we wtyk DC 2.5/5.5 lub 2.1/5.5 (taki jak widoczny na zdjęciu obok) lub bezpośrednio wyprowadzone przewodu.
W przypadku zastosowania wtyku DC będzie możliwe łatwe odłączanie zasilania od przetwornicy. Jednak konieczne jest wtedy użycie także odpowiedniego gniazda DC.

\begin{ProTip}{\normalfont{\strong{Uwaga}}}
  \textbf{Niektóre przetwornice nie są odporne na niepoprawną polaryzację wejścia. Odwrotne podłączenie napięcia do wejścia takiej przetwornicy doprowadzi do jej zniszczenia. Dlatego odłącz przetwornicę od kabli zasilających zanim sprawdzisz ich polaryzację.}
\end{ProTip}

\vspace{13pt}\noindent
Niezależnie od tego, czy do zasilania przetwornicy używamy baterii, czy zasilacza wtyczkowego, musimy~\textbf{sprawdzić
  polaryzację przewodów, które będziemy podłączać do przetwornicy}. 
\\

\noindent W tym celu:
\begin{itemize}
	\item \textbf{odłączamy przewody zasilające od przetwornicy}.
	\item wkładamy wtyk DC do gniazda DC zasilacza.
	\item włączamy multimetr i nastawiamy go na pomiar napięcia stałego (DC) w zakresie do 20V.
	\item dbając o to, aby przewody zasilające nie zetknęły się ze sobą (aby uniknąć zwarcia) podłączamy zasilanie, czyli wkładamy zasilacz wtyczkowy do gniazdka lub podłączamy baterię do złącza baterii.
	\item dokonujemy pomiaru napięcia na przewodach zasilających, podłączając czerwoną sondę (dodatnią) do czerwonego przewodu, a masę (czarną sondę) do czarnego przewodu.
        \item Multimetr powinien wskazać dodatnie napięcie. Jeżeli wskazuje ujemne, to znaczy, że polaryzacja naszego przewodu~zasilającego jest odwrotna. W takiej sytuacji \textbf{czerwony} przewód od zasilacza (lub baterii) jest \textbf{ujemny} a \textbf{czarny} jest \textbf{dodatni}. Jeśli multimetr wskazuje napięcie dodatnie, oznacza to, że kolory przewodów są zgodne z polaryzacją. 
	\item zapisujemy jaki kolor przewodu odpowiada masie, a jaki biegunowi dodatniemu.
	\item \textbf{wyłączamy zasilanie} – wyjmujemy wtyk DC z gniazdka DC, odłączamy baterię od złącza lub wyjmujemy zasilacz wtyczkowy z gniazdka.
\end{itemize}
Teraz możemy przykręcić przewody zasilające do naszej przetwornicy. Należy zwrócić szczególną uwagę aby podłączyć je do zacisków wejściowych (oznaczonych jako \texttt{IN}) z zachowaniem ustalonej przez nas polaryzacji:
\begin{itemize}
	\item przewód na którym mieliśmy masę (biegun ujemny) podłączamy do IN- / GND
	\item przewód na którym mieliśmy biegun dodatni podłączamy do IN+
\end{itemize}

\begin{wrapfigure}{r}{0.4\textwidth}
  \begin{center}
    \vspace{-30pt}
    \includegraphics[width=0.4\textwidth]{warsztat_elektroniczny/moduł_XL4015_LCD_2.jpg}
    \vspace{-50pt}
  \end{center}
\end{wrapfigure}

\subsubsection{Regulacja przetwornicy}

W celu sprawdzenia działania przetwornicy:
\begin{itemize}
	\item do wyjścia przetwornicy podłączamy multimetr nastawiony na zakres pomiaru napięcia stałego (DC) do 20V.
	\item włączamy zasilanie przetwornicy – wkładamy wtyk DC do gniazdka DC, podłączamy baterię od złącza lub wkładamy zasilacz wtyczkowy z gniazdka.
	\item multimetr wyświetla napięcie na wyjściu przetwornicy. Jeżeli nasza przetwornica posiada wbudowany woltomierz to wskazania obu przyrządów powinny być podobne (mogą się różnić o ułamkowe części wolta).
\end{itemize}
\vspace*{\baselineskip}


Przetwornica posiada dwa potencjometry – jeden służy do regulacji napięcia, a drugi do regulacji ograniczenia prądowego. Zazwyczaj są podpisane, ale jeżeli nie są, to możemy łatwo ustalić, który za co odpowiada.
W tym celu, przy pomocy odpowiedniego wkrętaka, obróć śrubkę regulacyjną potencjometru od napięcia i zobacz jak wpływa to na napięcie wyjściowe.

Nastaw napięcie wyjściowe na 5V. Drugi potencjometr odpowiedzialny jest za regulację prądu. Ustaw go w okolicy wartości minimalnej:
\begin{itemize}
	\item pokręć nim aż do oporu lub kliknięcia w tę samą stronę, która odpowiadała za zmniejszanie napięcia w potencjometrze od regulacji napięcia.
	\item zrób pół obrotu w przeciwną stronę.
\end{itemize}
Następnie:
\begin{itemize}
	\item odłącz multimetr od wyjścia przetwornicy, nastaw na pomiar prądu do 10A lub 20A i podłącz sondy do odpowiednich gniazd w mierniku.
	\item przytknij na krótko (około 1 – 2 sekund) sondy do zacisków wyjściowych i zaobserwuj wskazanie multimetru.
\end{itemize}
Jeżeli wskazanie było duże (około ampera lub kilku) obrócić potencjometr odpowiedzialny za regulację prądu w przeciwnym kierunku i ponów procedurę pomiaru.
Jeżeli było ono poniżej 0.2A przestaw multimetr na zakres 200mA, przełóż sondy do odpowiednich gniazd i ponownie wykonaj pomiar.

Nastaw ograniczenie prądowe na około 40mA.
