% Copyright (c) 2017-2020 Matematyka dla Ciekawych Świata (http://ciekawi.icm.edu.pl/)
% Copyright (c) 2017-2020 Robert Ryszard Paciorek <rrp@opcode.eu.org>
% 
% MIT License
% 
% Permission is hereby granted, free of charge, to any person obtaining a copy
% of this software and associated documentation files (the "Software"), to deal
% in the Software without restriction, including without limitation the rights
% to use, copy, modify, merge, publish, distribute, sublicense, and/or sell
% copies of the Software, and to permit persons to whom the Software is
% furnished to do so, subject to the following conditions:
% 
% The above copyright notice and this permission notice shall be included in all
% copies or substantial portions of the Software.
% 
% THE SOFTWARE IS PROVIDED "AS IS", WITHOUT WARRANTY OF ANY KIND, EXPRESS OR
% IMPLIED, INCLUDING BUT NOT LIMITED TO THE WARRANTIES OF MERCHANTABILITY,
% FITNESS FOR A PARTICULAR PURPOSE AND NONINFRINGEMENT. IN NO EVENT SHALL THE
% AUTHORS OR COPYRIGHT HOLDERS BE LIABLE FOR ANY CLAIM, DAMAGES OR OTHER
% LIABILITY, WHETHER IN AN ACTION OF CONTRACT, TORT OR OTHERWISE, ARISING FROM,
% OUT OF OR IN CONNECTION WITH THE SOFTWARE OR THE USE OR OTHER DEALINGS IN THE
% SOFTWARE.

\section{Zakupy, czyli co warto kupić na początek}

Praktyczna zabawa z elektroniką wymaga posiadania przynajmniej minimalnego zaplecza sprzętowego.
W jego skład powinno wejść co najmniej:
	uniwersalny miernik wielkości elektrycznych (multimetr),
	źródło zasilania (najlepiej z regulacją napięcia i ograniczenia prądowego),
	płytka prototypowa wraz z różnymi kabelkami i podstawowymi narzędziami (przynajmniej śrubokrętem),
	moduł mikrokontrolera wraz z programatorem
	oraz jakiś zestaw podstawowych elementów od których zaczniemy naszą zabawę.
W tym rozdziale powiemy na co warto zwrócić uwagę przy tych zakupach oraz przedstawiami kilka godnych uwagi propozycji dostępnych na rynku.

