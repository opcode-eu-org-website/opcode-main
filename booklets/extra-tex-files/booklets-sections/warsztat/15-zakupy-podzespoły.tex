% Copyright (c) 2017-2020 Matematyka dla Ciekawych Świata (http://ciekawi.icm.edu.pl/)
% Copyright (c) 2017-2020 Robert Ryszard Paciorek <rrp@opcode.eu.org>
% 
% MIT License
% 
% Permission is hereby granted, free of charge, to any person obtaining a copy
% of this software and associated documentation files (the "Software"), to deal
% in the Software without restriction, including without limitation the rights
% to use, copy, modify, merge, publish, distribute, sublicense, and/or sell
% copies of the Software, and to permit persons to whom the Software is
% furnished to do so, subject to the following conditions:
% 
% The above copyright notice and this permission notice shall be included in all
% copies or substantial portions of the Software.
% 
% THE SOFTWARE IS PROVIDED "AS IS", WITHOUT WARRANTY OF ANY KIND, EXPRESS OR
% IMPLIED, INCLUDING BUT NOT LIMITED TO THE WARRANTIES OF MERCHANTABILITY,
% FITNESS FOR A PARTICULAR PURPOSE AND NONINFRINGEMENT. IN NO EVENT SHALL THE
% AUTHORS OR COPYRIGHT HOLDERS BE LIABLE FOR ANY CLAIM, DAMAGES OR OTHER
% LIABILITY, WHETHER IN AN ACTION OF CONTRACT, TORT OR OTHERWISE, ARISING FROM,
% OUT OF OR IN CONNECTION WITH THE SOFTWARE OR THE USE OR OTHER DEALINGS IN THE
% SOFTWARE.

\subsection{Podzespoły elektroniczne}

Będzie potrzebny też zestaw drobnych podzespołów elektronicznych:
\begin{itemize}
	\item rezystory 1kΩ i 22kΩ, po około 10 sztuk
	\item potencjometr / rezystor nastwany 5kΩ, który da się włożyć w płytkę stykową, najlepiej wieloobrotowy, 1-2 sztuki
	\item kondensator elektrolityczny 100uF, kilka sztuk
	
	\item dioda prostownicza, około 10 sztuk
	\item dioda świecąca, około 10 sztuk
	\item dioda Zenera 3.3V, około 5 sztuk
	\item tranzystor NPN (np. BC337) i PNP (np. BC327), po kilka sztuk
	
	\item układ logiczny z serii 4000 lub 7400: NAND (np. CD4011BE) lub NOR (np. CD4001BP)
	\item rejestr przesuwny z serii 4000 lub 7400 (np. CD4094 lub 74HC595)
	\item rejestr pz interfejsem I2C (np. PCF8574A lub MCP23008)
\end{itemize}
