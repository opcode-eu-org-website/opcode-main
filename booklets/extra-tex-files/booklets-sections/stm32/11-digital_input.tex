% Copyright (c) 2020 Matematyka dla Ciekawych Świata (http://ciekawi.icm.edu.pl/)
% Copyright (c) 2020 Robert Ryszard Paciorek <rrp@opcode.eu.org>
% Copyright (c) 2020 Krzysztof Lasocki <krz.lasocki@gmail.com>
% 
% MIT License
% 
% Permission is hereby granted, free of charge, to any person obtaining a copy
% of this software and associated documentation files (the "Software"), to deal
% in the Software without restriction, including without limitation the rights
% to use, copy, modify, merge, publish, distribute, sublicense, and/or sell
% copies of the Software, and to permit persons to whom the Software is
% furnished to do so, subject to the following conditions:
% 
% The above copyright notice and this permission notice shall be included in all
% copies or substantial portions of the Software.
% 
% THE SOFTWARE IS PROVIDED "AS IS", WITHOUT WARRANTY OF ANY KIND, EXPRESS OR
% IMPLIED, INCLUDING BUT NOT LIMITED TO THE WARRANTIES OF MERCHANTABILITY,
% FITNESS FOR A PARTICULAR PURPOSE AND NONINFRINGEMENT. IN NO EVENT SHALL THE
% AUTHORS OR COPYRIGHT HOLDERS BE LIABLE FOR ANY CLAIM, DAMAGES OR OTHER
% LIABILITY, WHETHER IN AN ACTION OF CONTRACT, TORT OR OTHERWISE, ARISING FROM,
% OUT OF OR IN CONNECTION WITH THE SOFTWARE OR THE USE OR OTHER DEALINGS IN THE
% SOFTWARE.


\section{Obsługa wejść}
W tym ćwiczeniu pokażę, jak odczytywać stany logiczne pinów GPIO procesora oraz jak na ich podstawie podejmować
decyzje. Skompiluj ten kod podobnie jak w poprzednim przykładzie (weź pod uwagę, że znajduje się on w innym katalogu: \Verb$11_di$).
Plik \Verb$main.c$ wygląda następująco:

\inputminted[frame=single,firstline=11]{c}{stm32-examples/11_di/main.c}

Na płytce stykowej zbuduj układ wg. poniższego schematu:

\begin{center}\includegraphics[width=0.8\textwidth]{img/stm32/10_di}\end{center}

Po naciśnięciu przycisku na płytce stykowej, powinna zapalić się zielona dioda led. Omówmy teraz powyższy program krok po kroku.

Podobnie jak w poprzednim ćwiczeniu, musimy włączyć używane peryferia włączając ich sygnał zegara. Tym razem używamy
dwóch portów GPIO - portu C (do którego jest podłączona dioda) oraz portu A (do którego podłączony jest przycisk).
Z tego powodu trzeba uruchomić je oba (kolejność nie ma znaczenia):

\begin{CodeFrame*}[c]{}
  rcc_periph_clock_enable(RCC_GPIOA);
  rcc_periph_clock_enable(RCC_GPIOC);
\end{CodeFrame*}

Podobnie jak pin C13, pin A0 też należy skofigurować\footnotemark. Ustawiamy go jako wejście, bez podciągania
(wymuszania jakiegoś stanu, gdy nic nie jest podłączone - mówimy wtedy, że wejście jest pływające):

\begin{CodeFrame*}[c]{}
  gpio_set_mode(GPIOC, GPIO_MODE_OUTPUT_2_MHZ, GPIO_CNF_OUTPUT_PUSHPULL, GPIO13);
  gpio_set_mode(GPIOA, GPIO_MODE_INPUT, GPIO_CNF_INPUT_FLOAT, GPIO0);
\end{CodeFrame*}

\footnotetext{Wejście pływające (\textit{Input, Input floating}) jest domyślną kofiguracją wejścia. To wywołanie
  funkcji nic nie zmienia, ale umieściliśmy je tutaj w ramach przykładu.
  Jeżeli chcielibyśmy ustawić wewnętrzne podciąganie to zamiast \Verb$GPIO_CNF_INPUT_FLOAT$ należałoby podać \Verb$GPIO_CNF_INPUT_PULL_UPDOWN$
  i ustawić pin w stanie wysokim lub niskim (w zależności czy chcemy podciągać do stanu wysokiego czy do masy) przy pomocy \Verb$gpio_set$ lub \Verb$gpio_clear$.
}

Same w sobie, wejścia pływające nie mają niczego, co mogło by wymusić na nich stabilny stan logiczny. Zjawiska indukcji
elektromagnetycznej (spowodowane m.in. stałym, 50Hz szumem od sieci elektrycznej) indukują na nich losowe napięcia
(w pewnym małym zakresie, lecz wystarczająco dużym, aby powodować nieustalone stany logiczne). Z tego powodu
nieużywane wejścia z reguły podłącza się do znanego potencjału (GND lub napięcia zasilania układu) aby zapobiegać
stanom nieustalonym. W mikrokontrolerach można też włączyć wbudowane rezystory podciągające.
%
Pływającego wejścia można użyć jako źródła liczb losowych\footnote{Jest to jedna z prostszych wersji takiego
  generatora, wcale nie pozbawiona innych wad}

Podobnie jak w poprzednim przykładzie, po zakończeniu inicjalizacji mikrokontrolera (ustawienia GPIO), program
wchodzi w nieskończoną pętlę. Za każdą iteracją wczytujemy do niej stan portu A do 16 bitowej zmiennej
(w przypadku STM32 port ma 16 bitów szerokości, więc użycie 16-bitowej liczby jest najsensowniejsze):

\begin{CodeFrame*}[c]{}
  int16_t stan_a;

  while(1){
    stan_a = gpio_port_read(GPIOA);
\end{CodeFrame*}

W instrukcji warunkowej program sprawdza, czy najmłodszy bit zmiennej jest równy 1. \texttt{0x01} to tzw. maska bitowa.
Zauważ, że wynik operacji \texttt{\Verb$stan_a & 0x01$} jest 1 jeden gdy najmłodszy bit jest równy 1. Tego typu
konstrukcje są powszechnie spotykane w programach mikrokontrolerowych.\footnote{
	Jeśli chcielibyśmy sprawdzić inny bit (np. bit na 5 miejscu), użylibyśmy maski, która składa się z samych zer oprócz piątego bitu.
	Można ją wyrazić jako \Verb$0x01 << 5$ (jedynka przesunięta o 5 miejsc w lewo). Kompilator zamieni to wyrażenie na stałą.
}
Jeśli warunek wynosi jeden, wykonuje się funkcja \Verb$gpio_set$, która gasi diodę. W przeciwnym wypadku, funkcja
\Verb$gpio_clear$ przełącza pin C13 w stan niski, powodując zapalenie diody.

\begin{CodeFrame*}[c]{}
    if(stan_a & 0x01){
      gpio_set(GPIOC, GPIO13);
    }else{
      gpio_clear(GPIOC, GPIO13);
    }
\end{CodeFrame*}

Następnie wymuszamy opóźnienie (zbyt szybko działający program mógłby mieć problem z drganiem styków):

\begin{CodeFrame*}[c]{}
    for (int i = 0; i < 150000; i++) __asm__("nop");
\end{CodeFrame*}

W ten sposób doszliśmy do końca programu. Pętla jest nieskończona, więc mikroprocesor będzie cyklicznie wykonywał
jej zawartość, sterując diodą LED odpowiednio do stanu na A0.
