% Copyright (c) 2020 Matematyka dla Ciekawych Świata (http://ciekawi.icm.edu.pl/)
% Copyright (c) 2020 Robert Ryszard Paciorek <rrp@opcode.eu.org>
% Copyright (c) 2020 Krzysztof Lasocki <krz.lasocki@gmail.com>
% 
% MIT License
% 
% Permission is hereby granted, free of charge, to any person obtaining a copy
% of this software and associated documentation files (the "Software"), to deal
% in the Software without restriction, including without limitation the rights
% to use, copy, modify, merge, publish, distribute, sublicense, and/or sell
% copies of the Software, and to permit persons to whom the Software is
% furnished to do so, subject to the following conditions:
% 
% The above copyright notice and this permission notice shall be included in all
% copies or substantial portions of the Software.
% 
% THE SOFTWARE IS PROVIDED "AS IS", WITHOUT WARRANTY OF ANY KIND, EXPRESS OR
% IMPLIED, INCLUDING BUT NOT LIMITED TO THE WARRANTIES OF MERCHANTABILITY,
% FITNESS FOR A PARTICULAR PURPOSE AND NONINFRINGEMENT. IN NO EVENT SHALL THE
% AUTHORS OR COPYRIGHT HOLDERS BE LIABLE FOR ANY CLAIM, DAMAGES OR OTHER
% LIABILITY, WHETHER IN AN ACTION OF CONTRACT, TORT OR OTHERWISE, ARISING FROM,
% OUT OF OR IN CONNECTION WITH THE SOFTWARE OR THE USE OR OTHER DEALINGS IN THE
% SOFTWARE.

\section{Komunikacja $I^2C$}
$I^2C$ jest popularnym protokołem szeregowym wykorzystywanym do komunikacji między cyfrowymi układami scalonymi. Jego odmiana, SMBus, jest m. in.
stosowana do komunikacji z inteligentnymi bateriami. W tym ćwiczeniu zademonstrujemy implementację mastera oraz slave'a $I^2C$ na STM32.\\

W modelu STM32 używanym przez nas są dwa peryferia $I^2C$. Połączymy je ze sobą i oprogramujemy jedno jako master, a drugie jako slave.
Dzięki temu nasz mikrokontroler będzie mógł „rozmawiać” sam ze sobą\footnote{
	Oczywiście w realnych zastosowaniach nie stosuje się takiego połączenia, jednak jest to dobry przykład do pokazania obsługi $I^2C$
	zarówno od strony mastera jak i slave'a, bez konieczności używania jakichkolwiek innych układów.
} poprzez magistralę $I^2C$.
Nasz układ wygląda tak:

\begin{center}\includegraphics[width=0.8\textwidth]{img/stm32/40_i2c}\end{center}

Rezystory podciągające powinny mieć wartość kilku k$\Omega$.

Kod do tego ćwiczenia znajduje się w katalogu \Verb$40_i2c$.
Pliki \Verb$uart.*$ nie uległy zmianie od kiedy zaczęliśmy ich używać i dostarczają obsługę wypisywania informacji na port szeregowy.
Całość obsługi $I^2C$ znajduje się w pliku \Verb$main.c$ - przyjrzyjmy mu się bliżej.

Zaczynamy od zdefiniowania adresu naszego slave'a - zdefiniujemy makro:

\begin{CodeFrame*}[c]{}
#define SLAVE_ADDR 0x0F
\end{CodeFrame*}

Następnie należy skonfigurować oba peryferia. Funkcja inicjalizująca może zostać zaimplementowana w taki sposób:

\begin{CodeFrame*}[c]{}
void i2c_setup(){
  rcc_periph_clock_enable(RCC_GPIOB);
  rcc_periph_clock_enable(RCC_I2C1);
  rcc_periph_clock_enable(RCC_I2C2);


  /* I2C1 - master; SDA=B7, SCL=B6 */
  i2c_reset(I2C1);
  i2c_peripheral_disable(I2C1);
  i2c_set_speed(I2C1, i2c_speed_sm_100k, 8);

  gpio_set_mode(GPIOB, GPIO_MODE_OUTPUT_2_MHZ,
		  GPIO_CNF_OUTPUT_ALTFN_OPENDRAIN, GPIO6 | GPIO7);
  
  i2c_peripheral_enable(I2C1);


  /* I2C2 - slave; SDA=B11, SCL=B10 */
  i2c_reset(I2C2);
  i2c_peripheral_disable(I2C2);
  i2c_set_speed(I2C2, i2c_speed_sm_100k, 8);

  i2c_set_own_7bit_slave_address(I2C2, SLAVE_ADDR);

  nvic_enable_irq(NVIC_I2C2_EV_IRQ);
  
  gpio_set_mode(GPIOB, GPIO_MODE_OUTPUT_2_MHZ,
		  GPIO_CNF_OUTPUT_ALTFN_OPENDRAIN, GPIO10 | GPIO11);

  i2c_enable_interrupt(I2C2, I2C_CR2_ITEVTEN );
  i2c_peripheral_enable(I2C2);

  
  i2c_enable_ack(I2C2);
}
\end{CodeFrame*}

Powyższa funkcja konfiguruje oba interfejsy. Konfigurację zaczynamy włączenia taktowania, resetu oraz wyłączenia peryferiów (konfiguracje wstępne
powinny odbywać się przy wyłączonych peryferiach). Pierwszy interfejs (I2C1) będzie masterem,
zaś drugi (I2C2) - slave'm.\\

W przypadku mastera ustawiamy tylko prędkość magistrali za pomocą funkcji
\Verb$i2c_set_speed(I2C1, i2c_speed_sm_100k, 8)$, która jako argumenty pobiera peryferium (I2C1), prędkość (jako wartość typu wyliczeniowego enum),
oraz prędkość zegara taktującego peryferium w MHz (domyślnie jest to \textit{HSI Clock}, wewnętrzny oscylator - w naszym przypadku 8MHz).\\

Dla I2C2 konfiguracja wymaga także ustawienia adresu (funkcja \Verb$i2c_set_own_7bit_slave_address(I2C2, SLAVE_ADDR)$), włączenia przerwania
\Verb$nvic_enable_irq(NVIC_I2C2_EV_IRQ)$ i skonfigurowania go (\Verb$i2c_enable_interrupt(I2C2, I2C_CR2_ITEVTEN )$, generowanie przerwania dla wszystkich wydarzeń na szynie danych) oraz włączenia odpowiedzi na własny adres (odsyłanie \textit{ACK}, funkcja \Verb$i2c_enable_ack(I2C2)$,
wywoływana po włączeniu peryferium).

Oprócz wymienionych kroków należy też skonfigurować piny GPIO na których (domyślnie) znajdują się sygnały $I^2C$, w trybie
\Verb$GPIO_CNF_OUTPUT_ALTFN_OPENDRAIN$ (\textit{open-drain}, alternatywna funkcja).

Ponieważ nasz procesor jest jednowątkowy, obsługa slave'a będzie odbywać się całkowicie w funkcji obsługi przerwania. W związku z tym musimy ją
zdefiniować. Nasza biblioteka oraz skrypty linkera oczekują, że będzie nazywać się \Verb$i2c2_ev_isr$.

\begin{CodeFrame*}[c]{}
void i2c2_ev_isr(void){
  uint16_t sr1, sr2;

  sr1 = I2C_SR1(I2C2);
  printf("I2C ISR: sr1=0x%04x slavebyte=%d\n", sr1, slavebyte); // for demonstration / debug only

  // Address matched (Slave)
  if (sr1 & I2C_SR1_ADDR){
    // Clear the ADDR sequence by reading SR2.
    sr2 = I2C_SR2(I2C2);
    (void) sr2;
  }

  // Write request from master
  if (sr1 & I2C_SR1_RxNE){
    slavebyte = I2C_DR(I2C2);
    printf("I2C ISR: received %d\n", slavebyte); // for demonstration / debug only
    slavebyte *= 2; // slave logic is multiply by 2 ;)
  }

  // Read request from master
  if ((sr1 & I2C_SR1_TxE)){
    I2C_DR(I2C2) = slavebyte;
  }

  // Stop sequence detected (Slave)
  if (sr1 & I2C_SR1_STOPF){
    // Clear by write I2C_CR1
    I2C_CR1(I2C2) = I2C_CR1(I2C2);
  }
}
\end{CodeFrame*}

Ta procedura obsługi przerwania wykonuje się dla wszystkich wydarzeń na szynie danych. W związku z tym należy najpierw sprawdzić, jakie wydarzenie
spowodowało wystąpienie przerwania. Służy do tego rejestr \Verb$I2C_SR1$ (\textit{Status Register 1}). Jego adres zwraca makro \Verb$I2C_SR1(x)$,
które oblicza go na postawie adresu bazowego peryferium. W zależności od tego, które wydarzenie na szynie danych wygenerowało przerwanie, ustawiane
są odpowiednie bity w tym rejestrze.\\

Jeżeli przerwanie zostało wygenerowane poprzez odebranie własnego adresu, spełniony będzie warunek \Verb$sr1 & I2C_SR1_ADDR$ (w rejestrze ustawiony
będzie bit \Verb$I2C_SR1_ADDR$). W takim wypadku peryferium oczekuje od programu odczytania rejestru SR2.

\begin{ProTip}{\normalfont{\strong{Uwaga}}}
Odczytanie rejestru SR2 powoduje wysłanie ACK i przejście interfejsu do stanu gotowości do odebrania danych - inaczej peryferium będzie
trzymać linię SCL nisko, blokując szynę. Tego typu model programowy, w którym odczytanie lub zapisanie wartości rejestru powoduje
jakąś akcję peryferium jest często spotykany. Narzuca on pewnie ograniczenia ale ma też swoje zalety, m. in. pozwala na szybszą obsługę takich
interfejsów.

W przypadku interfejsu $I^2C$ trzeba odczytać i/lub zapisać pewne rejestry po większości operacji na szynie.
\end{ProTip}

Następnie procedura sprawdza, czy rejestr danych (DR) nie jest pusty (ustawiony bit \Verb$I2C_SR1_RxNE$ w SR1, \textit{Rx Not Empty}).
Jeśli tak jest, to znaczy że odebrano bajt z szyny danych (od mastera). Należy wtedy wczytać ten bajt z DR

Tutaj także wykonujemy operację którą realizuje slave. W tym przypadku jest nią pomnożenie odebranego bajtu razy 2 i zachowanie go w pamięci.

Jeśli master wyśle żądanie odczytu ze slave'a, po otrzymaniu adresu, w SR1 ustawiony będzie bit \Verb$I2C_SR1_TxE$ (\textit{Tx Empty}) informujący
o tym, że takie żądanie nastąpiło a rejestr danych do wysłania jest pusty. Wtedy zapisanie bajtu danych do DR spowoduje wysłanie tych danych na szynę
do mastera. W naszym przypadku daną do wysłania jest poprzednio zachowana i pomnożona wartość.

Jeżeli master wykona na szynie sekwencję stopu, co jest to sygnalizowane ustawieniem bitu \Verb$I2C_SR1_STOPF$ w SR1,
to należy potwierdzić jej odebranie poprzez dokonanie zapisu do rejestru CR1.
Wykonuje to linijka \Verb$I2C_CR1(I2C2) = I2C_CR1(I2C2)$.
Mimo braku jej sensu z punktu widzenia programu, należy pamiętać, że makra rejestrów definiują je jako \texttt{volatile},
więc nie zostanie ona usunięta w procesie optymalizacji kodu, a kompilator wygeneruje kod zapisujący wartość do CR1.\footnote{
	Jeżeli tego nie zrobimy procesor będzie ciągle generować przerwanie obsługi I2C z ustawionym bitem \Verb$I2C_SR1_STOPF$ w SR1 i nasz program zawiesi się.
}

\begin{ProTip}{\normalfont{\strong{Uwaga}}}
Procedura dodatkowo wypisuje na porcie szeregowym informację o tym że została wywołana i jakie dane odebrała.
Ogólnie nie należy stosować tak powolnych operacji jak nasza funkcja printf w procedurach obsługi przerwań, gdyż może to prowadzić do ich niewłaściwego działania.
Wypisywanie zostało tutaj wprowadzone wyłącznie w celach demonstracyjnych i może (a nawet powinno) być usunięte przed jakimkolwiek poważniejszym użyciem tego kodu.
\end{ProTip}

Oprogramowanie interfejsu mastera jest mniej skomplikowane, ponieważ odbywa się w zwykłej funkcji. Poniżej funkcja zapisu bajtu do slave'a (jako
argument pobiera interfejs oraz wartość bajtu):

\begin{CodeFrame*}[c]{}
void i2c_send_write(uint32_t peryf, uint8_t dane){
  i2c_send_start(peryf);
  // Czekaj na wysłanie startu
  while (!((I2C_SR1(peryf) & I2C_SR1_SB)
	   & (I2C_SR2(peryf) & (I2C_SR2_MSL | I2C_SR2_BUSY))));
  
  i2c_send_7bit_address(peryf, SLAVE_ADDR, I2C_WRITE);
  //Czekaj na wysłanie adresu
  while (!(I2C_SR1(peryf) & I2C_SR1_ADDR));
  (void) I2C_SR2(peryf); //Wyczyść EV6
  
  i2c_send_data(peryf, dane);

  while (!(I2C_SR1(peryf) & (I2C_SR1_BTF))); //Czekaj na wyslanie danych

  i2c_send_stop(peryf);
}
\end{CodeFrame*}

Transmisja danych rozpoczyna START, którego wysłanie powoduje funkcja \Verb$i2c_send_start(peryf)$. Następnie oczekujemy na zakończenie wysyłania
START wykonując pustą pętle dopóki bit \Verb$I2C_SR1_SB$ (\textit{Start Bit}) w SR1 oraz bity \Verb$I2C_SR1_MSL$ (\textit{Master/slave}) i
\Verb$I2C_SR1_BUSY$ (oznaczający, że na szynie odbywa się komunikacja) w SR2 nie zostaną ustawione na 1. Ponieważ funkcja \Verb$i2c_send_start(peryf)$
(oraz inne funkcje związane z kontrolą I2C) ustawia tylko bity rejestrach, zakończy się ona wcześniej niż na szynie pojawi się jej oczekiwany
efekt. Z tego powodu zawsze należy poczekać, aż żądane działanie rzeczywiście się zakończy.\\

Po wysłaniu warunku START, wysyłamy adres z flagą R/W ustawioną na zapis, za pomocą funkcji
\Verb$i2c_send_7bit_address(peryf, SLAVE_ADDR, I2C_WRITE)$. Po tej operacji należy poczekać aż bit \Verb$I2C_SR1_ADDR$ w SR1 zostanie ustawiony, a
następnie wykonać odczyt SR2\footnotemark (linijka \Verb$(void) I2C_SR2(peryf)$ odczytuje tę wartość a następnie ją porzuca. Jest to obowiązkowe).
Potem wysyłamy bajt za pomocą funkcji \Verb$i2c_send_data(peryf, dane)$ i oczekujemy na ustawienie bitu \Verb$I2C_SR1_BTF$ (\textit{Byte Transfer
  Finished} w SR1. Operację zapisu kończymy wysyłając STOP za pomocą \Verb$i2c_send_stop(peryf)$ (tutaj wyjątkowo nie trzeba czekać na ustawienie żadnych flag).\\

\footnotetext{Dokładna kolejność działań na rejestrach dla peryferium $I^2C$ jest opisana w rozdziałach 26.3.2 (oprogramowanie slave'a) i 26.3.3
  (oprogrwamowanie mastera) w \hyperref[refman]{\textit{Reference manual}}}


Podobnie możemy stworzyć funkcję odczytu bajtu ze slave'a:

\begin{CodeFrame*}[c]{}
uint8_t i2c_send_read(uint32_t peryf){
  uint8_t dane;
  
  i2c_send_start(peryf);
  // Czekaj na wysłanie startu
  while (!((I2C_SR1(peryf) & I2C_SR1_SB)
	   & (I2C_SR2(peryf) & (I2C_SR2_MSL | I2C_SR2_BUSY))));
  
  i2c_send_7bit_address(peryf, SLAVE_ADDR, I2C_READ);

  //Czekaj na wysłanie adresu
  while (!(I2C_SR1(peryf) & I2C_SR1_ADDR));
  (void) I2C_SR2(peryf); //Wyczyść EV6

  //Czekaj aż otrzymasz 1 bajt danych
  while (!(I2C_SR1(peryf) & I2C_SR1_RxNE));
  dane = i2c_get_data(peryf);
  
  i2c_send_stop(peryf);

  return dane;
}
\end{CodeFrame*}

Z taka różnicą, że adres wysyłamy z flagą odczytu (\Verb$I2C_READ$), a po wysłaniu adresu czekamy na zapełnienie się rejestru danych bajtem od
slave'a (ustawienie bitu \Verb$I2C_SR1_RxNE$ w SR1). Funkcja \Verb$i2c_get_data(peryf)$ odczytuje rejestr DR. Operację kończymy wysyłając STOP. \\

Mając zdefiniowane funkcje - konfigurującą interfejsy, wysyłającą i odbierającą dane, oraz realizującą logikę slave'a (funkcja obsługi przerwania
I2C2) - możemy ich użyć do napisania programu testowego. Skorzystamy też z poprzednich funkcji do komunikacji po UART aby wyświetlić wyniki naszego
programu:

\begin{CodeFrame*}[c]{}
  uint8_t k = 0;
  i2c_setup();

  while(1){
    for (int i = 0; i < 500000; i++) __asm__("nop");
    
    printf("Wysylam %d\n", k);
    i2c_send_write(I2C1, k);    
    uint8_t wynik = i2c_send_read(I2C1);
    printf("Odebralem %d\n", wynik);
    k++;
  }
\end{CodeFrame*}

Ten kod wysyła kolejne liczby całkowite do slave'a i wypisuje jego odpowiedź na port szeregowy mikronotrolera.\\

Należy pamiętać, że wyjścia I2C są typu otwarty kolektor i potrzebują rezystorów podciągających!

\insertZadanie{booklets-sections/stm32/zadania.tex}{zmiana_i2c}{}
\insertZadanie{booklets-sections/stm32/zadania.tex}{i2c_rejestr}{}
