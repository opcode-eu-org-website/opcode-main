\subsection{Instalacja i przygotowanie narzędzi}

Do programowania mikrokontrolera STM32 użyjemy następujących narzędzi programistycznych:

\begin{itemize}
\item Toolchain\footnotemark  dla architektury \emph{arm-none-eabi} (\Verb$gcc-arm-none-eabi$ oraz \Verb$binutils-arm-none-eabi$)
  \footnotetext{Tooolchain to zbiorowe określenie na narzędzia do przekształcania kodu źródłowego na kod maszynowy. Zawiera m.in. kompilator, linker (konsolidator), narzędzia obsługujące pliki obiektowe oraz kopujące dane między formatami plików wykonywalnych (\texttt{objcopy}) itd.}
\item Implementację libC (\Verb$libstdc++-arm-none-eabi-newlib$)
\item Narzędzie do programowania przez UART (\Verb$stm32flash$)
\item Program terminalowy do obsługi portów~szeregowych (\Verb$picocom$)
\item Bibliotekę \Verb$libopencm3$
\end{itemize}
Wszystkie narzędzia, oprócz ostatniego, dostępne są w repozytoriach Debiana (i systemów na nim opartych). Możesz je zainstalować
za pomocą polecenia:

\begin{CodeFrame*}[bash]{}
sudo apt install gcc-arm-none-eabi binutils-arm-none-eabi libstdc++-arm-none-eabi-newlib stm32flash picocom git make
\end{CodeFrame*}

Instalacja i kompilacja \Verb$libopencm3$ odbywa się w następujący sposób:

\begin{CodeFrame*}[bash]{}
git clone https://github.com/libopencm3/libopencm3.git
cd libopencm3
make
\end{CodeFrame*}

Pamiętaj o ustawieniu ścież

\subsection{Połączenie mikrokontrolera}
Aby móc uruchomić nasz kod na mikrokontrolerze, musimy podłączyć go do komputera w celu wgrywania na niego skompilowanych programów.
STM32 można programować na wiele sposobów, my będziemy używać do tego interfejsu UART\footnote{Inne z nich to SWD oraz JTAG}.\\

Aby zaprogramować mikrokontroler musisz podłączyć go do przejściówki USB-UART (nie podłączaj jeszcze
przejściówki do komputera).
Za pomocą pasujących kabelków podłącz następujące piny przejściówki do pinów na mikrokontrolerze:
\begin{itemize}
\item masę (GND) do (dowolnej) masy miktrokontrolera (GND lub G)
\item RX do TX mikrokontrolera (pin A9)
\item TX do RX mikrokontrolera (pin A10)
\item 5V do 5V mikrokontrolera
\end{itemize}

Za pomocą obcążków lub pęsety przełóż górną (patrząc na mikrokontroler tak aby port USB był po lewej) zworkę na pozycję ``1''.

Sprawdź wszystkie połączenia i podłącz przejściówkę do portu USB komputera. Na mikrokontrolerze powinna zaświecić się tylko czerwona dioda \texttt{PWR}

Możesz sprawdzić połączenie z mikrokontrolerem używając \Verb$stm32flash$ do wyświetlenia informacji na temat podłączonego mikrokontrolera.
Jeżeli polecenie poniżej nie zgłosi błędów, oznacza to, że wszystko działa poprawnie.

\begin{CodeFrame*}[bash]{}
stm32flash /dev/ttyUSB0
\end{CodeFrame*}

Tak przygotowany mikrokontroler jest gotowy do pracy.
