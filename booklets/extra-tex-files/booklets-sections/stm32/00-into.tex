% Copyright (c) 2020 Matematyka dla Ciekawych Świata (http://ciekawi.icm.edu.pl/)
% Copyright (c) 2020 Robert Ryszard Paciorek <rrp@opcode.eu.org>
% Copyright (c) 2020 Krzysztof Lasocki <krz.lasocki@gmail.com>
% 
% MIT License
% 
% Permission is hereby granted, free of charge, to any person obtaining a copy
% of this software and associated documentation files (the "Software"), to deal
% in the Software without restriction, including without limitation the rights
% to use, copy, modify, merge, publish, distribute, sublicense, and/or sell
% copies of the Software, and to permit persons to whom the Software is
% furnished to do so, subject to the following conditions:
% 
% The above copyright notice and this permission notice shall be included in all
% copies or substantial portions of the Software.
% 
% THE SOFTWARE IS PROVIDED "AS IS", WITHOUT WARRANTY OF ANY KIND, EXPRESS OR
% IMPLIED, INCLUDING BUT NOT LIMITED TO THE WARRANTIES OF MERCHANTABILITY,
% FITNESS FOR A PARTICULAR PURPOSE AND NONINFRINGEMENT. IN NO EVENT SHALL THE
% AUTHORS OR COPYRIGHT HOLDERS BE LIABLE FOR ANY CLAIM, DAMAGES OR OTHER
% LIABILITY, WHETHER IN AN ACTION OF CONTRACT, TORT OR OTHERWISE, ARISING FROM,
% OUT OF OR IN CONNECTION WITH THE SOFTWARE OR THE USE OR OTHER DEALINGS IN THE
% SOFTWARE.

Skrypt opisuje podstawy programowania mikrokontrolerów STM32. Podczas kursu będziemy używać
popularnej, dostępnej i prostej płytki ``Blue Pill''. Programy będą pisane w języku
C z pomocą biblioteki \Verb$libopencm3$ (\url{https://github.com/libopencm3/libopencm3}).
Każde z ćwiczeń to oddzielny program, więc jego kod znajduje się w oddzielnym katalogu.

Kod przykładów znajduje się w repozytorium git. Można je sklonować do podkatalogu za pomocą:

\begin{CodeFrame*}[bash]{}
  git clone https://bitbucket.org/OpCode-eu-org/stm32-examples.git
\end{CodeFrame*}

Po sklonowaniu repozytorium w jego katalogu głównym należy utworzyć link o nazwie \Verb$libopencm3$ wskazujący na katalog w którym znajduje się (skompilowana wcześniej) biblioteka:

\begin{CodeFrame*}[bash]{}
  cd stm32-examples
  ln -s SCIEŻKA/DO/libopencm3 libopencm3
\end{CodeFrame*}

Jeżeli nie masz jeszcze pobranej i skompilowanej biblioteki \Verb$libopencm3$, to zamiast tworzyć link możesz umieścić tutaj katalog z tą biblioteką.
Ważne jest zachowanie nazwy i umieszczenie go w głównym katalogu repozytorium z przykładami, gdyż pliki wspomagające kompilację przykładów zakładają że w tym miejscu znajdą katalog z tą biblioteką.

\begin{ProTip}{\normalfont{\strong{Ważne}}}
Każdy z przykładowych programów znajduje się w swoim własnym katalogu. Katalog taki zawiera wszystkie pliki źródłowe programu, plik \Verb#Makefile# i będzie też zawierać pliki tymczasowe i wynikowe powstałe w wyniku kompilacji.
Do kompilacji i wgrywanie programów do mikrokontrolera będziemy korzystać z narzędzia \Verb$make$, które z odpowiednimi argumentami należy wywołać w katalogu danego przykładu. Typowo będzimey używać 3 wywołań programu \Verb$make$:
\begin{itemize}
\item \Verb$make$ – skompiluje i zlinkuje z potrzebnymi fragmentami biblioteki nasz program
\item \Verb$make install$ – skompiluje i zlinkuje z potrzebnymi fragmentami biblioteki nasz program oraz wgra go do mikrokontrolera
\item \Verb$make run$ – uruchomi program znajdujący się w mikrokontrolerze po jego restarcie (jest to szybsze i bardziej przyjazne dla pamięci mikrokontrolera niż używanie \Verb$make install$ z tym samym kodem)
\end{itemize}

\vspace{5mm}

Make jest narzędziem automatyzującym proces kompilacji oprogramowania, wykorzystującym pliki \Verb#Makefile#. Więcej na jego temat można znaleźć w skrypcie \href{http://www.opcode.eu.org/Narzędzia\_deweloperskie.pdf}{narzędzia deweloperskie}. Pomimo iż nasze programy są małe to używamy go aby ułatwić sobie kompilację kodu dla mikrokontrolera - poprzez zastosowanie odpowiedniego kompilatora (który wygeneruje kod dla procesora ARM, a nie dla x86 pod kontrolą którego działa), przekazanie do niego opcji ustawiających jaki mikrokontroler używamy oraz dolinkowanie odpowiednich bibliotek.
\end{ProTip}
