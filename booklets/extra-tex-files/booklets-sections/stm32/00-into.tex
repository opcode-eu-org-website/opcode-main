% Copyright (c) 2020 Matematyka dla Ciekawych Świata (http://ciekawi.icm.edu.pl/)
% Copyright (c) 2020 Robert Ryszard Paciorek <rrp@opcode.eu.org>
% Copyright (c) 2020 Krzysztof Lasocki <krz.lasocki@gmail.com>
% 
% MIT License
% 
% Permission is hereby granted, free of charge, to any person obtaining a copy
% of this software and associated documentation files (the "Software"), to deal
% in the Software without restriction, including without limitation the rights
% to use, copy, modify, merge, publish, distribute, sublicense, and/or sell
% copies of the Software, and to permit persons to whom the Software is
% furnished to do so, subject to the following conditions:
% 
% The above copyright notice and this permission notice shall be included in all
% copies or substantial portions of the Software.
% 
% THE SOFTWARE IS PROVIDED "AS IS", WITHOUT WARRANTY OF ANY KIND, EXPRESS OR
% IMPLIED, INCLUDING BUT NOT LIMITED TO THE WARRANTIES OF MERCHANTABILITY,
% FITNESS FOR A PARTICULAR PURPOSE AND NONINFRINGEMENT. IN NO EVENT SHALL THE
% AUTHORS OR COPYRIGHT HOLDERS BE LIABLE FOR ANY CLAIM, DAMAGES OR OTHER
% LIABILITY, WHETHER IN AN ACTION OF CONTRACT, TORT OR OTHERWISE, ARISING FROM,
% OUT OF OR IN CONNECTION WITH THE SOFTWARE OR THE USE OR OTHER DEALINGS IN THE
% SOFTWARE.

Skrypt opisuje podstawy programowania mikrokontrolerów STM32. Podczas kursu będziemy używać
popularnej, dostępnej i prostej płytki ``Blue Pill''. Programy będą pisane w języku
C z pomocą biblioteki \Verb$libopencm3$ (\url{https://github.com/libopencm3/libopencm3}).
Każde z ćwiczeń to oddzielny program, więc jego kod znajduje się w oddzielnym katalogu.

Kod przykładów znajduje się w \hyperref[repo]{repozytorium}. Można je sklonować do podkatalogu za pomocą:

\begin{CodeFrame*}[bash]{}
  git clone https://bitbucket.org/OpCode-eu-org/stm32-examples.git
\end{CodeFrame*}

Repozytorium zawiera link symboliczny \Verb$libopencm3$ wskazujący na \Verb$../libopencm3-master$.
Pliki wspomagające kompilację przykładów zakładają że w tym miejscu znajdą katalog z (skompilowaną wcześniej) biblioteką \Verb$libopencm3$.
Jeżeli bibliotekę tę masz w innej lokalizacji zastąp ten link wskazującym na poprawną lokalizację katalogu z skompilowaną biblioteką.
