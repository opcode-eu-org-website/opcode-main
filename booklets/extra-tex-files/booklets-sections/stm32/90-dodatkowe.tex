% Copyright (c) 2020 Matematyka dla Ciekawych Świata (http://ciekawi.icm.edu.pl/)
% Copyright (c) 2020 Robert Ryszard Paciorek <rrp@opcode.eu.org>
% Copyright (c) 2020 Krzysztof Lasocki <krz.lasocki@gmail.com>
% 
% MIT License
% 
% Permission is hereby granted, free of charge, to any person obtaining a copy
% of this software and associated documentation files (the "Software"), to deal
% in the Software without restriction, including without limitation the rights
% to use, copy, modify, merge, publish, distribute, sublicense, and/or sell
% copies of the Software, and to permit persons to whom the Software is
% furnished to do so, subject to the following conditions:
% 
% The above copyright notice and this permission notice shall be included in all
% copies or substantial portions of the Software.
% 
% THE SOFTWARE IS PROVIDED "AS IS", WITHOUT WARRANTY OF ANY KIND, EXPRESS OR
% IMPLIED, INCLUDING BUT NOT LIMITED TO THE WARRANTIES OF MERCHANTABILITY,
% FITNESS FOR A PARTICULAR PURPOSE AND NONINFRINGEMENT. IN NO EVENT SHALL THE
% AUTHORS OR COPYRIGHT HOLDERS BE LIABLE FOR ANY CLAIM, DAMAGES OR OTHER
% LIABILITY, WHETHER IN AN ACTION OF CONTRACT, TORT OR OTHERWISE, ARISING FROM,
% OUT OF OR IN CONNECTION WITH THE SOFTWARE OR THE USE OR OTHER DEALINGS IN THE
% SOFTWARE.

%\section{Assembler}
%\textit{Co to \Verb$__asm__$ i kiedy się go używa. Pułapki oszukiwania kompilatora}


\section{Lektury uzupełniające}
\begin{itemize}

  % Link za długi
  %https://www.st.com/resource/en/reference_manual/cd00171190-stm32f101xx-stm32f102xx-stm32f103xx-stm32f105xx-and-stm32f107xx-advanced-arm-based-32-bit-mcus-stmicroelectronics.pdf
  \label{refman}
\item Tzw. \emph{reference manual} dla STM32F103 (\url{http://ln.opcode.eu.org/stm_rm0008}) - obszerny dokument opisujący w jaki sposób programować
  mikrokontroler. Zawiera szczegółowe opisy działania peryferiów, listę rejestrów i pól bitowych wraz ich funkcjami oraz
  adresami \footnotemark. Najważniejszy dokument przy programowaniu mikrokontrolera
  
  \footnotetext{W STM32 opis adresów rejestrów jest rozłożony pomiędzy \textit{reference manual} i kartę katalogową. Informacje dotyczące
    programowania mikrokontrolera są w \textit{Programming manual} (programowanie samego SCB - bloku głównego procesora) oraz
    \textit{Reference manual} (programowanie poszczególnych peryferiów)}

\item \emph{Karta katalogowa STM32F103} (\url{https://www.st.com/resource/en/datasheet/stm32f103c8.pdf}) opisująca pinout i
  parametry mikrokontrolera.

\item \emph{Dokumentacja \Verb$libopencm3$} (\url{http://libopencm3.org/docs/latest/html/}) opisująca funkcje i makra dostępne
  w bibliotece.
  
\item \emph{Przykładowy kod napisany z użyciem \Verb$libopencm3$}\\ (\url{https://github.com/libopencm3/libopencm3-examples}) 
  
\item \emph{Vademecum informatyki praktycznej} (\url{http://vip.opcode.eu.org/}) - zbiór materiałów na temat elektroniki i programowania.
  
\item \emph{Hacker's Delight} - Henry S. Warren, Jr. - książka opisująca dużą ilość algorytmów realizowalnych za pomocą operacji bitowych.
\end{itemize}

\section{Zadania}

\begin{Zadanie}{}{opeartory}
  W programowaniu mikrokontrolerów często zachodzi potrzeba ustawienia pojedynczego bitu rejestru na 0 lub 1, albo zmiany jego wartości.
  Sprawdź (np. rozpisując ich działanie), które wyrażenie w rejestrze \Verb$rejestr$, na podstawie maski \Verb$maska$:
  \begin{itemize}
  \item Ustawia te bity na 1,
  \item Ustawia te bity na 0,
  \item Odwraca wartości tych bitów.
  \end{itemize}

  \begin{CodeFrame*}[c]{}
    /* Wyrażenie 1 */
    rejestr |= maska;

    /* Wyrażenie 2 */
    rejestr ^= maska;

    /* Wyrażenie 3 */
    rejestr &= ~maska;
    \end{CodeFrame*}
\end{Zadanie}

\begin{Zadanie}{}{ustawienie_bitow}
  Napisz wyrażenia, które, nie znając poprzedniej wartości 8-bitowego rejestru \Verb$XYZZY$, wykona operacje:
  \begin{enumerate}
  \item Ustawi jego drugi najmłodszy bit jako 1
  \item Ustawi jego piąty najmłodszy bit jako 0
  \item Odwróci jego najstarszy bit
  \item Wyzeruje jego dolną połowę
  \end{enumerate}
  \textit{Wskazówka: Możesz wygenerować maskę z ustawionym bitem n za przesuwając jedynkę o n miejsc w lewo: \Verb$(1<<n)$}
\end{Zadanie}

\begin{Zadanie}{}{}
W jaki sposób zmienić częstotliwość migania LEDa w pierwszym programie? Zmień ten program tak aby LED migał (około) dwa razy szybciej.
\end{Zadanie}

\begin{Zadanie}{}{}
% Tutaj chodzi o użycie printf-a, putchar-a lub uart_send_blocking
Napisz program, który za pomocą zaimplementowanej w ćwiczeniu UART funkcji wejścia/wyjścia wypisze na UART
``trójkąt z gwiazdek'' jak poniżej

\begin{CodeFrame*}[text]{}
*
**
***
****
*****
******
*******
********
\end{CodeFrame*}
\end{Zadanie}

\begin{Zadanie}{}{adc_volty}
Wiedząc, że wartość 4096 odpowiada napięciu 3.3V, a 0 napięciu 0V, zmień przykładowy program ADC tak, aby zamiast surowej
wartości wypisywał wartość napięcia.\\
\textit{Wskazówka: Aby uprościć obliczenia (uniknąć działań na liczbach zmiennoprzecinkowych), Twój program może podawać wartość w mV.}
\end{Zadanie}

\begin{Zadanie}{}{}
Zmodyfikuj rozwiązanie zadania \ref{adc_volty} tak aby wypisywało wynik w V, bez stosowania arytmetyki zmiennoprzecinkowej.\\
\textit{Wskazówka: Wypisz osobno część całkowitą i część ułamkową}
\end{Zadanie}

\begin{Zadanie}{}{zmiana_i2c}
  Zmień funkcję realizującą logikę slave'a w przykładowym kodzie I2C tak, aby zamiast mnożyć otrzymaną liczbę przez 2, dodawał do niej jakąś
  (dowolną) stałą.
\end{Zadanie}
