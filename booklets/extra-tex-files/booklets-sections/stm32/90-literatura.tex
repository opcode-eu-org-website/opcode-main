% Copyright (c) 2020 Matematyka dla Ciekawych Świata (http://ciekawi.icm.edu.pl/)
% Copyright (c) 2020 Robert Ryszard Paciorek <rrp@opcode.eu.org>
% Copyright (c) 2020 Krzysztof Lasocki <krz.lasocki@gmail.com>
% 
% MIT License
% 
% Permission is hereby granted, free of charge, to any person obtaining a copy
% of this software and associated documentation files (the "Software"), to deal
% in the Software without restriction, including without limitation the rights
% to use, copy, modify, merge, publish, distribute, sublicense, and/or sell
% copies of the Software, and to permit persons to whom the Software is
% furnished to do so, subject to the following conditions:
% 
% The above copyright notice and this permission notice shall be included in all
% copies or substantial portions of the Software.
% 
% THE SOFTWARE IS PROVIDED "AS IS", WITHOUT WARRANTY OF ANY KIND, EXPRESS OR
% IMPLIED, INCLUDING BUT NOT LIMITED TO THE WARRANTIES OF MERCHANTABILITY,
% FITNESS FOR A PARTICULAR PURPOSE AND NONINFRINGEMENT. IN NO EVENT SHALL THE
% AUTHORS OR COPYRIGHT HOLDERS BE LIABLE FOR ANY CLAIM, DAMAGES OR OTHER
% LIABILITY, WHETHER IN AN ACTION OF CONTRACT, TORT OR OTHERWISE, ARISING FROM,
% OUT OF OR IN CONNECTION WITH THE SOFTWARE OR THE USE OR OTHER DEALINGS IN THE
% SOFTWARE.

\section{Lektury uzupełniające}
\begin{itemize}

  % Link za długi
  %https://www.st.com/resource/en/reference_manual/cd00171190-stm32f101xx-stm32f102xx-stm32f103xx-stm32f105xx-and-stm32f107xx-advanced-arm-based-32-bit-mcus-stmicroelectronics.pdf
  \label{refman}
\item Tzw. \emph{reference manual} dla STM32F103 (\url{http://ln.opcode.eu.org/stm_rm0008}) - obszerny dokument opisujący w jaki sposób programować
  mikrokontroler. Zawiera szczegółowe opisy działania peryferiów, listę rejestrów i pól bitowych wraz ich funkcjami oraz
  adresami \footnotemark. Najważniejszy dokument przy programowaniu mikrokontrolera
  
  \footnotetext{W STM32 opis adresów rejestrów jest rozłożony pomiędzy \textit{reference manual} i kartę katalogową. Informacje dotyczące
    programowania mikrokontrolera są w \textit{Programming manual} (programowanie samego SCB - bloku głównego procesora) oraz
    \textit{Reference manual} (programowanie poszczególnych peryferiów)}

\item \emph{Karta katalogowa STM32F103} (\url{https://www.st.com/resource/en/datasheet/stm32f103c8.pdf}) opisująca pinout i
  parametry mikrokontrolera.

\item \emph{Dokumentacja \Verb$libopencm3$} (\url{http://libopencm3.org/docs/latest/html/}) opisująca funkcje i makra dostępne
  w bibliotece.
  
\item \emph{Przykładowy kod napisany z użyciem \Verb$libopencm3$}\\ (\url{https://github.com/libopencm3/libopencm3-examples}) 
  
\item \emph{Vademecum informatyki praktycznej} (\url{http://vip.opcode.eu.org/}) - zbiór materiałów na temat elektroniki i programowania.
  
\item \emph{Hacker's Delight} - Henry S. Warren, Jr. - książka opisująca dużą ilość algorytmów realizowalnych za pomocą operacji bitowych.
\end{itemize}
