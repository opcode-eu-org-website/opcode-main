% Copyright (c) 2020 Matematyka dla Ciekawych Świata (http://ciekawi.icm.edu.pl/)
% Copyright (c) 2020 Robert Ryszard Paciorek <rrp@opcode.eu.org>
% Copyright (c) 2020 Krzysztof Lasocki <krz.lasocki@gmail.com>
% 
% MIT License
% 
% Permission is hereby granted, free of charge, to any person obtaining a copy
% of this software and associated documentation files (the "Software"), to deal
% in the Software without restriction, including without limitation the rights
% to use, copy, modify, merge, publish, distribute, sublicense, and/or sell
% copies of the Software, and to permit persons to whom the Software is
% furnished to do so, subject to the following conditions:
% 
% The above copyright notice and this permission notice shall be included in all
% copies or substantial portions of the Software.
% 
% THE SOFTWARE IS PROVIDED "AS IS", WITHOUT WARRANTY OF ANY KIND, EXPRESS OR
% IMPLIED, INCLUDING BUT NOT LIMITED TO THE WARRANTIES OF MERCHANTABILITY,
% FITNESS FOR A PARTICULAR PURPOSE AND NONINFRINGEMENT. IN NO EVENT SHALL THE
% AUTHORS OR COPYRIGHT HOLDERS BE LIABLE FOR ANY CLAIM, DAMAGES OR OTHER
% LIABILITY, WHETHER IN AN ACTION OF CONTRACT, TORT OR OTHERWISE, ARISING FROM,
% OUT OF OR IN CONNECTION WITH THE SOFTWARE OR THE USE OR OTHER DEALINGS IN THE
% SOFTWARE.

\section{Przetwornik analogowo-cyfrowy}
Przetwornik analogowo-cyfrowy w STM32 należy do jednych z bardziej zaawansowanych. W tym ćwiczeniu pokażemy prosty przykład
użycia przetwornika ADC do odczytu napięcia z jednego z pinów mikrokontrolera.

Zbuduj układ zgodnie z poniższym schematem:

\begin{center}\includegraphics[width=0.8\textwidth]{img/stm32/30_adc}\end{center}

Program będzie wysyłał wynik konwersji na UART. Do wysyłania komunikatów na uart użyjemy kodu z poprzedniego ćwiczenia.
Kod źródłowy znajduje się w katalogu \Verb$30_adc$ i wszystkie elementy związane z obsługą ADC umieszczone zostały w pliku \Verb$main.c$.

Najpierw musimy zainicjalizować peryferium ADC. W tym celu stworzyliśmy funkcję \Verb$adc_setup()$:

\inputminted[frame=single,firstline=13,lastline=35]{c}{stm32-examples/30_adc/main.c} % funkcja 

Powyższy fragment kodu ustawia pin A0 w tryb wejścia analogowego i konfiguruje ADC w sposób pozwalający na przeprowadzanie pojedynczych konwersji. Następnie włącza ADC, czeka pewien okres czasu aż ADC się ustabilizuje wykonuje procedurę kalibracji.\\

Teraz możemy zdefiniować funkcję, która zwróci nam wynik pojedynczej konwersji.

\inputminted[frame=single,firstline=36,lastline=48]{c}{stm32-examples/30_adc/main.c} % funkcja adc_read()

Peryferium ADC w mikrokontrolerach STM32 pobiera listę kanałów na których ma się odbyć konwersja. W naszym przypadku chcemy pobrać
wartość tylko z jedengo kanału. Pin A0 to kanał zerowy, więc umieszczemy zero w pierwszym elemencie tablicy którą przekażemy
funkcji ustawiąjącej kolejność konwersji. Do tej funkcji przekazujemy również ilość kanałów dla których przeprowadzana będzie
konwersja - w tym przypadku 1.

Następnie uruchamiamy konwersję i oczekujemy na jej zakończenie (\Verb$adc_eoc()$ zwraca fałsz dopóki trwa konwersja).
Na końcu naszej funkcji zwracamy wynik konwersji (wynik funkcji \Verb$adc_read_regular(ADC1)$.

Mając już gotowe funkcje obsługujące ADC pozostaje nam tylko włączyć jego zegar i użyć w jakiś sposób wyniku naszej konwersji:

\inputminted[frame=single,firstline=49]{c}{stm32-examples/30_adc/main.c} % funkcja main()

W \Verb$main$ pojawiło się wywołanie \Verb$rcc_periph_clock_enable$ uruchamiające zegar dla ADC, wywołanie naszej funkcji
konfigurującej ADC oraz wypisanie wyniku konwersji w pętli za pomocą \Verb$printf()$.

Po uruchomieniu programu powinien zacąć migać zielony LED, a po uruchomieniu \Verb$picocom$ powinniśmy widzieć wyniki konwersji
przesyłane przez UART. Obracająć potencjometr zmieniamy napięcie na pinie A0, co zmienia wynik konwersji ADC.
