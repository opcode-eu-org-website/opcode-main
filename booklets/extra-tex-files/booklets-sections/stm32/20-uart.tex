% Copyright (c) 2020 Matematyka dla Ciekawych Świata (http://ciekawi.icm.edu.pl/)
% Copyright (c) 2020 Robert Ryszard Paciorek <rrp@opcode.eu.org>
% Copyright (c) 2020 Krzysztof Lasocki <krz.lasocki@gmail.com>
% 
% MIT License
% 
% Permission is hereby granted, free of charge, to any person obtaining a copy
% of this software and associated documentation files (the "Software"), to deal
% in the Software without restriction, including without limitation the rights
% to use, copy, modify, merge, publish, distribute, sublicense, and/or sell
% copies of the Software, and to permit persons to whom the Software is
% furnished to do so, subject to the following conditions:
% 
% The above copyright notice and this permission notice shall be included in all
% copies or substantial portions of the Software.
% 
% THE SOFTWARE IS PROVIDED "AS IS", WITHOUT WARRANTY OF ANY KIND, EXPRESS OR
% IMPLIED, INCLUDING BUT NOT LIMITED TO THE WARRANTIES OF MERCHANTABILITY,
% FITNESS FOR A PARTICULAR PURPOSE AND NONINFRINGEMENT. IN NO EVENT SHALL THE
% AUTHORS OR COPYRIGHT HOLDERS BE LIABLE FOR ANY CLAIM, DAMAGES OR OTHER
% LIABILITY, WHETHER IN AN ACTION OF CONTRACT, TORT OR OTHERWISE, ARISING FROM,
% OUT OF OR IN CONNECTION WITH THE SOFTWARE OR THE USE OR OTHER DEALINGS IN THE
% SOFTWARE.

\section{UART}
UART to jeden z popularniejszych protokołów przesyłania danych. Jest podstawą wielu innych protokołów, i obecny praktycznie wszędzie
W tym ćwiczeniu dowiesz się jak zaprogramować peryferium UARTu w swoim mikrokontrolerze. Pokażemy też w jaki sposób ``połączyć'' znane Ci funkcje
wejścia/wyjścia w taki sposób aby odbywało się ono przez UART. Pliki źródłowe znajdują się w katalogu \Verb$20_uart$, pierwszy z nich (\Verb$main.c$) ma postać:

\inputminted[frame=single,firstline=6]{c}{stm32-examples/20_uart/main.c}

Na pierwszy rzut oka ten program wydaje się być bardzo podobny do ćwiczenia pierwszego. Pojawiły się jednak dwa nowe pliki nagłówkowe
oraz dwa wywołania nowych funkcji. Pierwszy plik nagłówkowy to znany Ci \Verb$stdio.h$, który definiuje funkcje wejścia/wyjścia. Drugi, \Verb$uart.h$
znajduje się w cudzysłowiach, co oznacza że jest to plik lokalny. Poniżej jest jego zawartość:

\inputminted[frame=single]{c}{stm32-examples/20_uart/uart.h}

W pliku zadeklarowane są dwie funkcje, \Verb$_write$ oraz \Verb$usart_setup$. Przyjrzyjmy się plikowi \Verb$uart.c$ zawierającemu definicje tych funkcji:

\inputminted[frame=single]{c}{stm32-examples/20_uart/uart.c}

Funkcja \Verb$usart_setup$ konfiguruje peryferium USART.
%
Najpierw należy ustawić pin A9 w trybie \Verb$GPIO_CNF_OUTPUT_ALTFN_PUSHPULL$, czyli włączyć go jako
pin wyjściowy w trybie jego drugiej funkcji (\emph{ALTFN - Alternative Function}). W STM32 większość z pinów pełni dwie funkcje - domyślną z nich jest
GPIO, a drugą odpowiednia funkcja peryferium. W celu użycia drugiej funkcji danego pinu, musimy go w taki sposób skonfigurować. W tym przypadku
TX znajduje się na pinie A9 (do którego podłączasz programator - ponieważ programowanie STM32 odbywa się przez UART\footnote{Można również przez
SWD lub JTAG, ale do tego potrzebne są inne programatory. Twoja przejściówka USB-UART pełni tutaj także drugą rolę jako programator}):

\begin{CodeFrame*}[c]{}
  gpio_set_mode(GPIOA, GPIO_MODE_OUTPUT_50_MHZ,
                GPIO_CNF_OUTPUT_ALTFN_PUSHPULL, GPIO_USART1_TX);
\end{CodeFrame*}

Następnie konfigurowany jest sam UART. Tutaj ustawiamy typowe wartości - prędkość 9600 baud\footnotemark, 8 bitów danych, 1 bit stopu.
Ustawiamy USART w trybie nadawania (odbieranie jest wyłączone). Ustawiamy brak bitu parzystości i brak kontroli przepływu:

\begin{CodeFrame*}[c]{}
  /* Setup UART parameters. */
  usart_set_baudrate(USART1, 9600);
  usart_set_databits(USART1, 8);
  usart_set_stopbits(USART1, USART_STOPBITS_1);
  usart_set_mode(USART1, USART_MODE_TX);
  usart_set_parity(USART1, USART_PARITY_NONE);
  usart_set_flow_control(USART1, USART_FLOWCONTROL_NONE);
\end{CodeFrame*}

\footnotetext{\emph{baud} (czyt. bod) to określenie jednostki symbol/sek. Prędkość transmisji w tej jednostce nazywa się \emph{baudrate}. W naszym
  przypadku symbolem jest stan niski/wysoki (bit). Z reguły baudrate nie jest równy przepustowości łącza, ponieważ wlicza się do niego też symbole
  które nie przenoszą danych(w przypadku UARTa są to bity startu,
  stopu i ew. parzystości).}
  
Na końcu procedury uruchamiamy USART:

\begin{CodeFrame*}[c]{}
  usart_enable(USART1);
\end{CodeFrame*}

Przejdźmy teraz do funkcji \Verb$_write$. Ta funkcja to nasz STM-owy odpowiednik funkcji \Verb$write$ z jądra linuksa. Służy ona do wypisania ciągu
bajtów o podanej długości na podany deskryptor pliku.

W naszym przypadku ignoruje ona wszystkie deskryptory powyżej 2 (w normalnym środowisku do tych deskryptorów podłączone były by otwarte pliki w systemie
plików. Jeśli chcielibyśmy dodać w naszym programie obsługę plików, np. przez kartę SD albo jakiś wirtualny system plików, w tym miejscu
należało by wprowadzić pierwszą zmianę):

\begin{CodeFrame*}[c]{}
  if (fd > 2) {
    return -1;
  }
\end{CodeFrame*}

Następnie przy pomocy pętli while, która wykona się \Verb$len$ razy lub aż dojdzie do znaku zerowego w napisie\footnote{Prawdziwy \Verb$write$ nie
  sprawdza tego drugiego warunku, ale ta wersja będzie służyć tylko do wypisywania napisów na UART}, wywołujemy funkcję \Verb$usart_send_blocking$
z parametrem USART1 i kolejnymi znakami z wypisywanego napisu. W ten sposób napis podany jako parametr będzie wypisany znak po znaku.

\begin{CodeFrame*}[c]{}
  while (*ptr && (i < len)) {
    usart_send_blocking(USART1, *ptr);
    if (*ptr == '\n') {
      usart_send_blocking(USART1, '\r');
    }
    i++; 
    ptr++;
  }
\end{CodeFrame*}

Warunek porównujący znak ze znakiem nowej linii i wstawiający znak powrotu karetki jest potrzeby aby po znaku nowej linii kursor w terminalu powrócił
na początek linii.

Na końcu zwraca ilość wypisanych znaków (tak jak prawdziwa funkcja \Verb$write$):

\begin{CodeFrame*}[c]{}
  return i;
\end{CodeFrame*}

Ta funkcja jest potrzebna (mimo tego, że nigdzie jej bezpośrednio nie wywołujemy) do tego aby funkcje z \Verb$stdio.h$ mogły komunikować się ze światem.
Jak widać jest to ograniczona wersja prawdziwej funkcji write, która dostarcza ``sztuczną'' obsługę plików \Verb$stdin$, \Verb$stdout$, i \Verb$stderr$
w naszym środowisku. 

To właśnie tą funkcję będą wywoływać \Verb$printf$, \Verb$puts$ w ramach ostatecznego wypisania danych na wyjście.
\\

Wracając do pliku \Verb$main.c$:

\begin{CodeFrame*}[c]{}
  while(1){
    for (int i = 0; i < 150000; i++) __asm__("nop");
    gpio_toggle(GPIOC, GPIO13);
    printf("Hello, World!\n");
  }
\end{CodeFrame*}

Ta pętla oprócz znanego już migania diodą wypisuje ``Hello, World!'' na standardowe wyjście (teraz podłączone do UARTa) za każdą iteracją.


Upewnij się, że wszystkie 3 pliki są zapisane a następnie skompiluj i wgraj ten program za pomocą:

\begin{CodeFrame*}[bash]{}
  make
  make install
\end{CodeFrame*}

a następnie uruchom \Verb$picocom$ aby zobaczyć co dzieje się na porcie szeregowym:

\begin{CodeFrame*}[bash]{}
  picocom /dev/ttyUSB0
\end{CodeFrame*}

\begin{ProTip}{Porada}
  Możesz automatycznie uruchamiać picocom po udanym wgraniu programu łącząc te dwa polecenia w następujący sposób:
  \begin{minted}[frame=none]{bash}
  make install && picocom /dev/ttyUSB0
  \end{minted}
\end{ProTip}

\subsection{Odbieranie danych poprzez UART}

W ostatnim ćwiczeniu nauczyliśmy się wysyłać dane poprzez interfejs UART. Uruchomiliśmy także wygodny\footnote{
	Ta wygoda ma jednak swoją cenę.
	Jeżeli zauważyliście że ostatni program wgrywał się dłużej niż poprzednie to jest to efektem zwiększenia jego rozmiaru (a zatem i zapotrzebowania na pamięć flash) na skutek dodania funkcji \Verb$printf$.
	Nasz mikrokontroler posiada jej na tyle dużo iż nie jest to dla nas istotnym problemem, ale warto pamiętać iż użycie \Verb$printf$ na słabszych mikrokontrolerach może być praktycznie nie do zrealizowania.
}
sposób wypisywania informacji na port szeregowy poprzez standardową funkcję \Verb$printf$.

UART możemy wykorzystać także do odbierania danych. W tym ćwiczeniu uruchomimy odbiornik UART, a jako że jego działanie oprzemy na przerwaniach (a nie aktywnym czekaniu na dane w wyznaczonym punkcie programu) to zapoznamy się także z podstawą obsługi przerwań.

\begin{ProTip}{Przerwania}
  Przerwanie jest to (sprzętowy lub programowy) sygnał dla procesora, powodujący zmianę przepływu sterowania, polegającą na przerwaniu aktualnie wykonywanego kodu programu i rozpoczęcie wykonywania procedury obsługi przerwania.
  Zazwyczaj odbywa się to z użyciem wektora przerwań, czyli tablicy używanej do mapowania numeru przerwania na adres pod którym umieszczona jest procedura obsługi danego przerwania.
  
  Na wcześniejszych zajęciach wspomnieliśmy już o przerwaniu zegarowym, które jest używane przez system operacyjny m.in. do okresowego przerywania działania programów celem ustalenia który z czekających procesów powinien rozpocząć swoje (dalsze) wykonywanie.
  Mechanizm ten (z użyciem przerwania generowanego programowo) jest wykorzystywany także do wywoływania funkcji systemowych z poziomu kodu użytkownika.
\end{ProTip}

Kod źródłowy dla tego ćwiczenia znajduje się w katalogu \Verb$21_uart_receiver$. Pliki \Verb$uart.h$ i \Verb$uart.c$ mają postać dokładnie taką samą jak poprzednio i nie będziemy ich modyfikować (będą one nam dostarczały jednokierunkowy UART używany przez \Verb$printf$ w tym i w kolejnych ćwiczeniach). Zmianom uległ natomiast \Verb$main.c$ i wygląda następująco:

\inputminted[frame=single,firstline=6]{c}{stm32-examples/21_uart_receiver/main.c}

Pierwszą rzeczą konieczną do wykonania jest ustawienie adresu wektora przerwań na początkowy adres pamięci flash,
	gdyż nasz program wgrywany i uruchamiany jest właśnie z wbudowanej pamięci flash,
	a kompilator wektor przerwań umieszcza na początku kodu\footnote{
		W tablicy tej, na dobrze zdefiniowanej pozycji (bajty 4-7, odpowiadające wyjątkowi \textit{reset}), umieszczany jest także adres początku kodu związanego z naszym programem.
		To do tego adresu wykonywany jest skok gdy zrestartujemy nasz mikrokontroler z ustawioną zworkami BOOT opcją uruchamiania z pamięci flash
			lub wydamy wbudowanemu bootloaderowi polecenie uruchomienia od początku pamięci flash (opcja \Verb$-g 0x0$).
	}:
	
\begin{CodeFrame*}[c]{}
  SCB_VTOR = FLASH_BASE;
\end{CodeFrame*}

Następnie możemy i powinniśmy włączyć obsługę (odbieranie) przerwań związanych z USART1:

\begin{CodeFrame*}[c]{}
  nvic_enable_irq(NVIC_USART1_IRQ);
\end{CodeFrame*}
oraz aktywować generowanie przez układ interfejsu USART przerwań związanych z odebraniem bajtu:

\begin{CodeFrame*}[c]{}
  usart_enable_rx_interrupt(USART1);
\end{CodeFrame*}

W kolejnych krokach konfigurujemy UART tak jak w poprzednim ćwiczeniu, konfigurujemy pin A10 (UART1 RX) jako input oraz przełączamy tryb pracy naszego portu szeregowego na nadawanie i odbiór:

\begin{CodeFrame*}[c]{}
  usart_setup();
  gpio_set_mode(GPIOA, GPIO_MODE_INPUT, GPIO_CNF_INPUT_FLOAT, GPIO10);
  usart_set_mode(USART1, USART_MODE_TX_RX);
\end{CodeFrame*}

Po powitalnym odliczaniu funkcje \Verb$main$ kończymy nieskończoną, pustą (wykonującą \Verb$nop$) pętlą \Verb$while$:

\begin{CodeFrame*}[c]{}
  while(1)
      _asm__("nop");
\end{CodeFrame*}

Musimy dodać jeszcze jedną funkcję która będzie związana z obsługą przerwania od portu szeregowego.
Używana biblioteka wymaga aby funkcja ta nazywała się \Verb#usart1_isr#.
W ramach niej odczytujemy z bufora wejściowego odebrany bajt (przy użyciu funkcji \Verb$usart_recv()$) i w zależności od jego parzystości włączamy lub wyłączamy LED podłączony do pinu C13:

\begin{CodeFrame*}[c]{}
  void usart1_isr(void) {
    if ( USART_SR(USART1) & USART_SR_RXNE ) {
      uint8_t data = usart_recv(USART1);
      if (data%2)
        gpio_set(GPIOC, GPIO13);
      else
        gpio_clear(GPIOC, GPIO13);
    }
  }
\end{CodeFrame*}
