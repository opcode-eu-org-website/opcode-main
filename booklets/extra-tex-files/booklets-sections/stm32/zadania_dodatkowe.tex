% Copyright (c) 2020 Matematyka dla Ciekawych Świata (http://ciekawi.icm.edu.pl/)
% Copyright (c) 2020 Robert Ryszard Paciorek <rrp@opcode.eu.org>
% Copyright (c) 2020 Krzysztof Lasocki <krz.lasocki@gmail.com>
% 
% MIT License
% 
% Permission is hereby granted, free of charge, to any person obtaining a copy
% of this software and associated documentation files (the "Software"), to deal
% in the Software without restriction, including without limitation the rights
% to use, copy, modify, merge, publish, distribute, sublicense, and/or sell
% copies of the Software, and to permit persons to whom the Software is
% furnished to do so, subject to the following conditions:
% 
% The above copyright notice and this permission notice shall be included in all
% copies or substantial portions of the Software.
% 
% THE SOFTWARE IS PROVIDED "AS IS", WITHOUT WARRANTY OF ANY KIND, EXPRESS OR
% IMPLIED, INCLUDING BUT NOT LIMITED TO THE WARRANTIES OF MERCHANTABILITY,
% FITNESS FOR A PARTICULAR PURPOSE AND NONINFRINGEMENT. IN NO EVENT SHALL THE
% AUTHORS OR COPYRIGHT HOLDERS BE LIABLE FOR ANY CLAIM, DAMAGES OR OTHER
% LIABILITY, WHETHER IN AN ACTION OF CONTRACT, TORT OR OTHERWISE, ARISING FROM,
% OUT OF OR IN CONNECTION WITH THE SOFTWARE OR THE USE OR OTHER DEALINGS IN THE
% SOFTWARE.

\IfStrEq{\dbEntryID}{}{
	\section{Zadania dodatkowe}
	\insertZadanie{\currfilepath}{opeartory_bitowe}{}
	\insertZadanie{\currfilepath}{zadanie_maska_bitowa}{}
	\insertZadanie{\currfilepath}{ustawienie_bitow}{}
}

\dbEntryBegin{opeartory_bitowe}\if1\dbEntryCheckResults
  W programowaniu mikrokontrolerów często zachodzi potrzeba ustawienia pojedynczego bitu rejestru na 0 lub 1, albo zmiany jego wartości.
  Sprawdź (np. rozpisując ich działanie), które wyrażenie w rejestrze \Verb$rejestr$, na podstawie maski \Verb$maska$:

  \begin{CodeFrame}[c]{0.55\textwidth}
    rejestr |= maska;  // <- Wyrażenie 1
    rejestr ^= maska;  // <- Wyrażenie 2
    rejestr &= ~maska; // <- Wyrażenie 3
  \end{CodeFrame}
  \noindent\begin{minipage}[t]{0.43\textwidth}\vspace{-3pt}\begin{itemize}
    \item Ustawia te bity na 1,
    \item Ustawia te bity na 0,
    \item Odwraca wartości tych bitów.
  \end{itemize}\end{minipage}\hfill
\fi

\dbEntryBegin{zadanie_maska_bitowa}\if1\dbEntryCheckResults
  Jaką maską bitową można sprawdzić czy bit nr. 1 jest w stanie wysokim? A jaką można sprawdzić to samo, ale
  dla wszystkich bitów parzystych (na pozycjach 0, 2, 4 ... 14\footnote{16 bitowa liczba ma bity ``ponumerowane'' od 0 do 15})?
\fi

\dbEntryBegin{ustawienie_bitow}\if1\dbEntryCheckResults
  Napisz wyrażenia, które nie znając poprzedniej wartości 8-bitowego rejestru \Verb$XYZZY$, wykona operacje:
  \begin{enumerate}
  \item Ustawi jego drugi najmłodszy bit jako 1
  \item Ustawi jego piąty najmłodszy bit jako 0
  \item Odwróci jego najstarszy bit
  \item Wyzeruje jego dolną połowę
  \end{enumerate}
  \textit{Wskazówka: Możesz wygenerować maskę z ustawionym bitem n za przesuwając jedynkę o n miejsc w lewo: \Verb$(1<<n)$}
\fi

\dbEntryBegin{trojkat_via_uart_simple}\if1\dbEntryCheckResults
% Tutaj chodzi o użycie printf-a, putchar-a lub uart_send_blocking
Napisz program, który za pomocą zaimplementowanej w ćwiczeniu UART funkcji wejścia/wyjścia wypisze na UART ``trójkąt z gwiazdek'' jak poniżej

\begin{CodeFrame*}[text]{}
*
**
***
****
*****
******
\end{CodeFrame*}
\fi
