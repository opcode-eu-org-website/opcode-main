% Copyright (c) 2013-2021 Robert Ryszard Paciorek <rrp@opcode.eu.org>
% 
% MIT License
% 
% Permission is hereby granted, free of charge, to any person obtaining a copy
% of this software and associated documentation files (the "Software"), to deal
% in the Software without restriction, including without limitation the rights
% to use, copy, modify, merge, publish, distribute, sublicense, and/or sell
% copies of the Software, and to permit persons to whom the Software is
% furnished to do so, subject to the following conditions:
% 
% The above copyright notice and this permission notice shall be included in all
% copies or substantial portions of the Software.
% 
% THE SOFTWARE IS PROVIDED "AS IS", WITHOUT WARRANTY OF ANY KIND, EXPRESS OR
% IMPLIED, INCLUDING BUT NOT LIMITED TO THE WARRANTIES OF MERCHANTABILITY,
% FITNESS FOR A PARTICULAR PURPOSE AND NONINFRINGEMENT. IN NO EVENT SHALL THE
% AUTHORS OR COPYRIGHT HOLDERS BE LIABLE FOR ANY CLAIM, DAMAGES OR OTHER
% LIABILITY, WHETHER IN AN ACTION OF CONTRACT, TORT OR OTHERWISE, ARISING FROM,
% OUT OF OR IN CONNECTION WITH THE SOFTWARE OR THE USE OR OTHER DEALINGS IN THE
% SOFTWARE.

\section{Debugowanie}

\subsection{gdb}

GNU Debugger (\Verb$gdb$) jest narzędziem służącym do podglądania działania innego programu.
Umożliwia ustawianie pułapek (powodujących przerwanie wykonywania na wywołaniu określonej funkcji, uruchamianie programu linia po linii, podgląd wartości zmiennych, ... .
Umożliwia także sprawdzenie co program robił gdy się wywalił (na podstawie pliku core).
Jest to narzędzie programistyczne, które niewątpliwie warto poznać, gdyż znacznie ułatwia poprawianie błędów w programach.

Dla skorzystania z części możliwości konieczne jest wkompilowanie informacji debugerowych w program (np. \Verb$gcc$ z opcją \Verb$-g$).
Program uruchamiany jest przez wywołanie gdb, zazwyczaj z jedną lub dwiema opcjami - plikiem wykonywalny oraz (ewentualnie) plikiem core. Do najistotniejszych poleceń należy:

\begin{itemize}
	\item\Verb$break plik:linia$ - ustawienie breakpointa na wskazanej linii
	\item\Verb$break nazwa_funkcji$ - ustawiające breakpointa na daną funkcję
	\item\Verb$rbreak fragment_nazwy_funkcji$ - ustawiające breakpointa na podstawie wyrażenia regularnego (np. fragmentu nazwy funkcji)
	\item\Verb$info breakpoint$ - lista ustawionych breakpointow
	\item\Verb$watch wyrazenie$ - zatrzymanie gdy wartość wyrażenia ulegnie zmianie
\vspace{6pt}
	\item\Verb$run$ - uruchamiające program (zatrzymuje się na pierwszej napotkanej pułapce)
	\item\Verb$continue$ - powoduje kontynuację programu do następnej pułapki
	\item\Verb$step$ - powoduje wykonanie kolejnej linii programu, z wchodzeniem w kod funkcji
	\item\Verb$next$ - powoduje wykonanie kolejnej linii programu, bez wchodzenia w kod funkcji
	\item\Verb$stepi$ / \Verb$nexti$ - jak \Verb$step$ / \Verb$next$, tyle że nie operuje na liniach a instrukcjach
	\item\Verb$finish$ - powoduje dokończenie bieżącej funkcji i zatrzymanie się
\vspace{6pt}
	\item\Verb$print zmienna$ - wypisanie wartości zmiennej zmiennej/wyrażenia (do elementów składowych odwołujemy się jak w C++)
	\item\Verb$display zmienna$ - automatyczne wypisanie wartości zmiennej wyrażenia przy każdym zatrzymaniu (breakpoincie), odwołanie przez \Verb$undisplay numer$
	\item\Verb$ptype zmienna$ - wypisanie informacji o typie zmiennej (m.in. elementy składowe struktury)
	\item\Verb$info local$ - podgląd zmiennych lokalnych
	\item\Verb$info args$ - podgląd argumentów funkcji
	\item\Verb$backtrace$ - wypisanie historii stosu wywołań funkcji
	\item\Verb$backtrace num$ - wypisanie historii stosu wywołań funkcji, wypisuje num pozycji (gdy num mniejsze od 0 to początkowych, gdy większe to końcowych)
\vspace{6pt}
	\item\Verb$set zmienna=wartosc$ - ustawia wartość zmiennej na podaną
	\item\Verb$call funkcja(argumenty)$ - wywołuje podaną funkcję z podanymi argumentami
\vspace{6pt}
	\item\Verb$help$ wyświetlające pomoc "on-line"
	\item\Verb$list funkcja$ - wyświetlenie kodu funkcji
	\item\Verb$list linia1,linia2$ - wyświetlenie kodu programu od linii do linii we aktualnym pliku
	\item\Verb$list plik:linia1,plik:linia2$ - wyświetlenie kodu programu od linii do linii
	\item\Verb$quit$ kończące pracę gdb
\end{itemize}

Oczywiście to tylko niewielka część wszystkich poleceń gdb, polecam zapoznać się z poszczególnymi grupami poleceń opisywanymi przez \Verb$help$.

Uruchamianie poprzez GDB przydaje się także przy debugowaniu problemów typu "segmentation fault" – po zatrzymaniu programu z powodu takiego błędu możemy wydać w gdb polecenie "backtrace" aby zobaczyć w której funkcji wystąpił problem.

\subsection{analiza plików binarnych}

\subsubsection{listowanie symboli}

Przy problemach z linkowaniem przydatne może być wyświetlenie symboli które zawiera konkretna biblioteka. Możemy to zrobić poleceniami:

\begin{Verbatim}
# listowanie symboli w pliku wykonywalnym typu elf:
	readelf -sWc   PLIK

# listowanie symboli w pliku .so:
	nm -gC         PLIK.so

# listowanie symboli w pliku .a:
	nm             PLIK.a
\end{Verbatim}
