% Copyright (c) 2018-2019 Robert Ryszard Paciorek <rrp@opcode.eu.org>
% 
% MIT License
% 
% Permission is hereby granted, free of charge, to any person obtaining a copy
% of this software and associated documentation files (the "Software"), to deal
% in the Software without restriction, including without limitation the rights
% to use, copy, modify, merge, publish, distribute, sublicense, and/or sell
% copies of the Software, and to permit persons to whom the Software is
% furnished to do so, subject to the following conditions:
% 
% The above copyright notice and this permission notice shall be included in all
% copies or substantial portions of the Software.
% 
% THE SOFTWARE IS PROVIDED "AS IS", WITHOUT WARRANTY OF ANY KIND, EXPRESS OR
% IMPLIED, INCLUDING BUT NOT LIMITED TO THE WARRANTIES OF MERCHANTABILITY,
% FITNESS FOR A PARTICULAR PURPOSE AND NONINFRINGEMENT. IN NO EVENT SHALL THE
% AUTHORS OR COPYRIGHT HOLDERS BE LIABLE FOR ANY CLAIM, DAMAGES OR OTHER
% LIABILITY, WHETHER IN AN ACTION OF CONTRACT, TORT OR OTHERWISE, ARISING FROM,
% OUT OF OR IN CONNECTION WITH THE SOFTWARE OR THE USE OR OTHER DEALINGS IN THE
% SOFTWARE.

\section{Narzędzia pythonowe}

\subsection{PIP}

Pip jest narzędziem służącym do instalowania modułów pythonowych z dedykowanego repozytorium (managerem/instalatorem pakietów pythonowych). W przypadku Pythona w wersji 3 należy wywoływać go poleceniem \Verb{pip3}.
Po nazwie polecenia podaje się komendę określającą działania które ma wykonać manager, do najistotniejszych należy zaliczyć:\\
\begin{itemize}
	\item \texttt{install \textit{nazwa}} instaluje podany pakiet
	\item \texttt{install -U \textit{nazwa}} aktualizuje podany pakiet
	\item \texttt{uninstall \textit{nazwa}} usuwa podany pakiet
	\item \texttt{search \textit{słowo}} wyszukuje pakiety w oparciu o podane słowo kluczowe
	\item \texttt{list -v} wypisuje zainstalowane pakiety wraz z informacją o katalogu w którym zostały zainstalowane oraz czy pakiet był instalowany przy pomocy pip, czy w inny sposób (np. przez systemowego menagera pakietów)
\end{itemize}

Pip automatycznie instaluje zależności pakietu, w niektórych przypadkach może to wymagać zainstalowanych narzędzi i odpowiednich bibliotek C i/lub C++.

\begin{Zadanie}{}{}
Zainstaluj (lub zaktualizuj jeżeli jest obecnie zainstalowany) przy pomocy pip pakiet virtualenv
\end{Zadanie}

\subsection{Virtualenv}

Virtualenv jest narzędziem pozwalającym na tworzenie środowisk pythonowych udostępniających różne wersje modułów. W celu utworzenia nowego środowiska wywołuje się go poleceniem \Verb{virtualenv ścieżka}, gdzie \Verb{ścieżka} wskazuje na miejsce w którym ma zostać utworzone środowisko. Po utworzeniu środowiska można doinstalować wewnątrz niego wymagane wersje bibliotek przy pomocy komendy pip dostępnej w pod-katalogu \Verb{bin} katalogu utworzonego środowiska. Do wywołania programów pythonowych z użyciem danego środowiska należy korzystać z pliku wykonywalnego python3 w pod-katalogu \Verb{bin} danego środowiska. Można także skorzystać z pliku \Verb{bin/activate} w katalogu utworzonego środowiska, poprzez wczytanie go w bierzącej powłoce basha (\Verb{. scizeka_do_enva/bin/activate}) celem uzyskania powłoki w której domyślnym interpreterem pythona będzie ten z utworzonego środowiska ,,virtualenv''. Zmiany w środowisku wprowadzone w wyniku wczytania pliku \Verb{bin/activate} obowiązują tylko w bierzącej powłoce, można je także odwrócić przy pomocy polecenia \Verb{deactivate}.

\begin{CodeFrame*}[bash]{}
# tworzenie nowego środowiska
virtualenv /tmp/py1
# użycie pip wewnątrz środowiska
/tmp/py1/bin/pip3 list
# wywołanie pytana wewnątrz środowiska
/tmp/py1/bin/python3

# "przełączenie" na pythona z środowiska
. /tmp/py1/bin/activate
# teraz aby użyć pip lub wewnątrz środowiska wystarczy uruchomić
pip3 list
python3
\end{CodeFrame*}

\begin{ProTip}{Informacja {\Symbola 🤔}}
Virtualenv nie jest chroot'em - Python z tak utworzonego środowiska ,,widzi'' resztę systemu, a także ,,wie'' w jakiej ścieżce jest zainstalowany (zobacz zawartość zmiennej \Verb{sys.path}).
\end{ProTip}

\begin{Zadanie}{}{}
Utwórz środowisko pythonowe typu ,,virtualenv'' w katalogu ytdl i zainstaluj wewnątrz niego pakiet pythonowy youtube-dl.
\end{Zadanie}
