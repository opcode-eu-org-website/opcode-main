
\documentclass{pdfBooklets}
\usepackage[printwatermark]{xwatermark}
\title{Programowanie mikrokontrolerów STM32}
\author{%
	Projekt ,,Matematyka dla Ciekawych Świata'',\\
	Krzysztof Lasocki\\\normalsize\ttfamily <krz.lasocki@gmail.com>
}
\date  {2020-04-04}

\makeatletter\hypersetup{
	pdftitle = {\@title}, pdfauthor = {\@author}
}\makeatother


\begin{document}

\maketitle

\newwatermark[allpages,color=red!50,angle=45,scale=3,xpos=0,ypos=0]{Wersja robocza}

\textit{o czym to w ogole jest? napisac slowo wstepne}

\section{Przygotowanie mikronotrolera}
\textit{tutaj opisać co i jak polutować. Obrazki? to powinno być też pokazane przed właściwymi ćwiczeniami}
\begin{ProTip}{\normalfont{\strong{Ostrożnie}}}
  Podczas pracy grot (metalowa końcówka) lutownicy jest rozgrzany do 200-400 stopni Celsjusza. Zachowaj ostrożność
  podczas jej używania. Zapamiętaj:
  \begin{itemize}
  \item Zawsze odkładaj lutownicę do stojaka. Nigdy nie kładź jej luzem na stole.
  \item Nigdy nie łap za metalowy koniec lutownicy.
  \item Nie wolno łapać spadającej lutownicy. Nie martw się, zawszę można kupić nową
  \item Grot lutownicy jest gorący przez jakiś czas po wyłączeniu
  \item Nie zostawiaj włączonej lutownicy bez opieki
  \item Przed lutowaniem musisz odłączyć układ od zasilania
  \end{itemize}
\end{ProTip}

Jeżeli Twój mikrokontroler nie ma przylutowanych pinów, musisz przylutować je samemu. Przymierz i odetnij
obcążkami dwa odcinki listwy kołkowej pasujące do otworów na brzegach płytki mikrokontrolera. Włóż piny do otworków
na płytce. Odwiń lub wyciągnij odcinek cyny ze szpulki. Aby zalutować pin w otworze, najpierw dotknij grotem lutownicy
miejsca, które będziesz lutować. Po około sekundzie, dotknij pinu końcówką odcinka cyny.

Postaraj się aby listwy kołkowe były prostopadle do płytki. Jeżeli masz odcinek damskiej listwy, możesz za jego pomocą
umiejscowić piny które lutujesz. Możesz też użyć do tego płytki stykowej. Lutowanie zacznij od czterech pinów na rogach płytki.


\section{Płytka stykowa}
  Płytka stykowa pozwala na szybkie prototypowanie układów. Posiada ona macierz dziurek. Pod nimi są umieszczone
  blaszki które łączą sąsiednie dziurki w taki sposób, że każde 5 dziurek w pionie jest ze sobą połączone elektrycznie.
  
  Wyżłobienie w środku płytki służy do 





\end{document}
