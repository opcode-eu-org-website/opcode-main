\ifdefined\inputOnlyContent\else
\documentclass{pdfBooklets}

\title{Laboratorium programistyczne: metody numeryczne (interpolacja, różniczkowanie i całkowanie)}
\author{%
	Projekt ,,Matematyka dla Ciekawych Świata'',\\
	Robert Ryszard Paciorek\\\normalsize\ttfamily <rrp@opcode.eu.org>
}
\date  {2018-05-23}

\makeatletter\hypersetup{
	pdftitle = {\@title}, pdfauthor = {\@author}
}\makeatother


\begin{document}

\maketitle
\fi

\teacher{Skrypt przeznaczony do realizacji na dwóch ostatnich labach programistycznych. W przypadku braku czasu można pominąć wybrane zagadnienia. W przypadku przedwczesnego skończenia skryptu można pokazać inne algorytmy numeryczne lub wrócić do materiałów dodatkowych z pierwszego skryptu.}

\teacher{\todo{używanie dx w argumentach całkowania (co powoduje problem z granicą) wymaga dopracyzowania zadań (bo obecnie gdy (a-b)\%dx !=0 nie istnieje rozwiązanie) ... może jednak lepiej dawać ilość punktów i dx obliczać wewnętrznie}}

\section{Wielomiany}

Najczęściej spotykanym zapisem wielomianu jest postać sumy jednomainów:
$$w(x) = \sum_{i=0}^na_ix^i = a_0 + a_1x + \ldots + a_nx^n$$
Żeby wiedzieć, o jaki wielomian chodzi, wystarczy że okreslimy listę jego współczynników $a_0 \ldots a_n$. 
\teacher{Zanim przejdziemy dalej, można zapytać, jak obliczalibyśmy np. wartość wielomianu $[1,1,\ldots,1]$, gdzie $n=2, \, 10$ lub $100$, w jakimś wybranym punkcie. Jak zapisalibyśmy algorytm tego obliczenia?}


\subsection{Obliczanie wartości wielomianu}

Niech  \Verb{a} będzie listą kolejnych współczynników wielomianu $a_0 \ldots a_n$.
Do obliczenia wartości wielomianu w zadanym punkcie $x$ naturalne wydaje się zastosowanie powyższego wzoru w następujący sposób:

\begin{CodeFrame*}[python]{}
def oblicz(a, x):
    w = 0
    for i in range(len(a)):
        w += a[i] * (x**i)
    return w
\end{CodeFrame*}
Niestety podejście takie jest bardzo nieefektywne, gdyż w każdym kolejnym kroku sumy musimy wykonać coraz więcej mnożeń. W ogólności w kroku $n$-tym musimy wykonać $n$ mnożeń, co przekłada się na czasową złożoność obliczeniową rzędu $\mathcal{O}(n^2)$, co oznacza że czas wykonania rośnie z kwadratem ilości współczynników wielomianu (czyli, ogólnie rzecz biorąc, ze stopniem wielomianu). Innymi słowy, jeśli zwiększymy 10 razy stopień wielomianu, to czas liczenia zwiększy się nam $100=10^2$ razy. 
Ponieważ będziemy czasem potrzebowali wielomianów wysokiego stopnia  --- o czym za chwilę --- nie jest to dla nas najlepsza sytuacja. 

Jeżeli zauważymy, że potrzebujemy policzyć kolejno wszystkie potęgi wartości $x$, aż do $n$ to algorytm ten możemy znacznie przyspieszyć implementując liczenie potęgi w tej samej pętli co sumowanie:
\begin{CodeFrame*}[python]{}
def oblicz(a, x):
    p = 1
    w = a[0]
    for i in range(1, len(a)):
        p = p * x
        w += a[i] * p
    return w
\end{CodeFrame*}
Zauważ, że dodatkowo wyrzuciliśmy $a_0$ poza pętlę. W ten sposób potrzebujemy już tylko $n$ sumowań i $2n$ mnożeń (pomijając narzut związany z obsługą pętli itp).

Jeżeli chcemy jeszcze szybciej obliczyć wartość wielomianu w punkcie $x$ możemy skorzystać z \emph{schematu Hornera}, co pozwoli nam zredukować o połowę liczbę mnożeń.
Wymaga on przekształcenia postaci w której zapisujemy wielomian poprzez powyciąganie $x$-ów przed nawias:
$$a_0 + x(a_1 + x(a_2 + \ldots + x(a_{n-1} + xa_n)))$$
i rozpoczęcia obliczeń od $a_n$.

\begin{CodeFrame*}[python]{}
def oblicz(a, x):
    w = 0
    for i in range(len(a)-1, 0, -1):
        w = x * (a[i] + w)
    w += a[0]
    return w
\end{CodeFrame*}

\subsection{Postać Newtona}

Wielomian możemy przedstawić także w innych postaciach. Jedną z takich postaci jest \emph{postać Newtona}:
$$w(x) = b_0 + b_1(x-x_0) + b_2(x-x_1)(x-x_0) + \ldots + b_n(x-x_{n-1})\ldots(x-x_0)$$
w której wielomian stopnia $n$ określony jest przez $n+1$ współczynników $b_i$ oraz $n$ punktów $x_i$.

\begin{Zadanie}{}{}
Napisz funkcję \Verb{oblicz_wart(xx, x, b)} obliczającą wartość wielomianu w postaci Newtona.
Funkcja powinna otrzymywać argumenty: \Verb{xx} - punkt w którym ma zostać obliczona wartość, \Verb{x} - lista współczynników $x_i$, \Verb{b} - lista współczynników $b_i$.\\
\\
Wskazówka: do testowania funkcji możesz użyć wielomianu $w(x)$ zdefiniowanego przez
\Verb{x = [-1, 1]} i \Verb{b = [2, 1, -3]}, który przyjmuje następujące wartości:
$w(-2) = -8$, $w(-1) = 2$, $w(0) = 6$, $w(1) = 4$, $w(2) = -4$.

\begin{rozwiazanie}{python}
def oblicz_wart(xx, x, b):
	xxx = 1
	wyn = b[0]
	
	for i in range(1, len(b)):
		xxx *= (xx-x[i-1])
		wyn += b[i] * xxx
	
	return wyn
\end{rozwiazanie}
\end{Zadanie}

\section{Interpolacja}

Interpolacja to zgadywanie wartości nieznanych na podstawie znanych. 
Dokładniej, jest to wyznaczanie przybliżonych wartości nieznanej funkcji, 
w oparciu o znane wartości w tzw. punktach węzłowych, które też znamy. 
Realizujemy  to wyznaczając tzw. funkcję interpolacyjną,  
która  w punktach węzłowych przyjmuje ustalone wartości. 
Innymi słowy, wybieramy sobie pewną postać funkcji (np. decydujemy, że nasza szukana 
funkcja interpolacyjna będzie wielomianem pewnego stopnia) 
i dopasowujemy ją tak, by przechodziła przez znane wartości w znanych punktach. 

\subsection{Interpolacja wielomianowa}

Bardzo często jako funkcje interpolacyjne stosowane są funkcje wielomianowe. W tym wypadku zadanie interpolacji to po  prostu zadanie znalezienia wielomianu przechodzącego przez dane punkty.
\begin{teacherOnly}
Warto zwrócić uwagę że:
\begin{itemize}
	\item w przypadku gdybyśmy wiedzieli, że przybliżana funkcja jest wielomianem $n$-tego stopnia oraz dysponowani $n+1$ punktami węzłowymi, to uzyskane rozwiązanie byłoby dokładne;
	\item parzystość stopnia wielomianu drastycznie wpływa na zachowanie uzyskanej funkcji (zwłaszcza poza zakresem dla którego mamy dane wejściowe)
\end{itemize}
\end{teacherOnly}

\subsubsection{Obliczanie współczynników interpolacyjnych}

Możliwe jest obliczanie współczynników $a_i$ interpolowanego wielomianu zapisanego w postaci potęgowej, jednak ze względów obliczeniowych bardziej praktyczne jest posłużenie się postacią Newtona.
Realizowane jest to przy pomocy następującego algorytmu:
\begin{teacherOnly}
Uwaga! ten algorytm możemy potraktować jako czarną skrzynkę, której uczymy się używać.
\end{teacherOnly}
\begin{CodeFrame*}[python]{}
j = 0
while j <= n:
	b[j] = f[j]
	j += 1

j = 1
while j <= n:
	k = n
	while k >= j:
		b[k] = (b[k] - b[k-1]) / (x[k] - x[k-j])
		k -= 1
	j += 1
\end{CodeFrame*}
gdzie \Verb{n} jest stopniem wielomianu interpolacyjnego (obliczanego dla $n+1$ punktów węzłowych), \Verb{x[i]} jest wartością $x$ dla $i$-tego punktu, a \Verb{f[i]} wartością funkcji w $i$-tym punkcie węzłowym.

\begin{Zadanie}{}{}
Napisz funkcję \Verb{oblicz_wsp(x, y)} obliczającą współczynniki wielomianu interpolacyjnego dla danego zbioru punktów węzłowych (określonych przez listy \Verb{x, y}).
Wykorzystaj ją do narysowania wykresu wielomianu interpolującego następującą funkcję:

\begin{center}\begin{tabular}{|c|c|}
\hline\hspace{.75cm}$x$\hspace{.75cm}~&\hspace{.75cm}$f(x)$\hspace{.75cm}~\\\hline 
1 &  8\\
3 & -5\\
6 &  3\\
7 &  9\\
\hline
\end{tabular}\end{center}

\begin{rozwiazanie}{python}
def oblicz_wsp(x, y):
	n = len(y)-1
	b = [0]*(n+1)
	
	j = 0
	while j <= n:
		b[j] = y[j]
		j += 1
	
	j = 1
	while j <= n:
		k = n
		while k >= j:
			b[k] = (b[k] - b[k-1]) / (x[k] - x[k-j])
			k -= 1
		j += 1
	
	return b

x = [1, 3, 6, 7]
y = [8, -5, 3, 9]
b = oblicz_wsp(x, y)

xx, yy = [], []
for w in range(-5, 13):
	xx.append(w)
	yy.append(oblicz_wart(w, x, b))

import matplotlib
matplotlib.use('Agg')
import matplotlib.pyplot as plt
plt.plot(x, y, ".r")
plt.savefig('wykres.png')
\end{rozwiazanie}
\end{Zadanie}

\subsection{Interpolacja trygonometryczna {\Symbola 🤔}}

Interpolacja wielomianowa, ze względu na swoją naturę, nie za bardzo sprawdza się w przypadku funkcji okresowych. Stosowana jest wtedy często interpolacja trygonometryczna, w której funkcja interpolująca ma postać:
$$f(x) = a_0 + \sum_{k=1}^m a_k \cos(kx) + \sum_{k=1}^m b_k \sin(kx)$$
W szczególnym przypadku nieparzystej liczby $n$ równoodległych\footnote{
	W przypadku punktów równoodległych rozwiązaniem jest dyskretna transformata Fouriera
} punktów $x_i = i \cdot \frac{2\pi}{n}$ ta funkcja jest określona przez następujące wzory:
\begin{itemize}
\item $m = \frac{n-1}{2}$
\item $a_0 = \frac{1}{n} \sum_{k=0}^{n-1} f(x _{k})$
\item $a_j = \frac{1}{n} \sum_{k=0}^{n-1} f(x _{k}) \cdot \cos(j \cdot x _{k} )$
\item $b_j = \frac{1}{n} \sum_{k=0}^{n-1} f(x _{k}) \cdot \sin(j \cdot x _{k} )$
\end{itemize}

\section{Całkowanie}

Najprostszą podejściem do numerycznego obliczania całki oznaczonej z danej funkcji byłoby liczenie sumy pól prostokątów pod wykresem funkcji dla odpowiednio małego boku $dx$ takiego prostokąta w współrzędnej X.
Za wysokość prostokąta możemy przyjmować wartość funkcji w początku, środku lub końcu danego przedziału.
Prostym usprawnieniem tej metody, nie wymagającym znacznie większej ilości obliczeń ani nie powodującej znacznego wzrostu skomplikowania programu, jest zastosowanie trapezów zamiast prostokątów.

\begin{Zadanie}{}{}
Napisz funkcję \Verb{calka(f, a, b, dx = 0.3)} obliczającą wartość całki określonej na odcinku od \Verb{a} do \Verb{b} z funkcji \Verb{f}, korzystającą z metody trapezów.\\
Wskazówka, zauważ że dodając do \Verb{a} kolejne wartości \Verb{dx} możemy przekroczyć \Verb{b}.

\begin{rozwiazanie}{python}
def calka(f, a, b, dx = 0.3):
	x = a
	fb = f(x)
	pole = 0
	while x < b:
		x += dx
		if x > b:
			dx -= x-b
			x = b
		
		fa, fb = fb, f(x)
		pole += (fa + .5 * (fb-fa)) * dx
	return pole

calka(f, 0, 1, 0.3)
\end{rozwiazanie}
\end{Zadanie}

\section{Różniczkowanie}

Najprostszą metodą obliczania (przybliżania) wartości pochodnej funkcji w danym punkcie jest obliczenie jej jako: $f'_x=  \frac{f(x+dx) - f(x)}{dx}$, gdzie $dx$ jest odpowiednio małą odległością pomiędzy dwoma punktami.

\begin{Zadanie}{}{}
Napisz funkcję obliczającą pochodną funkcji $sin(x)$, w oparciu o powyższą zależność. Funkcja powinna posiadać argumenty pozwalające na wskazanie punktu $x$ oraz wartości $dx$ używanej do obliczeń. Zobacz jak obliczane przybliżenie pochodnej zalezy od wartości $dx$.

\begin{rozwiazanie}{python}
import math

def pochodnaSin(x, dx):
	return (math.sin(x+dx) - math.sin(x)) / dx

for d in range(1,10):
	print(pochodnaSin(0.5, 0.1/d))
\end{rozwiazanie}
\end{Zadanie}

Główny problem takiego podejścia polega na znalezieniu odpowiednio małego $dx$. Można go jednak ominąć traktując $f'_x$ jako funkcję zależną od $dx$, obliczając jej wartości dla kilku $dx$ i interpolując jej wartość w zerze.

\begin{Zadanie}{}{pochodna_funkcj_z_interpolacja}
Napisz funkcję \Verb{pochodna(f, x, d = 0.3)} obliczającą wartość pochodnej funkcji $f(x)$ (przekazanej w argumencie \Verb{f}) w zadanym punkcie \Verb{x}.
Funkcja powinna obliczać pochodną interpolując wartości $f'_x(dx)$ obliczonej dla $dx = -2d$, $dx = -d$, $dx = d$ i $dx = 2d$.
\end{Zadanie}

\section{Regulator PID {\Symbola 🤔}}

Regulator PID jest to algorytm regulacji parametru procesu działający w pętli sprzężenia zwrotnego posiadający człony: proporcjonalny (P), całkujący (I) i różniczkujący (D).

Wejściem algorytmu jest wartość mierzona (bądź od razu różnica wartości zadanej i mierzonej). Jeżeli kierunek zmiany sterowania jest zgodny ze zmianą wartości mierzonej (zwiększenie wartości sygnału sterującego powoduje zwiększenie wartości mierzonej) to należy odejmować wartość mierzoną od zadanej, w przeciwnym razie od zadanej mierzoną.

Wyjściem algorytmu jest wartość zmiany jakiegoś sygnału sterującego - może być wykorzystana bezpośrednio przy sterowaniu krokowym lub akumulowana celem uzyskania stałej wartości sygnału sterującego.

Typowa algorytm ten działa w nieskończonej pętli, często z dodatkowym krokiem czasowym (odstępem pomiędzy wykonaniami kolejnych obiegów pętli) - deklarowanym jawnie lub wynikłym z kostrukcji układu sterującego.

Przykładowa implementacja algorytmu:
\begin{CodeFrame*}[python]{}
class PID:
  # nastawa - wartość zadana
  setPoint = 0
  
  # wartość wyjścia dla sterowania krokowego
  # (gdy akumulacja w układzie realizującym)
  outStep = 0
  
  # wartość wyjścia dla sterowania sygnałem
  outValue = 0
  
  # parametry regulatora PID
  Kp, Ki, Kd = 0, 0, 0
  
  # limity wartości sterowanej
  outValueMin, outValueMax = 0, 0
  
  # poprzednia różnica między wartością zadaną a otrzymaną
  # (poprzedni błąd regulacji / uchyb)
  lastDiff = 0
  
  # poprzednia wartość otrzymana (zmienna procesu)
  lastVal = 0
  
  # część całkująca, akumulowana pomiędzy krokami
  integral = 0
  
  def doStep(self, inputVal, stepTime):
    # obliczmy aktualny błąd regulacji
    # (na podstawie odczytanej wartości wejściowej)
    diff = self.setPoint - inputVal
    
    # wyłączamy regulację gdy prowadziłby do przesterowania
    if (self.outValue > self.outValueMax and diff > 0):
      return 1
    if (self.outValue < self.outValueMin and diff < 0):
      return -1
    
    # całkowanie przybliżamy jako jako suma pól trapezów
    self.integral += (diff + self.lastDiff) / 2 * stepTime
    
    # różniczkowanie przybliżamy jako tangens nachylenia
    # prostej pomiędzy poprzednim krokiem a obecnym
    # celem złagodzenia odpowiedzi na zmiany wartości zadanej
    # różniczkujemy sygnał wejściowy a nie błąd regulacji
    derivative = -(inputVal - self.lastVal) / stepTime
    
    # obliczenie wartości sygnału sterującego na podstawie tego kroku
    self.outStep   = self.Kp*diff + self.Ki*self.integral + \
                     self.Kd*derivative
    
    # akumulacja sygnału sterującego
    self.outValue += self.outStep
    
    # zapamiętanie aktualnego błędu regulacji
    # jako poprzedni dla następnego kroku
    self.lastDiff = diff
    
    # zapamiętanie aktualnej wartości wejściowej
    # jako poprzedniej dla następnego kroku
    self.lastVal  = inputVal
    
    return 0
\end{CodeFrame*}

Implementacja ta wymaga ustawienia parametrów pracy algorytmu takich jak \Verb{setPoint}, \Verb{outValueMin}, \Verb{outValueMax}, współczynniki \Verb{Kp}, \Verb{Ki}, \Verb{Kd}.
Przydatne może być też zainicjowanie \Verb{lastVal} na obecny stan wejścia. Następnie działanie odbywa się w pętli:
\begin{enumerate}
\item odczyt wejścia
\item obliczenie wartości sterującej z użyciem PID (wywołanie metody \Verb{doStep})
\item wystawienie wartości sterującej
\item opcjonalne odczekanie jakiegoś czasu
\end{enumerate}

Do testowania algorytmu posłuży nam prosty model zbiornika, w którym chcemy utrzymać zadany poziom wody, a z którego cały czas odpływa jakaś jej ilość:
\begin{CodeFrame*}[python]{}
poziom = 15
wyplyw = 2

def getInput():
	global poziom
	return poziom;

def setOutput(doplyw):
	global poziom
	poziom -= wyplyw
	
	if poziom < 0:
		poziom = 0
	
	if doplyw > 100:
		doplyw = 100
	if doplyw < 0:
		doplyw = 0
	
	poziom += 0.3 * doplyw
\end{CodeFrame*}

\begin{Zadanie}{}{testuj_pid}
Napisz funkcję \Verb{testuj(steps, kp, ki, kd)} testującą działanie algorytmu PID. Funkcja powinna wykonać \Verb{steps} iteracji algorytmu dla zadanych parametrów \Verb{kp}, \Verb{ki}, \Verb{kd} i narysować wykres zmian wartości poziomu wody w zbiorniku (zwracanej przez \Verb{getInput()} w danym kroku) oraz wartości sterującej zaworem dopływowym (przekazywanej do \Verb{setOutput()} w danym kroku).
\end{Zadanie}

\begin{Zadanie}{}{}
Korzystając z funkcji \Verb{testuj(steps, kp, ki, kd)} poeksperymentuj z doborem wartości współczynników algorytmu.

\begin{teacherOnly}
Pokazać róznicę pomiędzy:
\begin{CodeFrame*}[python]{}
testuj(200, 0.01, 0.00, 0.00)
testuj(200, 0.1, 0.00, 0.00)
testuj(200, 0.3, 0.00, 0.00)
\end{CodeFrame*}
i omówić znaczenie członu proporcjonalnego (wskazać kiedy algorytm zaczyna zamykanie dolewania wody oraz jaki jest skutek trwania tego procesu)

Wspomnieć o znaczeniu kroku czasowego algorytmu w stosunku co do szybkości procesu.

Pokazać róznicę pomiędzy:
\begin{CodeFrame*}[python]{}
testuj(200, 0.3, 0.00, 0.01)
testuj(200, 0.3, 0.00, 0.1)
\end{CodeFrame*}
i omówić znaczenie oraz wpływ członu różniczkującego (teraz prędkość sterowania zaworem dopływu zależy od prędkości zmiany poziomu wody w zbiorniku)

Można też zasugerować zmodyfikowanie zachowania zbiornika w taki sposób aby od pewnego kroku zmianie uległa ilość wypływającej wody.
\end{teacherOnly}
\end{Zadanie}

\section{Zadania dodatkowe}

\begin{Zadanie}{}{}
Napisz funkcję \Verb{oblicz_wart_tryg (xx, a, b)} obliczającą wartość interpolacji trygonometrycznej w punkcie \Verb{xx}, dla podanych list współczynników \Verb{a} i \Verb{b}.
\end{Zadanie}

\begin{Zadanie}{}{}
Napisz funkcję \Verb{oblicz_wsp_tryg (y)} obliczającą i zwracającą współczynniki \Verb{a} i \Verb{b} interpolacji trygonometrycznej  dla podanego w skrypcie przypadku.\\
Wskazówka: zauważ że nie obliczany i nie wykorzystywany jest współczynnik \Verb{b[0]}
\end{Zadanie}

\begin{Zadanie}{\domowe{1}}{calka_z_prostokatow}
Napisz funkcje obliczającą całkę jako sumę pól prostokątów o wysokości określonej przez prawy koniec przedziału o długości dx.
\end{Zadanie}

\begin{Zadanie}{\domowe{2}}{interpolacja_zbioru_punktow}
Napisz program dokonujący interpolacji następującego zbioru punktów:
\begin{center}\begin{tabular}{|c|c|}
\hline\hspace{.75cm}$x$\hspace{.75cm}~&\hspace{.75cm}$f(x)$\hspace{.75cm}~\\\hline 
-3 & 58 \\
-2 & 39 \\
-1 & -4 \\
1  & 6 \\
2  & -37 \\
3  & -56 \\
\hline
\end{tabular}\end{center}
Narysuj wykres funkcji interpolacyjnej w przedziale [-3.7 3.7] oraz podaj wzór wielomianu interpolacyjnego w postaci potęgowej.
\end{Zadanie}

\begin{Zadanie}{\domowe{3}}{interpolacja_cos}
Napisz program który będzie dokonywał interpolacji wielomianowej funkcji $cos(x)$ opierając się kolejno na: 3, 6, 13 i 16 wybranych punktach z przedziału $[0, 4\pi]$
Program powinien narysować wykresy kolejnych funkcji interpolującyh oraz wykres $cos(x)$ na wspólnym wykresie.
\end{Zadanie}


\begin{Zadanie}{\domowe{1}}{pochodna_dwustronna}
Na zajęciach pisaliśmy funckję przybliżającą wartoć pochodnej w oparciu o zależność $f'_x=  \frac{f(x+dx) - f(x)}{dx}$, innym podejściem jest obliczanie jej jako $f'_x=  \frac{f(x+dx) - f(x-dx)}{2 \cdot dx}$.
Napisz funkcją obliczającą przybliżenie pochodnej funkcji $cos(x)$ obliczane jako $f'_x=  \frac{f(x+dx) - f(x-dx)}{2 \cdot dx}$. Która z tych metod wydaje się lepsza.
\end{Zadanie}

\begin{Zadanie}{\domowe{2}}{wykres_pochodnej}
Napisz program która oblicza numerycznie wartość pochodnej danej funkcji (np. $cos(x)$) w danym przedziale (z jakimś ustalonym krokiem) i rysuje wykres funcji oraz tej pochodnej w zadanym przedziale.
Do obliczania pochodnej możesz użyć dowolnej metody omawianej na zajęciach.
\end{Zadanie}

\begin{Zadanie}{\domowe{3}}{wykres_sin_ze_styczna}
Narysuj wykres funkcji $sin(x)$ dla x z przedziału [1,2] oraz styczne w punkcie $x=1.4$ obliczone jako krzywe o nacyleniu odpowiadającemu przybliżeniu pochodnej obliczonemu jako: $f'_x=  \frac{f(x+dx) - f(x)}{dx}$ i jako $f'_x=  \frac{f(x+dx) - f(x-dx)}{2 \cdot dx}$ dla dx=0.13.
\end{Zadanie}

\rozwiazania

\ifdefined\inputOnlyContent\else
\copyrightFooter{
	© Matematyka dla Ciekawych Świata, 2018.\\
	© Robert Ryszard Paciorek <rrp@opcode.eu.org>, 2003-2018.\\
	Kopiowanie, modyfikowanie i redystrybucja dozwolone pod warunkiem zachowania informacji o autorach.
}
\end{document}
\fi
