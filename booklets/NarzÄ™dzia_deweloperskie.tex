% Copyright (c) 2018-2019 Robert Ryszard Paciorek <rrp@opcode.eu.org>
% 
% MIT License
% 
% Permission is hereby granted, free of charge, to any person obtaining a copy
% of this software and associated documentation files (the "Software"), to deal
% in the Software without restriction, including without limitation the rights
% to use, copy, modify, merge, publish, distribute, sublicense, and/or sell
% copies of the Software, and to permit persons to whom the Software is
% furnished to do so, subject to the following conditions:
% 
% The above copyright notice and this permission notice shall be included in all
% copies or substantial portions of the Software.
% 
% THE SOFTWARE IS PROVIDED "AS IS", WITHOUT WARRANTY OF ANY KIND, EXPRESS OR
% IMPLIED, INCLUDING BUT NOT LIMITED TO THE WARRANTIES OF MERCHANTABILITY,
% FITNESS FOR A PARTICULAR PURPOSE AND NONINFRINGEMENT. IN NO EVENT SHALL THE
% AUTHORS OR COPYRIGHT HOLDERS BE LIABLE FOR ANY CLAIM, DAMAGES OR OTHER
% LIABILITY, WHETHER IN AN ACTION OF CONTRACT, TORT OR OTHERWISE, ARISING FROM,
% OUT OF OR IN CONNECTION WITH THE SOFTWARE OR THE USE OR OTHER DEALINGS IN THE
% SOFTWARE.

\ifdefined\inputOnlyContent\else
\documentclass{pdfBooklets}

\title{Narzędzia deweloperskie: git, make i cmake, pip i virtualenv}
\author{%
	%Projekt ,,Matematyka dla Ciekawych Świata'',\\
	Robert Ryszard Paciorek\\\normalsize\ttfamily <rrp@opcode.eu.org>
}
\date  {2020-03-30}

\makeatletter\hypersetup{
	pdftitle = {\@title}, pdfauthor = {\@author}
}\makeatother


\begin{document}

\maketitle
\fi

\section{Systemem kontroli wersji}

Podstawowym zadaniem systemów kontroli wersji jest przechowywanie historii zmian w plikach (np. źródłach jakiegoś projektu, plikach konfiguracyjnych, itp) oraz umożliwianie śledzenia tych zmian (porównywanie wersji, uzyskiwanie informacji o autorach i datach zmian, itd).

W systemach rozproszonych wszystkie kopie repozytoriów są równoprawne (zawierają pełną historię, do każdego z nich mogą być commitowane zmiany, itd), w systemach scentralizowanych występuje pojedynczy serwer do którego commitowane są zmiany i z którego są one pobierane (także operacje związane z historią odwołują się do serwera, a nie lokalnej kopii).

Najpopularniejszymi systemami kontroli wersji są: systemy rozproszone Git (polecenie \Verb{git}), Mercurial (polecenie \Verb{hg}) i Bazaar (polecenie \Verb{bzr}) oraz scentralizowany system kontroli wersji Subversion (polecenie \Verb{svn}). Nadal można spotkać używane repozytoria \Verb{cvs}.

\subsection{Git}

\subsubsection{Inicjalizacja repozytorium}

Mamy dwie podstawowe metody utworzenia repozytorium na którym będziemy pracowali:
\begin{itemize}
	\item utworzenie nowego, pustego repozytorium: \Verb{git init}
	\item pobranie repozytorium z zewnętrznego źródła \Verb{git clone adres}
\end{itemize}
Adres repozytorium może być m.in. postaci: \Verb{https://[login@]serwer/sciezka/} lub\\ \Verb{ssh://login@serwer/sciezka/}.

\subsubsection{Kopia robocza i poczekalnia}

\begin{itemize}
	\item aktualizacja kopii roboczej: \Verb{git checkout [opcje] [sciezka]}\\
		szczególnie przydatne jest wywołanie \Verb{git checkout -f .} umożliwiająca odrzucenie
		istniejących zmian w kopii roboczej (reset kopii roboczej do stanu repozytorium)
	
	\vspace{6pt}
	
	\item dodawanie plików/zmian (do poczekalni): \Verb{git add sciezka}
	\item usuwanie zmian (z poczekalni): \Verb{git reset HEAD sciezka}
	\item usuwanie pliku: \Verb{git rm sciezka}\\
		(jest to równoważne z \Verb{rm sciezka && git add sciezka})
	\item zmiana nazwy (położenia) pliku: \Verb{git mv staraSciezka nowaSciezka}\\
		(jest to równoważne z \Verb{git rm staraSciezka; git add nowaSciezka})
	
	\vspace{6pt}
	
	\item podgląd zmian kopii roboczej i poczekalni: \Verb{git status}\\
		(aby ignorować jakieś niedodane do repozytorium pliki katalogi należy skorzystać z pliku \Verb{.gitignore})
	\item różnica kopii roboczej do poczekalni: \Verb{git diff [sciezka]}
	\item różnica kopii roboczej do wskazanej rewizji: \Verb{git diff rewizja [sciezka]}
	\item różnica poczekalni do ostatniej / wskazanej rewizji: \Verb{git diff [rewizja] --cached [sciezka]}
	
	\vspace{6pt}
	
	\item zatwierdzanie zmian z poczekalni (tworzenie rewizji): \Verb{git commit [opcje] [sciezka]}, przydatne opcje:\\
		\Verb{-a} powoduje automatyczne dodanie zmian w śledzonych plikach\\
		\Verb{-m "OPIS"} pozwala na podanie opisu zmian z linii poleceń\\
		\Verb{--amend} pozwala na poprawienie ostatniego commitu
\end{itemize}

\subsubsection{Historia zmian}

\begin{itemize}
	\item przeglądanie historii zmian: \Verb{git log [opcje] [sciezka]}, przydatne opcje:\\
		\Verb{-p} pokazuje w logu zmian diff pomiędzy rewizjami\\
		\Verb{--name-status} pokazuje nazwy i status (dodanie, usunięcie, modyfikacja) zmienianych plików\\
		\Verb{--graph} pokazuje wykres gałęzi
	\item listowanie pliku z danej rewizji: \Verb{git show rewizja:sciezka}
	\item listowanie struktury repozytorium z danej rewizji: \Verb{git ls-tree rewizja}
\end{itemize}

\subsubsection{Gałęzie}

\begin{itemize}
	\item tworzenie nowej gałęzi: \Verb{git branch nazwa}
	\item scalanie wskazanej gałęzi do aktualnej:\Verb{git merge [zasob_zdalny] nazwa}
	
	\vspace{6pt}
	
	\item utworzenie gałęzi po stronie zdalnej: \Verb{git push --set-upstream origin nazwa}
	\item pobranie / aktualizacja informacji (m.in.) o zdalnych gałęziach: \Verb{git fetch}
	\item listowanie gałęzi: \Verb{git branch -av}
	
	\vspace{6pt}
	
	\item przełączanie się pomiędzy gałęziami (zmiany w kopii roboczej nie zostanę utracone, w przypadku konfliktu przed przełączeniem będzie konieczność ich zaakceptowania lub odrzucenia):\\ \Verb{git checkout nazwa}
\end{itemize}

\subsubsection{Zasoby zdalne}

\begin{itemize}
	\item listowanie zasobów zdalnych: \Verb{git remote -v}
	\item dodawanie nowego zasobu zdalnego: \Verb{git remote add nazwa URL}
	
	\vspace{6pt}
	
	\item wysyłanie zmian do (wskazanego) zasobu zdalnego: \Verb{git push [nazwa]}
	\item pobieranie zmian z (wskazanego) zasobu zdalnego: \Verb{git pull [nazwa]}
\end{itemize}

\subsection{Zadania}

\begin{Zadanie}{}{git_init}
\begin{enumerate}
\item utwórz katalog \Verb#repo1# zawierający dwa pliki tekstowe z dowolną zawartością (różną)
\item utwórz w tym katalogu repozytorium git i sprawdź jego stan (status)
\item dodaj do niego wszystkie pliki z tego katalogu i sprawdź stan repozytorium
\item zakomituj zmiany i sprawdź stan repozytorium
\end{enumerate}
\end{Zadanie}

\begin{Zadanie}{}{git_modyfikacja}
Działając w \Verb#repo1# utworzonym w zadaniu \ref{git_init}
\begin{enumerate}
\item zmodyfikuj zawartość jednego z plików
\item sprawdź stan repozytorium
\item zmodyfikuj zawartość drugiego z plików
% \item wyświetl różnicę pomiędzy zawartośią kopii roboczej a zawartością repozytorium
\item wyświetl tę różnicę tylko dla jednego z plików
\item zakomituj zmiany
\item wyświetl historię zmian w repozytorium
\end{enumerate}
\end{Zadanie}

\begin{Zadanie}{}{git_clone}
Sklonuj repozytorium \Verb#repo1# jako \Verb#repo2#
\end{Zadanie}

\begin{Zadanie}{}{git_rm}
Działając w \Verb#repo2# utworzonym w zadaniu \ref{git_clone}
\begin{enumerate}
\item usuń z repozytorium jeden z plików tekstowych
\item zmień nazwę drugiego z plików tekstowych
\item sprawdź stan repozytorium
\item zakomituj zmiany
\item wyświetl historię zmian w repozytorium, z pokazaniem informacji o nazwach zmienianych plików
\end{enumerate}
\end{Zadanie}

\begin{Zadanie}{}{git_pull1}
Zaktualizuj stan \Verb#repo1#, tak aby odzwierciedlał zmiany dokonane w \Verb#repo2#
\end{Zadanie}

\begin{Zadanie}{}{git_merge}
\begin{enumerate}
\item zmodyfikuj plik tekstowy zarówno w \Verb#repo1#, jak i w \Verb#repo2# (w różny - sprzeczny sposób)
\item zakomituj zmiany w \Verb#repo1# i w \Verb#repo1#
\item spróbuj zaktualizować stan \Verb#repo1#, tak aby odzwierciedlał zmiany dokonane w \Verb#repo2#
\item sprawdź stan repozytorium
\item rozwiąż konflikt
\item sprawdź stan repozytorium
\item wyświetl historię zmian pokazującą wykres gałęzi w których były dokonywane zmiany
\item spróbuj zaktualizować stan \Verb#repo2#, tak aby odzwierciedlał zmiany dokonane w \Verb#repo1#
\end{enumerate}
\end{Zadanie}

\begin{Zadanie}{}{git_checkout}
\begin{enumerate}
\item przywrócić stan kopii roboczej do stanu repozytorium z przed zadania \ref{git_merge}
\item sprawdź stan repozytorium
\item wyświetl zawartość pliku modyfikowanego w zadaniu \ref{git_merge}
\item powróć do normalnego stanu repozytorium
\item sprawdź stan repozytorium
\end{enumerate}
\end{Zadanie}

\begin{Zadanie}{}{git_show}
Wyświetl zawartość pliku modyfikowanego w \ref{git_merge} przed dokonaniem tych modyfikacji bez przełączania kopii roboczej na ówczesny stan repozytorium.
\end{Zadanie}

\begin{Zadanie}{}{git_diff}
Wyświetl różnicę stanu obecnego pliku modyfikowanego w \ref{git_merge} i stanu przed dokonaniem tych modyfikacji
\end{Zadanie}

\begin{Zadanie}{}{git_branch}
\begin{enumerate}
\item utwórz nowy branch w \Verb#repo1# o nazwie \Verb#test# i przełącz się na niego
\item zmodyfikuj plik tekstowy w ramach tego brancha i zakomituj zmiany
\item wyświetl informacje na temat branchy
\item wróć do domyślnej gałęzi "master" i wyświetl zawartość modyfikowanego pliku oraz stan repozytorium
\end{enumerate}
\end{Zadanie}

\begin{Zadanie}{}{git_pull2}
\begin{enumerate}
\item zaciągnij zmiany dokonane w zadaniu \ref{git_branch} z \Verb#repo1# do \Verb#repo2# tak aby znalazły się one w brachu o nazwie test
\item wyświetl stan repozytorium, informacje na temat branchy oraz dwa najnowsze wpisy z historii repozytorium
\end{enumerate}
\end{Zadanie}

\section{make}

Make jest narzędziem służącym do automatyzowania kompilacji. Jego zadaniem jest ustalenie które pliki potrzebują kompilacji po zmianach w projekcie na podstawie czasu modyfikacji plików źródłowych, wynikowych oraz reguł zapisanych w pliku \Verb{Makefile}.

\subsection{Wywołanie polecenia make}

Wywołanie polecenia make ma postać \Verb{make [opcje] [akcja]}.

Wśród opcji warto zwrócić uwagę na:\\
\Verb{-j} pozwalającą określić liczbę równolegle uruchamianych procesów\\
\Verb{-i} powodującą ignorowanie błędów (normalnie make przerywa pracę gdy któreś z zadań nie powiodło się)\\
\Verb{-d} wypisywanie informacji dla debugowania.

Akcja określa regułę z pliku \Verb{Makefile} która ma zostać wykonana.


\subsection{Plik Makefile}

Plik składa się z reguł następującej postaci:
\begin{CodeFrame*}[make]{}
nazwa: zaleznosci
	polecenia
\end{CodeFrame*}

Pierwsza linia określa iż plik wynikowy (lub akcja) \Verb{nazwa} zależy od podanych po dwukropku plików.
Druga linia (i ewentualnie kolejne) zawierają polecenia służące do wygenerowania pliku \Verb{nazwa} lub realizacji tej akcji. Linie te muszą zaczynać się od znaku tabulacji.

\subsubsection{Proste przykłady}

\begin{CodeFrame}[make]{.5\textwidth}
all:
	echo "Hello world"
\end{CodeFrame}
\begin{minipage}[t]{.45\textwidth}
\vspace{6pt}\noindent Każde uruchomienie \Verb{make} lub \Verb{make all} spowoduje wypisanie "Hello world".
\end{minipage}

\begin{CodeFrame}[make]{.5\textwidth}
all: witaj.txt

co.txt:
	echo cosiu > co.txt

witaj.txt: co.txt
	echo -n "Witaj " > witaj.txt
	cat co.txt >> witaj.txt
\end{CodeFrame}
\begin{minipage}[t]{.45\textwidth}
\vspace{12pt}\noindent Uruchomienie \Verb{make} spowoduje nadpisanie zawartości pliku \Verb{witaj.txt} jeżeli plik \Verb{co.txt} jest od niego nowszy. W przypadku braku pliku \Verb{co.txt} zostanie on utworzony.
\end{minipage}

\subsubsection{Bardziej zaawansowane pliki Makefile}

Poniższy przykład ilustruje kilka rozwiązań stosowanych w plikach Makefile (dla GNU Make), takich jak ustawianie i korzystanie ze zmiennych, iterowanie po plikach, reguły generyczne:
\begin{CodeFrame*}[make]{}
# ustawiamy zmienną CONFDIR
CONFDIR=$(HOME)/.xyz/lib

# sprawdzamy czy podany plik/katalog istniej
ifeq ($(wildcard $(CONFDIR)),)
    # a jeżeli nie to zmieniamy wartość zmiennej CONFDIR
    CONFDIR=$(HOME)/.xyz-lib
endif

# ustawienie zmiennej w oparciu o standardowe wyjście innego programu
DATE=$(shell date --iso)

# wartość tej zmiennej może być nadpisana przez wywołanie:
# make CONFDIR=prawdziwa/sciezka

# target installConfig kopiuje wszystkie katalogi src-conf/* do $(CONFDIR)
# zauważ odwłonie do zmiennej makefile'owej CONFDIR vs bashowej inDir w pętli
installConfig:
    @ for inDir in src-conf/*; do \
        install -m 644 -Dt $(CONFDIR)/`basename $$inDir` $$inDir/*" ; \
    done

# target installGSym odnajduje ścieżkę z której będziemy kopiować i kopiuje ...
installGSym:
    # pobieramy output komendy do zmiennej makefilowej w ramach konkretnego targetu
    $(eval GEDASYSDIR := $(shell gschem -c '(display geda-data-path)(gschem-exit)'))
    # modyfikujemy zmienną, a jeżeli jest pusta to ustawiamy wartość domyślną
    $(eval GEDASYMDIR := $(if $(GEDASYSDIR), "$(GEDASYSDIR)/sym", "extra/sym"))
    install -m 644 -Dt $(CONFDIR)/sym $(GEDASYMDIR)/*"

# uzyskanie pliku xyz.o wymaga pliku xyz.c i wykonania komendy:
#   $(CC) -c -o xyz.o xyz.c $(CFLAGS)
# dla dowolnego xyz ...
%.o: %.c %.h
    $(CC) -c -o $@ $< $(CFLAGS)
# % po lewej stronie : zastępuje dowolny ciąg znaków
#   (podobnie jak * w shellu, ale objemuje też ściezki z katalogami)
# % po prawej stronie : oznacza podstawienie ciągu dopasowanego do 
#   znaku % po lewej stronie :
# pod $@ podstawiane jest to co zostało dopasowane do
#   całości napisu po lewj stronie :
# pod $< podstawiany jest pierwszy element z listy zalezności
#   (tego co po prawej stronie :)
# pod $^ podstawiana jest całość prawej strony od :
#   (co jest przydatne np. przy linkowaniu)
# standardowo $(CC) zawiera ścieżkę do komilatora C,
#   a $(CFLAGS) zawiera flagi kompilacji
\end{CodeFrame*}

\subsection{Zadania}

\begin{Zadanie}{}{makefile1}
Poniższe polecenia utworzą prosty program w C:
\begin{Verbatim}
echo 'void wypisz();' > wypisz.h
echo '#include <stdio.h>' > wypisz.c
echo '#include "wypisz.h"' >> wypisz.c
echo 'void wypisz() { printf("Hello world!\n"); }' >> wypisz.c
echo '#include "wypisz.h"' > main.c
echo 'int main() { wypisz(); return 0; }' >> main.c
\end{Verbatim}

Może on zostać skompilowany przy pomocy poleceń:
\begin{Verbatim}
gcc -o wypisz.o -c wypisz.c
gcc -o main.o -c main.c
gcc -o hello main.o wypisz.o
\end{Verbatim}

Napisz plik \Verb#Makefile#, który zautomatyzuje kompilację tego projektu. Plik powinien uwzględniać zależności pomiędzy elementami projektu (plik hello wymaga dwóch plików .o, pliki .o wymagają odpowiednich plików .c i .h). Plik powinien pozawalać na określenie używanego kompilatora przy pomocy zmiennej środowiskowej.
\end{Zadanie}

\begin{Zadanie}{}{makefile2}
Dodaj do pliku \Verb#Makefile# z zadania \ref{makefile1} regułę \Verb#clean#, która usunie wszystkie utworzone w wyniku kompilacji pliku. Reguła powinna działać nawet jeżeli w katalogu znajdzie się plik o nazwie \Verb#clean#.
\end{Zadanie}

\section{Generatory plików Makefile}

W przypadku większych projektów często stosowane są narzędzia służące do automatycznego generowania plików Makefile (a często także plików nagłówkowych, itp) w oparciu o dostępne biblioteki i automatyczną detekcję zależności pomiędzy plikami. 

Wywołanie komendy \Verb{make} często musi zostać poprzedzane wywołaniem skryptu \Verb{./configure}. Niekiedy także ten skrypt musi zostać poprzedzony wywołaniem polecenia \Verb{autoreconf} (z pakietu autoconf/automake).

Aktualnie wiele projektów korzysta do konfiguracji kompilacji (i utworzenia plików Makefile) z \Verb{cmake}.

\subsection{cmake}

Cmake jest narzędziem do zarządzania procesem kompilacji. Jest on niezależny od używanej platformy, kompilatora a także narzędzia automatyzacji kompilacji (oprócz omawianego wcześniej \Verb{make} wspiera także kilka innych tego typu narzędzi).

Głównym plikiem konfiguracyjnym programu \Verb{cmake} zawierającym reguły budowania danego oprogramowania jest \Verb{CMakeLists.txt}.

\subsubsection{Wywołanie polecenia cmake}

Wywołanie polecenia cmake ma postać \Verb{cmake [opcje] sciezka}, gdzie \Verb{cmake sciezka} wskazuje katalog zawierający plik \Verb{CMakeLists.txt} (typowo główny katalog ze źródłami projektu).

Konstrukcja taka pozwala na łatwe budowanie projektu w innym katalogu niż ten zawierający źródła\footnote{ pozwala to uniknąć mieszania źródeł z plikami .o, generowanymi plikami Makefile, itp} np.:
\begin{Verbatim}
cd sciezka/do/projektu
mkdir build
cd build
cmake ..
\end{Verbatim}

Wśród opcji należy zwrócić szczególną uwagą na opcję \Verb{-D} pozwalającą na ustawianie flag budowania i kompilacji. Warto także zrócić uwagę na program \Verb{ccmake} pozwalający na przeglądanie i modyfikowanie ustawień kompilacji.

\subsubsection{Plik CMakeLists.txt}

W pliku tym umieszczane są m.in komendy:
\begin{itemize}
	\item ustawiające zmienne cmake'owe, które mogą być użyte np. do generowania \Verb{#define} w plikach nagłówkowych czy też if'owania fragmentów pliku CMakeLists.txt (np. w celu wymagania lub nie danych bibliotek) -- \Verb{set()} i polecenia \Verb{option()}
	\item operujące na listach (np. plików) i plikach -- \Verb{list()} i \Verb{file()}
	\vspace{6pt}
	\item ustawiające opcje kompilacji -- \Verb{add_compile_options()}
	\item dodające katalogi z plikami nagłówkowymi -- \Verb{include_directories()}
	\item dodające linkowane biblioteki -- \Verb{link_libraries()}
	\vspace{6pt}
	\item dodające cele budowania (akcje w Makefile) związane z kompilacją i linkowaniem -- \Verb{add_executable()}, \Verb{add_library()}
	\item dodające cele budowania (akcje w Makefile) -- \Verb{add_custom_command()} \Verb{add_custom_target()}
	\item dodające biblioteki linkowane tylko z wskazanym celem budowania -- \Verb{target_link_libraries()}
	\item ustalające zależności pomiędzy celami -- polecenia \Verb{add_dependencies()}
	\vspace{6pt}
	\item realizujące pętle, sprawdzające warunki, definiujące funkcje czy też wypisujące komunikaty -- polecenia takie jak , \Verb{foreach()}, \Verb{if()}, \Verb{macro()}, \Verb{message()}
\end{itemize}

\noindent Przykład:
\begin{CodeFrame*}[cmake]{}
# wymagamy minimalnej wersji cmake
cmake_minimum_required (VERSION 3.0)

# określamy nazwę naszego projektu i język/kompilator
project("PoznajemyCMAKE" CXX)

option(USE_XLIB "Use X11 (if possible ...)" ON)

# ustawiamy flagi kompilatora
add_compile_options("-Wall")

# jeżeli działamy na platformie unix'owej ...
if(UNIX)
    # dodajemy opcję linkowania z biblioteką matematyczną
    # będzie ona dodana do wszystkich wywołań linkera
    link_libraries(m)
    
    # wyszukujemy bibliotekę X11
    if(${USE_XLIB})
        find_package(X11)
        
        # jeżeli nie ma to przełączamy USE_XLIB na OFF
        if(NOT ${X11_FOUND})
            set(USE_XLIB OFF)
        # a jeżeli jest to dodajemy do listy Libraries
        else()
            list(APPEND Libraries ${X11_X11_LIB})
        endif()
    endif()
endif(UNIX)

# tworzymy config.h w oparciu o szablon
# wpis #cmakedefine USE_XLIB zostanie zamieniony na #define USE_XLIB
# jeżeli USE_XLIB jest ustawione
configure_file (
    "${PROJECT_SOURCE_DIR}/config.h.in"
    "${PROJECT_BINARY_DIR}/config.h"
)

# wyszukujemy rekurencyjnie pliki z rozszerzeniem .cpp
# w pod katalogu src katalogu źródłowego projektu
file(GLOB_RECURSE Sources "${PROJECT_SOURCE_DIR}/src/*.cpp")

# wypisujemy co znaleźliśmy - debug output ;-)
message(STATUS "Sources:")
foreach(ff ${Sources})
    message(STATUS "  ${ff}")
endforeach()

# ustawiamy że katalogiem bazowm dla plików nagłówkowych
# (podawnym w opcji -I gcc/clang) będzie podkatalog src naszego projektu
include_directories("${PROJECT_SOURCE_DIR}/src")
# typowo robimy to także dla znalezionych nagłówków bibliotek ...

# dodajemy target w postaci tworzenia pliku wykonywalnego o nazwie Run
# powstałego z kompilacji i linkowania źródęł z listy Sources
add_executable(Run ${Sources})

# ustawiamy dodatkowe biblioteki z którymi będzie linkowany plik
# wykonywalny Run w oparciu o listę Libraries
target_link_libraries(Run ${Libraries})
\end{CodeFrame*}


\subsection{SCons}

SCons jest alternatywnym rozwiązaniem wobec zestawu \Verb{cmake} + \Verb{make},
pozwalającym na tworzenie plików zarządzających procesem budowania w postaci skryptów Pythona. 


\section{Narzędzia pythonowe}

\subsection{PIP}

Pip jest narzędziem służącym do instalowania modułów pythonowych z dedykowanego repozytorium (managerem/instalatorem pakietów pythonowych). W przypadku Pythona w wersji 3 należy wywoływać go poleceniem \Verb{pip3}.
Po nazwie polecenia podaje się komendę określającą działania które ma wykonać manager, do najistotniejszych należy zaliczyć:\\
\begin{itemize}
	\item \texttt{install \textit{nazwa}} instaluje podany pakiet
	\item \texttt{install -U \textit{nazwa}} aktualizuje podany pakiet
	\item \texttt{uninstall \textit{nazwa}} usuwa podany pakiet
	\item \texttt{search \textit{słowo}} wyszukuje pakiety w oparciu o podane słowo kluczowe
	\item \texttt{list -v} wypisuje zainstalowane pakiety wraz z informacją o katalogu w którym zostały zainstalowane oraz czy pakiet był instalowany przy pomocy pip, czy w inny sposób (np. przez systemowego menagera pakietów)
\end{itemize}

Pip automatycznie instaluje zależności pakietu, w niektórych przypadkach może to wymagać zainstalowanych narzędzi i odpowiednich bibliotek C i/lub C++.

\begin{Zadanie}{}{}
Zainstaluj (lub zaktualizuj jeżeli jest obecnie zainstalowany) przy pomocy pip pakiet virtualenv
\end{Zadanie}

\subsection{Virtualenv}

Virtualenv jest narzędziem pozwalającym na tworzenie środowisk pythonowych udostępniających różne wersje modułów. W celu utworzenia nowego środowiska wywołuje się go poleceniem \Verb{virtualenv ścieżka}, gdzie \Verb{ścieżka} wskazuje na miejsce w którym ma zostać utworzone środowisko. Po utworzeniu środowiska można doinstalować wewnątrz niego wymagane wersje bibliotek przy pomocy komendy pip dostępnej w pod-katalogu \Verb{bin} katalogu utworzonego środowiska. Do wywołania programów pythonowych z użyciem danego środowiska należy korzystać z pliku wykonywalnego python3 w pod-katalogu \Verb{bin} danego środowiska. Można także skorzystać z pliku \Verb{bin/activate} w katalogu utworzonego środowiska, poprzez wczytanie go w bierzącej powłoce basha (\Verb{. scizeka_do_enva/bin/activate}) celem uzyskania powłoki w której domyślnym interpreterem pythona będzie ten z utworzonego środowiska ,,virtualenv''. Zmiany w środowisku wprowadzone w wyniku wczytania pliku \Verb{bin/activate} obowiązują tylko w bierzącej powłoce, można je także odwrócić przy pomocy polecenia \Verb{deactivate}.

\begin{CodeFrame*}[bash]{}
# tworzenie nowego środowiska
virtualenv /tmp/py1
# użycie pip wewnątrz środowiska
/tmp/py1/bin/pip3 list
# wywołanie pytana wewnątrz środowiska
/tmp/py1/bin/python3

# "przełączenie" na pythona z środowiska
. /tmp/py1/bin/activate
# teraz aby użyć pip lub wewnątrz środowiska wystarczy uruchomić
pip3 list
python3
\end{CodeFrame*}

\begin{ProTip}{Informacja {\Symbola 🤔}}
Virtualenv nie jest chroot'em - Python z tak utworzonego środowiska ,,widzi'' resztę systemu, a także ,,wie'' w jakiej ścieżce jest zainstalowany (zobacz zawartość zmiennej \Verb{sys.path}).
\end{ProTip}

\begin{Zadanie}{}{}
Utwórz środowisko pythonowe typu ,,virtualenv'' w katalogu ytdl i zainstaluj wewnątrz niego pakiet pythonowy youtube-dl.
\end{Zadanie}

\ifdefined\inputOnlyContent\else
\rozwiazania

\copyrightFooter{
	© Robert Paciorek, 2015-2020.\\
	Kopiowanie, modyfikowanie i redystrybucja dozwolone pod warunkiem zachowania informacji o autorach.
}
\end{document}
\fi
