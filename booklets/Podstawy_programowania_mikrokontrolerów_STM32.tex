% Copyright (c) 2017-2019 Matematyka dla Ciekawych Świata (http://ciekawi.icm.edu.pl/)
% Copyright (c) 2017-2019 Robert Ryszard Paciorek <rrp@opcode.eu.org>
% Copyright (c) 2017-2019 Krzysztof Lasocki <krz.lasocki@gmail.com>
% 
% MIT License
% 
% Permission is hereby granted, free of charge, to any person obtaining a copy
% of this software and associated documentation files (the "Software"), to deal
% in the Software without restriction, including without limitation the rights
% to use, copy, modify, merge, publish, distribute, sublicense, and/or sell
% copies of the Software, and to permit persons to whom the Software is
% furnished to do so, subject to the following conditions:
% 
% The above copyright notice and this permission notice shall be included in all
% copies or substantial portions of the Software.
% 
% THE SOFTWARE IS PROVIDED "AS IS", WITHOUT WARRANTY OF ANY KIND, EXPRESS OR
% IMPLIED, INCLUDING BUT NOT LIMITED TO THE WARRANTIES OF MERCHANTABILITY,
% FITNESS FOR A PARTICULAR PURPOSE AND NONINFRINGEMENT. IN NO EVENT SHALL THE
% AUTHORS OR COPYRIGHT HOLDERS BE LIABLE FOR ANY CLAIM, DAMAGES OR OTHER
% LIABILITY, WHETHER IN AN ACTION OF CONTRACT, TORT OR OTHERWISE, ARISING FROM,
% OUT OF OR IN CONNECTION WITH THE SOFTWARE OR THE USE OR OTHER DEALINGS IN THE
% SOFTWARE.

\documentclass{pdfBooklets}
\usepackage[printwatermark]{xwatermark}
\title{Programowanie mikrokontrolerów STM32}
\author{%
	Projekt ,,Matematyka dla Ciekawych Świata'',\\
	Krzysztof Lasocki\\\normalsize\ttfamily <krz.lasocki@gmail.com>
}
\date  {2020-04-04}

\makeatletter\hypersetup{
	pdftitle = {\@title}, pdfauthor = {\@author}
}\makeatother


\begin{document}

\maketitle

\newwatermark[allpages,color=red!50,angle=45,scale=3,xpos=0,ypos=0]{Wersja robocza}

Skrypt opisuje podstawy programowania mikrokontrolerów STM32. Podczas kursu będziemy używać
popularnej, dostępnej i prostej płytki ``Blue Pill''. Programy będą pisane w języku
C z pomocą biblioteki \Verb$libopencm3$.




\section{Pierwszy program}
\textit{zintegrować z repo przykładów kodu}\\
Odpowiednikiem programu ``Hello, world!'' w elektronice jest program migający diodą LED. Zapisz poniższy kod źródłowy w pliku z
rozszerzeniem \Verb$.c$

\begin{CodeFrame*}[c]{}
#ifndef STM32F1
#define STM32F1
#endif


#include <libopencm3/stm32/rcc.h>
#include <libopencm3/stm32/gpio.h>

int main(){
  // Uruchomienie peryferium portu C
  // Włączenie sygnału zegara dla portu C
  rcc_periph_clock_enable(RCC_GPIOC);
  // Ustawienie pinu C13 w trybie wyjścia
  gpio_set_mode(GPIOC, GPIO_MODE_OUTPUT_2_MHZ,
		GPIO_CNF_OUTPUT_PUSHPULL, GPIO13);

  while(1){
    // Poczekaj chwilkę
    for (int i = 0; i < 150000; i++) __asm__("nop");
    // Przełącz stan pinu 13 w porcie C
    gpio_toggle(GPIOC, GPIO13);
      
  }
}
\end{CodeFrame*}

Skompiluj i wgraj program za pomocą

\begin{CodeFrame*}[bash]{}
make
make install
\end{CodeFrame*}

Jeśli \Verb$stm32flash$ wyszedł bez błędu, przełóż górną zworkę na pozycję zero i wciśnij przycisk reset. Możesz też odłączyć i podłączyć ponownie
zasilanie do płytki. Jeśli wszystko poszło dobrze, zielona dioda na płytce powinna zacząć migać


\begin{ProTip}{\normalfont{\strong{Uwaga}}}
  Zworka, którą przestawiasz, decyduje w jakim trybie uruchomi się procesor. Pozycja ``0'' oznacza normalny start i rozpoczęcie
  wykonywania programu z pamięci.
  
  Pozycja ``1'' uruchamia procesor w trybie bootloadera, za pomocą którego wgrywamy program. Zawsze, gdy chcesz uruchomić program po
  wgraniu, upenij się, że zworka jest w pozycji ``0''. Nie zapomnij wcisnąć przycisku reset po zaprogramowaniu mikrokontrolera.
\end{ProTip}



Przyjrzyjmy się powyższemu plikowi linia po linii, aby zrozumieć, dlaczego nasz program działa.


\begin{CodeFrame*}[c]{}
#include <libopencm3/stm32/rcc.h>
#include <libopencm3/stm32/gpio.h>

\end{CodeFrame*}

Dołączamy dwa pliki nagłówkowe z biblioteki \Verb$libopencm3$ aby móc używać jej funkcji. Możliwe jest pisanie kodu w ``czystym'' C
(lub nawet w asemblerze), ale kod w ten sposób napisany będzie mniej czytelny i mniej przenośny.
\footnote{Oraz niewiele szybszy.}

\begin{CodeFrame*}[c]{}
int main() { 
\end{CodeFrame*}

Jak każdy program w C, funkcją początkową jest \verb$main$. W tym przypadku nie bierze ona żadnych argumentów. Mimo \Verb$int$ w
definicji, nie zwraca ona żadnej wartości. W elektronice, \Verb$main$ z reguły nigdy nie kończy pracy (powrót z niej najczęściej
kończy się skokiem do wektora resetu i zresetowaniem mikronontrolera)

\begin{CodeFrame*}[c]{}
  rcc_periph_clock_enable(RCC_GPIOC);
  gpio_set_mode(GPIOC, GPIO_MODE_OUTPUT_2_MHZ,
		GPIO_CNF_OUTPUT_PUSHPULL, GPIO13);
\end{CodeFrame*}

Po rozpoczęciu programu konfigurujemy peryferia. W tym programie używamy portu C aby migać diodą, która jest podłączona do pinu C13
(13. bit portu C). Przed rozpoczęciem jakichkolwiek działań z tym peryferium musimy uruchomić jego zegar\footnotemark (wywołaniem makra
\Verb$rcc_periph_clock_enable$ z odpowiednim parametrem)
\footnotetext{Taka dowolność we włączaniu lub wyłączaniu sygnału zegara do peryferiów pozwala projektantom oszczędzać energię.
  Jest to bardzo ważne np. przy układach zasilanych bateriami. W układach CMOS, kiedy nie następują zmiany stanów, pobór energii
  jest praktycznie znikomy, więc to, czy sygnał zegara nieużywanego peryferium jest zatrzymany lub nie, znacznie wpływa na pobór prądu}

Następnie konfigurujemy pin C13 jako wyjście \textit{push-pull}. Domyślnie wszytkie piny GPIO są skonfigurowane jako wejścia.
\textit{może coś więcej?}


\begin{CodeFrame*}[c]{}
  while(1){
    // Poczekaj chwilkę
    for (int i = 0; i < 150000; i++) __asm__("nop");
    // Przełącz stan pinu 13 w porcie C
    gpio_toggle(GPIOC, GPIO13); 
  }
\end{CodeFrame*}

Jak mówiłem wcześniej, procedura main z reguły nie wychodzi. Zamiast tego kończy się nieskończoną pętlą. W pętli, procesor najpierw
wykonuje \textit{nop}, czyli tzw. pustą instrukcję 150000 razy\footnotemark. Następnie funkcja \Verb$gpio_toggle$ zmienia stan pinu 13
w porcie C na przeciwny, co powoduje zapalenie lub zgaszenie LEDa.
\footnotetext{W brew pozorom ta funkcja nie zabierze 150 tys. cykli procesora, tylko znacznie więcej. Zwiększenie wartości zmiennej,
  porównanie i skok warunkowy zajmują czas. Nie jest to precyzyjna metoda odmierzania czasu.}



\section{Obsługa standardowego wejścia/wyjścia}


\textit{tbd}

\section{Lektury uzupełniające}
\begin{itemize}

  % Link za długi?
  %https://www.st.com/resource/en/reference_manual/cd00171190-stm32f101xx-stm32f102xx-stm32f103xx-stm32f105xx-and-stm32f107xx-advanced-arm-based-32-bit-mcus-stmicroelectronics.pdf
\item \emph{Tzw. \textit{reference manual} dla STM32F103} (\url{link za dlugi}) - obszerny dokument opisujący w jaki sposób programować
  mikrokontroler. Zawiera szczegółówe opisy działania peryferiów, listę rejestrów i pól bitowych wraz ich funkcjami oraz
  adresami \footnotemark. Najważniejszy dokument przy programowaniu mikrokontrolera
  
  \footnotetext{W STM32 opis adresów jest rozbity pomiędzy \textit{reference manual} i kartę katalogową.}

\item \emph{Karta katalogowa STM32F103}\\ (\url{https://www.st.com/resource/en/datasheet/stm32f103c8.pdf}) opisująca pinout i
  parametry mikrokontrolera.

\item \emph{Dokumentacja \Verb$libopencm3$}\\ (\url{http://libopencm3.org/docs/latest/html/}) opisująca funkcje i makra dostępne
  w bibliotece.
  
\item \emph{Przykładowy kod napisany z użyciem \Verb$libopencm3$} (\url{https://github.com/libopencm3/libopencm3-examples}) 
  
\item \emph{Vademecum informatyki praktycznej} (\url{http://vip.opcode.eu.org/}) - zbiór materiałów na temat elektroniki i programowania.

\end{itemize}




\clearpage
\section{Zadania domowe}
\newtheorem{ZadanieDomowe}{Zadanie domowe}

\begin{ZadanieDomowe} [1pkt]
W jaki sposób zmienić częstotliwość migania LEDa w pierwszym programie? Zmień ten program tak aby LED migał (około) dwa razy szybciej.
\end{ZadanieDomowe}

\begin{ZadanieDomowe} [2pkt]
  
\end{ZadanieDomowe}

\begin{ZadanieDomowe} [3pkt]
  % Tutaj chodzi o użycie printf-a, putchar-a lub uart_send_blocking
  Napisz program, który za pomocą zaimplementowanej w ćwiczeniu \textit{XXX} funkcji wejścia/wyjścia wypisze na UART
  ``trójkąt z gwiazdek'' jak poniżej
  
  \begin{CodeFrame*}[text]{}
    *
    **
    ***
    ****
    *****
    ******
    *******
    ********
  \end{CodeFrame*}
\end{ZadanieDomowe}


\copyrightFooter{
	© Matematyka dla Ciekawych Świata, 2017-2019.\\
	© Robert Ryszard Paciorek <rrp@opcode.eu.org>, 2003-2019.\\
	© Krzysztof Lasocki <krz.lasocki@gmail.com>, 2003-2019.\\
	Kopiowanie, modyfikowanie i redystrybucja dozwolone pod warunkiem zachowania informacji o autorach.
}



\end{document}


