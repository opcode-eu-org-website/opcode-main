% Copyright (c) 2017-2019 Matematyka dla Ciekawych Świata (http://ciekawi.icm.edu.pl/)
% Copyright (c) 2017-2019 Robert Ryszard Paciorek <rrp@opcode.eu.org>
% 
% MIT License
% 
% Permission is hereby granted, free of charge, to any person obtaining a copy
% of this software and associated documentation files (the "Software"), to deal
% in the Software without restriction, including without limitation the rights
% to use, copy, modify, merge, publish, distribute, sublicense, and/or sell
% copies of the Software, and to permit persons to whom the Software is
% furnished to do so, subject to the following conditions:
% 
% The above copyright notice and this permission notice shall be included in all
% copies or substantial portions of the Software.
% 
% THE SOFTWARE IS PROVIDED "AS IS", WITHOUT WARRANTY OF ANY KIND, EXPRESS OR
% IMPLIED, INCLUDING BUT NOT LIMITED TO THE WARRANTIES OF MERCHANTABILITY,
% FITNESS FOR A PARTICULAR PURPOSE AND NONINFRINGEMENT. IN NO EVENT SHALL THE
% AUTHORS OR COPYRIGHT HOLDERS BE LIABLE FOR ANY CLAIM, DAMAGES OR OTHER
% LIABILITY, WHETHER IN AN ACTION OF CONTRACT, TORT OR OTHERWISE, ARISING FROM,
% OUT OF OR IN CONNECTION WITH THE SOFTWARE OR THE USE OR OTHER DEALINGS IN THE
% SOFTWARE.

\documentclass{pdfBooklets}
\usepackage[printwatermark]{xwatermark}
\title{Programowanie mikrokontrolerów STM32}
\author{%
	Projekt ,,Matematyka dla Ciekawych Świata'',\\
	Krzysztof Lasocki\\\normalsize\ttfamily <krz.lasocki@gmail.com>
}
\date  {2020-04-04}

\makeatletter\hypersetup{
	pdftitle = {\@title}, pdfauthor = {\@author}
}\makeatother


\begin{document}

\maketitle

\newwatermark[allpages,color=red!50,angle=45,scale=3,xpos=0,ypos=0]{Wersja robocza}

Skrypt opisuje podstawy programowania mikrokontrolerów STM32. Będziemy używać
popularnej, dostępnej i prostej płytki ``Blue Pill''. Programy będą pisane w języku
C z pomocą biblioteki \Verb$libopencm3$.

\section{Instalacja i przygotowanie narzędzi}

Przed rozpoczęciem pracy z STM32 należy zainstalować następujące narzędzia i biblioteki:

\begin{itemize}
  \item Toolchain arm-none-eabi (\Verb$gcc-arm-none-eabi$) oraz (\Verb$binutils-arm-none-eabi$)
  \item Implementację libC (\Verb$libstdc++-arm-none-eabi-newlib$)
  \item Narzędzie do programowania przez UART (\Verb$stm32flash$)
  \item Bibliotekę \Verb$libopencm3$
\end{itemize}
Wszystkie oprócz ostatniego dostępne są w repozytoriach Debiana (i systemów na nim opartych)
\begin{CodeFrame*}[bash]{}
sudo apt install gcc-arm-none-eabi binutils-arm-none-eabi libstdc++-arm-none-eabi stm32flash
\end{CodeFrame*}

Instalacja i kompilacja \Verb$libopencm3$ odbywa się w następujący sposób

\begin{CodeFrame*}[bash]{}
git clone https://github.com/libopencm3/libopencm3.git
make
\end{CodeFrame*}

\section{Przygotowanie mikronotrolera}
\textit{tutaj opisać co i jak polutować. Obrazki? to powinno być też pokazane przed właściwymi ćwiczeniami}
\begin{ProTip}{\normalfont{\strong{Ostrożnie}}}
  Podczas pracy grot (metalowa końcówka) lutownicy jest rozgrzany do 200-400 stopni Celsjusza. Zachowaj ostrożność
  podczas jej używania. Zapamiętaj:
  \begin{itemize}
  \item Zawsze odkładaj lutownicę do stojaka. Nigdy nie kładź jej luzem na stole.
  \item Nigdy nie łap za metalowy koniec lutownicy.
  \item Nie wolno łapać spadającej lutownicy. Nie martw się, zawszę można kupić nową
  \item Grot lutownicy jest gorący przez jakiś czas po wyłączeniu
  \item Nie zostawiaj włączonej lutownicy bez opieki
  \item Przed lutowaniem musisz odłączyć układ od zasilania
  \end{itemize}
\end{ProTip}

Jeżeli Twój mikrokontroler nie ma przylutowanych pinów, musisz przylutować je samemu. Przymierz i odetnij
obcążkami dwa odcinki listwy kołkowej pasujące do otworów na brzegach płytki mikrokontrolera. Włóż piny do otworków
na płytce. Odwiń lub wyciągnij odcinek cyny ze szpulki. Aby zalutować pin w otworze, najpierw dotknij grotem lutownicy
miejsca, które będziesz lutować. Po około sekundzie, dotknij pinu końcówką odcinka cyny.

Postaraj się aby listwy kołkowe były prostopadle do płytki. Jeżeli masz odcinek damskiej listwy, możesz za jego pomocą
umiejscowić piny które lutujesz. Możesz też użyć do tego płytki stykowej. Lutowanie zacznij od czterech pinów na rogach płytki.


\section{Płytka stykowa}
  Płytka stykowa pozwala na szybkie prototypowanie układów. Posiada ona macierz dziurek. Pod nimi są umieszczone
  blaszki które łączą sąsiednie dziurki w taki sposób, że każde 5 dziurek w pionie jest ze sobą połączone elektrycznie.
  
  Wyżłobienie w środku płytki służy do 



\section{Połączenie mikronotrolera}
\textit {tutaj warto by powrzucać zdjęcia lub rysunki}

\begin{ProTip}{\normalfont{\strong{Uwaga}}}
  Przed podłączaniem lub dokonywaniem jakichkolwiek zmian w układzie, odłącz go od zasilania. Unikniesz
  w ten sposób ryzyka zwarcia i uszkodzenia elementów.
\end{ProTip}

Aby zaprogramować mikrokontroler musisz podłączyć go do przejściówki USB-UART. Umieść płytkę ``okrakiem'' nad przerwą
w płytce stykowej. Za pomocą pasujących kabelków podłącz następujące piny przejściówki (nie podłączaj jeszcze
przejściówki do komputera) do pinów na mikrokontrolerze:
\begin{itemize}
\item masę (GND) do (dowolnej) masy miktrokontrolera (GND lub G)
\item RX do TX mikrokontrolera (pin A9)
\item TX do RX mikrokontrolera (pin A10)
\item 3v3 do 3v3 mikrokontrolera (lub, jeżeli Twoja przejściówka nie ma pinu 3v3, podłącz pin 5V do pinu 5V na
  mikrokontrolerze)
\end{itemize}

Za pomocą obcążków lub pęsety przełóż górną (patrząc na mikrokontroler tak aby port USB był po lewej) zworkę na
pozycję ``1''

Sprawdź wszystkie połączenia i podłącz przejściówkę do portu USB komputera. Powinna zaświecić się tylko czerwona dioda
\textit{PWR}





\end{document}


