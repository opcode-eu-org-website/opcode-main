% Copyright (c) 2020 Robert Ryszard Paciorek <rrp@opcode.eu.org>
% 
% MIT License
% 
% Permission is hereby granted, free of charge, to any person obtaining a copy
% of this software and associated documentation files (the "Software"), to deal
% in the Software without restriction, including without limitation the rights
% to use, copy, modify, merge, publish, distribute, sublicense, and/or sell
% copies of the Software, and to permit persons to whom the Software is
% furnished to do so, subject to the following conditions:
% 
% The above copyright notice and this permission notice shall be included in all
% copies or substantial portions of the Software.
% 
% THE SOFTWARE IS PROVIDED "AS IS", WITHOUT WARRANTY OF ANY KIND, EXPRESS OR
% IMPLIED, INCLUDING BUT NOT LIMITED TO THE WARRANTIES OF MERCHANTABILITY,
% FITNESS FOR A PARTICULAR PURPOSE AND NONINFRINGEMENT. IN NO EVENT SHALL THE
% AUTHORS OR COPYRIGHT HOLDERS BE LIABLE FOR ANY CLAIM, DAMAGES OR OTHER
% LIABILITY, WHETHER IN AN ACTION OF CONTRACT, TORT OR OTHERWISE, ARISING FROM,
% OUT OF OR IN CONNECTION WITH THE SOFTWARE OR THE USE OR OTHER DEALINGS IN THE
% SOFTWARE.

\documentclass{pdfBooklets}

\title{Linux i sieci: Sieci komputerowe}
\author{%
	Projekt ,,Matematyka dla Ciekawych Świata'',\\
	Robert Ryszard Paciorek\\\normalsize\ttfamily <rrp@opcode.eu.org>
}
\date  {2020-07-08}

\makeatletter\hypersetup{
	pdftitle = {\@title}, pdfauthor = {\@author}
}\makeatother

% Copyright (c) 2017-2020 Matematyka dla Ciekawych Świata (http://ciekawi.icm.edu.pl/)
% Copyright (c) 2017-2020 Robert Ryszard Paciorek <rrp@opcode.eu.org>
% Copyright (c) 2020 Krzysztof Lasocki <krz.lasocki@gmail.com>
% 
% MIT License
% 
% Permission is hereby granted, free of charge, to any person obtaining a copy
% of this software and associated documentation files (the "Software"), to deal
% in the Software without restriction, including without limitation the rights
% to use, copy, modify, merge, publish, distribute, sublicense, and/or sell
% copies of the Software, and to permit persons to whom the Software is
% furnished to do so, subject to the following conditions:
% 
% The above copyright notice and this permission notice shall be included in all
% copies or substantial portions of the Software.
% 
% THE SOFTWARE IS PROVIDED "AS IS", WITHOUT WARRANTY OF ANY KIND, EXPRESS OR
% IMPLIED, INCLUDING BUT NOT LIMITED TO THE WARRANTIES OF MERCHANTABILITY,
% FITNESS FOR A PARTICULAR PURPOSE AND NONINFRINGEMENT. IN NO EVENT SHALL THE
% AUTHORS OR COPYRIGHT HOLDERS BE LIABLE FOR ANY CLAIM, DAMAGES OR OTHER
% LIABILITY, WHETHER IN AN ACTION OF CONTRACT, TORT OR OTHERWISE, ARISING FROM,
% OUT OF OR IN CONNECTION WITH THE SOFTWARE OR THE USE OR OTHER DEALINGS IN THE
% SOFTWARE.

\usepackage{tikz}
\usetikzlibrary{circuits.ee.IEC}

\newtcolorbox{Ramka}[1][]{
	colback=white,
	colbacktitle=white,
	coltitle=black,
	colframe=white!50!black,
	fontupper=\small,
	enhanced,
	before skip=13pt plus 2pt,
	#1
}

% Symbol DC
\newcommand{\mathdirectcurrent}{\mathrel{\mathpalette\mathdirectcurrentinner\relax}}
\newcommand{\mathdirectcurrentinner}[2]{%
  \settowidth{\dimen0}{$#1=$}%
  \vbox to .85ex {\offinterlineskip
    \hbox to \dimen0{\hss\leaders\hrule\hskip.85\dimen0\hss}
    \vskip.35ex
    \hbox to \dimen0{\hss
      \leaders\hrule\hskip.17\dimen0
      \hskip.17\dimen0
      \leaders\hrule\hskip.17\dimen0
      \hskip.17\dimen0
      \leaders\hrule\hskip.17\dimen0
    \hss}
    \vfill
  }%
}
% symbol diody
\newcommand\esymbol[1]{\tikz[circuit ee IEC] \draw (0,0) -- (.1,0) node [#1,anchor=west,name=s] {} (s.east) -- +(.1,0);}

\newcommand{\textdirectcurrent}{\mathdirectcurrentinner{\textstyle}{}}

\newcommand\zaleta{\item[\textbf{\ttfamily +}]}
\newcommand\wada{\item[\textbf{\ttfamily -}]}
\newcommand\info{\item[\textbf{\ttfamily *}]}
\newcommand\uwaga{\item[\textbf{\ttfamily !}]}


\begin{document}

\maketitle

% Copyright (c) 2017-2020 Matematyka dla Ciekawych Świata (http://ciekawi.icm.edu.pl/)
% Copyright (c) 2017-2020 Robert Ryszard Paciorek <rrp@opcode.eu.org>
% 
% MIT License
% 
% Permission is hereby granted, free of charge, to any person obtaining a copy
% of this software and associated documentation files (the "Software"), to deal
% in the Software without restriction, including without limitation the rights
% to use, copy, modify, merge, publish, distribute, sublicense, and/or sell
% copies of the Software, and to permit persons to whom the Software is
% furnished to do so, subject to the following conditions:
% 
% The above copyright notice and this permission notice shall be included in all
% copies or substantial portions of the Software.
% 
% THE SOFTWARE IS PROVIDED "AS IS", WITHOUT WARRANTY OF ANY KIND, EXPRESS OR
% IMPLIED, INCLUDING BUT NOT LIMITED TO THE WARRANTIES OF MERCHANTABILITY,
% FITNESS FOR A PARTICULAR PURPOSE AND NONINFRINGEMENT. IN NO EVENT SHALL THE
% AUTHORS OR COPYRIGHT HOLDERS BE LIABLE FOR ANY CLAIM, DAMAGES OR OTHER
% LIABILITY, WHETHER IN AN ACTION OF CONTRACT, TORT OR OTHERWISE, ARISING FROM,
% OUT OF OR IN CONNECTION WITH THE SOFTWARE OR THE USE OR OTHER DEALINGS IN THE
% SOFTWARE.

% BEGIN: Sieci - intro
\section{Podstawy TCP/IP}

Sieci komputerowe działają na zasadzie przesyłania informacji w postaci porcji, z których każda posiada co najmniej informację o adresie odbiorcy (zwykle też nadawcy), nazywanych ramkami lub pakietami. Kierowanie pakietów w odpowiednie miejsce odbywa się na podstawie adresu pakietu i nie jest związane z fizycznym zestawianiem łącza pomiędzy nadawcą a odbiorcą - każdy pakiet jest kierowany niezależnie, a w ramach pojedynczego łącza (kanału transmisji) mogą być przekazywane pakiety adresowane do różnych odbiorców. Nazywane jest to komutacją pakietów, w odróżnieniu od komutacji łącza (która występowała np. w klasycznej, analogowej telefonii, gdzie przekaźniki w centralach dokonywały zestawienia połączeń elektrycznych między dwoma aparatami telefonicznymi).

\subsection{Struktura warstwowa}

Komunikacja sieciowa typowo posiada strukturę warstwową. W modelu OSI wyróżnia się 7 warstw:
\begin{enumerate}
	\item fizyczną (pierwszą) definiującą aspekty związane z fizycznym przesyłem sygnału takie jak częstotliwości radiowe, poziomy napięć, etc.;
		określa sposób transmisji kolejnych bajtów
	\item łącza danych (drugą) definiującą aspekty związane z formatem ramki, protokoły ustalania zasad dostępu do medium transmisyjnego, itd.;
		określa sposób transmisji porcji danych pomiędzy hostami w jednej sieci
	\item sieciową (trzecią) definiującą aspekty związane z formatem pakietu, adresacją i zasady routingu umożliwiające zapewnienie łączności pomiędzy różnymi sieciami;
		określa sposoby transmisji porcji danych pomiędzy sieciami
	\item transportową (czwartą) odpowiedzialną za podział strumienia na porcje informacji, kontrolę nad poprawnością transmisji, adresację usług w ramach hosta
	\item sesji (piątą)
	\item prezentacji (szóstą)
	\item aplikacji (siódmą)
\end{enumerate}

\inputSingleAsFigure[.9]{booklets-sections/network/ilustracje/10-warstwy.tex}{Struktura warstwowa protokołów sieciowych}{ilustracja-warstwy}

\noindent
W modelu TCP/IP wyróżnia się 4 warstwy:
\begin{enumerate}
	\item Dostępu do sieci - obejmującą warstwy 1 i 2 modelu OSI
	\item Internetu - obejmującą warstwę 3 modelu OSI
	\item Transportową - obejmującą warstwę 4 modelu OSI
	\item Aplikacji - obejmującą warstwy 5, 6 i 7 modelu OSI
\end{enumerate}

\noindent
Z punktu widzenia modelu TCP/IP można powiedzieć o enkapsulacji danych kolejnych warstw w ramach warstwy niższej, czyli „surowe” dane (np. strona HTML) obudowywane są strukturą opisywaną przez warstwę aplikacji (np. nagłówkami HTTP), następnie całość ta umieszczana jest w polu danych pakietu warstwy transportowej (np. TCP), ten z kolei w polu danych pakietu IP (warstwy sieciowej), na koniec pakiet IP jest umieszczany w polu danych ramki warstwy dostępu do sieci (np. ramki ethernetowej). W ramach podróży przez kolejne sieci pakiet IP jest wyjmowany i wkładany w kolejne ramki warstwy dostępu do sieci, na ogół tylko z niewielkimi ingerencjami w zawartość tego pakietu (prawie zawsze nie dochodzącymi do pola danych pakietu TCP lub datagramu UDP, czyli nie wykraczającymi poza warstwę 4 OSI).

\begin{teacherOnly} Można porównać do podróży listu różnymi środkami transportu:
	\begin{easylist}[itemize]
		& list ma jakąś zawartość - dane które na ogół nie są sprawdzane po drodze
		& koperta ma adres nadawcy i docelowy
		& pociąg którym podróżuje ma "lokalnuy"/"sprzętowy" adres nadawcy i docelowy (stacja początkowa, końcowa)
		& później koperta może być przeniesiona do innego pociągu (ten sam typ sieci fizycznej, ale inna sieć), lub np. na statek (inny typ sieci fizycznej)
	\end{easylist}
\end{teacherOnly}

% END: Sieci - intro

% BEGIN: sieci IP
\subsection{Protokół IP}

Protokół IP (Internet Protocol) odpowiedzialny jest przede wszystkim za sposób adresacji hostów oraz reguły komutacji pakietów (routing). Jest on wspomagany przez kolejny protokół z tej rodziny - ICMP (Internet Control Message Protocol), którego zadaniem jest przekazywanie informacji kontrolnych np. o nieosiągalności hosta docelowego, odrzuceniu przetwarzania pakietu ze względu na zbyt dużą liczbę skoków (gdy wartość pola TTL z nagłówka IP wyniesie zero) a także pingi (zarówno żądanie jak i odpowiedź).

\inputSideBySideAsFigure
	{Pakiet IPv4}{booklets-sections/network/ilustracje/10-ipv4.tex}
	{Pakiet IPv6}{booklets-sections/network/ilustracje/10-ipv6.tex}
	{Struktura pakietów IP}{ilustracja_pakiety_ip}
% END: sieci IP

% Copyright (c) 2017-2020 Matematyka dla Ciekawych Świata (http://ciekawi.icm.edu.pl/)
% Copyright (c) 2017-2020 Robert Ryszard Paciorek <rrp@opcode.eu.org>
% 
% MIT License
% 
% Permission is hereby granted, free of charge, to any person obtaining a copy
% of this software and associated documentation files (the "Software"), to deal
% in the Software without restriction, including without limitation the rights
% to use, copy, modify, merge, publish, distribute, sublicense, and/or sell
% copies of the Software, and to permit persons to whom the Software is
% furnished to do so, subject to the following conditions:
% 
% The above copyright notice and this permission notice shall be included in all
% copies or substantial portions of the Software.
% 
% THE SOFTWARE IS PROVIDED "AS IS", WITHOUT WARRANTY OF ANY KIND, EXPRESS OR
% IMPLIED, INCLUDING BUT NOT LIMITED TO THE WARRANTIES OF MERCHANTABILITY,
% FITNESS FOR A PARTICULAR PURPOSE AND NONINFRINGEMENT. IN NO EVENT SHALL THE
% AUTHORS OR COPYRIGHT HOLDERS BE LIABLE FOR ANY CLAIM, DAMAGES OR OTHER
% LIABILITY, WHETHER IN AN ACTION OF CONTRACT, TORT OR OTHERWISE, ARISING FROM,
% OUT OF OR IN CONNECTION WITH THE SOFTWARE OR THE USE OR OTHER DEALINGS IN THE
% SOFTWARE.

% BEGIN: adresacja IP
\subsection{Adresacja IP}

Adresy hostów (nazywane adresami IP) są to 32-bitowe (w IPv4) lub 128-bitowe (w IPv6) liczby.
Adresy IPv4 zapisywane są najczęściej w notacji kropkowo-dziesiętnej, gdzie każdy bajt (ciąg 8 bitów) zapisywany jest jako liczba dziesiętna rozdzielana kropką od pozostałych. Adresy IPv6 zapisywane są zazwyczaj w notacji dwukropokowej, polegającej na zapisywaniu 16 bitowych części adresu liczbami szesnastkowymi oddzielanymi dwukropkiem, dodatkowo jeden ciąg zer (o długości będącej wielokrotnością 16 bitów) może być skompresowany (pominięty) co daje w zapisie dwa dwukropki \Verb$::$.

\subsubsection{Długość prefixu i maska}

Adresy hostów grupuje się w adresy sieci, bazując na jednakowym (bitowo) początku takiego adresu (zwanym adresem sieci lub prefixem). Ilość bitów stanowiących adres sieci w danym adresie IP nazywana jest długością prefixu i zapisywana jest zazwyczaj po ukośniku\footnote{
	Jest to notacja \textit{CIDR}. Przed wprowadzeniem tego mechanizmu w IPv4 funkcjonował klasowy sposób routingu (\textit{classful}), gdzie wielkość maski była deteminowana wartością pierwszych bitów adresu – ale to już historia.
}. Na przykład zapis \Verb$2001:db8::a17/48$ oznacza że pierwsze 48 bity stanowią adres sieci a kolejne $128-48 = 80$ bitów stanowi adres hosta w tej sieci.

Długość prefixu jednoznacznie określa maskę danej podsieci, czyli liczbę odpowiadającą długości adresu (32 bity lub 128 bitów), złożoną z ciągu jedynek o długości prefixu oraz ciągu zer (o długości adresu hosta). W przypadku IPv4 spotykane jest także podawanie maski sieci w notacji kropkowo-dziesiętnej zamiast długości prefixu.

\begin{Verbatim}[frame=single,label=\textrm{\bfseries Przykład},commandchars=\\\{\},commentchar=\%]
adres IPv4 zapisany z informacją o długości prefixu: 10.23.45.56/{\color{violet}13}, czyli:
adres: 10.23.45.56 = {\color{blue}0000101000010}{\color{brown}1110010110100111000}
maska: 255.248.0.0 = {\color{blue}1111111111111}{\color{brown}0000000000000000000}
                     {\color{blue}prefix sieci }{\color{brown}adres hosta w sieci}
adres sieci        = {\color{blue}0000101000010}{\color{brown}0000000000000000000} = 10.16.0.0\vspace{4pt}
{\footnotesize{}Adres sieci obliczany jest jako bitowy AND pomiędzy adresem i maską.}\vspace{4pt}
{\footnotesize{}Maska zawsze jest złożona z ciągu samych bitów o wartości {\color{blue}1} a następnie o wartości {\color{brown}0}.}
{\footnotesize{}Bitów o wartości {\color{blue}1} jest tyle ile wynosi długość prefixu (podawana po /), czyli w tym przykładzie {\color{violet}13}.}\vspace{4pt}
{\footnotesize{}W IPv6 działa to analogicznie, tyle że adres ma 128 bitów długości, stosuje się notacje dwukropkową}
{\footnotesize{}zamiast kropkowo-dziesiętnej i nie stosuje się jawnego zapisu maski (a jedynie długość prefixu).}
\end{Verbatim}

Sieć może zostać podzielona na mniejsze sieci (z większą wartością prefixu), jak też grupa sieci może zostać zagregowana w jedną większą ($2^n$ raza) sieć (z prefixem mniejszym o n). Agregacja hostów i sieci w większe całości jest wykorzystywana w mechanizmach routingu, co pozwala na redukcję wielkości tablic routingowych.

\subsubsection{Przynależność do sieci}
Adres sieci zapisuje się typowo z wyzerowanymi bitami stanowiącymi adres hosta (czyli po dokonaniu bitowego \emph{and} z maską danej sieci) oraz podaną informacją o długości prefixu, dla powyższego przykładu będzie to \Verb$2001:db8::/48$. Informacja taka jest wystarczająca do sprawdzenia czy dowolny inny adres IP należy do tej sieci czy nie.

\label{czy_w_sieci_kod}\begin{CodeFrame*}[python][fontsize=\footnotesize]{}
from ipaddress import *

# adres

adr     = ip_interface("2001:0db8::17:15")
adr_int = int(adr.ip)
print("Adress IPv6 jest 128 bitową liczbą całkowitą np.:")
print("  " + str(adr.ip) + " == " + hex(adr_int) + "\n")

# sieć - maska i długość prefixu

net         = ip_interface("::/112");
netmask     = net.network.netmask
netmask_int = int(netmask)
net_preflen = net.network.prefixlen

print("Maska podsieci IPv6 jest 128 bitową liczbą całkowitą np.:")
print("  " + str(netmask) + " == " + hex(netmask_int) + "\n")
print("Jako że maska jest liczbą, która zapisana binarnie, zawsze zawiera ciągły ciąg bitów")
print("o wartości 1, a po nim ciągły ciąg bitów o wartości 0 (mogą być zerowej długości), to")
print("często stosowany jest zapis polegający na podawaniu długości prefiksu: /" + str(net_preflen))
print("jest to ilość bitów o wartości 1 w masce, czyli im większy prefix tym mniejsza sieć.\n")

# adres w sieci

adr2     = ip_interface("2001:0db8::17:15/112");
net2     = adr2.network
net2_int = int(net2.network_address)

print("Aby obliczyć adres sieci (czyli wspólną dla wszystkich hostów w danej sieci część")
print("adresu IP) należy wykonać binarny AND pomiędzy adresem IP hosta a maską podsieci.")
print("Dla powyższego przykładu:")
print("  " + hex(netmask_int & adr_int) + " == " + str(net2) + " == " + hex(net2_int) + "\n")

# aby sprawdzić czy adres IP należy do danej sieci trzeba obliczyć adres sieci tego hosta
# w oparciu o maskę sieci którą sprawdzamy
def sprawdzSiec(n, a):
	nn = int(a) & int(n.netmask)
	if nn == int(n.network_address):
		print(str(a) + " należy do sieci " + str(n))
	else:
		print(str(a) + " NIE należy do sieci " + str(n))

sprawdzSiec(net2, ip_interface("2001:0db8::17:ab13").ip)
sprawdzSiec(net2, ip_interface("2001:0db8::13:a").ip)
\end{CodeFrame*}
% END: adresacja IP

\setcounter{subsubsection}{0}
\insertZadanie{booklets-sections/network/zadania-10_20_30.tex}{czy_w_sieci}{}

\ifdefined\inputOnlyContent\else
	\documentclass[tikz]{standalone}
	\usepackage{tikzPackets}
	\InputIfFileExists{booklets-sections/network/preambule.tex}{}{}
	\begin{document}
\fi


\providecommand{\addrouter}[5][]{
	\node[router, #1]  (#2_0) {\bfseries #3};
	\foreach \txt[remember=\n as \lastn (initially 0), count=\n from 1] in {#4} {
		\node[riface] (#2_\n) [] at (#2_\lastn.south west) {eth\n: \txt};
	}
	\node[rtable] (#2_TABLE)  [] at (#2_\lastn.south west) {#5};
}

\providecommand{\addhost}[5][]{
	\node[base, #1]  (#2) {\textbf{#3}\\#4};
}

\begin{tikzpicture}[semithick]
	\tikzstyle{base}=[draw, minimum height = 0.8cm, anchor = north west, outer sep = 0pt, align=center]
	\tikzstyle{router}=[base, minimum width=4.5cm]
	\tikzstyle{riface}=[router, node font=\footnotesize]
	\tikzstyle{rtable}=[router, node font=\ttfamily\scriptsize, align=left, minimum height = 1.3cm]
	
	\addrouter[yshift=2.5cm,xshift=-6.5cm]{RA}{Router A}{
		10.0.4.1/24,
		10.9.1.2/29
	}{
		10.0.3.0/24 via 10.9.1.1\\
		10.0.5.0/24 via 10.9.1.1\\
		10.0.8.0/24 via 10.9.1.1\\
		default via 10.9.1.3
	}

	\addrouter[yshift=-2.5cm,xshift=-6.5cm]{RB}{Router B}{
		10.0.5.1/24,
		10.9.2.2/30,
		10.3.0.1/22
	}{
		10.0.3.0/24 via 10.9.2.1\\
		default via 10.3.1.1
	}
	
	\addrouter{RC}{Router C}{
		10.9.1.1/29,
		10.9.3.2/30,
		10.9.2.1/30,
		10.9.4.2/30
	}{
		10.0.3.0/24 via 10.9.3.1\\
		10.0.4.0/24 via 10.9.1.2\\
		10.0.5.0/24 via 10.9.3.2
	}

	\addrouter[yshift=2.5cm,xshift=6.5cm]{RD}{Router D}{
		10.0.3.1/24,
		10.9.3.1/30
	}{
		default via 10.0.3.100\\
		10.0.4.1/23 via 10.9.3.2
	}

	\addrouter[yshift=-2.5cm,xshift=6.5cm]{RE}{Router E}{
		10.0.8.1/24,
		10.9.4.1/30
	}{
		default via 10.9.4.2
	}
	
	\draw (RA_2.east) -- (RC_1.west);
	\draw (RB_2.east) -- (RC_3.west);
	\draw (RD_2.west) -- (RC_2.east);
	\draw (RE_2.west) -- (RC_4.east);
	
	\node[base, yshift=2.5cm, xshift=-10.5cm] (HA1) {HA1\\\footnotesize 10.0.4.33/24};
	\draw (HA1.east) -- (RA_1.west);
	\node[base, xshift=-10.5cm] (HA2) {HA2\\\footnotesize 10.0.4.71/24};
	\draw (HA2.east) -- (RA_1.west);
	
	\node[base, yshift=-2.5cm, xshift=-10.5cm] (HB1) {HB1\\\footnotesize 10.0.5.13/24};
	\draw (HB1.east) -- (RB_1.west);
	
	\node[base, yshift=2.5cm, xshift=12.5cm] (HD1) {HD1\\\footnotesize 10.0.3.5/24};
	\draw (HD1.west) -- (RD_1.east);
	
	\node[base, yshift=-2.5cm, xshift=12.5cm] (HE1) {HE1\\\footnotesize 10.0.8.8/24};
	\draw (HE1.west) -- (RE_1.east);
	
	\node[yshift=-8.1cm, xshift=2.25cm, text width=22cm] {\small
		Uwaga: W tablicach routingu na ilustracji pominięto wpisy związane z pokazanymi interfejsami i ich adresami IP.
		Pominięto też tablice routingowe hostów H*, należy zakładać że oprócz wpisu związanego z adresem IP ustawionym na interfejsie mają one jedynie wpis default wskazujący na „ich” router np. tablica HA1 ma dwa wpisy:
		\Verb$default via 10.0.4.1$ i \Verb$10.0.4.0/24 dev eth0$.
	};
\end{tikzpicture}

\ifdefined\inputOnlyContent\else\end{document}\fi


% Copyright (c) 2017-2020 Matematyka dla Ciekawych Świata (http://ciekawi.icm.edu.pl/)
% Copyright (c) 2017-2020 Robert Ryszard Paciorek <rrp@opcode.eu.org>
% 
% MIT License
% 
% Permission is hereby granted, free of charge, to any person obtaining a copy
% of this software and associated documentation files (the "Software"), to deal
% in the Software without restriction, including without limitation the rights
% to use, copy, modify, merge, publish, distribute, sublicense, and/or sell
% copies of the Software, and to permit persons to whom the Software is
% furnished to do so, subject to the following conditions:
% 
% The above copyright notice and this permission notice shall be included in all
% copies or substantial portions of the Software.
% 
% THE SOFTWARE IS PROVIDED "AS IS", WITHOUT WARRANTY OF ANY KIND, EXPRESS OR
% IMPLIED, INCLUDING BUT NOT LIMITED TO THE WARRANTIES OF MERCHANTABILITY,
% FITNESS FOR A PARTICULAR PURPOSE AND NONINFRINGEMENT. IN NO EVENT SHALL THE
% AUTHORS OR COPYRIGHT HOLDERS BE LIABLE FOR ANY CLAIM, DAMAGES OR OTHER
% LIABILITY, WHETHER IN AN ACTION OF CONTRACT, TORT OR OTHERWISE, ARISING FROM,
% OUT OF OR IN CONNECTION WITH THE SOFTWARE OR THE USE OR OTHER DEALINGS IN THE
% SOFTWARE.

% BEGIN: TCP / UDP
\section{Komunikacja TCP/IP}

\begin{teacherOnly}
	\begin{easylist}[itemize]
		& protokoły warstwy transportowej - TCP i UDP
		&& czym się różnią?
		&& numery portów – identyfikacja usługi / procesu na hoście
	\end{easylist}
\end{teacherOnly}

W oparciu o protokół IP działają protokoły warstwy transportowej takie jak UDP, TCP, czy też (mniej znany protokół dedykowany dla  strumieniowych transmisji czasu rzeczywistego) SCTP.

Jednym z zadań tych protokołów jest identyfikowanie usługi (procesu) w ramach systemu posiadającego dany adres IP, do którego mają trafić dane.
W tym celu zarówno UDP jak i TCP na każdym z hostów wyróżniają numeryczny identyfikator dla aplikacji/procesu/usługi będącego odbiorcą czy też nadawcą informacji zwany numerem portu.

Najprostszym protokołem warstwy transmisji wydaje się być UDP, protokół ten umożliwia przesłanie informacji pomiędzy dwoma hostami IP i nie kontroluje on tego czy została ona przesłana poprawnie.
Natomiast TCP, w odróżnieniu od UDP, kontroluje to czy przesłana informacja dotarła do adresata i nie została uszkodzona, a w przypadku problemów informacja wysyłana jest ponownie. TCP w związku z tym w przeciwieństwie do UDP musi otworzyć połączenie i wykorzystywać je do kontroli poprawności przesłania informacji, wymaga zatem przesłania większej liczby pakietów (co może prowadzić do pewnych opóźnień itp).
W związku z tym TCP używany jest tam gdzie konieczna jest kontrola poprawności transmisji (oraz ponowne wysłanie zgubionego pakietu), UDP tam gdzie nie jest to potrzebne (a liczy się czas).

\inputSideBySideAsFigure
	{Datagram UDP}{booklets-sections/network/ilustracje/20-udp.tex}
	{Pakiet TCP}{booklets-sections/network/ilustracje/20-tcp.tex}
	{Struktura pakietów UDP i TCP}{ilustracja_pakiety_udp_tcp}
% END: TCP / UDP

% BEGIN: Polularne usługi
\subsection{Popularne usługi}

W ramach sieci mogą być realizowane różne usługi w oparciu o różne protokoły warstwy aplikacyjnej. Standardowe usługi posiadają zdefiniowane domyślne adresy portów dla swoich protokołów. Wśród usług i protokołów sieciowych należy wymienić przynajmniej:
\begin{itemize}
	\item DNS (Domain Name System) - odpowiedzialny za system mapujący nazwy alfanumeryczne hostów na adresy IP.
	\item mechanizmy auto konfiguracji hostów - DHCP, rozgłaszanie informacji routingowej poprzez ICMPv6 (protokół warstwy 3)
	\item WWW - udostępnianie treści z użyciem protokołu HTTP
	\item pocztę elektroniczną - przesyłanie wiadomości (protokoły SMTP, IMAP, POP)
	\item komunikację natychmiastową i telefonię IP (protokoły SIP, XMPP, IAX)
	\item SSH - zdalny, szyfrowany dostęp do systemów IT, przesył plików oraz tunelowanie innych usług
\end{itemize}

\subsubsection{Domain Name System}

DNS umożliwia mapowanie nazwy na adres IP (lub wiele adresów IP) oraz przechowywanie dodatkowych informacji na temat domeny i znajdujących się w niej usług.

Domeny posiadają budowę hierarchiczną / drzewiastą:
\begin{itemize}
	\item precyzja rośnie od prawej do lewej
	\item kolejne poziomy oddzielane są kropkami
	\item najwyższym poziomem jest kropka będąca ostatnim znakiem w pełnej nazwie domenowej (np. \texttt{ciekawi.icm.edu.pl\textbf{\color{red}{.}}}), którą najczęściej pomija się w zapisie
	\item hierarchia ta jest niezależna od hierarchii routingu i wynika z faktu posiadania/użytkowania danej (pod)domeny)
\end{itemize}
Realizacja odpowiedzi na zapytanie DNS wygląda następująco:
\begin{enumerate}
	\item host kieruje zapytanie do określonego w jego konfiguracji serwera "rozwijającego" DNS (DNS resolver),
	\item serwer taki sprawdza w swojej pamięci podręcznej czy zna odpowiedź na to zapytanie (i nie jest ona przeterminowana - nie upłynął czas TTL od odnalezienia), jeżeli nie ma jej w swojej pamięci to
	\item serwer taki zna adresy głównych serwerów DNS (root serwerów) zawierających informacje na temat serwerów obsługujących domeny najwyższego rzędu i kieruje do jednego z nich zapytanie o serwer obsługujący skrajnie prawą część adresu (np. \textit{.org}),
	\item do otrzymanego serwera kierowane jest zapytanie o większą część adresu (np. \textit{eu.org}),
	\item itd. aż do uzyskania odpowiedzi o pytany adres
\end{enumerate}

\begin{teacherOnly}
pokazać jak działa serwer dns robiąc ręcznie zapytania dig'iem o kolejne poziomy:

\begin{Verbatim}
dig NS .
dig @g.root-servers.net. NS pl.
dig @a-dns.pl. NS edu.pl.
dig @a-dns.pl. NS icm.edu.pl.
dig @ns1.agh.edu.pl. NS ciekawi.icm.edu.pl.
# dostaliśmy   CNAME www2.icm.edu.pl.  oraz brak wpisu NS
dig @ns1.agh.edu.pl. A www2.icm.edu.pl.
# dostaliśmy 213.135.59.55
\end{Verbatim}
\end{teacherOnly}

\inputSingleAsFigure[.9]{booklets-sections/network/ilustracje/20-dns.tex}{Realizacja zapytania o rekord DNS}{ilustracja_dns}

DNS przechowuje informacje w postaci rekordów mających określony typ (w większości przypadków dla danej nazwy domenowej może być zdefiniowanych wiele rekordów, tego samego lub innych typów).
Wśród najważniejszych typów rekordów należy wymienić:
\begin{itemize}
	\item \Verb@NS@   – informacja o serwerach obsługujących DNS danej domeny
	\item \Verb@A@    – mapowanie nazwy na adres IPv4
	\item \Verb@AAAA@ – mapowanie nazwy na adres IPv4
	\item \Verb@MX@   – informacja o serwerach obsługujących pocztę danej domeny
	\item \Verb@SRV@  – informacje o hoście świadczącym usługę w tej domenie (usługa określana jest w nazwie domeny o którą pytamy)
	\item \Verb@PTR@  – mapowanie adresów IP na nazwy domenowe, realizowane w specjalnym drzewie \Verb@in-addr.arpa@ (dla IPv4) lub \Verb@ip6.arpa@ (IPv6),
	                    gdzie adres IP zapisywany jest w odwróconej kolejności po bajcie dla IPv4 lub cyfrze szesnastkowej dla IPv6
	                    \teacher{Pokazać wynik polecenia \texttt{host} z adresem IPv4 i IPv6}
	\item \Verb@TXT@  – informacje dodatkowe (np. jakie serwery pocztowe, są upoważnione do wysyłania poczty z tej domeny)
\end{itemize}

\subsubsection{Standardowe numery portów}

Popularne usługi (np. www) posiadają ustalone standardowe numery portów na których nasłuchiwac będzie serwer takiej usługi (np. dla wspomnainego www jest to port 80). Informacja o numerze portu usługi może być umieszczona także w rekordzie SRV systemu DNS.
% END: Polularne usługi

% Copyright (c) 2017-2020 Matematyka dla Ciekawych Świata (http://ciekawi.icm.edu.pl/)
% Copyright (c) 2017-2020 Robert Ryszard Paciorek <rrp@opcode.eu.org>
% 
% MIT License
% 
% Permission is hereby granted, free of charge, to any person obtaining a copy
% of this software and associated documentation files (the "Software"), to deal
% in the Software without restriction, including without limitation the rights
% to use, copy, modify, merge, publish, distribute, sublicense, and/or sell
% copies of the Software, and to permit persons to whom the Software is
% furnished to do so, subject to the following conditions:
% 
% The above copyright notice and this permission notice shall be included in all
% copies or substantial portions of the Software.
% 
% THE SOFTWARE IS PROVIDED "AS IS", WITHOUT WARRANTY OF ANY KIND, EXPRESS OR
% IMPLIED, INCLUDING BUT NOT LIMITED TO THE WARRANTIES OF MERCHANTABILITY,
% FITNESS FOR A PARTICULAR PURPOSE AND NONINFRINGEMENT. IN NO EVENT SHALL THE
% AUTHORS OR COPYRIGHT HOLDERS BE LIABLE FOR ANY CLAIM, DAMAGES OR OTHER
% LIABILITY, WHETHER IN AN ACTION OF CONTRACT, TORT OR OTHERWISE, ARISING FROM,
% OUT OF OR IN CONNECTION WITH THE SOFTWARE OR THE USE OR OTHER DEALINGS IN THE
% SOFTWARE.

\begin{teacherOnly}
\noindent Porozmawiać także o:
\begin{easylist}[itemize]
	& adresach URL
	& tunelowaniu ruchu – pakiet IP może zawierać inny pakiet IP
		&& pakiety zagnieżdżane są jeden w drugim ... danymi pakietu UDP lub TCP może być cokolwiek ... może to być też pakiet IP
		&& \Verb#ssh -L port:hostB:portB  host#\\
			połączenie do \Verb#localhost:port#  zostanie przekierowane przez SSH do \Verb#hostB:portB#  dostępnego z host (serwera ssh)
		&& \Verb#ssh -R port:hostB:portB  host#\\
			połączenie do \Verb#host:port#  zostanie przekierowane przez SSH do \Verb#hostB:portB# dostępnego z localhosta (klienta ssh)
			(domyślnie \Verb#host# będzie odbierał połączenia na ten port tylko od samego siebie – będzie słuchał tylko na swoim localhost)
		&& \Verb#ssh -D 8080  host#\\
			dynamiczny – proxy SOCKS4 / SOCKS5 
		&&\Verb# ssh -w 1:3  host#\\
			utworzy na kliencie \Verb#tun1# i serwerze urządzenia \Verb#tun3# i użyje ich do zestawienia tunelu (ruch wysłany na \Verb#tun1# bedzie docierał do \Verb#tun3# i na odwrót tak jakby były fizycznie połączone kablem ... rolę tego kabla pełni ssh)
		&& VPN - analogicznie do ssh -w tylko bardziej automatyczne zestawianie ... np. nie trzeba się martwić o konflikty numerków tunX ... dba o to serwer VPN ...
	& czemu służą, jak działają (ogólnie) protokoły routingu dynamicznego
	& standardach Internetu - dokumenty RFC
		&& standardy de facto
		&& poważne i mniej (gołębie, ...)
		&& dość dobrze się czyta
\end{easylist}
\end{teacherOnly}

% Copyright (c) 2017-2020 Matematyka dla Ciekawych Świata (http://ciekawi.icm.edu.pl/)
% Copyright (c) 2017-2020 Robert Ryszard Paciorek <rrp@opcode.eu.org>
% 
% MIT License
% 
% Permission is hereby granted, free of charge, to any person obtaining a copy
% of this software and associated documentation files (the "Software"), to deal
% in the Software without restriction, including without limitation the rights
% to use, copy, modify, merge, publish, distribute, sublicense, and/or sell
% copies of the Software, and to permit persons to whom the Software is
% furnished to do so, subject to the following conditions:
% 
% The above copyright notice and this permission notice shall be included in all
% copies or substantial portions of the Software.
% 
% THE SOFTWARE IS PROVIDED "AS IS", WITHOUT WARRANTY OF ANY KIND, EXPRESS OR
% IMPLIED, INCLUDING BUT NOT LIMITED TO THE WARRANTIES OF MERCHANTABILITY,
% FITNESS FOR A PARTICULAR PURPOSE AND NONINFRINGEMENT. IN NO EVENT SHALL THE
% AUTHORS OR COPYRIGHT HOLDERS BE LIABLE FOR ANY CLAIM, DAMAGES OR OTHER
% LIABILITY, WHETHER IN AN ACTION OF CONTRACT, TORT OR OTHERWISE, ARISING FROM,
% OUT OF OR IN CONNECTION WITH THE SOFTWARE OR THE USE OR OTHER DEALINGS IN THE
% SOFTWARE.

% BEGIN: Diagnostyka sieci
\section{Diagnostyka sieci}

Istnieje wiele poleceń służących do diagnozowania ewentualnych problemów sieciowych lub mogących być w tym przydatnymi.
Poniżej znajduje się zestawienie najbardziej popularnych / użytecznych narzędzi z podziałem wg zastosowań.

\subsection{Adresy}
\begin{itemize}
	\item \Verb#ipcalc# oraz \Verb#sipcalc# –
		kalkulator IP (pozwalający na obliczanie adresów sieci rozgłoszeniowych, zmianę notacji itd)
	\item \Verb#whois# –
		informacje z bazy whois (o domenie lub adresie IP)
\end{itemize}
				
\subsection{Dostępność i trasy do hostów}
\begin{itemize}
	\item \Verb#ping [opcje] host# lub \Verb#ping6 [opcje] host# –
		sprawdzanie dostępności hosta z użyciem protokołu ICMP
		(obecnie komenda \Verb@ping6@ najczęściej jest równoważna poleceniu ping z opcją \Verb@-6@ wymuszającą używanie jedynie IPv6,
		na starszych systemach komenda ping może nie obsługiwać adresów IPv6 i wtedy konieczne jest stosowanie do nich polecenia \Verb@ping6@),
		ważniejsze opcje:\\
		\Verb@-c n@ wykonaj n zapytań (domyślnie pyta do momentu przerwania przy pomocy np. Ctrl-C, lub sygnału wysłanego z uzyciem komendy \Verb@kill@)\\
		\Verb@-n@ nie zamieniaj adresu IP hosta który odpowiedział na nazwę domenową
		
	\item \Verb#traceroute#, \Verb#traceroute6#, \Verb#tracepath#, \Verb#tracepath6#, \Verb#tcptraceroute# lub \Verb#tcptraceroute6# – 
		sprawdzanie ścieżki do hosta (wypisanie listy routerów przez które przechodzi pakiet w drodze do wskazanego hosta)\\
		Istnieją różne warianty tych poleceń (nawet pod tą samą nazwą), różnią się one stosowanymi mechanizmami i domyślnymi opcjami.
		Generalnie wszystkie uruchamia się na zasadzie \Verb@polecenie [opcje] host@.
		Warianty z \Verb@6@ na końcu nazwy będą używały jedynie adresów IPv6, natomiast polecenia bez \Verb@6@ na końcu nazwy mogą potrafić ich używać lub nie.
		Wszystkie popularne warianty pozwalają na podanie opcji \Verb@-n@ wyłączającej zamienianie adresu IP hosta który odpowiedział na jego nazwę domenową.\\
		Może zdarzyć się że śledzenie urwie się na jakimś hoście (np. z powodu jego konfiguracji lub błędów w jego oprogramowaniu sieciowym),
		może się zdarzyć że przy użyciu innej komendy z tej grupy (lub zmianie opcji) uda się prześledzić dalszą trasę pakietu.
	\item \Verb#mtr [opcje] host# –
		sprawdzanie ścieżki do hosta (czyli podobnie jak traceroute i tracepath) w trybie ciągłym (z ciągłym odświeżaniem)
		wraz z wypisywaniem informacji o stratach pakietów i opóźnieniach na poszczególnych odcinkach, ważniejsze opcje:\\
		\Verb@-n@ nie zamieniaj adresu IP hosta który odpowiedział na nazwę domenową
	
	\item \Verb#nmap# –
		skaner sieciowy - sprawdzanie dostępnych hostów w sieci, otwartych portów, itd
	\item \Verb#arping# –
		narzędzie do pingowania z wykorzystaniem zapytań ARP zamiast ICMP\\
		istnieją dwie zasadnicze odmiany: z iputils oraz z synscan; ta druga zawarta w debianowym pakiecie "arping" umożliwia także pingowanie po adresie MAC (ale nie przez RARP, bo on nie do tego służy), aby to jednak działało host docelowy nie może ignorować pingów rozgłoszeniowych, metoda obejścia opisana jest w README arping'a
	
	\item \Verb#arp-scan# –
		wyszukiwanie hostów w oparciu o zapytania ARP (można powiedzieć że jest to równoważne uruchamianiu komendy arping w pętli)
\end{itemize}

\subsection{DNS}
\begin{itemize}
	\item \Verb#dig [opcje] nazwa [typRekordu]# –
		narzędzia do uzyskania informacji z DNS,
		pozwala na określenie poprzez \Verb#@adres# serwera który chcemy odpytać oraz na określenie (poprzez drugi argument) typu rekordu który chcemy uzyskać,
		zamiast typu rekordu można podać: \Verb@ANY@ (powoduje odpytanie o wszystkie rekordy) lub \Verb@AXFR@ (powoduje wysłanie prośby o transfer całej strefy, działa jeżeli dany host ma prawo transferu całej strefy z danego serwera)
	\item \Verb#host [opcje] nazwa|ip [server]# –
		narzędzia do zamiany adresów domenowych na IP i odwrotnie oraz wyciągania innych informacji z DNS (np. rekordy MX)
	\item \Verb#nslookup [opcje] nazwa|ip [server]# –
		narzędzia do zamiany adresów domenowych na IP i odwrotnie oraz wyciągania innych informacji z DNS (np. rekordy MX)
	
	\item \Verb#dnstracer# –
		śledzenie trasy zapytań DNS
	\item \Verb#dnswalk# –
		debuger DNS
\end{itemize}

\subsection{IPv6}
\begin{itemize}
	\item \Verb#ndisc6# –
		testowanie ICMPv6 Neighbor Discovery
	\item \Verb#rdisc6# –
		testowanie ICMPv6 Router Discovery
	\item \Verb#rltraceroute6# –
		trasa pakietów do danego hosta IPv6 z użyciem UDP/ICMP
	\item \Verb#tcpspray6# –
		pomiar prędkości łącza z użyciem TCP/IP Discard/Echo
	
	\item \Verb#na6# / \Verb#ns6# –
		wysyłanie pakietów Neighbor Advertisement / Solicitation
	\item \Verb#ra6# / \Verb#rs6# –
		wysyłanie pakietów Router Advertisement / Solicitation
	\item \Verb#ni6# / \Verb#rd6# –
		wysyłanie pakietów ICMPv6 Node Information / Redirect
	\item \Verb#scan6# –
		skanowanie sieci IPv6
\end{itemize}

\subsection{debugowanie łączności sieciowej}
\begin{itemize}
	\item \Verb#netcat# lub \Verb#nc# lub \Verb#netcat6# –
		program pozwalający na wysyłanie pakietów TCP i UDP z definiowaną przez nas zawartością, oraz odbiór pakietów TCP i UDP (słuchanie na wskazanym porcie), umożliwia m.in. testowanie usług sieciowych (takich jak smtp, www, jabber, ...); uwaga: występuje w kilku wersjach różniących się opcjami
	\item \Verb#telnet# –
		program umożliwiający zdalny (nieszyfrowany, łącznie z hasłem!) dostęp do powłoki, a także (podobnie jak netcat) m.in. testowanie usług sieciowych
	
	\item \Verb#swaks# –
		narzędzie do testowania SMTP
	
	\item \Verb#tcpdump# –
		przechwytuje komunikację sieciową celem analizy nagłówków lub pełnej zawartości pakietów
		(wsparcie dla niektórych z protokołów warst wyższych wymaga doinstalowania - np. obsługę DHCP zapewnia dhcpdump)
	\item \Verb#wireshark# lub \Verb#tshark# –
		przechwytuje komunikację sieciową celem analizy nagłówków lub pełnej zawartości pakietów, wspiera dekodowanie wielu protokołów warstwy aplikacyjnej, wireshark posiada graficzny interfejs użytkownika, tshark jest wersją konsolową
\end{itemize}

\subsection{informacje o wykorzystaniu i prędkości sieci}
\begin{itemize}
	\item \Verb#netstat# –
		 informacje o sieci
		 (np. \Verb#netstat -l46np | sort -t / -k 2# wypisze informację o nasłuchujących (po IPv4 lub IPv6) usługach posortowane po nazwie procesu)
	\item \Verb#iptraf# –
		monitor IP LAN
	\item \Verb#nload# –
		graficzne (ncurses) pokazywanie wykorzystania (prędkości) interfejsów sieciowych
	
	\item \Verb#ttcp# –
		testuje prędkość połączenia sieciowego (strona domowa, najnowsza wersja oraz mutacja)
	\item \Verb#iperf# –
		pomiar prędkości połączenia sieciowego
\end{itemize}
% END: Diagnostyka sieci

\ifdefined\inputOnlyContent\else
	\documentclass[tikz]{standalone}
	\usepackage{tikzPackets}
	\InputIfFileExists{booklets-sections/network/preambule.tex}{}{}
	\begin{document}
\fi

\begin{minipage}[t]{45\packetsBitWidth}\hspace{-0.5\packetsBitWidth}\begin{tikzpicture}[semithick]
	\packetsInit
	\packetsBitWidth=6mm
	\tikzstyle{protocolsField}=[protocolsFieldBase, draw, minimum height = 1.5cm, align=center]
	\tikzstyle{info}=[protocolsFieldBase, draw, minimum height = 0.5cm, font=\footnotesize]
	
	\packetsPrintBitScale{21}
		\node [bitScaleInfo, minimum width = 6\packetsBitWidth] (bitScaleInfo_22) [] at (bitScaleInfo_21.north east) {\packetsPrintBitNumber{....}};
		\draw[thin] (bitScaleInfo_22.south west) |- (bitScaleInfo_22.north west);
		
		\foreach \n[remember=\n as \lastn (initially 22)] in {23,...,39} {
			\node [bitScaleInfo] (bitScaleInfo_\n) [] at (bitScaleInfo_\lastn.north east) { };
			\draw[thin] (bitScaleInfo_\n.south west) |- (bitScaleInfo_\n.north west);
		}
	
	
	\packetsPutField{8}{Preambuła + SFD}
	\packetsPutField{6}{Adres docelowy}
	\packetsPutField{6}{Adres źródłowy}
	\packetsPutField{2}{Typ}
	\packetsPutField{6}{Zawartość}
	\packetsPutField{4}{Suma\\Kontrolna}
	\packetsPutField{12}{Przerwa międzypakietowa}
	\packetsNextLine
	
	\packetsPutField[info]{8}{64}
	\packetsPutField[info]{6}{48}
	\packetsPutField[info]{6}{48}
	\packetsPutField[info]{2}{16}
	\packetsPutField[info]{6}{46·8 – 1500·8}
	\packetsPutField[info]{4}{32}
	\packetsPutField[info]{12}{96}
	\packetsEndLine{długość\\(bity)}
	
	\packetsPutField[protocolsFieldBase, minimum height = 0.5cm]{8}{}
	\packetsPutField[info]{24}{ramka L2}
	
	\packetsNextLine
	\packetsPutField[protocolsFieldBase, align=left]{1}{
		\\\vspace{-10pt}\\
		Preambuła wraz z SFD (start frame delimiter) i przerwa międzypakietowa są transmitowane na kablu (są elementem ramki L1),\\
			ale nie wchodzą w skład ramki L2.
		\\\vspace{-7pt}\\
		Typ identyfikuje zawartość przenoszoną w pakiecie (gdy ≥ 0x600) lub określa długość pakietu.\\
		Jeżeli wynosi 0x8100 to ramka należy do tagowanego VLANu i ma postać:
		\\\vspace{-13pt}\\
	}
	\packetsNextLine
	
	\packetsPutField{8}{Preambuła + SFD}
	\packetsPutField{6}{Adres docelowy}
	\packetsPutField{6}{Adres źródłowy}
	\packetsPutField{2}{\footnotesize 0x81\\\footnotesize 0x00}
	\packetsPutField{2}{Tag}
	\packetsPutField{2}{Typ}
	\packetsPutField{6}{Zawartość}
	\packetsPutField{4}{Suma\\Kontrolna}
	\packetsPutField{12}{Przerwa międzypakietowa}
	\packetsNextLine
	
	\packetsPutField[info]{8}{64}
	\packetsPutField[info]{6}{48}
	\packetsPutField[info]{6}{48}
	\packetsPutField[info]{2}{16}
	\packetsPutField[info]{2}{16}
	\packetsPutField[info]{2}{16}
	\packetsPutField[info]{6}{46·8 – 1500·8}
	\packetsPutField[info]{4}{32}
	\packetsPutField[info]{12}{96}
	\packetsEndLine{długość\\(bity)}
	
	\packetsNextLine
	\packetsPutField[protocolsFieldBase, align=left]{1}{
		\\\vspace{-7pt}\\
		Gdzie Typ identyfikuje właściwą zawartość przenoszoną w pakiecie (gdy ≥ 0x600) lub określa długość pakietu.\\
		Natomiast Tag zawiera m.in. 12 bitowy numer VLANu 802.1q\\
		W analogiczny sposób mogą zostać dodane kolejne tagi używane w standardzie IEEE 802.1ad.
	}
	\packetsNextLine
	
\end{tikzpicture}\end{minipage}

\ifdefined\inputOnlyContent\else\end{document}\fi

% Copyright (c) 2017-2020 Matematyka dla Ciekawych Świata (http://ciekawi.icm.edu.pl/)
% Copyright (c) 2017-2020 Robert Ryszard Paciorek <rrp@opcode.eu.org>
% 
% MIT License
% 
% Permission is hereby granted, free of charge, to any person obtaining a copy
% of this software and associated documentation files (the "Software"), to deal
% in the Software without restriction, including without limitation the rights
% to use, copy, modify, merge, publish, distribute, sublicense, and/or sell
% copies of the Software, and to permit persons to whom the Software is
% furnished to do so, subject to the following conditions:
% 
% The above copyright notice and this permission notice shall be included in all
% copies or substantial portions of the Software.
% 
% THE SOFTWARE IS PROVIDED "AS IS", WITHOUT WARRANTY OF ANY KIND, EXPRESS OR
% IMPLIED, INCLUDING BUT NOT LIMITED TO THE WARRANTIES OF MERCHANTABILITY,
% FITNESS FOR A PARTICULAR PURPOSE AND NONINFRINGEMENT. IN NO EVENT SHALL THE
% AUTHORS OR COPYRIGHT HOLDERS BE LIABLE FOR ANY CLAIM, DAMAGES OR OTHER
% LIABILITY, WHETHER IN AN ACTION OF CONTRACT, TORT OR OTHERWISE, ARISING FROM,
% OUT OF OR IN CONNECTION WITH THE SOFTWARE OR THE USE OR OTHER DEALINGS IN THE
% SOFTWARE.

\section{Konfiguracja sieci w Linuxie}

Konfigurację interfejsów sieciowych w systemie Linux umożliwia polecenie \Verb$ip$. Przykłady użycia (ta lista w żaden sposób nie wyczerpuje dostępnych możliwości i dodatkowych opcji):
\begin{itemize}
	\item wyświetlanie i ustawianie adresów IP
		\begin{itemize}
			\item \Verb{ip addr} – wypisuje obecną konfigurację adresów i informacje o stanie interfejsu
			                 (\Verb{UP}/\Verb{DOWN} – interfejs włączony/wyłączony,
			                  \Verb{LOWER_UP}/\Verb{LOWER_DOWN} – link warstwy niższej na interfejsie / jego brak)
			\item \Verb{ip addr add ADDRESS dev INTERFACE} – dodaje adres \Verb{ADDRESS} do interfejsu \Verb{INTERFACE}
			\item \Verb{ip addr del ADDRESS dev INTERFACE} – usuwa adres \Verb{ADDRESS} z interfejsu \Verb{INTERFACE}
		\end{itemize}
	\item włączanie i wyłaczanie interfejsów
		\begin{itemize}
			\item \Verb{ip link set INTERFACE up} / \Verb{ip link set INTERFACE down} – włączenie / wyłączenie interfejsu \Verb{INTERFACE}
			\item \Verb{ip link set INTERFACE address ADDRESS} – ustawienie adresu sprzętowego urządzenia \Verb{INTERFACE} na \Verb{ADDRESS}
		\end{itemize}
	\item konfiguracja tagowanych VLANów
		\begin{itemize}
			\item \Verb{ip link add link INTERFACE name INTERFACE.VLANID type vlan id VLANID} – dodanie interfejsu związanego z tagowanym VLANem o numerze \Verb{VLANID} na interfejsie \Verb{INTERFACE}, moduł 8021q powinien zostać załadowany automatycznie
			\item \Verb{ip link del INTERFACE.VLANID type vlan} – usunięcie interfejsu INTERFACE.VLANID (związanego z tagowanym VLANem VLANID na interfejsie INTERFACE)
		\end{itemize}
	\item konfiguracja BRIDGE (programowego switcha)
		\begin{itemize}
			\item \Verb{ip link add INTERFACE type bridge} – dodanie interfejsu bridge'owego o nazwie INTERFACE
			\item \Verb{ip link set SLAVE master INTERFACE}  – włączenie interfejsu SLAVE w skład bridge'owego INTERFACE
			\item \Verb{ip link set SLAVE nomaster} – wyłaczenie interfejsu SLAVE z bridge'a
			\item \Verb{ip link show master INTERFACE} – wyświetlanie portów składowych bridge'a o nazwie INTERFACE
			\item przydatne może być także polecenie \Verb{bridge}
		\end{itemize}
	\item konfiguracja BONDów (interfejsów agregujących inne w grupę celem zwiększenia prędkości lub niezawodności)
		\begin{itemize}
			\item \Verb{ip link add INTERFACE type bond} – dodanie interfejsu bondingowego o nazwie INTERFACE
			\item \Verb{ip link set SLAVE master INTERFACE}  – włączenie interfejsu SLAVE w skład bondingu INTERFACE
			\item \Verb{ip link set SLAVE nomaster} - wyłaczenie interfejsu SLAVE z bondingu
			\item \Verb{ip link show master INTERFACE} – wyświetlanie portów składowych interfejsu bondingowego o nazwie INTERFACE
		\end{itemize}
	\item konfiguracja routingu
		\begin{itemize}
			\item \Verb{ip [-6] route} – wyświetlanie informacji na temat tras routingowych dla IPv4 (gdy wywołany bez opcji \Verb{-6}) / IPv6 (gdy wywołany z opcją \Verb{-6})
			\item \Verb{ip route add NETWORK via GATEWAY dev INTERFACE} – dodanie trasy routingowej do sieci \Verb{NETWORK} poprzez router o adresie \Verb{GATEWAY} na interfejsie \Verb{INTERFACE}
			\item \Verb{ip route del NETWORK via GATEWAY dev INTERFACE} – usunięcie trasy routingowej do sieci \Verb{NETWORK} ...
		\end{itemize}
\end{itemize}

Często dostępne są także klasyczne polecenia:
\begin{itemize}
	\item \Verb{ifconfig}
		włączanie i wyłączanie interfejsów sieciowych (up i down), ustawianie adresu IP i wyświetlanie informacji o interfejsach.
	\item \Verb{route}
		konfiguracja tras routingowych
	\item \Verb{vconfig}
		dodawanie i usuwanie obsługi wskazanych VLANów z danego interfejsu
	\item \Verb{brctl}
		konfiguracja programowego switcha ethernetowego pomiędzy interfejsami (bridge)
	\item \Verb{ifenslave}
		konfiguracja bondów
\end{itemize}

Innym przydatnym poleceniem z pakietu iproute2 jest \Verb#tc#, które służy do konfiguracji ustawień kontroli przepływu (np. kolejkowania) na interfejsach sieciowych.

Warto zaznaczyć iż konfiguracja dokonywana poleceniami takimi jak \Verb#ip#, \Verb#tc#, \Verb#ifconfig#, itp. jest konfiguracją typu \textit{runtime}, czyli jest tracona po wyłączeniu systemu.
Aby konfiguracja sieci była trwała należy polecenia takie zapisać w postaci skryptu uruchamianego przy starcie systemu lub skorzystać z systemowych plików konfiguracyjnych związanych z siecią.


\subsection{Konfiguracja DNS}

Za zamianę nazw domenowych na adresy IP odpowiadają funkcje biblioteki standardowej C. Korzysta ona do tego celu z konfiguracji zawartej w pliku \Verb#/etc/resolv.conf#.
Powinien on zawierać co najmniej jeden wpis postaci \Verb#nameserver ip_serwera_dns#, określający serwer rozwiązujący nazwy DNS do którego będziemy kierować nasze zapytania.
Wpisów tych może być kilka co pozwala na określenie serwerów uzywany w przypadku niedostępności podstawowego (obecnie używane są maksymalnie 3).

Dodatkowo plik ten może posiadać wpisy
	\Verb#domain# określający domenę lokalną (jeżeli nie jest tu określona a hostname zawiera domenę to używana jest ta z hostname; jeżeli nie chcemy używać można określić na \Verb#.#) oraz
	\Verb#search# określający listę domen do przeszukiwania.
Określają one domeny, króre będą dodawane jako surfix do domeny o którą się pytamy. Na przykład gdy mamy \Verb#domain abc.def#, a pytamy się o \Verb#xyz# (bez kropki w środku lub na końcu), biblioteka najpierw spróbuje ustalić adres \Verb#xyz.abc.def.# a potem \Verb#xyz.#

Plik ten pozwala ustawić także inne opcje związane z odpytywaniem DNS - szczegóły w \Verb#man 5 resolv.conf#.

Innym plikiem związanym z rozwijaniem nazw jest \Verb#/etc/hosts#, który stanowi bazę mapowań nazw na numery IP.
Jest on użyteczny dla lokalnie definiowanych nazw i adresów.
Wpisy w nim zawarte mają priorytet wyższy od informacji z DNS (jeżeli host został znaleziony w tym pliku nie jest wykonywane zapytanie do serwera rozwijającego DNS).


\subsection{konfiguracja automatyczna}

W zależności od ustawień sieci do której podłączony jest konfigurowany host możliwe jest także skorzystanie z konfiguracji automatycznej DHCP i/lub autokonfiguracji IPv6.

\subsubsection{DHCP}

DHCP jest protokołem typu klient-serwer, pozwalającym klientowi uzyskać informacje na temat konfiguracji sieci takie jak adres ip, długość prefixu, trasy routingowe (w szczególności adres bramki domyślnej), adresy serwerów DNS zarówno dla IPv4, jak i IPv6.

Do pobrania konfiguracji z serwera DHCP i jej ustawienia służy najczęściej polecenie \Verb#dhclient# (dostępne są inne implementacje klienta dhcp, np: \Verb#udhcpc#, \Verb#dhcpcd#).
Z ważniejszych opcji należy wspomnieć o:\\
	\hspace*{1cm} \Verb#-6# – korzystanie z DHCPv6, czyli DHCP dla protokołu IPv6,\\
	\hspace*{1cm} \Verb#-n# – nie ustawianie / używanie pobranej konfiguracji,\\
	\hspace*{1cm} \Verb#-d# – nie przechodzenie w tło (włącza też \Verb#-v#),\\
	\hspace*{1cm} \Verb#-v# – wypisywanie większej informacji o działaniu programu.

Dostępne są też różne narzędzia diagnostyczne związane z DHCP, np: \Verb#dhcping#, \Verb#dhcp-probe#.
Linux może pełnić także funkcję serwera DHCP, przy pomocy aplikacji takich jak np.: 
	\textit{isc-dhcp-server}, \textit{udhcpd}, \textit{dnsmasq}, \textit{odhcp6c}, \textit{dhcpy6d}, \textit{wide-dhcpv6}.

\subsubsection{IPv6 autoconf}

Innym sposobem automatycznej konfiguracji interfejsów sieciowych, wprowadzonym w IPv6 jest autokonfiguracja w oparciu o adresy link-local generowane w oparciu o MAC adres karty sieciowej.
	Polega ona na tym że dla podsieci będących LAN'em przydzielana jest pula z maską /64 co umożliwia tworzenie unikalnych numerów IP w oparciu o (niepowtarzalne) numery sprzętowe MAC.
	64 bitowy prefiks sieci jest informacją rozgłaszaną przy pomocy ICMPv6 przez routery (mechanizm radvd), a host dokleja do niego część go identyfikującą związaną z adresem link-local.
	Radvd rozgłasza także informacje routingowe (takie jak adres bramy - dhcpv6 tego nie potrafi), niestety nie da się rozgłaszać w ten sposób innej od standardowej dla LAN długości prefixu.

Linux domyślnie ma włączony ten mechanizm, można go jednak wyłączyć poprzez \Verb#echo 0 > /proc/sys/net/ipv6/conf/${IFACE}/autoconf#, gdzie \Verb#${IFACE}# oznacza interfejs na którym chcemy wyłączyć ten mechanizm.


\subsection{Konfiguracja w proc}

Konieczne / przydatne może być dokonywanie pewnych ustawień poprzez jądrowe systemy plików \Verb$/proc$ i \Verb$/sys$.
Najczęstszym przypadkiem jest włączenie przekazywania pakietów pomiędzy interfejsami poprzez:

\begin{minted}{bash}
for f in /proc/sys/net/ipv*/conf/*/forwarding; do echo 1 > $f; done
\end{minted}
(powyższy jednolinijkowiec włącza forwading pakietów IP dla IPv4 i IPv6 na wszystkich interfejsach)

Innym przykładem jest pokazane wcześniej wyłączenie automatycznej konfiguracji IPv6, przydatne gdy chcemy korzystać tylko z ręcznie przydzielanych adresów.

Z opisem poszczególnych ustawień w ramach systemu \Verb$/proc/sys$ (w tym tych poświęceonych obsłudze sieci z \Verb$/proc/sys/net$) można zapoznać się m.in. na stronie \url{https://sysctl-explorer.net/}.


\subsection{Filtracja pakietów}

Oprócz wyżej omówionej konfiguracji interfejsów i tras routingowych, często potrzebna jest konfiguracja jądrowych mechanizmów filtracji pakietów.

Filtracja pakietów umożliwia m.in. ignorowanie (\textit{drop}) lub odrzucenie z komunikatem błędu (\textit{reject}) niepożądanego ruchu sieciowego
	– zarówno wchodzącego, wychodzącego, jak i przekazywanego (jeżeli uruchomiona jest funkcjonalność routera poprzez wpisanie wartości 1 do \Verb$/proc/sys/net/ipv*/conf/*/forwarding$).
Pozwala także na śledzenie połączeń (np. w celu innego traktowania połączeń nawiązanych niż nowych)
	i manipulowanie przechodzącymi pakietami (np. modyfikację adresów IP i numerów portów w ramach mechanizmów \textit{snat} modyfikującego adresy źródłowe i \textit{dnat} modyfikujące adresy docelowe).

Do konfiguracji filtracji pakietów służy polecenie \Verb#nft#.
Na starszych systemach \Verb#nft# może być niedostępny, wtedy można korzystać z poleceń:
\begin{itemize}
	\item \Verb{iptables}, \Verb{ip6tables}
		konfiguracja filtrów działających na pakietach IP (\Verb{iptables} dla IPv4, \Verb{ip6tables} dla IPv6), filtracja może odbywać się m.in. w oparciu o źródłowe i docelowe adresy IP, numery portów, protokół warstwy transportowej, interfejsy oraz mechanizm śledzenia połączeń; umożliwia także konfigurację translacji adresów (NAT).
	\item \Verb{ebtables}
		konfiguracja filtrów działających na poziomie switcha ethernetowego, filtracja może odbywać się m.in. w oparciu o źródłowe i docelowe interfejsy i adresy sprzętowe.
	\item \Verb{arptables}
		konfiguracja filtrów związanych z protokołem ARP (zamiany adresów IP na adresy sprzętowe)
\end{itemize}

Konfiguracja filtracji pakietów dokonywana z użyciem poleceń \Verb#nft#, \Verb#iptables# jest konfiguracją \textit{runtime} i jest tracona po wyłączeniu systemu.

% Copyright (c) 2017-2020 Matematyka dla Ciekawych Świata (http://ciekawi.icm.edu.pl/)
% Copyright (c) 2017-2020 Robert Ryszard Paciorek <rrp@opcode.eu.org>
% 
% MIT License
% 
% Permission is hereby granted, free of charge, to any person obtaining a copy
% of this software and associated documentation files (the "Software"), to deal
% in the Software without restriction, including without limitation the rights
% to use, copy, modify, merge, publish, distribute, sublicense, and/or sell
% copies of the Software, and to permit persons to whom the Software is
% furnished to do so, subject to the following conditions:
% 
% The above copyright notice and this permission notice shall be included in all
% copies or substantial portions of the Software.
% 
% THE SOFTWARE IS PROVIDED "AS IS", WITHOUT WARRANTY OF ANY KIND, EXPRESS OR
% IMPLIED, INCLUDING BUT NOT LIMITED TO THE WARRANTIES OF MERCHANTABILITY,
% FITNESS FOR A PARTICULAR PURPOSE AND NONINFRINGEMENT. IN NO EVENT SHALL THE
% AUTHORS OR COPYRIGHT HOLDERS BE LIABLE FOR ANY CLAIM, DAMAGES OR OTHER
% LIABILITY, WHETHER IN AN ACTION OF CONTRACT, TORT OR OTHERWISE, ARISING FROM,
% OUT OF OR IN CONNECTION WITH THE SOFTWARE OR THE USE OR OTHER DEALINGS IN THE
% SOFTWARE.

\subsubsection{nft (nftables)}

Polecenie \Verb#nft list ruleset# pozwala na wylistowanie wszystkich reguł.

\begin{figure}[h!]\begin{center}\begin{adjustbox}{scale=.9}
\inputFileContent{booklets-sections/network/ilustracje/51-nft.tex}
\end{adjustbox}\end{center}
\caption{Trasa pakietu przez filtry nftables. Wskazano punkty zaczepień dla łańcuchów reguł.}
\end{figure}

\paragraph{Tabele, łańcuchy i reguły}
\begin{itemize}
	\item Reguły (\Verb#rule#) grupowane są w łańcuchy (\Verb#chains#) w ramach których przetwarzane są kolejno (do momentu napotkania reguły kończącej przetwarzanie pakietu).
	\item Łańcuchy grupowane są w tabele (\Verb#table#).
	\item Każda tabela ma określoną rodzinę obsługiwanych adresów (\Verb#family#), mogą to być:
	\begin{itemize}
		\item \Verb#inet#   (osobne lub wspólne reguły dla IPv4 i IPv6),
			\item \Verb#ip#  (reguły tylko dla IPv4),
			\item \Verb#ip6# (reguły tylko dla IPv6),
		\item \Verb#arp#    (reguły dla warstwy L2 przetwarzane przed uruchomieniem procesowania IP),
		\item \Verb#bridge# (reguły przetwarzane dla pakietów przechodzących przez softwerowy bridge),
		\item \Verb#netdev# (reguły przetwarzane w momencie wejścia ruchu na urządzenie sieciowe, urządzenie musi być określone dla łańcucha reguł, może być alternatywą dla \Verb#tc#).
		% https://wiki.nftables.org/wiki-nftables/index.php/Nftables_families
	\end{itemize}
	\item Tabel danej rodziny może być wiele, stosowane będą łańcuchu z wszystkich tych tabel (odpowiednio do ich parametrów).
	\item Tabele dla różnych rodzin mogą mieć taką samą nazwę.
\end{itemize}

\paragraph{Kierowanie ruchu do reguł}
\begin{itemize}
	\item Ruch do łańcucha może być kierowany jawnie przez regułę w innym łańcuchu lub automatycznie w oparciu o parametry danego łańcucha: typ (\Verb#type#), punkt zaczepienia (\Verb#hook#) i priorytet (\Verb#priority#).
	\item Pasujące łańcuchy (o tym samym punkcie zaczepienia) będą przetwarzane kolejno wg priorytetów do momentu napotkania reguły kończącej przetwarzanie pakietu w którymś z tych łańcuchów (lub przetworzenia wszystkich reguł).
	\item Podstawowym typem łańcucha jest \Verb#filter#. Dodatkowo mogą być użyte typy:
	\begin{itemize}
		\item \Verb#nat# –
			translacja adresów sieciowych w oparciu o śledzenie połączenie (\Verb#conntrack#),
			reguły przetwarzają tylko pierwszy pakiet połączenia, pozostałe przetwarza utworzony wpis \Verb#conntrack#,
			typ może być użyty jedynie w łańcuchach tabel związanych z protokołami IP (inet, ip, ip6) z wyjątkiem łańcucha \Verb#forward#
		\item \Verb#route# –
			zaakceptowanie w takim powoduje wyszukanie nowej trasy routingowej,
			typ może być użyty jedynie w łańcuchach wyjściowych (zaczepionych w \Verb#output#) tabel związanych z protokołami IP (inet, ip, ip6)
	\end{itemize}
	\item Dostępne punkty zaczepienia reguł zależą od rodziny:
	\begin{itemize}
		\item dla \Verb#inet#, \Verb#ip#, \Verb#ip6# i \Verb#bridge# są to:
			prerouting
			input
			forward
			output
			postrouting
		\item dla \Verb#arp# są to:
			input
			output
		\item dla \Verb#netdev# są to:
			ingress
	\end{itemize}
	\item Priorytet jest określany swobodnie i może być wartością ujemny lub dodatnią.
		Warto mieć świadomość iż śledzenie pakietów (\Verb#conntrack#) na wejściu ma priorytet -200 (jest robione przed większością innych reguł) a na wyjściu 300 (jest robione po większości innych reguł).
\end{itemize}

\paragraph{Konstrukcja poleceń}
\ 

Polecenia nft można konstruować na kilka sposobów.
%
Możemy dla każdego elementu tworzonego firewalla wywoływać polecenie nft – np poniższe polecenie utworzy tablicę ABC, w niej łańcuch XYZ i w nim jedną dwie reguły (akceptację ruchu wchodzącego interfejsem "eth0" i eth1):

\begin{CodeFrame*}[bash]{}
nft 'add table ip ABC';  nft 'add chain ip ABC XYZ'
nft 'add rule  ip ABC XYZ iifname "eth0" accept'
nft 'add rule  ip ABC XYZ iifname "eth1" accept'
\end{CodeFrame*}

Możemy też kilka poleceń nft podać w jenym uruchomieniu komendy nft, rozdzielając je średnikiem\footnote{
	Zauważ że popzrednio średnik był poza cudzysłowem (aby bash zinterpretował go jako koniec pierwszej komendy),
	a teraz musi być wewnątrz cudzyłowia lub być zabezpieczony w inny sposób (aby bash nie zinterpretował go jako koniec komendy).
}:

\begin{CodeFrame*}[bash]{}
nft 'add table ip ABC; add chain ip ABC XYZ
     add rule ip ABC XYZ iifname "eth0" accept
     add rule ip ABC XYZ iifname "eth1" accept'
\end{CodeFrame*}

Zauważ że w pierwszej linii mamy dwa polecenia nft rozdielone średnikiem, a w kolejne nie są już nim rozdzielane.
Wynika to z tego że w składni nft – podobnie jak w bashu – średnik może zostać zastąpiony nową linią.

W obu tych wypadkach dodając kolejną regułem musimy każdorazowo powtarać określenie jej lokalizacji (typ tablicy, jej nazwę i nazwę łańcucha).
Aby tego uniknąć można zastoswać zapis z blokami wydzielanymi przy pomocy nawiasów klamrowych:

\begin{CodeFrame*}[bash]{}
nft 'table ip ABC { chain XYZ { iifname "eth0" accept; iifname "eth1" accept; }; };'
\end{CodeFrame*}

Zauważ średniki po każdym z wewnętrznych poleceń i po klamerkach kończących bardziej zewnętrzne polecenia.
W przypadku zapisu wieloliniowego, gdyby występował tam znak nowej linii mogłyby być one pominięte.

Wszystkie powyższe zapisy generują identyczny układ reguł firewalla.
Zapis ten można jeszcze bardziej skompresować, ale uzyskamy wtedy też bardziej skompresowaną regułę firewalla:

\begin{CodeFrame*}[bash]{}
nft 'table ip ABC { chain XYZ { iifname {"eth0", "eth1"} accept; }; };'
\end{CodeFrame*}

\paragraph{Pliki konfiguracyjne}

\begin{Verbatim}
#!/usr/sbin/nft -f
flush ruleset

table inet filter {
	chain INPUT {
		type filter hook input priority 0; policy drop;
		
		# lo and established / invalid connections
		iifname "lo" accept
		ct state {established, related} accept
		ct state invalid reject
		
		# icmp, igmp
		meta l4proto icmp icmp type timestamp-request reject
		meta l4proto {icmp, ipv6-icmp, igmp} accept
		
		# ssh
		ip  saddr 10.40.0.0 tcp dport ssh accept
		ip6 saddr {
			2001:db8:0:a17::123,
			2001:db8:0:1313::/64
		} tcp dport ssh accept
		
		# reject all other packets with ICMP error
		reject
	}
}
\end{Verbatim}

\noindent Zauważ że zamiast powtarzać regułę dla każdego adresu:
\begin{Verbatim}
	ip6 saddr 2001:db8:0:a17::123 tcp dport ssh accept
	ip6 saddr 2001:db8:0:1313::/64 tcp dport ssh accept
\end{Verbatim}
możemy podać zbiór parametrów (np. adresów) w klamerkach w ramach jednej reguły (tak jak pokazano powyżej).
Możliwe jest także definiowanie zbiorów adresów (\Verb$set$) i odwoływanie się do nich z użyciem \Verb$@nazwa$.

Podobnie jak przy wpisywaniu poleceń nftables bespośrednio w argumentach polecenia nft, także w plikach konfiguracyjnych możemy zapisywać je zarówno w „notacji klamerkowej” (jak powyżej) jak i ciągu kolejnych poleceń - np.:

\begin{Verbatim}
#!/usr/sbin/nft -f
add table ip filter
add chain ip filter POST {type nat hook postrouting priority 100; policy accept;}
add rule ip filter POST oifname "ens4" ip saddr 10.40.0.0/24 snat to 213.135.50.250
\end{Verbatim}

\noindent Co tworzy maskowanie adresów IP z 10.40.0.0/24 na 213.135.50.250 dla ruchu wychodzącego interfejsem ens4 i jest równoważne:

\begin{Verbatim}
#!/usr/sbin/nft -f
table ip filter {
	chain POST {
		type nat hook postrouting priority 100; policy accept;
		oifname "ens4" ip saddr 10.40.0.0/24 snat to 213.135.50.250
	}
}
\end{Verbatim}

\vspace{5pt}\noindent
Od wersji 0.9.2 nftables możliwe jest też tworzenie wspólnych reguł dla udp i tcp w następujący sposób:
\\\Verb$add rule inet filter INPUT meta l4proto {tcp, udp} th dport domain$

\insertZadanie{booklets-sections/network/zadania-50_60.tex}{wlacz_forward}{}

\ifdefined\inputOnlyContent\else
	\documentclass[tikz]{standalone}
	\usetikzlibrary{positioning,arrows.meta}
	\begin{document}
\fi

\begin{tikzpicture}[->, >={Stealth[length=8pt,width=6pt]}, node distance=0.6cm, semithick]
	\tikzstyle{invisibleNode}=[inner sep=0, outer sep = 0pt, minimum size=0]
	\tikzstyle{info1}=[above, align=center]
	\tikzstyle{info2}=[below=0.8cm, align=center]
	\tikzstyle{base}=[align=center, minimum height=3.3em, minimum width=8.6em]
	\tikzstyle{inout}=[base]
	\tikzstyle{routing}=[draw, base]
	\tikzstyle{table_raw}=[draw, base]
	\tikzstyle{table_mangle}=[draw, base]
	\tikzstyle{table_nat}=[draw, base]
	\tikzstyle{table_filter}=[draw, base]
	
	\node[inout] (INPUT) {pakiet\\przychodzący};
	\node[table_raw] (RAW_PREROUTING) [below = of INPUT] {raw\\PREROUTING};
	\node[routing] (TRACK1) [below = of RAW_PREROUTING] {śledzenie\\stanu połączeń};
	\node[table_mangle] (MANGLE_PREROUTING) [below = of TRACK1] {mangle\\PREROUTING};
		\node[table_nat] (NAT_PREROUTING) [right = 2.5cm of MANGLE_PREROUTING] {nat\\PREROUTING};
		\node[routing] (ROUTING1) [below = of NAT_PREROUTING] {wybór trasy\\routingowej};
	
	\node[table_mangle] (MANGLE_INPUT) [below = 2.7cm of MANGLE_PREROUTING] {mangle\\INPUT};
		\node[table_mangle] (MANGLE_FORWARD) [below = 1.7cm of ROUTING1] {mangle\\FOWRARD};
	\node[table_filter] (FILTER_INPUT) [below = of MANGLE_INPUT] {filter\\INPUT};
	\node[table_nat] (NAT_INPUT) [below = of FILTER_INPUT] {nat\\INPUT};
	\node[inout] (RECIVE) [below = of NAT_INPUT] {przetwarzanie\\lokalne};
		\node[table_filter] (FILTER_FORWARD) [below = of MANGLE_FORWARD] {filter\\FORWARD};
	
	\node[inout] (GENERATE) [right = 7.6cm of INPUT] {pakiet\\generowany lokalnie};
	\node[routing] (ROUTING2) [below = of GENERATE] {wybór trasy\\routingowej};
	\node[table_raw] (RAW_OUTPUT) [below = of ROUTING2] {raw\\OUTPUT};
	\node[routing] (TRACK2) [below = of RAW_OUTPUT] {śledzenie\\stanu połączeń};
	\node[table_mangle] (MANGLE_OUTPUT) [below = of TRACK2] {mangle\\OUTPUT};
	\node[table_nat] (NAT_OUTPUT) [below = of MANGLE_OUTPUT] {nat\\OUTPUT};
	\node[routing] (ROUTING3) [below = of NAT_OUTPUT] {wybór trasy\\routingowej};
	\node[table_filter] (FILTER_OUTPUT) [below = of ROUTING3] {filter\\OUTPUT};
	\node[table_mangle] (MANGLE_POSTROUTING) [below = of FILTER_OUTPUT] {mangle\\POSTROUTING};
	
	\node[table_nat] (NAT_POSTROUTING) [below = 1.7cm of MANGLE_POSTROUTING] {nat\\POSTROUTING};
	\node[inout] (OUTPUT) [below = of NAT_POSTROUTING] {pakiet\\wychodzący};
	
	\draw (INPUT) edge (RAW_PREROUTING);
	\draw (RAW_PREROUTING) edge (TRACK1);
	\draw (TRACK1) edge (MANGLE_PREROUTING);
		\draw (MANGLE_PREROUTING) edge node[info1] {nie z\\localhost} (NAT_PREROUTING);
		\draw (MANGLE_PREROUTING) node[info2] {z localhost} edge (MANGLE_INPUT);
	\draw (NAT_PREROUTING) edge (ROUTING1);
		\draw (ROUTING1) node[info2] {nie do localhost} edge (MANGLE_FORWARD);
		\node[invisibleNode] (ROUTING1a) [left = 3.6cm of ROUTING1] {};
		\draw[-] (ROUTING1) edge node[info1] {do localhost} (ROUTING1a);
		\draw (ROUTING1a) -| (MANGLE_INPUT);
	\draw (MANGLE_INPUT) edge (FILTER_INPUT);
	\draw (FILTER_INPUT) edge (NAT_INPUT);
	\draw (NAT_INPUT) edge (RECIVE);
	
	\draw (MANGLE_FORWARD) edge (FILTER_FORWARD);
	\draw (FILTER_FORWARD) |- (MANGLE_POSTROUTING);
	
	\draw (GENERATE) edge (ROUTING2);
	\draw (ROUTING2) edge (RAW_OUTPUT);
	\draw (RAW_OUTPUT) edge (TRACK2);
	\draw (TRACK2) edge (MANGLE_OUTPUT);
	\draw (MANGLE_OUTPUT) edge (NAT_OUTPUT);
	\draw (NAT_OUTPUT) edge (ROUTING3);
	\draw (ROUTING3) edge (FILTER_OUTPUT);
	\draw (FILTER_OUTPUT) edge (MANGLE_POSTROUTING);
	
	\draw (MANGLE_POSTROUTING) node[info2] {nie do localhost} edge (NAT_POSTROUTING);
	\node[invisibleNode] (MANGLE_POSTROUTINGa) [right = 0.7cm of MANGLE_POSTROUTING] {};
	\node[invisibleNode] (OUTPUTa) [right = 0.7cm of OUTPUT] {};
	\draw[-] (MANGLE_POSTROUTINGa) -| (MANGLE_POSTROUTING);
	\draw[-] (MANGLE_POSTROUTINGa) edge node[below,rotate=90] {do localhost} (OUTPUTa);
	\draw (OUTPUTa) |- (OUTPUT);
	\draw (NAT_POSTROUTING) edge (OUTPUT);
\end{tikzpicture}

\ifdefined\inputOnlyContent\else\end{document}\fi

% Copyright (c) 2017-2020 Matematyka dla Ciekawych Świata (http://ciekawi.icm.edu.pl/)
% Copyright (c) 2017-2020 Robert Ryszard Paciorek <rrp@opcode.eu.org>
% 
% MIT License
% 
% Permission is hereby granted, free of charge, to any person obtaining a copy
% of this software and associated documentation files (the "Software"), to deal
% in the Software without restriction, including without limitation the rights
% to use, copy, modify, merge, publish, distribute, sublicense, and/or sell
% copies of the Software, and to permit persons to whom the Software is
% furnished to do so, subject to the following conditions:
% 
% The above copyright notice and this permission notice shall be included in all
% copies or substantial portions of the Software.
% 
% THE SOFTWARE IS PROVIDED "AS IS", WITHOUT WARRANTY OF ANY KIND, EXPRESS OR
% IMPLIED, INCLUDING BUT NOT LIMITED TO THE WARRANTIES OF MERCHANTABILITY,
% FITNESS FOR A PARTICULAR PURPOSE AND NONINFRINGEMENT. IN NO EVENT SHALL THE
% AUTHORS OR COPYRIGHT HOLDERS BE LIABLE FOR ANY CLAIM, DAMAGES OR OTHER
% LIABILITY, WHETHER IN AN ACTION OF CONTRACT, TORT OR OTHERWISE, ARISING FROM,
% OUT OF OR IN CONNECTION WITH THE SOFTWARE OR THE USE OR OTHER DEALINGS IN THE
% SOFTWARE.

\section{Programowanie usług sieciowych}

\subsection{wysyłanie danych po UDP}
\begin{CodeFrame*}[python]{}
import socket, sys

if len(sys.argv) != 3:
  print("USAGE: " + sys.argv[0] + " dstHost dstPort", file=sys.stderr)
  exit(1)

# pobieramy informacje o adresach na które rozwija się podana nazwa hosta / usługi
# dzięki tej funkcji jako dstHost możemy podać zarówno nazwę domenową jak i numer IP 
# (w którejś z standardowych notacji) hosta z którym chcemy się połączyć
# a jako dstPort numer portu lub nazwę usługi z /etc/services
dstAddrInfo = socket.getaddrinfo(sys.argv[1], sys.argv[2], type=socket.SOCK_DGRAM)

# funkcja ta nam zwraca listę dostępnych możliwości połączenia (np. nazwa domenowa
# może rozwijać się na wiele różnych adresów), przekazując do getaddrinfo
# opcjonalny argument type zawęziliśmy ta listę do połączeń UDP (SOCK_DGRAM)

# moglibyśmy próbować kolejnych adresów w pętli, ale w tym prostym przykładzie
# próbujemy użyć jedynie pierwszego ze zwróconych adresów
dstAddrInfo = dstAddrInfo[0]

# otwieramy gniazdo
sfd = socket.socket(dstAddrInfo[0], socket.SOCK_DGRAM)

# wysyłamy dane
sfd.sendto("Ala ma kota".encode(), dstAddrInfo[4])
\end{CodeFrame*}


\subsection{odbiór danych po UDP}
\begin{CodeFrame*}[python]{}
import socket, sys

if len(sys.argv) != 2:
  print("USAGE: " + sys.argv[0] + " listenPort", file=sys.stderr)
  exit(1)

# otwieramy gniazdo
sfd = socket.socket(socket.AF_INET6, socket.SOCK_DGRAM)

# ustawiamy opcję gniazda pozwalającą na jednoczesną obsługę IPv4 i IPv6
sfd.setsockopt(socket.IPPROTO_IPV6, socket.IPV6_V6ONLY, 0)

# ustawiamy adres i port na którym słuchamy
# adres zerowy oznacza słuchanie na wszystkich adresach IP danego hosta
sfd.bind(('::', int(sys.argv[1])))

while True:
  # czekamy na dane, gdy je otrzymamy to odbieramy
  data, sAddr, = sfd.recvfrom(4096)
  # i wypisujemy co i od kogo dostaliśmy
  print("odebrano od", sAddr, ":", data.decode());
\end{CodeFrame*}

\subsection{klient TCP}
\begin{CodeFrame*}[python]{}
import socket, select, sys

if len(sys.argv) != 3:
    print("USAGE: " + sys.argv[0] + " dstHost dstPort", file=sys.stderr)
    exit(1);

# struktura zawierająca adres na który wysyłamy
dstAddrInfo = socket.getaddrinfo(sys.argv[1], sys.argv[2], proto=socket.IPPROTO_TCP)

# mogliśmy uzyskać kilka adresów, więc próbujemy używać kolejnych do skutku
for aiIter in dstAddrInfo:
    try:
        print("try connect to:", aiIter[4])
        # utworzenie gniazda sieciowego ... SOCK_STREAM oznacza TCP
        sfd = socket.socket(aiIter[0], socket.SOCK_STREAM)
        # połączenie ze wskazanym adresem
        sfd.connect(aiIter[4])
    except:
        # jeżeli się nie udało ... zamykamy gniazdo
        if sfd:
            sfd.close()
        sfd = None
        # i próbujemy następny adres
        continue
    break;

if sfd == None:
    print("Can't connect", file=sys.stderr)
    exit(1);

# wysyłanie
sfd.sendall("Ala ma Kota\n".encode())

# czekanie na odbiór i odbiór
while True:
    rdfd, _, _ = select.select([sfd], [], [], 13.0)
    if sfd in rdfd:
        d = sfd.recv(4096)
        d = d.decode()
        print(d, end="")
        
        # odbiór pustego pakietu lub pakietu zawierającego
        # jedynie pustą linię kończy działanie
        if d == "" or d == "\n" or d == "\r\n":
            break
    else:
        # timeout kończy działanie
        break

# zamykanie połączenia
sfd.shutdown(socket.SHUT_RDWR)
sfd.close()
\end{CodeFrame*}

\subsection{serwer TCP}
\begin{CodeFrame*}[python]{}
import socket, select, signal, sys, os

MAX_CHILD = 5
QUERY_SIZE = 3
TIMEOUT = 13
BUF_SIZE = 4096

if len(sys.argv) != 2:
    print("USAGE: " + sys.argv[0] + " listenPort", file=sys.stderr)
    exit(1);

# obsługa sygnału o zakończeniu potomka
childNum = 0
def onChildEnd(s, f):
    print("odebrano sygnał o śmierci potomka")
    global childNum
    childNum -= 1
    os.waitpid(-1, os.WNOHANG);
signal.signal(signal.SIGCHLD, onChildEnd)

# utworzenie gniazd sieciowych ... SOCK_STREAM oznacza TCP
sfd_v4 = socket.socket(socket.AF_INET,  socket.SOCK_STREAM)
sfd_v6 = socket.socket(socket.AF_INET6, socket.SOCK_STREAM)

# ustawienie opcji gniazda ... IPV6_V6ONLY=1 wyłącza korzystanie
# z tego samego socketu dla IPv4 i IPv6
sfd_v6.setsockopt(socket.IPPROTO_IPV6, socket.IPV6_V6ONLY, 1)

# przypisanie adresów ...
# '0.0.0.0' oznacza dowolny adres IPv4 (czyli to samo co INADDR_ANY)
# '::' oznacza dowolny adres IPv6 (czyli to samo co in6addr_any)
sfd_v4.bind(('0.0.0.0', int(sys.argv[1])))
sfd_v6.bind(('::',      int(sys.argv[1])))

# określenie gniazd jako używanych do odbioru połączeń przychodzących
# (długość kolejki połączeń ustawiona na wartość QUERY_SIZE)
sfd_v4.listen(QUERY_SIZE)
sfd_v6.listen(QUERY_SIZE)

# czekanie na połączenia z użyciem select() w nieskończonej pętli
while True:
    sfd, _, _ = select.select([sfd_v4, sfd_v6], [], [])
    for fd in sfd:
        #  odebranie połączenia
        sfd_c, sAddr = fd.accept()
        
        # weryfikacja ilości potomków
        if childNum >= MAX_CHILD:
            print("za dużo potomków - odrzucam połączenie od:", sAddr);
            sfd_c.send("Internal Server Error\r\n".encode())
            sfd_c.close()
            return
        
        # aby móc obsługiwać wiele połączeń rozgałęziamy proces
        pid = os.fork()
        if pid > 0:
            # rodzic - zwiększamy licznik potomków
            childNum += 1
        else:
            # potomek - obsługa danego połączenia
            print("połączenie od:", sAddr)
            while True:
                # czekanie na dane z timeout'em
                # aby zabezpieczyć się przed atakiem DoS
                rd, _, _ = select.select([sfd_c], [], [], TIMEOUT)
                if sfd_c in rd:
                    data = sfd_c.recv(BUF_SIZE)
                    if data:
                        print("odebrano od", sAddr, ":", data.decode());
                        sfd_c.send(data)
                    else:
                        print("koniec połączenia od:", sAddr)
                        break
                else:
                    print("timeout połączenia od:", sAddr)
                    break
            # zamykanie połączenia
            sfd_c.shutdown(socket.SHUT_RDWR)
            sfd_c.close()
            sys.exit()
\end{CodeFrame*}


\section{Zadania}
% Copyright (c) 2017-2020 Matematyka dla Ciekawych Świata (http://ciekawi.icm.edu.pl/)
% Copyright (c) 2017-2020 Robert Ryszard Paciorek <rrp@opcode.eu.org>
% 
% MIT License
% 
% Permission is hereby granted, free of charge, to any person obtaining a copy
% of this software and associated documentation files (the "Software"), to deal
% in the Software without restriction, including without limitation the rights
% to use, copy, modify, merge, publish, distribute, sublicense, and/or sell
% copies of the Software, and to permit persons to whom the Software is
% furnished to do so, subject to the following conditions:
% 
% The above copyright notice and this permission notice shall be included in all
% copies or substantial portions of the Software.
% 
% THE SOFTWARE IS PROVIDED "AS IS", WITHOUT WARRANTY OF ANY KIND, EXPRESS OR
% IMPLIED, INCLUDING BUT NOT LIMITED TO THE WARRANTIES OF MERCHANTABILITY,
% FITNESS FOR A PARTICULAR PURPOSE AND NONINFRINGEMENT. IN NO EVENT SHALL THE
% AUTHORS OR COPYRIGHT HOLDERS BE LIABLE FOR ANY CLAIM, DAMAGES OR OTHER
% LIABILITY, WHETHER IN AN ACTION OF CONTRACT, TORT OR OTHERWISE, ARISING FROM,
% OUT OF OR IN CONNECTION WITH THE SOFTWARE OR THE USE OR OTHER DEALINGS IN THE
% SOFTWARE.

\IfStrEq{\dbEntryID}{}{
	\insertZadanie{\currfilepath}{czy_w_sieci}{}
	\insertZadanie{\currfilepath}{routing}{}
	\insertZadanie{\currfilepath}{ping1}{}
	\insertZadanie{\currfilepath}{traceroute1}{}
	\insertZadanie{\currfilepath}{dns1}{}
	\insertZadanie{\currfilepath}{tcpdump}{}
	\insertZadanie{\currfilepath}{http1}{}
	\insertZadanie{\currfilepath}{http2}{}
	\insertZadanie{\currfilepath}{ping_utf}{}
}


\dbEntryBegin{czy_w_sieci}\if1\dbEntryCheckResults
Ustal czy adres \Verb$2001:db8:0:a17::123$ należy do sieci \Verb$2001:db8::/48$. Możesz posłużyć się narzędziami do obliczania zakresów sieci IP (np. \Verb#sipcalc#) lub obliczyć to ręcznie.
\fi

\dbEntryBegin{czy_w_sieci-rozwiazanie}\if1\dbEntryCheckResults
tak --- zakres sieci 2001:06a0:0000:0000:0000:0000:0000:0000 - 2001:06a0:0000:003f:ffff:ffff:ffff:ffff
\fi


\dbEntryBegin{routing}\if1\dbEntryCheckResults
Wynik polecenia \Verb#ip -6 r# pokazującego tablicę rotingową wygląda następująco:
\begin{Verbatim}
2001:db8:0:21::/64 dev eth1  proto kernel  metric 256 
2001:db8::/32 via 2001:db8:0:21::100 dev eth2  metric 1024 
2001:db8:fff:21::/64 dev eth2  proto kernel  metric 256 
2001:db8:abc:21::/64 via 2001:db8:fff:21::1 dev eth2  metric 1024 
fe80::/64 dev eth1  proto kernel  metric 256 
fe80::/64 dev eth2  proto kernel  metric 256 
default via 2001:db8:0:21::1 dev eth1  metric 1024 
\end{Verbatim}

Ustal gdzie zostanie skierowany pakiet adresowany do \Verb$2001:db8:abc:21:123::ff3$.
\fi

\dbEntryBegin{routing-rozwiazanie}\if1\dbEntryCheckResults
Adres 2001:db8:abc:21:123::ff3 zawiera się w zakresie sieci następujących sieci występujących w tablicy routingu:
\begin{itemize}
	\item 2001:db8::/32
	\item 2001:db8:abc:21::/64 
	\item default (0::/0)
\end{itemize}

Najdłuższy prefix spośród tych sieci ma 2001:db8:abc:21::/64 (64 bity), zatem zostanie użyty wpis w tablicy routingowej dla tej sieci.

W efekcie pakiet zostanie skierowany do routera 2001:db8:fff:21::1 poprzez interfejs eth2.
\fi


\dbEntryBegin{ping1}\if1\dbEntryCheckResults
Korzystając z narzędzi służących do diagnozowania sieci sprawdź czy host \Verb#ciekawi.icm.edu.pl# jest dostępny.
Jakiego polecenia użyłeś(aś) w tym celu? Co jeszcze mówi wynik tego polecenia?

\begin{teacherOnly}
Rozwiązaniem powinno opierać się o użycie polecenia \Verb#ping#, inne działające rozwiązania też akceptujemy, ale komentując/dyskutując naprowadzamy na to że standardowo do tego używa się komendy \Verb#ping#.
\end{teacherOnly}
\fi

\dbEntryBegin{ping1-rozwiazanie}\if1\dbEntryCheckResults
\begin{Verbatim}
$ ping -c4 ciekawi.icm.edu.pl
PING www2.icm.edu.pl (213.135.59.55) 56(84) bytes of data.
64 bytes from www2.icm.edu.pl (213.135.59.55): icmp_seq=1 ttl=60 time=3.91 ms
64 bytes from www2.icm.edu.pl (213.135.59.55): icmp_seq=2 ttl=60 time=3.63 ms
64 bytes from www2.icm.edu.pl (213.135.59.55): icmp_seq=3 ttl=60 time=2.94 ms
64 bytes from www2.icm.edu.pl (213.135.59.55): icmp_seq=4 ttl=60 time=5.03 ms

--- www2.icm.edu.pl ping statistics ---
4 packets transmitted, 4 received, 0% packet loss, time 6ms
rtt min/avg/max/mdev = 2.942/3.875/5.025/0.752 ms
\end{Verbatim}

Polecenie na standardowym wyjściu wypisuje m.in.:
\begin{itemize}
	\item odpytywany adres IP i wynik odwrotnego dns'u dla niego
	\item ilość danych wysyłanych (może być użyteczne do testowania problemów z MTU)
	\item dla każdego zapytania, które dostało odpowiedź:
	\begin{itemize}
		\item ilość przesłanych otrzymanych w pakiecie
		\item serwer który odpowiedział (rev-dns i ip)
		\item numer kolejny zapytania/odpowiedzi
		\item TTL pakietu z odpowiedzią (wskazuje na ilość routerów przez które przeszedł pakiet)
		\item czas potrzebny na uzyskanie odpowiedzi
	\end{itemize}
	\item podsumowujące statystyki związane z ilością zagubionych pakietów i czasami odpowiedzi
\end{itemize}

Zamiast informacji o poszczególnych odpowiedziach polecenie ping może nic nie wypisywać (brak odpowiedzi):
\begin{Verbatim}
$ ping -c 1 1.5.2.1
PING 1.5.2.1 (1.5.2.1) 56(84) bytes of data.

--- 1.5.2.1 ping statistics ---
1 packets transmitted, 0 received, 100% packet loss, time 0ms
\end{Verbatim}
lub podawać komunikat błędu wraz z jego nadawcą (jeżeli taki został zwrócony):
\begin{Verbatim}
$ ping -c1 192.168.6.55
PING 192.168.6.55 (192.168.6.55) 56(84) bytes of data.
From 192.168.6.3 icmp_seq=1 Destination Host Unreachable

--- 192.168.6.55 ping statistics ---
1 packets transmitted, 0 received, +1 errors, 100% packet loss, time 0ms
\end{Verbatim}

Kod powrotu polecenie ping informuje o tym czy host jest osiągalny czy nie, dzięki czemu polecenie to może zostać uzyte np. w bash'owym warunku \Verb#if#:
\begin{Verbatim}
if ping -c 1 10.0.1.1 >& /dev/null; then
	echo "10.0.1.1 jest dostępny"
else
	echo "10.0.1.1 nie jest dostępny"
fi
\end{Verbatim}
\fi


\dbEntryBegin{traceroute1}\if1\dbEntryCheckResults
Korzystając z narzędzi służących do diagnozowania sieci ustal jaką trasą podróżują pakiety z Twojego komputera do \Verb#www.opcode.eu.org# oraz do \Verb#www.example.org#.
Jakiego lub jakich poleceń użyłeś(aś) w tym celu? Co jeszcze mówi ich wynik? Co możesz powiedzieć porównując uzyskane trasy?
\fi
\dbEntryBegin{traceroute1-rozwiazanie}\if1\dbEntryCheckResults
Należy skorzystać z polecenia \Verb#treacroute#, \Verb#tracepath#, \Verb#mtr# lub podobnych.

Wynik polecenia host może wyglądać następująco (ale nie będzie nic dziwnego jeżeli otrzymasz inny – trasa routingowa zależy od miejsca z którego się łączysz, może też zmieniać się z czasem)
\begin{Verbatim}
traceroute  www.example.org
traceroute to www.example.org (93.184.216.34), 30 hops max, 60 byte packets
 1  funbox.home (192.168.6.1)  0.570 ms  9.648 ms  9.624 ms
 2  192.0.0.1 (192.0.0.1)  7.981 ms  8.007 ms  8.082 ms
 3  195.117.0.225 (195.117.0.225)  8.120 ms  8.189 ms  8.223 ms
 4  poz-r1.tpnet.pl (194.204.175.94)  18.474 ms  18.512 ms  18.547 ms
 5  hbg-b1-link.telia.net (213.248.96.144)  23.234 ms  23.268 ms  23.304 ms
 6  hbg-bb4-link.telia.net (213.155.135.86)  110.655 ms hbg-bb3-link.telia.net (213.155.135.80)  104.215 ms hbg-bb4-link.telia.net (213.155.135.86)  104.056 ms
 7  * * ldn-bb4-link.telia.net (62.115.122.161)  101.809 ms
 8  * * *
 9  nyk-b6-link.telia.net (80.91.254.36)  106.471 ms nyk-b6-link.telia.net (62.115.125.63)  104.446 ms  104.562 ms
10  edgecast-ic-317660-nyk-b5.c.telia.net (62.115.147.201)  109.616 ms  109.669 ms  112.148 ms
11  152.195.69.131 (152.195.69.131)  106.924 ms 152.195.68.141 (152.195.68.141)  109.509 ms 152.195.69.139 (152.195.69.139)  106.711 ms
12  93.184.216.34 (93.184.216.34)  103.418 ms  108.218 ms  106.420 ms
13  93.184.216.34 (93.184.216.34)  106.454 ms  106.587 ms  110.858 ms
\end{Verbatim}

Podobnie jak w przypadku polecenia ping wypisywana jest informacja o odpytywanym adresie IP (i wynik odwrotnego dns'u dla niego), ilość wysyłanych danych.
Wypisywane są kolejne routery (ip i rev-dns) wraz z czasami odpowiedzi (domyślnie traceroute na każdym poziomie robi 3 zapytania).
Gwiazdki oznaczają brak odpowiedzi. Jeżeli od pewnego momentu występują same gwiazdki oznacza to że wraz z dalszym zwiększaniem TTL nie dostajemy już kolejnych komunikatów o jego przekroczeniu.
	Może to wynikać z osiągnięcia hostu docelowego, który ma zablokowany testowany port bez wysyłania komunikatu o niedostępności portu lub błędu routera który nie przekazuje poprawnie komunikatów o zbyt małym TTL.
\fi



\dbEntryBegin{dns1}\if1\dbEntryCheckResults
Ustal (wszystkie) adresy IPv4 i IPv6 serwera \Verb#www.bitbucket.org#.
Zastanów się czemu może służyć to że niektóre nazwy domenowe rozwijają się na wiele różnych adresów IP.
\fi

\dbEntryBegin{dns1-rozwiazanie}\if1\dbEntryCheckResults
Należy użyć na przykład polecenia \Verb#host www.bitbucket.org# lub poleceń \Verb#dig AAAA www.bitbucket.org# ; \Verb#dig A www.bitbucket.org#.

Wynik polecenia host może wyglądać następująco (ale nie będzie nic dziwnego jeżeli otrzymasz inne adresy – dane w DNS ulegają zmianom, niekiedy nawet dynamicznie wprowadzanym).
\begin{Verbatim}
www.bitbucket.org is an alias for bitbucket.org.
bitbucket.org has address 104.192.141.1
bitbucket.org has IPv6 address 2406:da00:ff00::22c5:2ef4
bitbucket.org has IPv6 address 2406:da00:ff00::6b17:d1f5
bitbucket.org has IPv6 address 2406:da00:ff00::22e9:9f55
bitbucket.org has IPv6 address 2406:da00:ff00::34cc:ea4a
bitbucket.org mail is handled by 1 aspmx.l.google.com.
bitbucket.org mail is handled by 5 alt1.aspmx.l.google.com.
\end{Verbatim}
Warto zauważyć że zostaliśmy poinformowani także o tym że www.bitbucket.org jest aliasem (CNAME) na bitbucket.org.

Zwracanie wielu IP jest jednym ze sposobów rozkładania obciążenia i zapewnienia redundancji. Innymi rozwiązaniami jest udzielanie różnych odpowiedzi różnym kliemntom (tak robi np. www.google.com):
\begin{Verbatim}
$ host www.google.com
www.google.com has address 172.217.20.164
www.google.com has IPv6 address 2a00:1450:401b:802::2004
\end{Verbatim}
z innego hosta:
\begin{Verbatim}
$ host www.google.com
www.google.com has address 216.58.206.4
www.google.com has IPv6 address 2a00:1450:4001:821::2004
\end{Verbatim}
Jeszcze innym jest używanie mechanizmów routingowych takich jak anycast.
\fi


\dbEntryBegin{tcpdump}\if1\dbEntryCheckResults
Korzystając z dwóch instancji programu \Verb#nc# (\Verb#netcat#) – jednej w roli serwera, drugiej w roli klienta prześlij między nimi jakieś dane.
Użyj programu \Verb#tcpdump# (z odpowiednimi opcjami) aby podsłuchać komunikację sieciową między tymi programami i zobaczyć przesyłane dane.
\fi

\dbEntryBegin{tcpdump-rozwiazanie}\if1\dbEntryCheckResults
Należy uruchomić w osobnych oknach terminala kolejno polecenia:
\begin{enumerate}
	\item \Verb#tcpdump -i lo -A port 5555# - podsłuchujemy komunikację używającą portu 5555
	\item \Verb#nc -lp 5555# - uruchamiamy nasłuch na porcie 5555
	\item \Verb#nc localhost 5555# - łączymy się na port 5555
\end{enumerate}
Dane  wpisywane w jednym z uruchomionych netcat'ów będą wypisywane w drugim i odwrotnie.
Wszystkie te dane (wraz z informacjami zawartymi w nagłówkach IP i TCP) będą wyświetlane w okienku w którym działa tcpdump.

Oczywiście możemy użyć innego numeru portu, itd.
\fi


\dbEntryBegin{http1}\if1\dbEntryCheckResults
Korzystając bezpośrednio z poleceń protokołu HTTP i programu \Verb#nc# (\Verb#netcat#) lub \Verb#telnet#, pobierz i wyświetl kod strony \Verb#http://www.opcode.eu.org/#.

\textit{Wskazówka:
	Opis protokołu HTTP odnajdziesz bez problemu w sieci.\\
	Ogólnie żądanie HTTP składa się z pierwszej linii określającej typ wykonywanej operacji, ścieżkę oraz wersję protokołu - np. \texttt{GET /abc.txt HTTP/1.1} oznacza prośbę o zwrócenie zawartości pliku \texttt{/abc.txt}.
	Następnie podawane są nagłówki, w wersji HTTP 1.1 obowiązkowy jest nagłówek „Host” określający nazwę domenową serwera - np. \texttt{Host: www.example.org}.
	Po nagłówkach występuje pusta linia po której mogą być przesłane (przy niektórych typach żądań) jakieś dane (np. z wypełnionego na stronie formularza).
}
\fi

\dbEntryBegin{http1-rozwiazanie}\if1\dbEntryCheckResults
\begin{Verbatim}
netcat www.opcode.eu.org 80
GET / HTTP/1.1
Host: www.opcode.eu.org

HTTP/1.1 200 OK
Server: nginx/1.14.2
Date: Sun, 26 Apr 2020 09:12:09 GMT
Content-Type: text/html; charset=utf-8
Content-Length: 11490
Last-Modified: Wed, 08 Apr 2020 17:36:32 GMT
Connection: keep-alive
ETag: "5e8e0ba0-2ce2"
Accept-Ranges: bytes

<?xml version="1.0" encoding="UTF-8" ?>
<html xmlns="http://www.w3.org/1999/xhtml" xml:lang="pl" lang="pl">
<head>
        <title>OpCode.eu.org - strona główna</title>
\end{Verbatim}
[...]

\vspace{7pt}
Jeżeli użyjemy innego nagłówka \Verb#Host:# w zapytaniu możemy dostać np. komunikat o przekierowaniu:
\begin{Verbatim}
netcat www.opcode.eu.org 80
GET / HTTP/1.1
Host: opcode.eu.org

HTTP/1.1 301 Moved Permanently
Server: nginx/1.14.2
Date: Tue, 21 Apr 2020 17:58:44 GMT
Content-Type: text/html
Content-Length: 185
Connection: keep-alive
Location: http://www.opcode.eu.org/

<html>
<head><title>301 Moved Permanently</title></head>
<body bgcolor="white">
<center><h1>301 Moved Permanently</h1></center>
<hr><center>nginx/1.14.2</center>
</body>
</html>
\end{Verbatim}
Przeglądarki WWW nie wyświetlają tych komunikatów tylko automatycznie przechodzą pod wskazany w nagłówku \Verb#Location:# adres.
\fi


\dbEntryBegin{http2}\if1\dbEntryCheckResults
Zadanie \ref{http1} można rozwiązać przy pomocy netcat'a bez dodatkowych opcji, jednak jeżeli stroną do pobrania byłoby np. \Verb#http://www.icm.edu.pl# to należałoby skorzystać z opcji \Verb#-C# netcat'a (w przeciwnym razie serwer zwraca błąd 400 "Bad Request").
Sprawdź co robi ta opcja i zastanów się dlaczego w przypadku niektórych serwerów jest konieczna a w przypadku innych nie? Co na ten temat mówi standard HTTP?
\fi

\dbEntryBegin{http2-rozwiazanie}\if1\dbEntryCheckResults
Wynika to z stosowania w większości protokołów sieciowych (w tym w HTTP) jako znaku linii sekwencji dwu bajtowej \Verb#\n\r# (nowa linia, powrót karetki).
Serwer WWW obsługujący domenę opcode.eu.org postępuje według filozofii nakazującej liberalne podejście do danych otrzymywanych i restrykcyjne do generowanych przez siebie (wysyłanych) i poprawnie interpretuje formalnie błędne żądanie zawierające znaki nowej linii w postaci \Verb#\n#.
Serwer obsługujący www.icm.edu.pl nie jest już tak liberalny i wymaga znaków nowej linii w postaci \Verb#\n\r#, dlatego do netcata musimy dodać opcję -C aby konwertował wprowadzone znaki nowej linii na taką postać przed wysłaniem.
\fi

\dbEntryBegin{ping_utf}\if1\dbEntryCheckResults
Zobacz czy rozwiązanie zadania \ref{ping1} zadziała gdy użyjesz nazwy serwera zawierającej polskie znaki: \Verb#licealiści.icm.edu.pl#.
Jak myślisz, dlaczego polskie znaki są tak rzadko używane w nazwach domenowych?
\fi
\dbEntryBegin{ping_utf-rozwiazanie}\if1\dbEntryCheckResults
\begin{Verbatim}
ping -c2 licealiści.icm.edu.pl
PING www2.icm.edu.pl (213.135.59.55) 56(84) bytes of data.
64 bytes from www2.icm.edu.pl (213.135.59.55): icmp_seq=1 ttl=60 time=4.20 ms
64 bytes from www2.icm.edu.pl (213.135.59.55): icmp_seq=2 ttl=60 time=3.26 ms

--- www2.icm.edu.pl ping statistics ---
2 packets transmitted, 2 received, 0% packet loss, time 2ms
rtt min/avg/max/mdev = 3.260/3.732/4.204/0.472 ms
\end{Verbatim}
Jak widać działa. Standard kodowania dowolnych znaków Unicode, tak aby mogły być użyte w nazwach domenowych został opracowany w 2003 roku (Punycode, RFC3492), jest dość powszechnie zaimplementowany.
Możemy użyć go także w pythonowej metodzie encode (np. \Verb#print( "żółw".encode("punycode") )#).

Ciężko powiedzieć dlaczego (przynajmniej w Polsce) znaki z poza ASCII są tak rzadko stosowane w nazwach domenowych.

\textit{Ciekawostka: kodowanie znaków nie ASCII w nazwie domenowej jest określone jednoznacznie, natomiast w dalszej części adresu URL niezbyt - zasadniczo zależy od konfiguracji serwera, ale standardem defacto jest tutaj UTF8).}
\fi

\insertZadanie{booklets-sections/network/zadania-10_20_30.tex}{skrypt_dostepnosc}{}
\insertZadanie{booklets-sections/network/zadania-10_20_30.tex}{ping_utf}{}
\insertZadanie{booklets-sections/network/zadania-10_20_30.tex}{mail_smtp}{}
% Copyright (c) 2017-2020 Matematyka dla Ciekawych Świata (http://ciekawi.icm.edu.pl/)
% Copyright (c) 2017-2020 Robert Ryszard Paciorek <rrp@opcode.eu.org>
% 
% MIT License
% 
% Permission is hereby granted, free of charge, to any person obtaining a copy
% of this software and associated documentation files (the "Software"), to deal
% in the Software without restriction, including without limitation the rights
% to use, copy, modify, merge, publish, distribute, sublicense, and/or sell
% copies of the Software, and to permit persons to whom the Software is
% furnished to do so, subject to the following conditions:
% 
% The above copyright notice and this permission notice shall be included in all
% copies or substantial portions of the Software.
% 
% THE SOFTWARE IS PROVIDED "AS IS", WITHOUT WARRANTY OF ANY KIND, EXPRESS OR
% IMPLIED, INCLUDING BUT NOT LIMITED TO THE WARRANTIES OF MERCHANTABILITY,
% FITNESS FOR A PARTICULAR PURPOSE AND NONINFRINGEMENT. IN NO EVENT SHALL THE
% AUTHORS OR COPYRIGHT HOLDERS BE LIABLE FOR ANY CLAIM, DAMAGES OR OTHER
% LIABILITY, WHETHER IN AN ACTION OF CONTRACT, TORT OR OTHERWISE, ARISING FROM,
% OUT OF OR IN CONNECTION WITH THE SOFTWARE OR THE USE OR OTHER DEALINGS IN THE
% SOFTWARE.

\IfStrEq{\dbEntryID}{}{
	\insertZadanie{\currfilepath}{ustaw_adres}{}
	\insertZadanie{\currfilepath}{ustaw_route}{}
	\insertZadanie{\currfilepath}{wlacz_forward}{}
	\insertZadanie{\currfilepath}{serwer_trojkat}{}
	\insertZadanie{\currfilepath}{udp_echo}{}
}

\IfStrEq{\dbEntryID}{rozwiazania}{
	\insertRozwiazanie{\currfilepath}{ustaw_adres}{}
	\insertRozwiazanie{\currfilepath}{ustaw_route}{}
	\insertRozwiazanie{\currfilepath}{wlacz_forward}{}
	\insertRozwiazanie{\currfilepath}{serwer_trojkat}{}
	\insertRozwiazanie{\currfilepath}{udp_echo}{}
}


\dbEntryBegin{ustaw_adres}\if1\dbEntryCheckResults
Napisz polecenie które ustawi adres ip \Verb#172.33.13.113# (maska sieci to \Verb#255.255.255.0#) na interfejsie \Verb#eth5#.
\fi

\dbEntryBegin{ustaw_adres-rozwiazanie}\if1\dbEntryCheckResults
\begin{Verbatim}
ip addr add 172.33.13.113/24 dev eth5
\end{Verbatim}
\fi


\dbEntryBegin{ustaw_route}\if1\dbEntryCheckResults
Napisz polecenie które ustawi trasę routingową do sieci \Verb#10.13.0.0/16# przez bramkę o adresie ip \Verb#172.33.13.13#.
\fi

\dbEntryBegin{ustaw_route-rozwiazanie}\if1\dbEntryCheckResults
\begin{Verbatim}
ip route add 10.13.0.0/16 via 172.33.13.13
\end{Verbatim}
\fi


\dbEntryBegin{wlacz_forward}\if1\dbEntryCheckResults
Napisz polecenia które włączą przekazywanie pakietów (routing) pomiędzy interfejsami \Verb#eth3# i \Verb#eth4#, ale nie zezwolą na przekazywanie pakietów innymi interfejsami (w tym pakietów inny interfejs $\leftrightarrow$ \Verb#eth3# / \Verb#eth4#).
\\
\textit{Wskazówka: skorzystaj z reguł filtracji pakietów}
\fi

\dbEntryBegin{wlacz_forward-rozwiazanie}\if1\dbEntryCheckResults
\begin{Verbatim}
for f in /proc/sys/net/ipv*/conf/*/forwarding; do echo 1 > $f; done

nft add table ip filter
nft add chain ip filter FORWARD '{ type filter hook forward priority 0; }'
nft add rule  ip filter FORWARD iifname "eth3" oifname "eth4" accept
nft add rule  ip filter FORWARD iifname "eth4" oifname "eth3" accept
nft add rule  ip filter FORWARD reject
\end{Verbatim}
\fi


\dbEntryBegin{serwer_trojkat}\if1\dbEntryCheckResults
Napisz (w Pythonie lub C/C++) serwer TCP, który oczekuje że klient wyśle mu liczbę, w odpowiedzi na którą serwer odeśle do tego klienta trójkąt z gwiazdek odpowiedniej wielkości.
Na przykład dla żądania klienta w postaci "3" powinien to być:
\begin{Verbatim}
*
**
***
\end{Verbatim}
\fi

\dbEntryBegin{serwer_trojkat-rozwiazanie}\if1\dbEntryCheckResults
\begin{Verbatim}
import socket, select, signal, sys, os

MAX_CHILD = 5
QUERY_SIZE = 3
TIMEOUT = 13
BUF_SIZE = 4096

if len(sys.argv) != 2:
    print("USAGE: " + sys.argv[0] + " listenPort", file=sys.stderr)
    exit(1);

# obsługa sygnału o zakończeniu potomka
childNum = 0
def onChildEnd(s, f):
    print("odebrano sygnał o śmierci potomka")
    global childNum
    childNum -= 1
    os.waitpid(-1, os.WNOHANG);
signal.signal(signal.SIGCHLD, onChildEnd)

# utworzenie gniazd sieciowych ... SOCK_STREAM oznacza TCP
sfd_v4 = socket.socket(socket.AF_INET,  socket.SOCK_STREAM)
sfd_v6 = socket.socket(socket.AF_INET6, socket.SOCK_STREAM)

# ustawienie opcji gniazda ... IPV6_V6ONLY=1 wyłącza korzystanie
# z tego samego socketu dla IPv4 i IPv6
sfd_v6.setsockopt(socket.IPPROTO_IPV6, socket.IPV6_V6ONLY, 1)

# przypisanie adresów ...
# '0.0.0.0' oznacza dowolny adres IPv4 (czyli to samo co INADDR_ANY)
# '::' oznacza dowolny adres IPv6 (czyli to samo co in6addr_any)
sfd_v4.bind(('0.0.0.0', int(sys.argv[1])))
sfd_v6.bind(('::',      int(sys.argv[1])))

# określenie gniazd jako używanych do odbioru połączeń przychodzących
# (długość kolejki połączeń ustawiona na wartość QUERY_SIZE)
sfd_v4.listen(QUERY_SIZE)
sfd_v6.listen(QUERY_SIZE)

# czekanie na połączenia z użyciem select() w nieskończonej pętli
while True:
    sfd, _, _ = select.select([sfd_v4, sfd_v6], [], [])
    for fd in sfd:
        #  odebranie połączenia
        sfd_c, sAddr = fd.accept()
        
        # weryfikacja ilości potomków
        if childNum >= MAX_CHILD:
            print("za dużo potomków - odrzucam połączenie od:", sAddr);
            sfd_c.send("Internal Server Error\r\n".encode())
            sfd_c.close()
            break
        
        # aby móc obsługiwać wiele połączeń rozgałęziamy proces
        pid = os.fork()
        if pid > 0:
            # rodzic - zwiększamy licznik potomków
            childNum += 1
        else:
            # potomek - obsługa danego połączenia
            print("połączenie od:", sAddr)
            while True:
                # czekanie na dane z timeout'em
                # aby zabezpieczyć się przed atakiem DoS
                rd, _, _ = select.select([sfd_c], [], [], TIMEOUT)
                if sfd_c in rd:
                    data = sfd_c.recv(BUF_SIZE)
                    if data:
                        print("odebrano od", sAddr, ":", data.decode());
\end{Verbatim}
\begin{minted}[frame=none]{python}
                        # zamist odsyłać odebrane dane poprzez sfd_c.send(data)
                        try:
                            liczba = int(data.decode())
                        except:
                            liczba = 0
                        if liczba < 2 or liczba > 10:
                            sfd_c.send("Błędne polecenie – podaj liczbę >=2 i <=10.\n".encode())
                        else:
                            trojkat = ""
                            for i in range(1, liczba+1):
                                trojkat += "*"*i+"\n"
                            sfd_c.send(trojkat.encode())
                        # generujemy na ich podstawie trójkąt i go odsyłamy używając sfd_c.send
\end{minted}
\begin{Verbatim}
                    else:
                        print("koniec połączenia od:", sAddr)
                        break
                else:
                    print("timeout połączenia od:", sAddr)
                    break
            # zamykanie połączenia
            sfd_c.shutdown(socket.SHUT_RDWR)
            sfd_c.close()
            sys.exit()
\end{Verbatim}
Kolorowaniem kodu wyróżniono zmianę w stosunku co do przykładowego kodu ze skryptu.
\fi


\dbEntryBegin{udp_echo}\if1\dbEntryCheckResults
Powyżej znajdują się przykładowe kody wysyłający dane po UDP ("klient UDP") i odbierający dane po UDP ("serwer UDP")
oraz kod serwera usługi "echo" (odsyłającej odebrane dane do nadawcy) w wariancie TCP, którą omawialiśmy na wykładzie.

W oparciu o te informacje napisz (w Pythonie lub C/C++) program realizujący funkcję serwera echo z użyciem UDP.
\fi

\dbEntryBegin{udp_echo-rozwiazanie}\if1\dbEntryCheckResults
\begin{minted}[frame=none]{python}
import socket, sys

if len(sys.argv) != 2:
    print("USAGE: " + sys.argv[0] + " listenPort", file=sys.stderr)
    exit(1);

sfd = socket.socket(socket.AF_INET6, socket.SOCK_DGRAM)
sfd.setsockopt(socket.IPPROTO_IPV6, socket.IPV6_V6ONLY, 0)
sfd.bind(('::', int(sys.argv[1])))

while True:
    data, sAddr, = sfd.recvfrom(4096)
    print("odebrano od", sAddr, ":", data.decode());
    sfd.sendto(data, sAddr)
\end{minted}
\fi



\section{Zadania dodatkowe}
\insertZadanie{booklets-sections/network/zadania_dodatkowe.tex}{czy_w_sieci_ipv4}{}
\insertZadanie{booklets-sections/network/zadania_dodatkowe.tex}{adresy_serwerow_dns}{}
\insertZadanie{booklets-sections/network/zadania_dodatkowe.tex}{trasy_pakietow}{}
\insertZadanie{booklets-sections/network/zadania_dodatkowe.tex}{kalkulator_ipv4}{}

\insertZadanie{booklets-sections/network/zadania_dodatkowe.tex}{ustaw_adres}{}
\insertZadanie{booklets-sections/network/zadania_dodatkowe.tex}{ustaw_route}{}
\insertZadanie{booklets-sections/network/zadania_dodatkowe.tex}{wlacz_forward}{}
\insertZadanie{booklets-sections/network/zadania_dodatkowe.tex}{ustaw_route}{}

\insertZadanie{booklets-sections/network/zadania_dodatkowe.tex}{rfc1924}{}


\rozwiazania

\copyrightFooter{
	© Matematyka dla Ciekawych Świata, 2017-2020.\\
	© Robert Ryszard Paciorek <rrp@opcode.eu.org>, 2003-2020.\\
	Wykorzystano grafiki należące do domeny publicznej.\\
	Kopiowanie, modyfikowanie i redystrybucja dozwolone pod warunkiem zachowania informacji o autorach.
}
\end{document}
