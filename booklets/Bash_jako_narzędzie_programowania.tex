\documentclass{pdfBooklets}

\title {Linux i sieci: Bash jako narzędzie programowania}
\author{%
	Projekt ,,Matematyka dla Ciekawych Świata'',\\
	Robert Ryszard Paciorek\\\normalsize\ttfamily <rrp@opcode.eu.org>
}
\date  {2020-04-08}

\makeatletter\hypersetup{
	pdftitle = {\@title}, pdfauthor = {\@author}
}\makeatother

\begin{document}

\maketitle

\section{Podstawy}
\input{booklets-sections/linux/podstawy_programowania_w_bash.tex}
\subsection{Zadania}
% Copyright (c) 2017-2020 Matematyka dla Ciekawych Świata (http://ciekawi.icm.edu.pl/)
% Copyright (c) 2017-2020 Robert Ryszard Paciorek <rrp@opcode.eu.org>
% 
% MIT License
% 
% Permission is hereby granted, free of charge, to any person obtaining a copy
% of this software and associated documentation files (the "Software"), to deal
% in the Software without restriction, including without limitation the rights
% to use, copy, modify, merge, publish, distribute, sublicense, and/or sell
% copies of the Software, and to permit persons to whom the Software is
% furnished to do so, subject to the following conditions:
% 
% The above copyright notice and this permission notice shall be included in all
% copies or substantial portions of the Software.
% 
% THE SOFTWARE IS PROVIDED "AS IS", WITHOUT WARRANTY OF ANY KIND, EXPRESS OR
% IMPLIED, INCLUDING BUT NOT LIMITED TO THE WARRANTIES OF MERCHANTABILITY,
% FITNESS FOR A PARTICULAR PURPOSE AND NONINFRINGEMENT. IN NO EVENT SHALL THE
% AUTHORS OR COPYRIGHT HOLDERS BE LIABLE FOR ANY CLAIM, DAMAGES OR OTHER
% LIABILITY, WHETHER IN AN ACTION OF CONTRACT, TORT OR OTHERWISE, ARISING FROM,
% OUT OF OR IN CONNECTION WITH THE SOFTWARE OR THE USE OR OTHER DEALINGS IN THE
% SOFTWARE.

\IfStrEq{\dbEntryID}{}{
	\insertZadanie{\currfilepath}{petla_linki_html}{}
	\insertZadanie{\currfilepath}{warunek_istnienie_pliku}{}
	\insertZadanie{\currfilepath}{funkcja_n_razy_napis}{}
	\insertZadanie{\currfilepath}{funkcja_liczba_kotow}{}
}

% BEGIN: podstawy programowania w bashu - zadania
\dbEntryBegin{petla_linki_html}\if1\dbEntryCheckResults
Napisz pętle, która wypisze wszystkie pliki nieukryte z bieżącego katalogu w postaci linków HTML, czyli:
dla pliku o nazwie \Verb{ABC} powinna wypisać \Verb{<a href="ABC">ABC</a>}. Przedstaw zarówno rozwiązanie z użyciem pętli \Verb{for}, jak i pętli \Verb{while}.
\fi

\dbEntryBegin{warunek_istnienie_pliku}\if1\dbEntryCheckResults
Napisz warunek, który sprawdzi czy \Verb{/tmp/abc} istnieje i jest katalogiem.
\fi

\dbEntryBegin{funkcja_n_razy_napis}\if1\dbEntryCheckResults
Napisać funkcję przyjmującą dwa argumenty - liczbę i napis; funkcja ma wypisać napis tyle razy ile wynosi podana liczba.
\fi

\dbEntryBegin{funkcja_liczba_kotow}\if1\dbEntryCheckResults
Napisać funkcję przyjmującą jeden argument - liczbę kotów i wypisującą:
\begin{itemize}
	\item "Ala ma kota" dla ilości kotów równej 1
	\item "Ala ma x koty" lub "Ala ma x kotów" gdzie dobrana jest poprawna forma, a pod x podstawiona podana w argumencie ilość kotów.
\end{itemize}
Dla uproszczenia należy założyć że podana ilość kotów jest w zakresie od 1 do 9.
\fi
% END: podstawy programowania w bashu - zadania


\section{Przetwarzanie napisów}
% Copyright (c) 2017-2020 Matematyka dla Ciekawych Świata (http://ciekawi.icm.edu.pl/)
% Copyright (c) 2017-2020 Robert Ryszard Paciorek <rrp@opcode.eu.org>
% 
% MIT License
% 
% Permission is hereby granted, free of charge, to any person obtaining a copy
% of this software and associated documentation files (the "Software"), to deal
% in the Software without restriction, including without limitation the rights
% to use, copy, modify, merge, publish, distribute, sublicense, and/or sell
% copies of the Software, and to permit persons to whom the Software is
% furnished to do so, subject to the following conditions:
% 
% The above copyright notice and this permission notice shall be included in all
% copies or substantial portions of the Software.
% 
% THE SOFTWARE IS PROVIDED "AS IS", WITHOUT WARRANTY OF ANY KIND, EXPRESS OR
% IMPLIED, INCLUDING BUT NOT LIMITED TO THE WARRANTIES OF MERCHANTABILITY,
% FITNESS FOR A PARTICULAR PURPOSE AND NONINFRINGEMENT. IN NO EVENT SHALL THE
% AUTHORS OR COPYRIGHT HOLDERS BE LIABLE FOR ANY CLAIM, DAMAGES OR OTHER
% LIABILITY, WHETHER IN AN ACTION OF CONTRACT, TORT OR OTHERWISE, ARISING FROM,
% OUT OF OR IN CONNECTION WITH THE SOFTWARE OR THE USE OR OTHER DEALINGS IN THE
% SOFTWARE.

% BEGIN: Wbudowane przetwarzanie napisów w bashu
\subsection{Wbudowane przetwarzanie napisów w bash'u}

Wbudowane przetwarzanie napisów w bashu opiera się na odwołaniach do zmiennych w postaci \Verb@${}@:

\begin{itemize}
	\item \shell@${zmienna:-"napis"}@ zwróci napis gdy zmienna nie jest zdefiniowana lub jest pusta
	\item \shell@${zmienna:="napis"}@ zwróci napis oraz wykona podstawienie zmienna="napis" gdy zmienna nie jest zdefiniowana lub jest pusta
	\item \shell@${zmienna:+"napis"}@ zwróci napis gdy zmienna jest zdefiniowana i nie pusta
	
	\vspace{6pt}
	
	\item \shell@${#str}@    zwróci długość napisu w zmiennej str
	\item \shell@${str:n}@   zwróci pod-napis zmiennej str od n do końca
	\item \shell@${str:n:m}@ zwróci pod-napis zmiennej str od n o długości m
	
	\vspace{6pt}
	
	\item \shell@${str/"n1"/"n2"}@  zwróci wartość str z zastąpionym pierwszym wystąpieniem n1 przez n2
	\item \shell@${str//"n1"/"n2"}@  zwróci wartość str z zastąpionymi wszystkimi wystąpieniami n1 przez n2
	
	\vspace{6pt}
	
	\item \shell@${str#"ab"}@ zwróci wartość str z obciętym "ab" z początku
	\item \shell@${str%"fg"}@ zwróci wartość str z obciętym "fg" z końca
\end{itemize}
W napisach do obcięcia możliwe jest stosowanie shellowych znaków uogólniających, czyli \Verb@*@, \Verb@?@, \Verb@[abc]@, itd operator \Verb@#@ i \Verb@%@ dopasowują minimalny napis do usunięcia, natomiast operatory \Verb@##@ i \Verb@%%@ dopasowują maksymalny napis do usunięcia.

\begin{teacherOnly}
Przykład do zademonstrowania:

\begin{CodeFrame*}[bash]{}
a=""; b=""; c=""
echo ${a:-"aa"} ${b:="bb"} ${c:+"cc"}
echo $a $b $c

a="x"; b="y"; c="z"
echo ${a:-"aa"} ${b:="bb"} ${c:+"cc"}
echo $a $b $c

x=abcdefg
echo ${#x} ${x:2} ${x:0:3} ${x:0:$((${#x}-2))}
echo ${x#"abc"} ${x%"efg"}
echo ${x#"ac"}  ${x%"eg"}

x=abcd.e.fg
echo ${x#*.} ${x##*.} ${x%.*} ${x%%.*}

y="aa bb cc bb dd bb ee"
echo ${y/"bb"/"XX"}
echo ${y//"bb"/"XX"}
\end{CodeFrame*}

\end{teacherOnly}
Możliwe jest także korzystanie z wyrażeń regularnych.
Polecenie \Verb@expr match $x 'wr1\(wr2\)wr3'@ zwróci na stdout (wypisze) część \Verb@$x@ pasującą do wyrażenia regularnego \Verb@wr2@,
wyrażenia regularne \Verb@wr1@ i \Verb@wr2@ pozwalają na określanie części napisu do odrzucenia.
Alternatywną składnią jest \Verb@expr $x : 'wr1\(wr2\)wr3'@

\begin{teacherOnly}
Przykład do zademonstrowania:

\begin{CodeFrame*}[bash]{}
z="ab=cd"
expr match $z '^\([^=]*\)='
expr $z : '^[^=]*=\(.*\)$'
\end{CodeFrame*}
\end{teacherOnly}

Możliwe jest też sprawdzanie dopasowań wyrażeń regularnych poprzez (zwróć uwagę na brak cytowania wyrażenia regularnego):

\begin{CodeFrame*}[bash]{}
[[ "$z" =~ ^([^=]*)= ]] && echo "OK"
\end{CodeFrame*}

Możliwe jest także zaawansowane formatowanie napisów, konwertowanie liczb na napisy,
w tym wypisywanie w różnych systemach liczbowych przy pomocy \Verb@printf@\footnote{
	Instrukcja \Verb@printf@ ma składnię opartą na tej funkcji z C, interpretuje ona także liczby zmiennoprzecinkowe.
}:

\begin{CodeFrame*}[bash]{}
printf "0o%o %d 0x%x\n" 0xf 010 3
\end{CodeFrame*}
% END: Wbudowane przetwarzanie napisów w bashu

\input{booklets-sections/linux/napisy-sed_grep_cut.tex}
\input{booklets-sections/linux/napisy-awk.tex}

\subsection{Zadania dodatkowe}
	% Copyright (c) 2017-2020 Matematyka dla Ciekawych Świata (http://ciekawi.icm.edu.pl/)
% Copyright (c) 2017-2020 Robert Ryszard Paciorek <rrp@opcode.eu.org>
% 
% MIT License
% 
% Permission is hereby granted, free of charge, to any person obtaining a copy
% of this software and associated documentation files (the "Software"), to deal
% in the Software without restriction, including without limitation the rights
% to use, copy, modify, merge, publish, distribute, sublicense, and/or sell
% copies of the Software, and to permit persons to whom the Software is
% furnished to do so, subject to the following conditions:
% 
% The above copyright notice and this permission notice shall be included in all
% copies or substantial portions of the Software.
% 
% THE SOFTWARE IS PROVIDED "AS IS", WITHOUT WARRANTY OF ANY KIND, EXPRESS OR
% IMPLIED, INCLUDING BUT NOT LIMITED TO THE WARRANTIES OF MERCHANTABILITY,
% FITNESS FOR A PARTICULAR PURPOSE AND NONINFRINGEMENT. IN NO EVENT SHALL THE
% AUTHORS OR COPYRIGHT HOLDERS BE LIABLE FOR ANY CLAIM, DAMAGES OR OTHER
% LIABILITY, WHETHER IN AN ACTION OF CONTRACT, TORT OR OTHERWISE, ARISING FROM,
% OUT OF OR IN CONNECTION WITH THE SOFTWARE OR THE USE OR OTHER DEALINGS IN THE
% SOFTWARE.

\IfStrEq{\dbEntryID}{}{
	\insertZadanie{\currfilepath}{passwd_warunek_na_uid}{}
	\insertZadanie{\currfilepath}{kopiowanie_tylko_plikow}{}
	\insertZadanie{\currfilepath}{rekurecyjne_wyszukaj_i_zastap}{}
	\insertZadanie{\currfilepath}{wyszukaj_napis_kopiuj}{}
	
	\insertZadanie{\currfilepath}{zmiana_rozszerzenia}{}
	\insertZadanie{\currfilepath}{pliki_zawierajace_napis}{}
	\insertZadanie{\currfilepath}{parsowanie_cmdline}{}
}



% bash

\dbEntryBegin{passwd_warunek_na_uid}\if1\dbEntryCheckResults
Wyświetl z /etc/passwd linie w których UID (3 pole) ma warość >= 1000 ... jeżeli ktoś ma pomysł to na dwa lub trzy sposoby
\fi

\dbEntryBegin{passwd_warunek_na_uid_noawk}\if1\dbEntryCheckResults
Wyświetl z /etc/passwd linie w których UID (3 pole) ma warość >= 1000 nie korzystając z AWK.
\fi

\dbEntryBegin{kopiowanie_tylko_plikow}\if1\dbEntryCheckResults
Napisz polecenie które skopiuje wszystkie pliki (nie katalogi ani linki symboliczne) z katalogu \Verb{/etc} do \Verb{/tmp}
% wyłączamy linki symboliczne aby uniknąć: cp /etc/* /tmp
\fi

\dbEntryBegin{rekurecyjne_wyszukaj_i_zastap}\if1\dbEntryCheckResults
Napisz funkcję która przyjmuje dwa argumenty - napis wyszukiwany i napis go zastępujący oraz dokonuje rekurencyjnego wyszukania i zamiany tych napisów w wszystkich plikach w bierzącym katalogu.

\textit{Wskazówka 1: polecenie \texttt{sed} z opcją \texttt{-i} i wskazaniem pliku modyfikuje zawartości tego pliku stosownie do poleceń wydanych sed'owi}\\
\textit{Wskazówka 2: dla uproszczenia możesz przyjąć że napisy te składają się jedynie z liter i cyfr.}
\fi

\dbEntryBegin{wyszukaj_napis_kopiuj}\if1\dbEntryCheckResults
Napisz polecenie które przekopiuje wszystkie pliki zawierające słowo \Verb{hostname} z katalogu \Verb{/etc} (wraz z jego podkatalogami) do katalogu \Verb{/tmp/etc}
zachowując strukturę katalogów (czyli plik /etc/a/b kopiowany jest do katalogu /tmp/etc/b).
Przyjmij że katalog \Verb{/tmp/etc} nie istnieje.
\fi

% PwES domowe (?):

\dbEntryBegin{zmiana_rozszerzenia}\if1\dbEntryCheckResults
Napisz polecenie które dla wszystkich plików z rozszerzeniem \Verb{.TXT}  w bierzącym katalogu (bez podkatalogów) dokona zmiany ich nazwy zmieniając rozszerzenie na \Verb{.txt}, zachowując podstawową część nazwy bez modyfikacji.
W rozwiązaniu nie korzystamy z polecenia \Verb{rename}.
\fi

\dbEntryBegin{pliki_zawierajace_napis}\if1\dbEntryCheckResults
Napisz polecenie które wyszuka i przekopiuje do katalogu \Verb{/tmp} pliki z katalogu \Verb{/etc} (wraz z jego podkatalogami), które (w swojej treści) zawierają napis \Verb{nameserver}.
\fi

\dbEntryBegin{parsowanie_cmdline}\if1\dbEntryCheckResults
Plik \Verb{/proc/cmdline} zawiera informację o opcjach przekazanych do jądra podczas startu. Kolejne opcje rozdzielane są spacją, a nazwę opcji od jej argumentu rozdziela znak rówwności.
Napisz polecenie które wypisze argument opcji root. Dla pliku \Verb{/proc/cmdline} postaci:\\
 \Verb{BOOT_IMAGE=/vmlinuz-4.9-amd64 root=UUID=cad866ab-aabd-4686-8376-e4b9f1c2ae9e rw}\\
polecenie powinno wypisać:\\
 \Verb{UUID=cad866ab-aabd-4686-8376-e4b9f1c2ae9e}

\textbf{Uwaga:} nie wolno zakładać że \Verb#root=# jest drugą opcją linii poleceń jądra, nie wolno zakładać że nie ma tam innej opcji kończącej się na \Verb#root=# (np. \Verb#nfsroot=#).
\fi


\vspace{6pt}

\rozwiazania

\copyrightFooter{
	© Matematyka dla Ciekawych Świata, 2017-2020.\\
	© Robert Ryszard Paciorek <rrp@opcode.eu.org>, 2003-2020.\\
	Kopiowanie, modyfikowanie i redystrybucja dozwolone pod warunkiem zachowania informacji o autorach.
}
\end{document}
