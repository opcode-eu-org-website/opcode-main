% Copyright (c) 2020 Robert Ryszard Paciorek <rrp@opcode.eu.org>
% 
% MIT License
% 
% Permission is hereby granted, free of charge, to any person obtaining a copy
% of this software and associated documentation files (the "Software"), to deal
% in the Software without restriction, including without limitation the rights
% to use, copy, modify, merge, publish, distribute, sublicense, and/or sell
% copies of the Software, and to permit persons to whom the Software is
% furnished to do so, subject to the following conditions:
% 
% The above copyright notice and this permission notice shall be included in all
% copies or substantial portions of the Software.
% 
% THE SOFTWARE IS PROVIDED "AS IS", WITHOUT WARRANTY OF ANY KIND, EXPRESS OR
% IMPLIED, INCLUDING BUT NOT LIMITED TO THE WARRANTIES OF MERCHANTABILITY,
% FITNESS FOR A PARTICULAR PURPOSE AND NONINFRINGEMENT. IN NO EVENT SHALL THE
% AUTHORS OR COPYRIGHT HOLDERS BE LIABLE FOR ANY CLAIM, DAMAGES OR OTHER
% LIABILITY, WHETHER IN AN ACTION OF CONTRACT, TORT OR OTHERWISE, ARISING FROM,
% OUT OF OR IN CONNECTION WITH THE SOFTWARE OR THE USE OR OTHER DEALINGS IN THE
% SOFTWARE.

\documentclass{pdfBooklets}

\title {Linux i sieci: Bash jako narzędzie programowania}
\author{%
	Projekt ,,Matematyka dla Ciekawych Świata'',\\
	Robert Ryszard Paciorek\\\normalsize\ttfamily <rrp@opcode.eu.org>
}
\date  {2020-07-09}

\makeatletter\hypersetup{
	pdftitle = {\@title}, pdfauthor = {\@author}
}\makeatother

\begin{document}

\maketitle

\section{Podstawy}
	% Copyright (c) 2017-2020 Matematyka dla Ciekawych Świata (http://ciekawi.icm.edu.pl/)
% Copyright (c) 2017-2020 Robert Ryszard Paciorek <rrp@opcode.eu.org>
% 
% MIT License
% 
% Permission is hereby granted, free of charge, to any person obtaining a copy
% of this software and associated documentation files (the "Software"), to deal
% in the Software without restriction, including without limitation the rights
% to use, copy, modify, merge, publish, distribute, sublicense, and/or sell
% copies of the Software, and to permit persons to whom the Software is
% furnished to do so, subject to the following conditions:
% 
% The above copyright notice and this permission notice shall be included in all
% copies or substantial portions of the Software.
% 
% THE SOFTWARE IS PROVIDED "AS IS", WITHOUT WARRANTY OF ANY KIND, EXPRESS OR
% IMPLIED, INCLUDING BUT NOT LIMITED TO THE WARRANTIES OF MERCHANTABILITY,
% FITNESS FOR A PARTICULAR PURPOSE AND NONINFRINGEMENT. IN NO EVENT SHALL THE
% AUTHORS OR COPYRIGHT HOLDERS BE LIABLE FOR ANY CLAIM, DAMAGES OR OTHER
% LIABILITY, WHETHER IN AN ACTION OF CONTRACT, TORT OR OTHERWISE, ARISING FROM,
% OUT OF OR IN CONNECTION WITH THE SOFTWARE OR THE USE OR OTHER DEALINGS IN THE
% SOFTWARE.

% BEGIN: podstawy programowania w bashu

\subsection{Zmienne}

Określanie typów zmiennych w bashu odbywa się na podstawie wartości znajdującej się w zmiennej.
Zasadniczo wszystkie zmienne są napisami, a interpretacja typu ma miejsce przy ich użyciu (a nie przy tworzeniu).
\teacher{Porównując do Pythona - jest to wyraźnie mniej silne typowanie. }
Obsługiwane są liczby całkowite oraz napisy, bash nie posiada wbudowanej obsługi liczb zmiennoprzecinkowych.

\begin{CodeFrame*}[bash]{}
zmiennaA=-91
zmiennaB="qa z"
zmiennaC=98.6 # to będzie traktowane jako napis a nie liczba
\end{CodeFrame*}
Zwróć uwagę na brak spacji pomiędzy nazwą zmiennej a znakiem równości w operacji przypisania - jest to wymóg składniowy.

Odwołanie do zmiennej odbywa się z użyciem znaku dolara, po którym występuje nazwa zmiennej. Nazwa może być ujęta w klamry, ale nie musi (jest to przydatne gdy nie chcemy dawać spacji pomiędzy nazwą zmiennej a np. fragmentem napisu). Rozwijaniu ulegają nazwy zmiennych znajdujące się w napisach umieszczonych w podwójnych cudzysłowach.

\begin{CodeFrame*}[bash]{}
echo  $zmiennaA ${zmiennaA}AA
echo "$zmiennaA ${zmiennaA}AA"
echo '$zmiennaA ${zmiennaA}AA'
\end{CodeFrame*}

Jeżeli chcemy aby zmienna była widoczna przez programy uruchamiane z naszej powłoki (w tym przez kolejne instancje bash'a, odpowiedzialne np. za wykonywanie kodu skryptu uruchamianego z pliku) należy ją wyeksportować za pomocą polecenia \Verb#export zmienna# (zwróć uwagę na brak dolara w tym miejscu).

\subsection{Podstawowe operacje}
Aby wykonać działania arytmetyczne należy umieścić je wewnątrz \Verb{$((} i \Verb{))}

Dodawanie, mnożenie, odejmowanie zapisuje się i działają one tak jak w normalnej matematyce, dzielenie zapisuje się przy pomocy ukośnika i jest ono zawsze dzieleniem całkowitym:

\begin{CodeFrame*}[bash]{}
a=12; b=3; x=5; y=6

e=$(( ($a + $b) * 4 - $y ))
c=$((  $x / $y ))

echo $e $c $z
\end{CodeFrame*}
Zauważ zachowanie przy odwołaniu do niezainicjalizowanej zmiennej \Verb{z}.

Operacje logiczne obsługiwane są komendą \Verb{test} lub operatorem \Verb{[ ]} wynik zwracany jest jako kod powrotu.
Należy zwrócić uwagę na escapowanie odwrotnym ukośnikiem nawiasów i na to że spacje mają znaczenie.
Negację realizuje \Verb{!}, należy pamiętać jednak że wynikiem negacji dowolnej liczby jest FALSE.

\begin{CodeFrame*}[bash]{}
[ \( $a -ge 0 -a $b -lt 2 \) -o $z -eq 5 ]; z=$?

echo $z
\end{CodeFrame*}
Wartość zmiennej \Verb{z} jest wynikiem warunku: \texttt{((a większe równe od zera) AND (b mniejsze od dwóch)) OR (z równe 5)}.
Bash stosuje logikę odwróconą 0 oznacza prawdę, coś nie zerowego to fałsz.

Jako operacje podstawowe powinniśmy patrzyć także na wykonanie innych programów i pobieranie ich standardowego wyjścia i/lub kodu powrotu.
Pobieranie standardowego wyjścia możemy realizować za pomocą ujęcia polecenia w \emph{backquotes} (\Verb{`}) lub operatora \Verb{$( )} (pozwala on na zagnieżdżanie takich operacji).
Natomiast kod powrotu ostatniej komendy znajduje się w zmiennej \Verb{$?} (używaliśmy tego już przy obliczaniu wyrażeń logicznych).

\begin{CodeFrame*}[bash]{}
a=`cat /etc/issuse`
b=$(cat /etc/issuse; cat /etc/resolv.conf)

echo  $a
echo  $b
echo "$b"
\end{CodeFrame*}
Zwróć uwagę na różnicę w wypisaniu zmiennej zawierającej znaki nowej linii objętej cudzysłowami i nie objętej nimi.

Bash nie obsługuje liczb zmiennoprzecinkowych ani operacji bitowych, nieobsługiwane operacje można wykonać za pomocą innego programu np:

\begin{CodeFrame*}[bash]{}
a=`echo 'print(3/2)' | python3`
b=$(echo '3/2' | bc -l)
echo $a $b
\end{CodeFrame*}

\begin{ProTip}{inne polecenia}
Programowanie w bashu w dużej mierze polega na wywoływaniu innych programów (np. takich jak sed, grep, find, awk).
Sam bash oferuje jedynie podstawowe konstrukcje składniowe, obsługę zmiennych i pewnych podstawowych operacji na nich.

Na te zewnętrzne polecenia można patrzeć trochę jak na biblioteki w innych językach programowania –
	komendy gwarantowane przez standard stanowią „bibliotekę standardową” basha,
	a inne (np. użyty w powyższym przykładzie artmetyki zmiennoprzecinkowej python) stanowią dodatkowe opcjonalne „biblioteki”, które pozwalają na łatwiejsze i szybsze rozwiązywanie problemów.
W zasadzie podobnie można patrzeć na wywołania zewnętrznych programów w ramach kodu Pythona, C czy innych języków (niekiedy łatwiej jest zrobić np. \Verb#system("mv plik nowyplik")# niż zakodować to bezpośrednio w Pythonie czy w C).
\end{ProTip}

\subsection{Uruchamianie kodu z pliku}

Dłuższe fragmenty kodu bashowego często wygodniej jest pisać w pliku tekstowym niż bezpośrednio w linii poleceń.
Plik taki może zostać wykonany przy pomocy polecenia: \Verb{./nazwa_pliku} pod warunkiem że ma prawo wykonalności
(powinien także zawierać w pierwszą linii komentarz określający program używany do interpretacji tekstowego pliku wykonywalnego,
w postaci: \Verb{#!/bin/bash}).
Może też być wykonany za pomocą wywołania: \Verb{bash nazwa_pliku}.

Przydatną alternatywą dla powyższych metod wykonania kodu zawartego w pliku jest włączenie go do aktualnej sesji basha przy pomocy \Verb{. ./nazwa_pliku}.
W odróżnieniu od poprzednich metod pozwala to na korzystanie z funkcji i zmiennych zdefiniowanych w tym pliku w kolejnych poleceniach.

\subsection{Pętle i warunki}

\subsubsection{Pętla for}

W bashu możemy korzystać z kilku wariantów pętli for. Jednym z najczęściej używanych jest przypadek iterowania po plikach\footnote{
	Dokładniej: iteracja odbywa się po liście napisów rozdzielanej spacjami - zobacz rezultat \Verb{echo /tmp/*}
}:

\begin{CodeFrame*}[bash]{}
for nazwa in /tmp/* ; do
	echo $nazwa;
done
\end{CodeFrame*}

Możliwe jest też iterowanie po wartościach całkowitych zarówno w stylu ,,shellowym'' \teacher{(zwróć uwagę na podobieństwo do Pythona) } jak i w stylu C

\begin{CodeFrame*}[bash]{}
for i in `seq 0 20`; do
	echo $i;
done

for (( i=0 ; $i<=20 ; i++ )) ; do
	echo $i;
done
\end{CodeFrame*}

\subsubsection{Pętla while}

Często używana jest pętla while w połączeniu z instrukcją \Verb$read$\footnote{
	Polecenie \Verb$read$ można także wykorzystać do wczytania danych podawanych przez użytkownika do jakiejś zmiennej – np. \Verb$read -p "wpisz coś >> " xyz$ wczyta tekst do zmiennej \Verb$xyz$.
	\Verb$read$ z opcją \Verb$-e$ potrafi kożystać z biblioteki readline, jednak np. współdzieli histiorię z historią basha. Dlatego często wygodniejsze może być zainstalowanie i użycie \Verb$rlwrap$, np:
	\Verb$xyz=`rlwrap -H historia.txt -S "wpisz coś >> " head -n1`$.
} co umożliwia przetwarzanie jakiegoś wejścia (wyniku komendy lub pliku) linia po linii (także z podziałem linii na słowa):

\begin{CodeFrame*}[bash]{}
cat /etc/fstab | while read slowo reszta; do
	echo $reszta;
done
\end{CodeFrame*}
Powyższa pętla wypisze po kolei wszystkie wiersze pliku \texttt{/etc/fstab} przekazanego przez stdin (przy pomocy komendy \texttt{cat})\footnote{
	Takie rozwiązanie nazywane jest \emph{martwym kotem} i powinno go się unikać.
	Lepszym rozwiązaniem jest przekazywanie pliku przez przekierowanie strumienia wejściowego przy pomocy \texttt{< plik},
	który w tym przypadku powinien znaleźć się za kończącym pętle słowem kluczowym \texttt{done}.
} z pominięciem pierwszego słowa (które wczytywane było do zmiennej slowo).

Słowa domyślnie rozdzielane są przy pomocy dowolnego ciągu spacji lub tabulatorów, separator można zmienić za pomocą zmiennej \texttt{IFS}, np:

\begin{CodeFrame*}[bash]{}
IFS=":"
while read a b; do echo $a; done < /etc/group
unset IFS # przywracamy domyślne zachowanie read poprzez usunięcie zmiennej IFS
\end{CodeFrame*}

\subsubsection{Instrukcja if}

Poznane wcześniej obliczanie wartości wyrażeń logicznych najczęściej stosowane jest w instrukcji warunkowej \Verb{if}\footnote{
	Może być też stosowane np. w pokazanej wcześniej pętli \Verb{while}
}.

\begin{CodeFrame*}[bash]{}
# instruikcja if - else
if [ "$xx" = "kot" -o "$xx" = "pies" ]; then
	echo  "kot lub pies";
elif [ "$xx" = "ryba" ];  then
	echo  "ryba"
else
	echo  "coś innego"
fi
\end{CodeFrame*}

Zauważ że spacje wokół i wewnątrz nawiasów kwadratowych przy warunku są istotne składniowo,
zawartość nawiasów kwadratowych to tak naprawdę argumenty dla komendy \Verb{test}.
Oprócz typowych warunków logicznych możemy sprawdzać np. istnienie plików, czy też ich typ (link, katalog, etc).
Szczegółowy opis dostępnych warunków które mogą być użyte w tej konstrukcji znajduje się w \Verb{man test}.
\teacher{Zwrócić szczególną uwagę na tą dokumentację.}

Jako warunek może wystąpić dowolne polecenie wtedy sprawdzany jest jego kod powrotu 0 oznacza prawdę / zakończenie sukcesem,
a wartość nie zerowa fałsz / błąd

\begin{CodeFrame*}[bash]{}
if grep '^root:' /etc/passwd > /dev/null; then
	echo /etc/passwd zawiera root-a;
fi
\end{CodeFrame*}

Istnieje możliwość skróconego zapisu warunków z użyciem łączenia instrukcji
przy pomocy \Verb{&&} (wykonaj gdy poprzednia zwróciła zero -- true)
lub \Verb{||} (wykonaj gdy poprzednia zwróciła nie zero -- false):

\begin{CodeFrame*}[bash]{}
[ -f /etc/issuse ] && echo "jest plik /etc/issuse"

grep '^root:' /etc/passwd > /dev/null && echo /etc/passwd zawiera root-a;
\end{CodeFrame*}

\subsubsection{Instrukcja case}
Instrukcja \Verb{case} służy do rozważania wielu przypadków opartych na równości zmiennej z podanymi napisami.
\teacher{Zwróć uwagę na podobieństwo do switch z C/C++ oraz różnicę - operuje na napisach a nie liczbach}

\begin{CodeFrame*}[bash]{}
case $xx in
	kot | pies)
		echo  "kot lub pies"
		;;
	ryba)
		echo  "ryba"
		;;
	*)
		echo  "cos innego"
		;;
esac
\end{CodeFrame*}

\subsection{Definiowanie funkcji}

W bashu każda funkcja może przyjmować dowolną ilość parametrów pozycyjnych
(w identyczny sposób obsługiwane są argumenty linii poleceń dla całego skryptu).
Ilość parametrów znajduje się w zmiennej \Verb{$#}, lista wszystkich parametrów w \Verb{$@},
a do kolejnych parametrów możemy odwoływać się z użyciem \Verb{$1}, \Verb#$2#, itd.

\begin{CodeFrame*}[bash]{}
f1() {
	echo "wywołano z $# parametrami, parametry to: $@"
	
	[ $# -lt 2 ] && return;
	
	echo -e "drugi: $2\npierwszy: $1"
	
	# albo kolejnych w pętli
	for a in "$@"; do  echo $a;  done
	
	# lub z użyciem polecenia shift
	for i in `seq 1 $#`; do
		echo $1
		shift # powoduje zapomnienie $1
		      # i przenumerowanie argumentów pozycyjnych o 1
		      # wpływa na wartości $@ $# itp
	done
	
	# funkcja może zwracać tylko wartość numeryczną -- tzw kod powrotu
	return 83
}
\end{CodeFrame*}

Zwróć uwagę że w nawiasach po nazwie funkcji nie podajemy przyjmowanych argumentów, natomiast puste nawiasy te są elementem składniowym i muszą wystąpić. Jeżeli zapisujesz definicję funkcji w jednej linii, np. \Verb@abc() { echo "abc"; }@ to pamiętaj, że spacje po otwierającym i przed kończącym nawiasem klamrowym są obowiązkowe, podobnie jak średniki występujące po każdej instrukcji w ciele funkcji.

Wywołanie funkcji nie różni się niczym od wywołania programów czy instrukcji wbudowanych
(możemy używać przekierowań strumieni wejścia, wyjścia, czy też przechwycić wyjście do zmiennej). Powyższą funkcję możemy wywołać np. w następujący sposób: \Verb{f1 a "b c"   d}

% END: podstawy programowania w bashu

\subsection{Zadania}
	% Copyright (c) 2017-2020 Matematyka dla Ciekawych Świata (http://ciekawi.icm.edu.pl/)
% Copyright (c) 2017-2020 Robert Ryszard Paciorek <rrp@opcode.eu.org>
% 
% MIT License
% 
% Permission is hereby granted, free of charge, to any person obtaining a copy
% of this software and associated documentation files (the "Software"), to deal
% in the Software without restriction, including without limitation the rights
% to use, copy, modify, merge, publish, distribute, sublicense, and/or sell
% copies of the Software, and to permit persons to whom the Software is
% furnished to do so, subject to the following conditions:
% 
% The above copyright notice and this permission notice shall be included in all
% copies or substantial portions of the Software.
% 
% THE SOFTWARE IS PROVIDED "AS IS", WITHOUT WARRANTY OF ANY KIND, EXPRESS OR
% IMPLIED, INCLUDING BUT NOT LIMITED TO THE WARRANTIES OF MERCHANTABILITY,
% FITNESS FOR A PARTICULAR PURPOSE AND NONINFRINGEMENT. IN NO EVENT SHALL THE
% AUTHORS OR COPYRIGHT HOLDERS BE LIABLE FOR ANY CLAIM, DAMAGES OR OTHER
% LIABILITY, WHETHER IN AN ACTION OF CONTRACT, TORT OR OTHERWISE, ARISING FROM,
% OUT OF OR IN CONNECTION WITH THE SOFTWARE OR THE USE OR OTHER DEALINGS IN THE
% SOFTWARE.

\IfStrEq{\dbEntryID}{}{
	\insertZadanie{\currfilepath}{petla_linki_html}{}
	\insertZadanie{\currfilepath}{warunek_istnienie_pliku}{}
	\insertZadanie{\currfilepath}{funkcja_n_razy_napis}{}
	\insertZadanie{\currfilepath}{funkcja_liczba_kotow}{}
}

\IfStrEq{\dbEntryID}{rozwiazania}{
	\insertRozwiazanie{\currfilepath}{petla_linki_html}{}
	\insertRozwiazanie{\currfilepath}{warunek_istnienie_pliku}{}
	\insertRozwiazanie{\currfilepath}{funkcja_n_razy_napis}{}
	\insertRozwiazanie{\currfilepath}{funkcja_liczba_kotow}{}
}

% BEGIN: podstawy programowania w bashu - zadania
\dbEntryBegin{petla_linki_html}\if1\dbEntryCheckResults
Napisz pętle, która wypisze wszystkie pliki nieukryte z bieżącego katalogu w postaci linków HTML, czyli:
dla pliku o nazwie \Verb{ABC} powinna wypisać \Verb{<a href="ABC">ABC</a>}. Przedstaw zarówno rozwiązanie z użyciem pętli \Verb{for}, jak i pętli \Verb{while}.
\fi
\dbEntryBegin{petla_linki_html-rozwiazanie}\if1\dbEntryCheckResults
\begin{CodeFrame*}[bash]{}
for f in *; do echo "<a href=\"$f\">$f</a>"; done
ls | while read f; do echo "<a href=\"$f\">$f</a>"; done
\end{CodeFrame*}

\noindent Zwróć uwagę że:
\begin{itemize}
\item rozwiązania te różnią się jedynie sposobem uzyskania listy plików ald której mają wypisać linki
\item pętla for w każdym obiegu podstawia pod f kolejny element z listy nazw dopasowanych do gwiazdki (czyli wszystkich plików nieukrytych)
\item pętla while listę plików dostaje na standardowe wejście (jeden plik na linię) i wczytuje każdą kolejną linię (czyli kolejną nazwę pliku) stosując komendę read – jest to bardzo standardowe rozwiązanie do przetwarzania standardowego wejścia linia po linii
\item w obu wypadkach używamy takiego echo z napisem w podwójnych cudzysłowach (aby móc umieścić w nim zmienną), cudzysłowa które mają być wypisane zabezpieczamy odwrotnym ukośnikiem
\item wypisywanie można rozwiązać na kilka innych sposobów np.: \shell{echo '<a href="'"$f"'">'"$f"'</a>'} – zadziała tak samo, ale wydaje się być to mniej czytelne
\end{itemize}
\fi


\dbEntryBegin{warunek_istnienie_pliku}\if1\dbEntryCheckResults
Napisz warunek, który sprawdzi czy \Verb{/tmp/abc} istnieje i jest katalogiem.
\fi
\dbEntryBegin{warunek_istnienie_pliku-rozwiazanie}\if1\dbEntryCheckResults
\begin{CodeFrame*}[bash]{}
if [ -d /tmp/abc ]; then echo "jest katalogiem"; else echo "nie";
\end{CodeFrame*}

lub krócej:

\begin{CodeFrame*}[bash]{}
[ -d /tmp/abc ] && echo "jest katalogiem";
\end{CodeFrame*}

\noindent Zwróć uwagę że:
\begin{itemize}
\item w celu warunkowego wypisania jakiejś informacji możemy użyć zarówno konstrukcji if, jak też łączenia poleceń,
	jednak w przypadku większej ilości poleceń objętych warunkiem konstrukcja z if jest bardziej czytelna
\item sprawdzenie czy podana ścieżka istnieje i jest katalogiem odbywa się przy pomocy opcji -d, informacja ta celowo nie była podana w treści skryptu – należało to sprawdzić w dokumentacji systemowej (\shell{man test}). \textbf{Czytanie dokumentacji jest ważne!}
\end{itemize}
\fi


\dbEntryBegin{funkcja_n_razy_napis}\if1\dbEntryCheckResults
Napisać funkcję przyjmującą dwa argumenty - liczbę i napis; funkcja ma wypisać napis tyle razy ile wynosi podana liczba.
\fi
\dbEntryBegin{funkcja_n_razy_napis-rozwiazanie}\if1\dbEntryCheckResults
\begin{CodeFrame*}[bash]{}
f() { for i in `seq 1 $1`; do echo $2; done; }
\end{CodeFrame*}

\noindent Zwróć uwagę że:
\begin{itemize}
\item w nawiasach po nazwie funkcje nie piszemy nic na temat jej argumentów - one są puste
\item do argumentów odwołujemy się poprzez dolar numer argumentu
\item do n krotnego powtórzenia czynności używamy pętli for która iteruje po liście liczb zwracanej przez seq
\item seq objęta jest znakami ` oznaczającymi że należy wykonać podany w nich kod i podstawić w to miejsce jego standardowe wyjście, nie należy ich mylić z apostrofami używanymi do napisów (')
\item spacja po { oraz średnik (lub nowa linia) przed } są istotne składniowo
\end{itemize}
\fi


\dbEntryBegin{funkcja_liczba_kotow}\if1\dbEntryCheckResults
Napisać funkcję przyjmującą jeden argument - liczbę kotów i wypisującą:
\begin{itemize}
	\item "Ala ma kota" dla ilości kotów równej 1
	\item "Ala ma x koty" lub "Ala ma x kotów" gdzie dobrana jest poprawna forma, a pod x podstawiona podana w argumencie ilość kotów.
\end{itemize}
Dla uproszczenia należy założyć że podana ilość kotów jest w zakresie od 1 do 9.
\fi
\dbEntryBegin{funkcja_liczba_kotow-rozwiazanie}\if1\dbEntryCheckResults
\begin{CodeFrame*}[bash]{}
koty() {
	case $1 in
		1) echo "Ala ma kota";;
		2|3|4) echo "Ala ma $1 koty";;
		*) echo "Ala ma $1 kotów";;
	esac
}
\end{CodeFrame*}
\fi
% END: podstawy programowania w bashu - zadania


\section{Przetwarzanie napisów}
	% Copyright (c) 2017-2020 Matematyka dla Ciekawych Świata (http://ciekawi.icm.edu.pl/)
% Copyright (c) 2017-2020 Robert Ryszard Paciorek <rrp@opcode.eu.org>
% 
% MIT License
% 
% Permission is hereby granted, free of charge, to any person obtaining a copy
% of this software and associated documentation files (the "Software"), to deal
% in the Software without restriction, including without limitation the rights
% to use, copy, modify, merge, publish, distribute, sublicense, and/or sell
% copies of the Software, and to permit persons to whom the Software is
% furnished to do so, subject to the following conditions:
% 
% The above copyright notice and this permission notice shall be included in all
% copies or substantial portions of the Software.
% 
% THE SOFTWARE IS PROVIDED "AS IS", WITHOUT WARRANTY OF ANY KIND, EXPRESS OR
% IMPLIED, INCLUDING BUT NOT LIMITED TO THE WARRANTIES OF MERCHANTABILITY,
% FITNESS FOR A PARTICULAR PURPOSE AND NONINFRINGEMENT. IN NO EVENT SHALL THE
% AUTHORS OR COPYRIGHT HOLDERS BE LIABLE FOR ANY CLAIM, DAMAGES OR OTHER
% LIABILITY, WHETHER IN AN ACTION OF CONTRACT, TORT OR OTHERWISE, ARISING FROM,
% OUT OF OR IN CONNECTION WITH THE SOFTWARE OR THE USE OR OTHER DEALINGS IN THE
% SOFTWARE.

% BEGIN: Wbudowane przetwarzanie napisów w bashu
\subsection{Wbudowane przetwarzanie napisów w bash'u}

Wbudowane przetwarzanie napisów w bashu opiera się na odwołaniach do zmiennych w postaci \Verb@${}@:

\begin{itemize}
	\item \shell@${zmienna:-"napis"}@ zwróci napis gdy zmienna nie jest zdefiniowana lub jest pusta
	\item \shell@${zmienna:="napis"}@ zwróci napis oraz wykona podstawienie zmienna="napis" gdy zmienna nie jest zdefiniowana lub jest pusta
	\item \shell@${zmienna:+"napis"}@ zwróci napis gdy zmienna jest zdefiniowana i nie pusta
	
	\vspace{6pt}
	
	\item \shell@${#str}@    zwróci długość napisu w zmiennej str
	\item \shell@${str:n}@   zwróci pod-napis zmiennej str od n do końca
	\item \shell@${str:n:m}@ zwróci pod-napis zmiennej str od n o długości m
	
	\vspace{6pt}
	
	\item \shell@${str/"n1"/"n2"}@  zwróci wartość str z zastąpionym pierwszym wystąpieniem n1 przez n2
	\item \shell@${str//"n1"/"n2"}@  zwróci wartość str z zastąpionymi wszystkimi wystąpieniami n1 przez n2
	
	\vspace{6pt}
	
	\item \shell@${str#"ab"}@ zwróci wartość str z obciętym "ab" z początku
	\item \shell@${str%"fg"}@ zwróci wartość str z obciętym "fg" z końca
\end{itemize}
W napisach do obcięcia możliwe jest stosowanie shellowych znaków uogólniających, czyli \Verb@*@, \Verb@?@, \Verb@[abc]@, itd operator \Verb@#@ i \Verb@%@ dopasowują minimalny napis do usunięcia, natomiast operatory \Verb@##@ i \Verb@%%@ dopasowują maksymalny napis do usunięcia.

\begin{teacherOnly}
Przykład do zademonstrowania:

\begin{CodeFrame*}[bash]{}
a=""; b=""; c=""
echo ${a:-"aa"} ${b:="bb"} ${c:+"cc"}
echo $a $b $c

a="x"; b="y"; c="z"
echo ${a:-"aa"} ${b:="bb"} ${c:+"cc"}
echo $a $b $c

x=abcdefg
echo ${#x} ${x:2} ${x:0:3} ${x:0:$((${#x}-2))}
echo ${x#"abc"} ${x%"efg"}
echo ${x#"ac"}  ${x%"eg"}

x=abcd.e.fg
echo ${x#*.} ${x##*.} ${x%.*} ${x%%.*}

y="aa bb cc bb dd bb ee"
echo ${y/"bb"/"XX"}
echo ${y//"bb"/"XX"}
\end{CodeFrame*}

\end{teacherOnly}
Możliwe jest także korzystanie z wyrażeń regularnych.
Polecenie \Verb@expr match $x 'wr1\(wr2\)wr3'@ zwróci na stdout (wypisze) część \Verb@$x@ pasującą do wyrażenia regularnego \Verb@wr2@,
wyrażenia regularne \Verb@wr1@ i \Verb@wr2@ pozwalają na określanie części napisu do odrzucenia.
Alternatywną składnią jest \Verb@expr $x : 'wr1\(wr2\)wr3'@

\begin{teacherOnly}
Przykład do zademonstrowania:

\begin{CodeFrame*}[bash]{}
z="ab=cd"
expr match $z '^\([^=]*\)='
expr $z : '^[^=]*=\(.*\)$'
\end{CodeFrame*}
\end{teacherOnly}

Możliwe jest też sprawdzanie dopasowań wyrażeń regularnych poprzez (zwróć uwagę na brak cytowania wyrażenia regularnego):

\begin{CodeFrame*}[bash]{}
[[ "$z" =~ ^([^=]*)= ]] && echo "OK"
\end{CodeFrame*}

Możliwe jest także zaawansowane formatowanie napisów, konwertowanie liczb na napisy,
w tym wypisywanie w różnych systemach liczbowych przy pomocy \Verb@printf@\footnote{
	Instrukcja \Verb@printf@ ma składnię opartą na tej funkcji z C, interpretuje ona także liczby zmiennoprzecinkowe.
}:

\begin{CodeFrame*}[bash]{}
printf "0o%o %d 0x%x\n" 0xf 010 3
\end{CodeFrame*}
% END: Wbudowane przetwarzanie napisów w bashu

	% Copyright (c) 2017-2020 Matematyka dla Ciekawych Świata (http://ciekawi.icm.edu.pl/)
% Copyright (c) 2017-2020 Robert Ryszard Paciorek <rrp@opcode.eu.org>
% 
% MIT License
% 
% Permission is hereby granted, free of charge, to any person obtaining a copy
% of this software and associated documentation files (the "Software"), to deal
% in the Software without restriction, including without limitation the rights
% to use, copy, modify, merge, publish, distribute, sublicense, and/or sell
% copies of the Software, and to permit persons to whom the Software is
% furnished to do so, subject to the following conditions:
% 
% The above copyright notice and this permission notice shall be included in all
% copies or substantial portions of the Software.
% 
% THE SOFTWARE IS PROVIDED "AS IS", WITHOUT WARRANTY OF ANY KIND, EXPRESS OR
% IMPLIED, INCLUDING BUT NOT LIMITED TO THE WARRANTIES OF MERCHANTABILITY,
% FITNESS FOR A PARTICULAR PURPOSE AND NONINFRINGEMENT. IN NO EVENT SHALL THE
% AUTHORS OR COPYRIGHT HOLDERS BE LIABLE FOR ANY CLAIM, DAMAGES OR OTHER
% LIABILITY, WHETHER IN AN ACTION OF CONTRACT, TORT OR OTHERWISE, ARISING FROM,
% OUT OF OR IN CONNECTION WITH THE SOFTWARE OR THE USE OR OTHER DEALINGS IN THE
% SOFTWARE.

% BEGIN: grep cut sed
\subsection{grep, cut, sed, ...}

Jako że większość operacji wykonywanych w powłoce takiej jak bash wiąże się z uruchamianiem zewnętrznych programów,
to także przetwarzanie napisów może być realizowane w ten sposób.
Opiera się na tym jedno z podejść do obsługi napisów w bashu,
którym jest korzystanie z standardowych komend POSIX, takich jak grep, cut, sed.

\begin{CodeFrame*}[bash]{}
# obliczanie długości napisu w znakach, w bajtach i ilości słów w napisie
echo -n "aąbcć 123" | wc -m
echo -n "aąbcć 123" | wc -c
echo -n "aąbcć 123" | wc -w

# obliczanie ilości linii (dokładniej ilości znaków nowej linii)
wc -l < /etc/passwd

# wypisanie 5 pola (rozdzielanego :) z pliku /etc/passwd  z eliminacją
# pustych linii oraz linii złożonych tylko ze spacji i przecinków
cut -f5 -d: /etc/passwd | grep -v '^[ ,]*$'
# komenda cut wybiera wskazane pola, opcja -d określa separator
\end{CodeFrame*}

Inną bardzo przydatną komendą jest sed pozwala ona m.in na zastępowanie
wyszukiwanego na podstawie wyrażenia regularnego tekstu innym:
\begin{CodeFrame*}[bash]{}
echo "aa bb cc bb dd bb ee" | sed -e 's@\([bc]\+\) \([bc]\+\)@X-\2-X@g'
\end{CodeFrame*}

Sedowe polecenie s przyjmuje 3 argumenty (oddzielane mogą być dowolnym znakiem który wystąpi za~\Verb@s@),
pierwszy to wyszukiwane wyrażenie, drugi tekst którym ma zostać zastąpione,
a trzeci gdy jest \Verb@g@ to powoduje zastępowanie wszystkich wystąpień a nie tylko pierwszego.

Należy zwrócić uwagę na różnicę w składni wyrażenia regularnego polegającą na poprzedzaniu
\Verb@(@, \Verb@)@ i \Verb@+@ odwrotnym ukośnikiem aby miały znaczenie specjalne
(jeżeli nie chcemy tego robić możemy włączyć obsługę ERE w sed poprzez opcję \Verb@-E@).

Innymi przydatnymi komendami przetwarzającymi (specyficznej postaci) napisy są polecaenia \Verb@basename@ i \Verb@dirname@.
Służą one do uzyskania nazwy najgłębszego elementu ścieżki oraz ścieżki bez tego najglębszego elementu. Zobacz wynik działania:

\begin{CodeFrame*}[bash]{}
basename /proc/sys/net/core/
dirname /proc/sys/net/core/
\end{CodeFrame*}
% END: grep cut sed

	% Copyright (c) 2017-2020 Matematyka dla Ciekawych Świata (http://ciekawi.icm.edu.pl/)
% Copyright (c) 2017-2020 Robert Ryszard Paciorek <rrp@opcode.eu.org>
% 
% MIT License
% 
% Permission is hereby granted, free of charge, to any person obtaining a copy
% of this software and associated documentation files (the "Software"), to deal
% in the Software without restriction, including without limitation the rights
% to use, copy, modify, merge, publish, distribute, sublicense, and/or sell
% copies of the Software, and to permit persons to whom the Software is
% furnished to do so, subject to the following conditions:
% 
% The above copyright notice and this permission notice shall be included in all
% copies or substantial portions of the Software.
% 
% THE SOFTWARE IS PROVIDED "AS IS", WITHOUT WARRANTY OF ANY KIND, EXPRESS OR
% IMPLIED, INCLUDING BUT NOT LIMITED TO THE WARRANTIES OF MERCHANTABILITY,
% FITNESS FOR A PARTICULAR PURPOSE AND NONINFRINGEMENT. IN NO EVENT SHALL THE
% AUTHORS OR COPYRIGHT HOLDERS BE LIABLE FOR ANY CLAIM, DAMAGES OR OTHER
% LIABILITY, WHETHER IN AN ACTION OF CONTRACT, TORT OR OTHERWISE, ARISING FROM,
% OUT OF OR IN CONNECTION WITH THE SOFTWARE OR THE USE OR OTHER DEALINGS IN THE
% SOFTWARE.

% BEGIN: AWK
\subsection{awk}

Awk jest interpreterem prostego skryptowego języka umożliwiający przetwarzanie tekstowych baz danych postaci \texttt{linia=rekord}, gdzie pola oddzielane ustalonym separatorem (można powiedzieć że łączy funkcjonalność komend takich jak grep, cut, sed z prostym językiem programowania).

\teacher{Około 30 minutowe wprowadzenie z pisaniem kodu na żywo}

Wyżej zaprezentowane wypisanie 5 pola (rozdzielanego :) z pliku \Verb@/etc/passwd@  z eliminacją pustych linii oraz
linii złożonych tylko ze spacji i przecinków, realizowane przy użyciu poleceń \Verb@cut@ i \Verb@grep@
może być zrealizowane za pomocą samego awk:

\begin{CodeFrame*}[bash]{}
awk -F: '$5 !~ "^[ ,]*$" {print $5}' /etc/passwd
\end{CodeFrame*}

Awk daje duże możliwości przy przetwarzaniu tego typu tekstowych baz danych -- możemy np.
wypisywać wypisywać pierwsze pole w oparciu o warunki nałożone na inne:

\begin{CodeFrame*}[bash]{}
awk -F: '$5 !~ "^[ ,]*$" && $3 >= 1000 {print $1}' /etc/passwd
\end{CodeFrame*}

Jak widać w powyższych przykładach do poszczególnych pól odwołujemy się poprzez \Verb@$n@,
gdzie \Verb@n@ jest numerem pola, \Verb@$0@ oznacza cały rekord

Program dla każdego rekordu przetwarza kolejne instrukcje postaci \Verb@warunek { komendy }@,
instrukcji takich może być wiele w programie (przetwarzane są kolejno),
komenda \Verb@next@ kończy przetwarzanie danego rekordu.

Separator pola ustawiamy opcją \Verb@-F@ (lub zmienną \Verb@FS@),
domyślnym separatorem pola jest dowolny ciąg spacji i tabulatorów
(w odróżnieniu od cut separator może być wieloznakowym napisem lub wyrażeniem regularnym).
Domyślnym separatorem rekordu jest znak nowej linii (można go zmienić zmienną RS).

Awk jest prostym językiem programowania obsługującym podstawowe pętle i instrukcje warunkowe
oraz funkcje wyszukujące i modyfikujące napisy:

\begin{oframed}\noindent\shell{echo "aba aab bab baa bba bba" | awk}\Verb@ '{@\vspace{-0.95em}
\begin{minted}{awk}
	# dla każdego pola w rekordzie
	for (i=1; i<=NF; ++i) {
		# jeżeli jego numer jest parzysty
		# to zastąp wszystkie ciągi b pojedynczym B
		if (i%2==0)
			gsub("b+", "B", $i);
		
		# wyszukaj pozycję pod-napisu B
		ii = index($i, "B")
		# jeżeli znalazł
		# to wypisz pozycję i pod-napis od tej pozycji do końca
		if (ii)
			printf("# %d %s\n", ii, substr($i, ii))
		# zwróć uwagę że w AWK liczy elementy napisy od 1 a nie od 0
	}
	print $0
\end{minted}
\vspace{-0.95em}\Verb@}'@\end{oframed}

\noindent
AWK obsługuje także tablice asocjacyjne pozwala to np. policzyć powtórzenia słów:

\begin{oframed}\noindent\shell{echo "aa bb aa ee dd aa dd" | awk }\Verb@ '@\vspace{-0.95em}
\begin{minted}{awk}
	BEGIN {RS="[ \t\n]+"; FS=""}
	{slowa[$0]++}
	# może być kilka bloków {} pasujących do rekordu
	# jeżeli nie użyjemy next przetworzone zostaną wszystkie
	# {printf("rekord: %d done\n", NR)}
	END {for (s in slowa) printf("%s: %s\n", s, slowa[s])}
\end{minted}
\vspace{-0.95em}\Verb@'@\end{oframed}

Podobny efekt możemy uzyskać stosując "uniq -c" (który wypisuje unikalne wiersze wraz z ich ilością)
na odpowiednio przygotowanym napisie (spacje zastąpione nową linią, a linie posortowane):

\begin{CodeFrame*}[bash]{}
echo "aa bb aa ee dd aa dd" | tr ' ' '\n' | sort | uniq -c
\end{CodeFrame*}
Jednak rozwiązanie awk można łatwo zmodyfikować aby wypisywało pierwsze wystąpienie linii bez sortowania pliku.

Innym użytecznym zastosowaniem AWK może być wypisanie pliku bez linii pasujących do wzorca oraz linii poprzednich:

\begin{oframed}\noindent\shell{echo -e "aa\nbb\nWZORZEC\ncc" | awk}\Verb@ '{@\vspace{-0.95em}
\begin{minted}{awk}
	# dla linii pasującej do wzorca ustwaiamy flagę print_last na zero i przechodzimy do następnej linii
	/WZORZEC/ {print_last=0; next}
	# jeżeli flaga print_last jest nie zero wypisujemy zapamiętaną poprzednią liniię
	print_last == 1 {print last}
	# zapamiętujemy bierzacą linię do wypisania przy przetwarzaniu kolejnej (jeżeli nie bedzie pasować do wzorca)
	{last=$0; print_last=1}
	# jeżeli osiągneliśmy koniec pliku i mamy linię do wypisania to ją wypisujemy
	END {if (print_last == 1) print last}
\end{minted}
\vspace{-0.95em}\Verb@}'@\end{oframed}

AWK pozwala także na definiowanie funkcji:
\begin{CodeFrame*}[bash]{}
awk 'function f(x) {return 2*x} { print f($1+$2) }'
\end{CodeFrame*}
% END: AWK

\subsection{Zadania}
	% Copyright (c) 2017-2020 Matematyka dla Ciekawych Świata (http://ciekawi.icm.edu.pl/)
% Copyright (c) 2017-2020 Robert Ryszard Paciorek <rrp@opcode.eu.org>
% 
% MIT License
% 
% Permission is hereby granted, free of charge, to any person obtaining a copy
% of this software and associated documentation files (the "Software"), to deal
% in the Software without restriction, including without limitation the rights
% to use, copy, modify, merge, publish, distribute, sublicense, and/or sell
% copies of the Software, and to permit persons to whom the Software is
% furnished to do so, subject to the following conditions:
% 
% The above copyright notice and this permission notice shall be included in all
% copies or substantial portions of the Software.
% 
% THE SOFTWARE IS PROVIDED "AS IS", WITHOUT WARRANTY OF ANY KIND, EXPRESS OR
% IMPLIED, INCLUDING BUT NOT LIMITED TO THE WARRANTIES OF MERCHANTABILITY,
% FITNESS FOR A PARTICULAR PURPOSE AND NONINFRINGEMENT. IN NO EVENT SHALL THE
% AUTHORS OR COPYRIGHT HOLDERS BE LIABLE FOR ANY CLAIM, DAMAGES OR OTHER
% LIABILITY, WHETHER IN AN ACTION OF CONTRACT, TORT OR OTHERWISE, ARISING FROM,
% OUT OF OR IN CONNECTION WITH THE SOFTWARE OR THE USE OR OTHER DEALINGS IN THE
% SOFTWARE.

\IfStrEq{\dbEntryID}{}{
	\insertZadanie{\currfilepath}{passwd_warunek_na_uid_awk}{}
	\insertZadanie{\currfilepath}{awk_last}{}
	\insertZadanie{\currfilepath}{passwd_warunek_na_uid_noawk}{}
	\insertZadanie{\currfilepath}{rekurecyjne_wyszukaj_i_zastap}{}
}

% napisy z uzyciem awk i/lub programowanie w bash

\dbEntryBegin{passwd_warunek_na_uid_awk}\if1\dbEntryCheckResults
Korzystając z AWK wyświetl z /etc/passwd linie w których UID (3 pole) ma warość >= 1000.
\fi

\dbEntryBegin{awk_last}\if1\dbEntryCheckResults
Polecenie \Verb{last} wypisuje informację o ostatnich zalogowaniach w systemie. Napisz polecenie (wykorzystujące \Verb{last}), które wypisze informację jak często logowali się poszczególni użytkownicy.
\fi

\dbEntryBegin{passwd_warunek_na_uid_noawk}\if1\dbEntryCheckResults
Wyświetl z /etc/passwd linie w których UID (3 pole) ma warość >= 1000 nie korzystając z AWK.
Jeżeli masz pomysł przedstw więcej niż jedno rozwiązanie.
\fi

\dbEntryBegin{rekurecyjne_wyszukaj_i_zastap}\if1\dbEntryCheckResults
Napisz funkcję która przyjmuje dwa argumenty - napis wyszukiwany i napis go zastępujący oraz dokonuje rekurencyjnego wyszukania i zamiany tych napisów w wszystkich plikach w bierzącym katalogu.

\textit{Wskazówka 1: polecenie \texttt{sed} z opcją \texttt{-i} i wskazaniem pliku modyfikuje zawartości tego pliku stosownie do poleceń wydanych sed'owi}\\
\textit{Wskazówka 2: dla uproszczenia możesz przyjąć że napisy te składają się jedynie z liter i cyfr.}
\fi

	\insertZadanie{booklets-sections/linux/zadania-60-napisy.tex}{passwd_warunek_na_uid_noawk}{}
	\insertZadanie{booklets-sections/linux/zadania-60-napisy.tex}{rekurecyjne_wyszukaj_i_zastap}{}

\section{Zadania dodatkowe}
	\dbEntryInsert{booklets-sections/linux/zadania_dodatkowe.tex}{zadania_bash}

\vspace{6pt}

\rozwiazania

\copyrightFooter{
	© Matematyka dla Ciekawych Świata, 2017-2020.\\
	© Robert Ryszard Paciorek <rrp@opcode.eu.org>, 2003-2020.\\
	Kopiowanie, modyfikowanie i redystrybucja dozwolone pod warunkiem zachowania informacji o autorach.
}
\end{document}
