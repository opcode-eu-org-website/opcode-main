\documentclass[fontSize=10pt, rmargin=3cm, extra]{pdfArticle}

\title {Advanced PDF Demo}
\author{Robert Ryszard Paciorek <rrp@opcode.eu.org>}
\date  {2019-05-13}

\makeatletter\hypersetup{
	pdftitle = {\@title},
	pdfauthor = {\@author}
}\makeatother


% set line stretch, paragraph indent and skip
\setstretch{1.1} \setlength{\parindent}{0pt} \setlength{\parskip}{5pt}

% include Examples related functions
\input{LaTeX-demos-examples.tex}

\usepackage{pdfpages} % włączanie wielustronnicowych PDFów
\usepackage{luatex85,attachfile2,embedfile}

% funkcje (załączniki, komentarze, ...) pdf-owe
\let\ulx\ul \let\stx\st \let\hlx\hl  \let\ul\relax \let\st\relax \let\hl\relax
\usepackage{pdfcomment}  % włącza solu :-(  ... bo używa całkiem ciekawego soulpos

\usepackage{lipsum}

\begin{document}

To demo prezentuje zagadnienia związane z funkcjami dokumentów PDF takimi jak załączanie plików, notatki, itp.
Dokument ten oparty jest na klasie \rrpPkgLink{pdfArticle},
	jednak większość prezentowanych zagadnień dotyczy pakietów nie wykorzystywanych w tej klasie.

\section{Zaawansowane funkcje PDFów}

\subsection{Załączanie plików}

Istnieją dwa sposoby załączania innych plików do plików pdf:
\begin{itemize}
\item globalnie (plik będzie dostepny np. w File → Embedded Files)
\item w postaci notatki zawierającej plik
\end{itemize}

Pakiet \pkgLink{embedfile} umożliwia osadzanie globalne pliku przy pomocy komendy \Verb$\embedfile$:

\vspace{7pt}
\begin{Example}
\embedfile[desc={definicja klasy dokumentu}]{pdfArticle.cls}
\end{Example}
\vspace{7pt}

Pakiet \pkgLink{attachfile2} umożliwia załączanie pliku w postaci notatki przy pomocy komendy \Verb$\attachfile$:

\vspace{7pt}
\begin{Example}
\attachfile[
		description={Plik źródłowy}, icon={Tag}
	]{latex-AdvancedPDF.tex}
\end{Example}
\vspace{7pt}

\subsection{Interaktywne formularze}
Pakiet \pkgLink{hyperref} zapewnia wsparcie dla tworzenia interaktywnych formularzy w PDFach. Poniżej pokazany jest przykład formularza stworzonego z użyciem tego pakietu.
\begin{ExampleVertical*}
\begin{Form}
	\vbox to 20pt{\vfill Ala ma:         \tabto{3cm}
		\TextField [bordercolor={}, default=bota, name=f1]{}
	}
	\vbox to 20pt{\vfill W wersji:       \tabto{3cm}
		\ChoiceMenu[bordercolor={}, combo, name=f2]{}{demo, v1.1, 2.0, innej}
	}
	\vbox to 30pt{\vfill Dodatkowe info: \tabto{3cm}
		\TextField [bordercolor={}, multiline=true, height=20pt, name=f3]{}
	}
	\vbox to 20pt{\vfill Akceptacja      \tabto{3cm}
		\CheckBox  [bordercolor={}, name=f4]{}
	}
\end{Form}
\end{ExampleVertical*}

\subsection{Inne funkcje}

\subsubsection{pakiet navigator}

Z pomocą pakietu \pkgLink{navigator} możliwe jest także tworzenie bardziej surowych linków PDFowych poprzez \Verb$\anchor{ID}$, odwoływać się do nich można poprzez: \Verb$\jumplink{ID}{tekst linku}$.

Umożliwia on także dodawanie pozycji do zakładek pdf'owych (czyli pdf'owego spisu treści) poprzez
	\Verb$\outline{x}[outlineID]{tekst}$ (definiuje etykietę ID w miejscu wstawienia) lub
	\Verb$\outline[anchor=ID]{x}{tekst}$ (zakładka wskazuje na etykietę ID),
	gdzie x określa poziom w menu zakładek.
Zakładki te zostaną umieszczone po zakładkach związanych z automatycznym spisem sekcji.


\subsubsection{Notatki PDF - pakiet pdfcomment}
\begin{pdfsidelinecomment}[color=red,author={Inny Autor},linewidth=3mm,linesep=1cm]{ notatka na obu marginesach }
Pakiet \pkgLink{pdfcomment} jest chyba najbardziej rozbudowanym z pakietów umożliwiających dodawanie do plików pdf różnego rodzaju notatek.

(trochę rozciągacza żeby notatka na marginesie ładnie wyglądała) \lipsum[1-2]
\end{pdfsidelinecomment}

\vspace{1cm}

Z użyciem pdfcomment możemy komentować bloki tekstu (tak jak powyżej po bokach), wstawiać notatki na marginesach oraz w \pdfcomment[icon=Note,color=yellow,opacity=0.5,author={Inny Autor}]{Notatka umieszczona w tekście} tekście.
\pdfmargincomment[icon=Insert,color=red,author={Jeszcze Inny Autor}]{Notatka umieszczona na marginesie}

Możliwe jest też oznaczanie i komentowanie fragmentów tekstu poprzez
\pdfmarkupcomment[color=blue,markup=Highlight,author={Inny Autor 2}]{podświetlenia}{notatka do oznaczonego fragmentu nr 1},
\pdfmarkupcomment[color=blue,markup=Underline,author={Inny Autor 2}]{podkreślenia}{notatka do oznaczonego fragmentu nr 2} i
\pdfmarkupcomment[color=blue,markup=StrikeOut,author={Inny Autor 2}]{przekreślenia}{notatka do oznaczonego fragmentu nr 3}.
%Pakiet pozwala \pdftooltip{także na dodawanie "tooltips"}{to jest właśnie dymek związany z tamtym tekstem.}

3 cm wyżej i 7 na prawo od tego > \pdffreetextcomment[subject={free text comment},height=1cm,width=3cm,voffset=5cm,hoffset=3cm,opacity=0.5,color=yellow,author={Jeszcze Inny Autor}]{Notatka w prostokącie} < miejsca będzie notatka w prostokącie.

\subsection{Włączanie wielostronicowych plików PDF}

Pakiet \pkgLink{pdfpages} umożliwia włączanie plików PDF lub ich części do generowanego dokumentu. Pozwala na wybór stron z włączanego pliku, umieszczanie kilku stron na jednej stronie, dodawanie stron z włączanych plików do spisu treści, automatyczne skalowanie dodawanych stron lub dostosowywanie rozmiarów stron do dodawanych stron i wiele więcej.

Następna strona została włączona poleceniem:

\begin{CatchExample}
\includepdf[fitpaper=true]{/usr/share/texlive/texmf-dist/tex/latex/mwe/example-image-a5-landscape.pdf}
\end{CatchExample}

\putExampleVerbatimAdjust

\putExampleTeX

\end{document}
