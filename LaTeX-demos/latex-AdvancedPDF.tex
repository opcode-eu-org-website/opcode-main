\documentclass[fontSize=10pt, rmargin=3cm, extra]{pdfArticle}

\title {Advanced PDF Demo}
\author{Robert Ryszard Paciorek <rrp@opcode.eu.org>}
\date  {2019-05-13}

\makeatletter\hypersetup{
	pdftitle = {\@title},
	pdfauthor = {\@author}
}\makeatother


% set line stretch, paragraph indent and skip
\setstretch{1.1} \setlength{\parindent}{0pt} \setlength{\parskip}{5pt}

% include Examples related functions
\input{LaTeX-demos-examples.tex}

\usepackage{pdfpages} % włączanie wielustronnicowych PDFów
\usepackage{luatex85,attachfile2,embedfile}

% funkcje (załączniki, komentarze, ...) pdf-owe
\let\ulx\ul \let\stx\st \let\hlx\hl  \let\ul\relax \let\st\relax \let\hl\relax
\usepackage{pdfcomment}  % włącza solu :-(  ... bo używa całkiem ciekawego soulpos

\usepackage{lipsum}

\begin{document}

To demo prezentuje zagadnienia związane z funkcjami dokumentów PDF takimi jak załączanie plików, notatki, itp.
Dokument ten oparty jest na klasie \pkgLink{pdfArticle},
	jednak większość prezentowanych zagadnień dotyczy pakietów nie wykorzystywanych w tej klasie.

\section{Zaawansowane funkcje PDFów}

\subsection{Załączanie plików}

Istnieją dwa sposoby załączania innych plików do plików pdf:
\begin{itemize}
\item globalnie (plik będzie dostepny np. w File → Embedded Files)
\item w postaci notatki zawierającej plik
\end{itemize}

Pakiet \pkgLink{embedfile} umożliwia osadzanie globalne pliku przy pomocy komendy \Verb$\embedfile$:

\vspace{7pt}
\begin{Example}
\embedfile[desc={definicja klasy dokumentu}]{pdfArticle.cls}
\end{Example}
\vspace{7pt}

Pakiet \pkgLink{attachfile2} umożliwia załączanie pliku w postaci notatki przy pomocy komendy \Verb$\attachfile$:

\vspace{7pt}
\begin{Example}
\attachfile[
		description={Plik źródłowy}, icon={Tag}
	]{latex-AdvancedPDF.tex}
\end{Example}
\vspace{7pt}

\subsection{Interaktywne formularze}
Pakiet \pkgLink{hyperref} zapewnia wsparcie dla tworzenia interaktywnych formularzy w PDFach. Poniżej pokazany jest przykład formularza stworzonego z użyciem tego pakietu.
\begin{ExampleVertical*}
\begin{Form}
	\vbox to 20pt{\vfill Ala ma:         \tabto{3cm}
		\TextField [bordercolor={}, default=bota, name=f1]{}
	}
	\vbox to 20pt{\vfill W wersji:       \tabto{3cm}
		\ChoiceMenu[bordercolor={}, combo, name=f2]{}{demo, v1.1, 2.0, innej}
	}
	\vbox to 30pt{\vfill Dodatkowe info: \tabto{3cm}
		\TextField [bordercolor={}, multiline=true, height=20pt, name=f3]{}
	}
	\vbox to 20pt{\vfill Akceptacja      \tabto{3cm}
		\CheckBox  [bordercolor={}, name=f4]{}
	}
\end{Form}
\end{ExampleVertical*}

\subsection{Inne funkcje}

\subsubsection{pakiet navigator}

Z pomocą pakietu \pkgLink{navigator} możliwe jest także tworzenie bardziej surowych linków PDFowych poprzez \Verb$\anchor{ID}$, odwoływać się do nich można poprzez: \Verb$\jumplink{ID}{tekst linku}$.

Umożliwia on także dodawanie pozycji do zakładek pdf'owych (czyli pdf'owego spisu treści) poprzez
	\Verb$\outline{x}[outlineID]{tekst}$ (definiuje etykietę ID w miejscu wstawienia) lub
	\Verb$\outline[anchor=ID]{x}{tekst}$ (zakładka wskazuje na etykietę ID),
	gdzie x określa poziom w menu zakładek.
Zakładki te zostaną umieszczone po zakładkach związanych z automatycznym spisem sekcji.


\subsubsection{Notatki PDF - pakiet pdfcomment}
\begin{pdfsidelinecomment}[color=red,author={Inny Autor},linewidth=3mm,linesep=1cm]{ notatka na obu marginesach }
Pakiet \pkgLink{pdfcomment} jest chyba najbardziej rozbudowanym z pakietów umożliwiających dodawanie do plików pdf różnego rodzaju notatek.

(trochę rozciągacza żeby notatka na marginesie ładnie wyglądała) \lipsum[1-2]
\end{pdfsidelinecomment}

\vspace{1cm}

Z użyciem pdfcomment możemy komentować bloki tekstu (tak jak powyżej po bokach), wstawiać notatki na marginesach oraz w \pdfcomment[icon=Note,color=yellow,opacity=0.5,author={Inny Autor}]{Notatka umieszczona w tekście} tekście.
\pdfmargincomment[icon=Insert,color=red,author={Jeszcze Inny Autor}]{Notatka umieszczona na marginesie}

Możliwe jest też oznaczanie i komentowanie fragmentów tekstu poprzez
\pdfmarkupcomment[color=blue,markup=Highlight,author={Inny Autor 2}]{podświetlenia}{notatka do oznaczonego fragmentu nr 1},
\pdfmarkupcomment[color=blue,markup=Underline,author={Inny Autor 2}]{podkreślenia}{notatka do oznaczonego fragmentu nr 2} i
\pdfmarkupcomment[color=blue,markup=StrikeOut,author={Inny Autor 2}]{przekreślenia}{notatka do oznaczonego fragmentu nr 3}.
%Pakiet pozwala \pdftooltip{także na dodawanie "tooltips"}{to jest właśnie dymek związany z tamtym tekstem.}

3 cm wyżej i 7 na prawo od tego > \pdffreetextcomment[subject={free text comment},height=1cm,width=3cm,voffset=5cm,hoffset=3cm,opacity=0.5,color=yellow,author={Jeszcze Inny Autor}]{Notatka w prostokącie} < miejsca będzie notatka w prostokącie.

\subsection{Włączanie wielostronicowych plików PDF}

Pakiet \pkgLink{pdfpages} umożliwia włączanie plików PDF lub ich części do generowanego dokumentu. Pozwala na wybór stron z włączanego pliku, umieszczanie kilku stron na jednej stronie, dodawanie stron z włączanych plików do spisu treści, automatyczne skalowanie dodawanych stron lub dostosowywanie rozmiarów stron do dodawanych stron i wiele więcej.

Następna strona została włączona poleceniem:

\begin{CatchExample}
\includepdf[fitpaper=true]{/usr/share/texlive/texmf-dist/tex/latex/mwe/example-image-a5-landscape.pdf}
\end{CatchExample}

\putExampleVerbatimAdjust

\putExampleTeX


\section{Zewnętrzne narzędzia do manipulowania plikami PDF}

\subsection{pdfjam}

Skrypt \textit{pdfjam} (dystrybuowany w ramach \textit{texlive-extra-utils}) pozwala na wykorzystanie możliwości pakietu \textit{pdfpages} do operowania plikami PDF z poziomu linii poleceń bez konieczności tworzenia dokumentów \textit{.tex}.

\subsubsection{łączenie i dzielenie plików}

\begin{minted}{bash}
pdfjam --outfile OUT.pdf -- IN1.pdf 1,3 IN2.pdf
\end{minted}
Połączy strony 1 i 3 z \Verb$IN1.pdf$ oraz wszystkie strony z \Verb$IN2.pdf$ w \Verb$OUT.pdf$.
Cały dokument będzie miał jednakowy rozmiar i orientację stron (ustawiane opcjami \Verb$--paper$ / \Verb$--papersize$ i \Verb$--landscape$ / \Verb$--no-landscape$).
Dodanie opcji \Verb$--fitpaper 'true' --rotateoversize 'true'$ spowoduje obrócenie o 90stopni stron, które dzięki temu będą mogły być mniej pomniejszone.

\begin{minted}{bash}
pdfjam --outfile OUT.pdf -- IN.pdf 2
\end{minted}
Zapiszę drugą stronę z \Verb$IN.pdf$ jako \Verb$OUT.pdf$.

\subsubsection{obracanie i odbijanie}

\begin{minted}{bash}
pdfjam --outfile OUT.pdf --angle '90' --fitpaper 'true' -- IN.pdf
\end{minted}
Utworzy plik \Verb$OUT.pdf$ złożony z obróconych o 90 stopni (w stronę przeciwną do ruchu wskazówek zegara) stron pliku \Verb$IN.pdf$. Można podać dowolną wartość obrotu.

\begin{minted}{bash}
pdfjam --outfile OUT.pdf --reflect true --fitpaper 'true' -- IN.pdf
\end{minted}
Utworzy plik \Verb$OUT.pdf$ złożony z odbitych lustrzane lewo-prawo stron pliku \Verb$IN.pdf$.

Operacje mogą być składnae, np. celem uzyskania dokumentu odbitego góra-dół:
\begin{minted}{bash}
pdfjam --outfile /dev/stdout --angle '90' --fitpaper 'true' -- IN.pdf |
	pdfjam --outfile /dev/stdout --fitpaper 'true' --reflect true |
	pdfjam --outfile OUT.pdf --angle '270' --fitpaper 'true'
\end{minted}

\subsubsection{kilka stron na jednej}

\begin{minted}{bash}
pdfjam --outfile OUT.pdf --nup 2x1 --landscape IN.pdf
\end{minted}
Utworzy plik \Verb$OUT.pdf$ gdzie każda strona będzie składała się z dwóch umieszczonych obok siebie stron pliku \Verb$IN.pdf$.

\begin{minted}{bash}
pdfjam --outfile OUT.pdf --nup 2x2 --frame true --delta "0.2cm 0.1cm" -- IN.pdf
\end{minted}
Utworzy plik \Verb$OUT.pdf$ gdzie każda strona będzie składała się z czterech stron pliku \Verb$IN.pdf$.
Strony będą odsunięte o 0.2cm w poziomie i 0.1cm w pionie (opcja \Verb$--delta$) i otoczone ramką (opcja \Verb$--frame$).

Domyślnie stosowany jest układ wierszowy (strona 1 i 2 w pierwszym wierszu, poniżej strona 3 i 4), można to zmienić dodając opcję \Verb$--column true$:
\begin{minted}{bash}
pdfjam --outfile OUT.pdf --nup 2x2 --column true -- IN.pdf
\end{minted}

Więcej możliwości dają także takie jak opcje: \Verb$--preamble$ (wstawia swój argument do preambuły), \Verb$--pagecommand$ (wstawia swój argument na każdej generowanej stronie).
Dostępne są też inne opcje pakietu \pkgLink{pdfpages}, takie jak np. \Verb$columnstrict$, czy też  \Verb$booklet$ i \Verb$signature$ pozwalające na tworzenie układu książeczkowego:

\begin{minted}{bash}
pdfjam --outfile OUT.pdf --booklet 'true' --signature '4' --landscape -- IN.pdf
\end{minted}

Przy pomocy \Verb$--preamble$ i pakietu \pkgLink{everyshi} można na przykład obrócić każdą parzystą / nieparzystą stronę pliku wynikowego:
\begin{minted}{bash}
pdfjam --outfile OUT.pdf --preamble '\usepackage{everyshi}\makeatletter \EveryShipout{\ifodd\c@page\pdfpageattr{/Rotate 180}\fi} \makeatother' -- IN.pdf
\end{minted}


\subsection{poppler-utils}

Zbiór narzędzi związany z obsługą PDFów dostarcza też pakiet \textit{poppler-utils}. Są to m.in.:

\begin{itemize}
	\item \Verb$pdfinfo$ – informacje na temat dokumentu PDF
	\item \Verb$pdfimages$, \Verb$pdfdetach$, \Verb$pdftotext$ – eksport obrazków, załączników i tekstu z plików PDF
	\item \Verb$pdfseparate$ – podział pliku PDF na strony
	\item \Verb$pdfunite$ – łączenie plików PDF
	\item \Verb$pdftops$, \Verb$pdftocairo$, \Verb$pdftoppm$, \Verb$pdftohtml$ – konwersja na PostScript, obrazki, HTML
\end{itemize}


\subsection{pdftk}

\subsubsection{łączenie i dzielenie plików}

\begin{minted}{bash}
pdftk A=IN1.pdf B=IN2.pdf cat A1 A3 B output OUT.pdf
\end{minted}
Połączy strony 1 i 3 z \Verb$IN1.pdf$ oraz wszystkie strony z \Verb$IN2.pdf$ w \Verb$OUT.pdf$.
Strony zachowają swój rozmiar i orientację.

\begin{minted}{bash}
pdftk IN.pdf cat 2 output OUT.pdf
\end{minted}
Zapiszę drugą stronę z \Verb$IN.pdf$ jako \Verb$OUT.pdf$.

Przy użyciu opcji \Verb$shuffle$ (zamiast \Verb$cat$) możliwe jest mieszanie stron z kilku zakresów, np.:
\begin{minted}{bash}
pdftk A=IN.pdf shuffle A1-8 A16-9 output out.pdf
\end{minted}

\subsubsection{obracanie i odbijanie}

\begin{minted}{bash}
pdftk IN.pdf cat 1-endeast output OUT.pdf
\end{minted}
Utworzy plik \Verb$OUT.pdf$ złożony z obróconych o 90 stopni (w stronę zgodną z ruchem wskazówek zegara) stron pliku \Verb$IN.pdf$. Można dla różnych stron stosować różne obroty, np.:
\begin{minted}{bash}
pdftk IN.pdf cat 1-2east 2-3south 4-5 6-endwest output OUT.pdf
\end{minted}

\subsubsection{zabezpieczanie hasłem}

\begin{minted}{bash}
pdftk IN.pdf output OUT.pdf encrypt_128bit user_pw TAJNE-HASLO
\end{minted}
Wygeneruje plik \Verb$OUT.pdf$, którego odczyt będzie wymagał podania hasła \Verb$TAJNE-HASLO$.

\begin{minted}{bash}
pdftk IN.pdf TAJNE-HASLO foopass output OUT.pdf
\end{minted}
W oparciu o plik \Verb$IN.pdf$ zabezpieczony hasłem \Verb$TAJNE-HASLO$ wygeneruje plik \Verb$OUT.pdf$, który nie będzie zabezpieczony hasłem.

Możliwe jest też zabezpieczanie wybranych operacji na pliku PDF (drukowania, kopiowania tekstu, itd) – przykłady w \Verb$man pdftk$.

\subsubsection{stemplowanie}

Pdftk pozwala też na stemplowanie dokumentu PDF jakimś obrazkiem / innym dokumentem PDF. Poniższy skrypt:
\begin{itemize}
	\item oblicza sumę kontrolną md5 pliku PDF
	\item generuje (przy pomocy \Verb$gs$) plik z informacją na temat tej sumy
	\item tworzy (przy pomocy \Verb$pdftk$) nowy plik poprzez nadrukowanie tej informacji na oryginalnym pliku
\end{itemize}
Zastosowaniem tego skryptu jest umożliwienie łatwego umieszczenia informacji identyfikującej źródło (wraz z wersją) na wydrukowanym pliku.

\begin{minted}{bash}
#!/bin/bash

# script to add md5 sum and other info to pdf for printing

if [ $# -lt 1 ]; then
	echo USAGE $0 inputfile.pdf [up]
	exit
fi

INPUT="$1"
UP=false; [ "$2" = "up" ] && UP=true

MD5=`md5sum "${INPUT}" | cut -f1 -d' '`
INFO=`pdfinfo "${INPUT}" | grep '^Page size:'`
INFO_X=`echo $INFO | awk '{print $3 * 10}' | cut -f1 -d.`
INFO_Y=`echo $INFO | awk '{print $5 * 10}' | cut -f1 -d.`
OFFSET_X="32"
if $UP; then
  OFFSET_Y=$[ $INFO_Y/10 - 64 ]
else
  OFFSET_Y="32"
fi

STAMPFILE=`mktemp`
gs -o ${STAMPFILE} -sDEVICE=pdfwrite -g${INFO_X}x${INFO_Y} \
  -c "/Helvetica-Bold findfont 8 scalefont setfont" \
  -c "0 .8 .8 0 setcmykcolor" -c "${OFFSET_X} ${OFFSET_Y} moveto" \
  -c "(MD5: ${MD5}  FILE: ${INPUT}) show" -c "showpage"

pdftk "${INPUT}" stamp ${STAMPFILE} output "${INPUT%.pdf}__MD5__.pdf"

\rm ${STAMPFILE}
\end{minted}

Możliwe jest też dodawanie tego typu nadruków w warstwie tła (opcje \Verb$background$).

\subsubsection{inne}

Więcej zastosowań opisanych jest w \Verb$man pdftk$.


\subsection{edycja PDFów, SVG}

\begin{itemize}
	\item \Verb$inkscape$ – edycja plików PDF, konwersja na SVG, ...
	\item \Verb$pdf2svg$ – konwersja na SVG
\end{itemize}

Inkscape jest nie tylko bardzo skutecznym edytorem plików PDF ale pozwala także m.in. na wsadową konwersję PDF do SVG poprzez:
\begin{minted}{bash}
inkscape --without-gui --file=IN.pdf --export-plain-svg="OUT.svg"
\end{minted}
Taka konwersja (w odróżnieniu od \Verb$pdf2svg$) pozwala na zachowanie tekstu z pliku PDF jako tekstu w SVG.

\subsection{PS → EPS → PDF}

Bezpośrednia konwerska pliku PS (PostScript) na PDF powoduje zachowanie rozmiaru strony, marginesów, itp.
Nie jest to pożądane jeżeli chcemy wykorzystać uzyskany plik PDF do umieszczenia jako obrazek wektorowy przy pomocy \Verb$\includegraphics$.
Należy wtedy przed konwersją na PDF przekonwertować plik PS najpierw na EPS a dopiero potem na PDF.

\begin{itemize}
	\item \Verb$ps2pdf$ (pakiet \textit{ghostscript}) – konwersja PS na PDF
	\item \Verb$ps2eps$, \Verb$ps2epsi$ (pakiet \textit{ghostscript}) – konwersja PS na EPS
	\item \Verb$epstopdf$ (pakiet \textit{texlive-font-utils}) – konwersja EPS na PDF
	\item \Verb$pdf2ps$ (pakiet \textit{ghostscript}) – konwersja PDF na PS
\end{itemize}

Pakiet \textit{ghostscript} dostarcza też polecenia do operowania na postscriptowych plikach PS i EPS, takie jak m.in. \Verb$eps2eps$, \Verb$ps2ps$.

\end{document}
