\documentclass[fontSize=10pt, extra]{pdfArticle}

\title {Lists and Tables Demo}
\author{Robert Ryszard Paciorek <rrp@opcode.eu.org>}
\date  {2019-05-13}

\makeatletter\hypersetup{
	pdftitle = {\@title},
	pdfauthor = {\@author}
}\makeatother


% set line stretch, paragraph indent and skip
\setstretch{1.1} \setlength{\parindent}{0pt} \setlength{\parskip}{7pt}

% include Examples related functions
\input{LaTeX-demos-examples.tex}

%\RequirePackage[ampersand]{easylist}   % easylist environment (for creating multilevel items list with ampersand (&) character
\usepackage[ampersand]{labels4easylist} % (auto next level via &&, &&&, etc) instead of using \item command ans nested environments)

% tabele i okolice
\usepackage{multirow,longtable} % lepsze, elastyczniejsze tabele
\usepackage{diagbox} % komórki dzielone na ukos
\usepackage{colortbl} % kolorowanie tabel, musi być PRZED ehhline (bo redefiniuje funkcje z pakietu hhline)
\usepackage{vtable,ehhline}
\usepackage{dashrule} % zaawansowane linie
\usepackage{tabu,spreadtab}

\usepackage{lipsum}

\begin{document}

To demo prezentuje zagadnienia związane z formatowaniem list oraz tabel.
Dokument ten oparty jest na klasie \rrpPkgLink{pdfArticle},
	jednak większość prezentowanych zagadnień dotyczy pakietów nie wykorzystywanych w tej klasie.

\vspace{-0.3cm}

\section{Listy numerowane i wypunktowywane}

Podstawowym środowiskiem do tworzenia różnego rodzaju list jest środowisko \Verb$list$ przyjmujące dwa argumerty:
	pierwszy określa etykietę dla poszczególnych punktów, a drugi określa formatownaie tekstu danego punktu.

\begin{Example}
\begin{list}{\color{red}→}{\color{blue}}
\item aa
\item bb
\end{list}
\end{Example}

Środowisko to żadko używane jest bezpośrednio dużo częściej stosowane są środowiska \Verb$itemize$, \Verb$enumerate$ oraz \Verb$description$.
Ich bardziej rozbudowane i konfigurowalne warianty dostarcza pakiet \pkgLink{enumitem}.

\begin{Example}
\begin{enumerate}
	\item aa
	\item bb
	\begin{enumerate}
		\item bb aa
		\begin{enumerate}
			\item bb aa aaa
			\item bb aa bbb
		\end{enumerate}
		\item bb bb
	\end{enumerate}
	\item cc
\end{enumerate}
\end{Example}

Powyżej zamieszczona jest domyślnie sformatowana lista numerowana, poniżej zamieszczam przykład własnej definicji listy numerowanej opartej na tym pakiecie z odmiennym formatowaniem:

\begin{CatchExample*}{1}
% definiujemy środowisko enumIII oferujące listę z:
%   3 poziomami numerowanymi cyframi arabskimi, jednym literami i jednym wypunktowywanym
\newlist{enumIII}{enumerate}{5}
\setlist[enumIII,1]{label=\arabic*., ref=\arabic*}
\setlist[enumIII,2]{label=\theenumIIIi.\arabic*., ref=\theenumIIIi.\arabic*}
\setlist[enumIII,3]{label=\theenumIIIii.\arabic*., ref=\theenumIIIii.\arabic*}
\setlist[enumIII,4]{label=\alph*), ref=\theenumIIIiii.\alph*}
\setlist[enumIII,5]{label=$\bullet$, ref=\theenumIIIiv.??}

% definiujemy komendę wstawiającą item wypisywany pogróbioną i powiększoną czcionką
\newcommand{\hitemB}[1]{\large\bfseries\item #1 \mdseries\normalsize}
\end{CatchExample*}

\begin{CatchExample*}{2}
\setlist[enumIII]{before={\upshape}}
\setlist[enumIII,2]{before={\itshape}, label=\theenumIIIi.\arabic*., ref=\theenumIIIi.\arabic*}
\end{CatchExample*}

\putExampleTeX[1]
\putExampleTeX[2]

\begin{Example*}
\begin{enumIII}
	\hitemB{ pierwszy nagłówek }
	\begin{enumIII}
		\item aa  \item bb  % dwa itemy w jednej linii ...
		\begin{enumIII}
			\item bb aa
			\begin{enumIII}
				\item bb aa aaa \label{e11}
				\begin{enumIII}
					\item xx xx
				\end{enumIII}
			\end{enumIII}
		\end{enumIII}
		\item ee \label{e12}
	\end{enumIII}
	\hitemB{ drugi nagłówek }
		\begin{enumIII}
		\item ref: e11 → \ref{e11}\\e12 → \ref{e12}
		\end{enumIII}
	\end{enumIII}
\end{Example*}

Środowisko \Verb$enumIII$ oraz polecenie \Verb$\hitemB$ zostało zdefiniowane w pakiecie \rrpPkgLink{styles4lists} w następujący sposób:
\putExampleVerbatimAdjust[1]

Definicja ta obejmuje konfigurację sposobu w jaki będą prezentowane referencje do etykiet ustawionych w ramach listy.
W tym wypadku jest to pełny numer punktu rozdzielany kropkami, jak pokazano na powyższym przykładzie.

Dodatkowo zostało włączone wypisywanie kursywą 2 poziomu listy poprzez modyfikację definicji \Verb$enumIII$:
\putExampleVerbatimAdjust[2]


Innym pakietem umożliwiającym zaawansowaną konfigurację list jest \pkgLink{easylist}.
Dodatkowo pozwala on także na prostsze tworzenie wielopoziomowych list bez konieczności tworzenia zagnieżdżonych środowisk
(poziom listy jest określany przez ilość znaków oznaczających punkt listy np. \Verb$&$).

\begin{CatchExample*}{1}
\def\initStdList{\NewList(
	% separator pomiędzy numerami oraz separatory końcowe
	Mark=.,FinalMark=.,
	% ustawienie odstępów między punktami
	Space=3pt,Space*=1pt,
	% ustawienia wcięć
	Hang=true,Align=move,FinalSpace=0.5em,Progressive*=3ex,Margin1=1ex,
)}
\def\initEnumIII{\initStdList\ListProperties(
	% na poziomie 4 uzywamy liter z separtorem końcowym ")"
	% i nie dołączamy numeracji wyższych poziomów
	Numbers4=l,Hide4=3,FinalMark4={)},Margin4=8ex,
	% na poziomie 5 używamy wypunktowywania z użyciem kropki
	Hide5=5,Style5*=$\bullet$ ,
)}
\end{CatchExample*}
\putExampleTeX[1]

\begin{Example*}
\begin{easylist}\initEnumIII\ListProperties(
	% formatowanie tekstu przypisane do 1go poziomu
	Style1=\large\bf
)
& pierwszy nagłówek
&& aa
&& bb
&&& bb aa
&&&& bb aa aaa \label{e21} \itemLabel{e21a}
&&&&& xx xx
&& ee  \label{e22}
& drugi nagłówek
&& ref: e21 → \ref{e21} vs \ref{e21a}\\e22 → \ref{e22}
\end{easylist}
\end{Example*}

Komenda konfiguracyjna \Verb$\initEnumIII$ została zdefiniowana w pakiecie \rrpPkgLink{styles4lists} w następujący sposób:
\putExampleVerbatimAdjust[1]

Pakiet \texttt{easylist} nie pozwala wpływać na sposób formatowania referencji do poszczególnych punktów (jest ona zawsze tożsama z wypisanym numerem danego punktu).
Dlatego w przykładzie została użyta \Verb$\itemLabel$ komenda z pakietu \rrpPkgLink{labels4easylist}, pozwalająca na tworzenie bardziej elastycznych etykiet.


\section{Wyrównanie tabulatorowe}

Pakiet \pkgLink{tabto} pozwala na stosowanie wyrównania względem tabulatora.
Pozycje tabulatora mogą być określane każdorazowo bądź z góry ustalone przy pomocy komendy \Verb$\tab$.
W połączeniu z komendą \Verb$\makebox$ możliwe jest wycentrowanie i wyrównanie do prawej tekstu umieszczanego na pozycji tabulatora.

\begin{ExampleVertical}
aa \tabto{3cm} bb \tabto{7cm}\makebox[0.1pt][c]{ccc ccc} \tabto{11cm}\makebox[0.1pt][r]{rrrr} \par
aaa a aaa aa a \tabto{3cm} bbb b bb \tabto{7cm}\makebox[0.1pt][c]{ccc cc c ccc}
	\tabto{11cm}\makebox[0.1pt][r]{rrr rr rrrr}
\end{ExampleVertical}


\section{Tabele}

\begin{ExampleVertical}
\begin{tabular}{ p{2.9cm} || l | r c | r@{.}l || D{.}{.}{-1} }
długi tekst który zostanie połamany &  bb & cc & dd & 2&718281 & 3.141592 \tabularnewline \hline
% dwie metody wyrównywania wg kropki dziesiętnej: dwie kolumny rozdzielane kropką, lub kolumna typu D
% dwie metody kończenia wiersza: \tabularnewline działa jak \\ ale bywa bezpieczniejszy dla formatowania
AAA & \multirow{2}{2.5cm}{dwu wierszowa, łamie}  & \multicolumn{2}{c|}{dwukolumnowa} & 112&57 & 13.16 \\
%        \multicolumn wymaga ustawienia stylu prawego obramowania  ^^^ w przeciwnym razie go nie będzie
BBB &                                     & xx & yy & 98765&12356 & 1716.13 \\ \hline\hline
%     ^^^ pusta komórka ze względu na \multirow powyżej (w tej kolumnie)
CCC & ale tylko wewnątrz niej ... & x2 & x3 & 12345 & & 67890
\end{tabular}
\end{ExampleVertical}

Powyżej pokazana jest prosta \LaTeX'owa tabelka, korzysta ona z następujących pakietów: \pkgLink{array}, \pkgLink{dcolumn} oraz \pkgLink{multirow}.
Oprócz nich także wiele innych pakietów rozszerza możliwości związane z tworzeniem tabel.
Poniższa tabelka korzysta dodatkowo z pakietów: \pkgLink{diagbox}, \pkgLink{colortbl}, \pkgLink{hhline} (wraz z rozszerzeniem w \pkgLink{ehhline}).
Użyty longtable pozwala na dzielenie tej tabelki pomiędzy strony.

\begin{ExampleVertical}
\begin{longtable}{
    || >{\raggedleft}p{3cm}      % z łamaniem linii o szerokości 3cm, wyrównana do prawej
    || r                         % wyrównana do prawej
    || >{\color{yellow}\columncolor{gray}[.5\tabcolsep]}c % żółty tekst na szarym tle
     | c
       !{\color{blue}\vrule width 3pt} % ta kolumna pełni rolę obramowania ...
 }
  \hhline{|t:==:t:==|}
    \diagbox[width=3.4cm]{AA}{BB}{CC} & pierwsza & \multicolumn{1}{c}{druga} & trzecia \\
  \hhline{|:=||=>{\arrayrulecolor{red}}=>{\arrayrulecolor{black}}=|}
    \textcolor{red}{aa bbb cccc dd eeee} &
    \multirow{2}{2.5cm}[-0.65cm]{xxx yy zzzz yy xxxx} &
    ccc ccc &
    112.5 \tabularnewline
  \hhline{#=#~||~|->{\color{blue}\vrule width 3pt}} % >{...} zapewnia odpowiednią linię po prawej
    QQQQ \linebreak od nowej linii &                  %       (zgodną z definicją w nagłówku)
    & % ta komórka powinna być putsta bo wyżej użyliśmy \multirow
    \diagbox[dir=SE,height=4\line]{AA AA}{CC CC} &
    \cellcolor[rgb]{0,1,1} 98765.12356 \tabularnewline
  \hhline{% !{...} jest rozszerzeniem \hhline zdefiniowanym w ehhline
     !{\leaders\hrule height 1pt\hfil}%    pozwala na tworzenie dowolnych
     !{\leaders\hbox{\tiny$\star$}\hfil}%  poziomych linii rozdzielających
     !{\hfil...\hfil...\hfil}%
     !{\leaders\hbox{\hdashrule{1.5mm}{1pt}{0.5mm 0.5mm 0.5mm 0mm}}\hfil}%
  }
\end{longtable}
\end{ExampleVertical}

\subsection{wyrównanie w pionie i poziomie}

Poniższa tabelka bazuje na zdefiniowanych w \rrpPkgLink{vtable} typach kolumn i komórek (wielowierszowych, wielokolumnowych, niezależnie formatowanych) - umożliwiają one poprawne wyrównywanie wysokości elementów w tabeli.

\begin{ExampleVertical}
\begin{tabular}{
    | C{2cm}{}{t} | C{2cm}{1.5cm}{m} | C{2cm}{}{b} | L{2cm}{}{m} |
    R{2cm}{}{t} | J{2cm}{}{t} I{\color{red}}{1pt}{0.5mm 0.5mm 0.5mm 0mm}
  }
  \hline
    top & middle & bottom & middle left & top right & top justify
  \nextRow \hline
    A \lb xxx \lb X & B & C &
      \setMultiRow{3}{- \lipsum[1][1] - \lb - \lipsum[1][2] -} &
      q & q
  \nextRow \hhline{---~--}
    D \lb d & E \lb xxx xx \lb X \lb X & F \lb f & & q & q
  \nextRow \hhline{---~--}
    G & H & I \lb xxx \lb Xj & & q & q
  \nextRow \hline
    G &
    \setMultiColumn{2}{4cm}{3cm}{c}{t}{}{|} {top, center: \lb \lipsum[1][3]} &
    xx & q & q
  \nextRow \hline
    G &
    \multicolumn{2}{Z{4cm}{3cm}{l}{2}{t}{1}|} {top, left: \lb \lb \lipsum[1][4]} &
    LL LL\lb xxx\lb X\lb X & RR RR\lb xxx\lb X\lb X & BB BB\lb xxx\lb X\lb X
  \nextRow \hline
    G &
    \tableFormatedCell{3.5cm}[3cm]{r}{b}I I \lb xxxxxxx \lb a a Xj &
    y & \lipsum[1][1] & \lipsum[1][1] & \lipsum[1][1]
  \nextRow \hline
\end{tabular}
\end{ExampleVertical}

\subsection{jeszcze więcej tabelek}

Istnieje także wiele innych pakietów bardziej gruntownie modyfikujących składanie tabel (jak choćby mdwtab, longtable czy tabularx). Najbardziej wszechstronnym i elastycznym wydaje się \href{http://mirrors.ctan.org/macros/latex/contrib/tabu/tabu.pdf}{tabu}. Powyższe tabelki są z nim kompatybilne (to znaczy pokazane efekty obramowań, wyrównań, rozciągniętych komórek itd możemy uzyskać w ten sam sposób).

\begin{ExampleVertical}
 \taburowcolors 2{gray!15 .. gray!50}
 \begin{tabu} {|[1pt blue]|[1pt green] m{2cm} || r || c | D{.}{.}{-1} }
  \tabucline[3pt on 1.5pt blue off 2pt red]{1-2}
  \tabucline[2pt white]{1-2}
  \tabucline[3pt on 1.5pt blue off 2pt red]{1-2}
    \diagbox[width=2.4cm]{AA}{BB}{CC} &
    BBB &
    ddd dd &
    0.12356 \\
  \hhline{|:=||=>{\arrayrulecolor{red}}=>{\arrayrulecolor{black}}=}
     aaaa aa bbb ddd eee &
     bbb aaaa aa  &
    ccc ccc &
    112.57 \\
  \tabucline[2pt red]{1-1}
  \tabucline[\hbox{\textcolor{blue}{x}}]{1-1}
  \tabucline[2pt red]{1-1}
    QQQQ &
    bbb ddd eee &
    \diagbox[dir=SE]{AA}{CC} &
    98765.12356 \\
 \end{tabu}
\end{ExampleVertical}

Jeżeli chcemy korzystać w ramach tabu z komórek rozciągniętych równocześnie na kilka kolumn i wierszy możemy deklarować je w sposób następujący:
\begin{MintedCode}
\multicolumn{2}{c|}{\multirow{3}{2\tabucolX}{\centering
	zawartość komórki
}}
% nawet przy samym multirow w kolumnach typu X należy jako drugi argument podawać \tabucolX
% tutaj mamy 2\tabucolX bo zajmujemy miejsce dwóch kolumnt typu X z powodu \multicolumn
\end{MintedCode}

Dzięki pakietowi \pkgLink{spreadtab} można nawet używać prostych arkuszy kalkulacyjnych.
\begin{Example}
 \begin{spreadtab}{{tabular}{ccc}}
  @A/1 & @B                  & @C       \\
  2    & 11                  & 23       \\
  3    & @ b2 + c2 =   & b2 + c2  \\
  4    & 54                  & 10       \\
  5    & :={c4} * c3 =       & c4 * c3
 \end{spreadtab}
\end{Example}

Tabele możemy umieszczać także poza głównym tekstem (analogicznie jak obrazki) poprzez wstawienie ich w środowisko "table". Dzięki użyciu \href{http://mirrors.ctan.org/macros/latex/contrib/threeparttable/threeparttable.pdf}{threeparttable} możliwe jest umieszczanie w takich tabelach niezależnych przypisów.
\end{document}
