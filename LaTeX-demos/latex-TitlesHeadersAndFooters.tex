\documentclass[fontSize=10pt,extra]{pdfArticle}

\title {Titles, Headers and Footers Demo}
\author{Robert Ryszard Paciorek <rrp@opcode.eu.org>}
\date  {2019-05-13}

\makeatletter\hypersetup{
	pdftitle = {\@title},
	pdfauthor = {\@author}
}\makeatother


% multilingual support
\RequirePackage{polyglossia}
\setdefaultlanguage{polish} % by default polish settings and names
                            % only this, NOT \RequirePackage{polski} NOR \RequirePackage[polish]{babel}
\setotherlanguages{english} % provide english language option too (for code printing)

% set line stretch, paragraph indent and skip
\setstretch{1.1} \setlength{\parindent}{0pt} \setlength{\parskip}{5pt}

% footnotes always on page bottom
\let\svfootnoterule\footnoterule
\renewcommand\footnoterule{\vfill\svfootnoterule}

% include Examples related functions
\input{LaTeX-demos-examples.tex}

\usepackage{titling}  % modyfikowanie \maketitle
\usepackage{titlesec} % kontrolowanie formatowania nagłówków sekcji

% naglowki, stopki, daty, ...
\usepackage{fancyhdr}  % nagłówek i stopka
\usepackage{lastpage}  % numer ostatniej strony wstawiany przez \pageref{LastPage}
\usepackage{datetime2} % formatowanie daty i czasu
\usepackage{lipsum}

\begin{document}

To demo prezentuje zagadnienia związane z formatowaniem nagłówków, tytułów i stopek.
Dokument ten oparty jest na klasie \pkgLink{pdfArticle},
	jednak większość prezentowanych zagadnień dotyczy pakietów nie wykorzystywanych w tej klasie.

\section{Tytuł dokumentu}

Stanadrowe polecenie \Verb$\maketitle$ pozwala na wstawienie tytułu zbudowanego w oparciu o informacje ustwaione poprzez polecenia \Verb$\title{}$, \Verb$\author{}$ i \Verb$\date{}$.

\begin{Example}
\title {To jest tytuł}
\author{A to jego autor}
\date  {1970-01-01}
% powyższe deklaracje mogą być zarówno w treści
% jak i preambule dokumentu (po lub przed \begin{document})
\maketitle
\end{Example}

Pakiet \pkgLink{hyperref} pozwala na ustawienie tytułu i autora umieszczanego w metadanych dokumentu. Umieszczając w preambule dokumentu:
\begin{MintedCode}
% wcześniej musi być ustawiony \title {...} i \author{...}
\makeatletter\hypersetup{
	pdftitle = {\@title},
	pdfauthor = {\@author}
}\makeatother
\end{MintedCode}
\vspace{-\parskip}
możemy ustawić je na wczesniej ustawione wartości tytułu i autora użuywane przez \Verb$\maketitle$.

Pakiet \pkgLink{titling} pozwala wpływac na formatowanie tytułu uzyskiwanego komendą \Verb$\maketitle$, np.:
\begin{Example}
% zmniejszamy odstęp przed tytułem
\renewcommand{\maketitlehooka}{\vspace{-44pt}}
% zmniejszamy odstęp pomiędzy tytułemn a autorem
\renewcommand{\maketitlehookb}{\vspace{-14pt}}
% zmniejszamy odstęp pomiędzy autorem a datą
\renewcommand{\maketitlehookd}{\vspace{-13pt}}
% zmniejszamy odstęp po dacie
\renewcommand{\maketitlehookd}{\vspace{-12pt}} 
\maketitle
\end{Example}

Jeżeli potrzebujemy bardzo dużych zmian w sposobie wyświetlania tutułu możemy redefiniować komendę \Verb$\maketitle$ np.
\begin{Example}
\makeatletter % powoduje że możemy używać @ w nazwach makr
\renewcommand{\maketitle}{
	Tytuł: \textcolor{red}{\@title}\\
	Autor: \textcolor{blue}{\@author}\\
	Data: \textcolor{green}{\@date}
}
\makeatother % przywracamy standardowe znaczenie @
\maketitle
\end{Example}

Należy zrócić uwagę na występowanie znaku \Verb$@$ w nazwach poleceń przechowujących tutuł, autora i datę.
Aby ich użyć konieczne jest obudowanie poprzez \Verb$\makeatletter$ i \Verb$\makeatother$.

\section{Tytuały sekcji}

\label{test1}

Standardowo w article na którym bazujemy dostępne są 3 poziomy nagłówków typowo z włączoną ich numeracją:\vspace{-20pt}
\begin{Example}
\section{sekcja}
\subsection{sub-sekcja}
\subsubsection{sub-sub-sekcja}
\end{Example}

Na fakt ich numerowania lub nie możemy wpływać np. poprzez ustawienie:\\ \Verb#\setcounter{secnumdepth}{2}# włączamy numerację do 2go poziomu włącznie.
Do bardziej zaawansowanego manipulowania numeracją można użyć redefiniowania komend \Verb#\thesection# \Verb#\thesubsection# itd. np.:
\begin{Example}
\renewcommand{\thesubsection}{
  \thesection-\arabic{subsection})
}
\renewcommand{\thesubsubsection}{
  \thesection-\arabic{subsection}-\arabic{subsubsection})
}
\subsection{subsection}
\subsubsection{subsubsection}
\end{Example}

Pakiet \pkgLink{titlesec} pozwala konfigurować sposób formatowania i wypisywania nagłówków sekcji.
Umożliwia konfigurowanie fontu, obramowań, zmianę sposobu wypisywania numeru sekcji
	(w odróżnieniu od modyfikowania \Verb$\thesubsection$ i jemu podobnych, te zmiany nie wpłyną na sposób wypisywania numerów w referencjach, spisach treści itp),
	zrezygnowanie z wypisywania numeru, itd.
Prosty przykład  użycia:
\begin{Example}
\titleformat{\subsection}
 {\large\color{red}} % format etykiety i tekstu
 {\arabic{subsection}.} % postać etykietu (tutaj numer z kropką)
 {.5em} % odległość etykiety od terkstu
 {\itshape}  % komenda do wykoania przed wstawieniem tekstu
\subsection{subsection}
\end{Example}

Polecenia \Verb$\section*$, \Verb$\subsection*$, \Verb$\subsubsection*$ wstawiają nienumerowaną sekcję bądź podsekcję.
Na numerację sekcji możemy wpływać także modyfikujac liczniki np.
\begin{Example}
\setcounter{section}{2}
\setcounter{subsection}{0}
\end{Example}

spowosuje że kolejna sub-sekcja będzie miała numer 2.1 zamiast 3.4.

\subsection{akapity i wyróżnienia tekstu}

Oprócz sekcji do wstawiania śródtytułów mamy także (standardowo nie numerowane i nie wstawiane do spisu treści) \Verb$\paragraph$ i \Verb$\subparagraph$.

Polecenia \Verb$\emph$ i \Verb$\strong$ pozwalają na wyróżnienie fragmentu tekstu.
Przypisane do nich formatowanie jest zależne od aktualnych ustawień.
Polecenia te możemy zagnieżdzać

\begin{Example}
\emfontdeclare{\itshape,\upshape\scshape,\slshape}
\emph{ABCdef \emph{ABCdef \emph{ABCdef}}}

\strongfontdeclare{\bfseries,\color{red}}
\strong{ABCdef \strong{ABCdef}}
\end{Example}

\vspace{8pt}
\section{Spisy treści}
Spis treści umieszczamy poprzez wstawienie \Verb$\tableofcontents$ aby nadać mu nazwę "Spis treści" korzystamy wcześniej z \Verb$\def\contentsname{Spis treści}$. Pozycję do spisu wprowadzają automatycznie polecenia takie jak: chapter, section, subsection, ..., możemy też wstawić ją ręcznie:
\begin{MintedCode}
  \addcontentsline{toc}  {nazwa_poziomu_np_section}  {tekst który ma zostać podany w spisie}
\end{MintedCode}
Aby decydować które poziomy są pokazywane w spisie treści można użyć \Verb$\setcounter{tocdepth}{x}$, gdzie x określa maksymalny numer poziomu pokazywanego w spisie.
Możemy też dodawać informacje dodatkowe np. dodatkowy odstęp: \Verb$\addtocontents{toc}{\protect\vspace{1ex}}$
Jeżeli chcielibyśmy mieć kropki (....) w spisie treści między tytułem sekcji a numerem strony możemy je uzyskać przy pomocy:
\begin{MintedCode}
\makeatletter   \renewcommand*\l@section{\@dottedtocline{1}{0cm}{0cm}}  \makeatother
\end{MintedCode}
Jeżeli chcielibyśmy pozbyć się numeracji w spisie treści możemy użyć:
\begin{MintedCode}
\makeatletter  \renewcommand*\mw@markandtoc{
  \addcontentsline{toc}{\mw@HeadingType}{\HeadingTOCText}
} \makeatother
\end{MintedCode}

\begin{Example}
\setcounter{tocdepth}{1}
\tableofcontents
\end{Example}

\section{Odnośniki}
Możemy tworzyć odnośniki wewnętrzne z wykorzystaniem: \Verb$\label{ID}$ (lub zdefiniowanego w \pkgLink{pdfArticle} \Verb$\namedLabel{ID}{nazwa}$) w miejscu na które wskazujemy. W miejscu w którym chcemy je przywołać (czyli umieścić link) możemy użyć: \Verb$\ref{ID}$ lub \Verb$\hyperref[ID]{tekst}$.
Możliwe jest także tworzenie odnośników zewnętrznych poprzez \Verb$\url{URL}$ (\url{http://www.opcode.eu.org/}) lub \Verb$\href{URL}{tekst}$.

\begin{Example}
Link do: \ref{test1} oraz \hyperref[test1]{
	inny link do tego samego miejsca
} w hyperref.
\end{Example}


\section{Przypisy}

Przypisy\footnote{to jest przypis}, możemy dodawać w tekście przy pomocy polecenia \Verb$\footnote{tekst przypisu}$. Są one umieszczane na dole strony, a w miejscu ich wstawienia umieszczany jest odpowiedni odnośnik.

Sposób wyświetlania przypisów możemy modyfikować np. redefiniując \@makefntext:
\begin{MintedCode}
\makeatletter\renewcommand{\@makefntext}[1]{%
  \setlength{\@tempdima}{\columnwidth}\addtolength{\@tempdima}{-9mm}%
  \protect\footnotesize\hspace{3mm}\makebox[6mm][l]{\@thefnmark.}\parbox[t]{\@tempdima}{#1}%
}\makeatother
\end{MintedCode}
W efekcie powyższej komendy przypisyb będą miały:
\begin{itemize}
\item dodatkowy lewy margines wielkości 3mm (zwiększona odległość numeru przypisu od lewej krawędzi strony)
\item stałej szerokości (6mm) pole przeznaczone na numer przypisu, następującą po nim kropkę
\item zmienny odstęp związany z niezapełnioną częścią pola stałej szerokości
\item tekst przypisu (złamane linie będą wyrównane do pierwszej, czyli wcięte o $3+6 = 9\textrm{\ mm}$
\end{itemize}

Przy pomocy poniższego kodu możemy wymusić aby przypisy były zawsze dociągane do dolnego marginesu:
\begin{MintedCode}
\let\svfootnoterule\footnoterule
\renewcommand\footnoterule{\vfill\svfootnoterule}
\end{MintedCode}

\section{Nagłówki i stopki}

Domyślnie nagłówek strony jest pusty a stopka zawiera numer strony.
Nagłówek i stopka mogą być zmieniano zarówno dla całego dokumentu (komendą \Verb$\pagestyle{}$) jak i dla danej strony (komendą \Verb$\thispagestyle{}$).
Argumentem tych poleceń jest styl nagłówka i stopki. Ustawienie w preambule \Verb$\pagestyle{empty}$ spowoduje że zarówno nagłówek jak i stopka będą puste.

Pakiet \pkgLink{fancyhdr} pozwala na swobodne konfigurowanie nagłówków i stopek. Numer bieżącej strony może zostać wstawiony poleceniem \Verb$\thepage$, a jeżeli chemy móc użyć numeru ostatniej strony przydatny będzie pakiet \pkgLink{lastpage}, pozwalający na wstawienie numeru ostatniej strony poprzez \Verb$\pageref*{LastPage}$ (wersja gwiazdkowana komendy \Verb$\pageref$ wstawia sam numer strony bez tworzenia klikalnego odnośnika).

\begin{CatchExample*}{1}
% date and time (iso 8601 format with spaces: %Y-%m-%d %H:%M:%S %z)
\renewcommand*{\DTMdisplaydate}[4]{\number#1-\DTMtwodigits{#2}-\DTMtwodigits{#3}}
\renewcommand*{\DTMDisplaydate}{\DTMdisplaydate}
\renewcommand*{\DTMdisplaytime}[3]{\DTMtwodigits{#1}:\DTMtwodigits{#2}:\DTMtwodigits{#3}}
\renewcommand*{\DTMdisplayzone}[2]{\ifnum#1<0-\else+\fi\DTMtwodigits{#1}\DTMtwodigits{#2}}
\renewcommand*{\DTMdisplay}[9]{
	\DTMdisplaydate{#1}{#2}{#3}{#4} 
	\DTMdisplaytime{#5}{#6}{#7} 
	\DTMdisplayzone{#8}{#9}
}
\renewcommand*{\DTMDisplay}{\DTMdisplay}
\end{CatchExample*}

\putExampleTeX[1]

Przykłąd użycia \pkgLink{fancyhdr} do zdefiniowania własnego stylu nagłówka i stopki:
\begin{CatchExample}
\fancypagestyle{stylSpecjalny}{%
  % usuwamy wszystkie formatowania
  \fancyhf{}%
  % w stopce na stronach niepatrzystych po prawej (RO), na parzystych po lewej (LE)
  % umieszczamy numer strony / numer ostatniej strony
  \fancyfoot[RO, LE]{\thepage~/~\pageref*{LastPage}}%
  % wyłączamy linię oddzielającą stopkę
  \renewcommand{\footrulewidth}{0pt}%
  % w ngłówku po środku umieszczamy datę i czas wygenerowania dokumentu
  \fancyhead[C]{ \DTMnow }%
  % ustawiamy grubosć linii oddzielającej nagłówek
  \renewcommand{\headrulewidth}{1pt}%
}
\thispagestyle{stylSpecjalny}
\end{CatchExample}

\putExampleTeX

\putExampleVerbatimAdjust

Użyta do wypisania daty i czasu komenda \Verb$\DTMnow$ pochodzi z pakietu \pkgLink{datetime2}.
Został on uprzednio skonfigurowany do wypisywania daty w stylu ISO8601 poprzez:

\putExampleVerbatimAdjust[1]

\section{Ustawienia językowe}

Przykładowy spis treści kilka stron temu został automatycznie zatytułowany w języku polskim dzięki ustawieniu w nagłówku dokumentu języka polskiego jako domyślnego:

\begin{MintedCode}
\RequirePackage{polyglossia}
\setdefaultlanguage{polish} % domyślne ustawienia dla języka polskiego
\setotherlanguages{english} % dostarczamy także ustawień dla języka angielskiego
                            % język moze zostać przełączony \begin{english} ... \end{english}
% używając pakietu polyglossia nie uzywamy już pakietu polski ani pakietu babel z opcją "polish"
\end{MintedCode}

Ustawienie to aktywuje także mechanizmy przenoszenia wyrazów odpowiednie dla wybranego języka.

W odróżnieniu od pakietu \texttt{polski} użyty pakiet \pkgLink{polyglossia} nie ustawia polonizacji środowiska matematycznego. Można ją uzyskać np. następującymi deklaracjami:

\begin{MintedCode}
\newcommand{\tg}{\ensuremath{\operatorfont tg}}
\newcommand{\ctg}{\ensuremath{\operatorfont ctg}}
\newcommand{\arctg}{\ensuremath{\operatorfont arctg}}
\newcommand{\arcctg}{\ensuremath{\operatorfont arcctg}}
\newcommand{\tgh}{\ensuremath{\operatorfont tgh}}
\newcommand{\ctgh}{\ensuremath{\operatorfont ctgh}}
\newcommand{\nwd}{\ensuremath{\operatorfont nwd}}
\AtBeginDocument{\let\le=\leqslant \let\ge=\geqslant}
\end{MintedCode}


\end{document}
